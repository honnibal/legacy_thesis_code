%
% File acl-ijcnlp2009.tex
%
% Contact  jshin@csie.ncnu.edu.tw
%%
%% Based on the style files for EACL-2009 and IJCNLP-2008...

%% Based on the style files for EACL 2006 by 
%%e.agirre@ehu.es or Sergi.Balari@uab.es
%% and that of ACL 08 by Joakim Nivre and Noah Smith

\documentclass[11pt]{article}

\addtolength{\floatsep}{-3mm}
\addtolength{\textfloatsep}{-10mm}
\addtolength{\dbltextfloatsep}{-6mm}

\usepackage{acl2010}
\usepackage{times}
\usepackage{url}
\usepackage{latexsym}
\usepackage{xspace}
\usepackage{natbib}

\usepackage{lcovington}
\usepackage{mattLI}
\usepackage{parsetree}
\usepackage{graphicx}

% 
% \newcommand{\cf}[1]{\mbox{$\it{#1}$}}   % category font
% \newcommand{\assign}{\vdash}
% 
% \newcommand{\gis}{\textsc{gis}\xspace}
% \newcommand{\iis}{\textsc{iis}\xspace}
% \newcommand{\pos}{\textsc{pos}\xspace}
% \newcommand{\wsj}{\textsc{wsj}\xspace}
% \newcommand{\ccg}{\textsc{ccg}\xspace}
% %\newcommand{\tag}{\textsc{tag}\xspace}
% \newcommand{\ltag}{\textsc{ltag}\xspace}
% \newcommand{\hpsg}{\textsc{hpsg}\xspace}
% \newcommand{\psg}{\textsc{psg}\xspace}
% \newcommand{\lbfgs}{\textsc{l-bfgs}\xspace}
% \newcommand{\bfgs}{\textsc{bfgs}\xspace}
% \newcommand{\cfg}{\textsc{cfg}\xspace}
% \newcommand{\pcfg}{\textsc{pcfg}\xspace}
% \newcommand{\nlp}{\textsc{nlp}\xspace}
% \newcommand{\dop}{\textsc{dop}\xspace}
% \newcommand{\lfg}{\textsc{lfg}\xspace}
% \newcommand{\pp}{\textsc{pp}\xspace}
% \newcommand{\cky}{\textsc{cky}\xspace}
% \newcommand{\gb}{\textsc{gb}\xspace}
% \newcommand{\mb}{\textsc{mb}\xspace}
% \newcommand{\ram}{\textsc{ram}\xspace}
% \newcommand{\mpi}{\textsc{mpi}\xspace}
% \newcommand{\cpu}{\textsc{cpu}\xspace}
% \newcommand{\cpp}{\textsc{c++}\xspace}
% \newcommand{\parseval}{\textsc{parseval}\xspace}
% \newcommand{\trec}{\textsc{trec}\xspace}
% \newcommand{\epsrc}{\textsc{epsrc}\xspace}
% \newcommand{\dt}{\textsc{dt}\xspace}
% \newcommand{\hmm}{\textsc{hmm}\xspace}
% \newcommand{\ghz}{\textsc{ghz}\xspace}
% \newcommand{\rasp}{\textsc{rasp}\xspace}
% \newcommand{\ldc}{\textsc{ldc}\xspace}
% \newcommand{\gr}{\textsc{gr}\xspace}
% \newcommand{\qa}{\textsc{qa}\xspace} 
% \newcommand{\jj}{\textsc{jj}\xspace}
% \newcommand{\vbn}{\textsc{vbn}\xspace}
% \newcommand{\cd}{\textsc{cd}\xspace}
% \newcommand{\rp}{\textsc{rp}\xspace}
% \newcommand{\cc}{\textsc{cc}\xspace}
% \newcommand{\susanne}{\textsc{susanne}\xspace}
% \newcommand{\bandc}{\textsc{b{\small \&}c}\xspace}
% \newcommand{\ctl}{\textsc{ctl}\xspace}
% \newcommand{\ccgbank}{CCGbank\xspace}
% \newcommand{\depbank}{DepBank\xspace}
% \newcommand{\markedup}{\emph{markedup}\xspace}
% \newcommand{\candc}{\textsc{c{\scriptsize\&}c}\xspace}
% \newcommand{\cg}{\textsc{cg}\xspace}
% \newcommand{\penn}{\textsc{ptb}\xspace}
% \newcommand{\develtwo}{\textsc{devel-2}}
% \newcommand{\deps}{\mbox{\em deps}}
% \newcommand{\cdeps}{\mbox{\em cdeps}}
% \newcommand{\dmax}{\mbox{\em dmax}}
% \newcommand{\unify}{\equiv}
% \newcommand{\nounify}{\neq}
% \newcommand{\dest}{\textsc{dest}\xspace}
% \newcommand{\nom}{\textsc{nom}\xspace}
% \newcommand{\tccg}{\textsc{tccg}\xspace}
% \newcommand{\abcg}{\textsc{ab} categorial grammar\xspace}
% \newcommand{\mmccg}{\textsc{mmccg}\xspace}
% \newcommand{\cfpsg}{\textsc{cf-psg}\xspace}
% \newcommand{\evalb}{evalb\xspace}
% \newcommand{\dol}{\textsc{dol}\xspace}
% \newcommand{\xtag}{\textsc{xtag}\xspace}
% \newcommand{\erg}{\textsc{erg}\xspace}
% \newcommand{\enju}{\textsc{enju}\xspace}
% \newcommand{\bbn}{\textsc{bbn}\xspace}
% \newcommand{\xle}{\textsc{xle}\xspace}
% \newcommand{\lingo}{\textsc{lingo}\xspace}
% 
% 
% \newcommand{\lexpunct}{LexPunct\xspace}
% \newcommand{\nattach}{NAttach\xspace}
% 
% %\newcommand{\unhat}{\textsc{unhat}\xspace}
% % Constants and things
% \newcommand{\calo}{\mathcal{O}}
% \newcommand{\caln}{\mathcal{N}}
% 
% % commands for xyling
% \newcommand{\unode}[2][]{\K{#1$_{#2}$}}
% \newcommand{\bnode}[2][]{\K{#1$_{#2}$}\V}
% \newcommand{\vpmod}{}
% \newcommand{\bks}{$\backslash$}
% 
% 
% \newcommand{\cn}{\emph{\[citation needed\]}\xspace}
% \newcommand{\eqnpsrule}[3]{#1\;\;#2 & \longrightarrow & #3}
% 
% \newcommand{\psbinary}[3]{\ensuremath{\langle #1\;\; #2\;\longrightarrow\;#3\rangle}}
% \newcommand{\psunary}[3]{\ensuremath{\langle #1\;\longrightarrow\;#2\rangle}}
% 
% \newcommand{\term}[1]{\emph{#1}}
% %\newcommand{\comment}[1]{\quote{#1}}
% 
% 
% 
% % Corpus names
% \newcommand{\nounary}{\textsc{no unary}\xspace}
% 
% % \newcommand{\noabsorb}{NoAbsorb\xspace}
% % \newcommand{\commaconj}{CommaConj\xspace}
% % \newcommand{\hattraise}{HatTRaise\xspace}
% % \newcommand{\rrcsubcat}{RRCSubcat\xspace}
% % \newcommand{\tcpunct}{TCPunct\xspace}
% 
% % Small-capped constants for AVM fields
% \newcommand{\ressc}{\textsc{res}\xspace}
% \newcommand{\argsc}{\textsc{arg}\xspace}
% \newcommand{\slashsc}{\textsc{dir}\xspace}
% \newcommand{\headsc}{\textsc{head}\xspace}
% \newcommand{\catsc}{\textsc{cat}\xspace}
% \newcommand{\featsc}{\textsc{feat}\xspace}
% 
% \newcommand{\citepos}[1]{\citeauthor{#1}'s~(\citeyear{#1})\xspace}
% \newcommand{\scare}[1]{#1}
% 
% \newcommand{\wikieval}{\textsc{wiki200}\xspace}
% \newcommand{\hatsys}{\textsc{hat}\xspace}
% \newcommand{\trsys}{\textsc{hat+tr}\xspace}
% \newcommand{\lexpunctsys}{\textsc{lex-punct}\xspace}
% \newcommand{\puresys}{\textsc{no~unary}\xspace}
% 
% \newcommand{\sew}{\textsc{sew}\xspace}
% \newcommand{\few}{\textsc{few}\xspace}
% 
% %\newcommand{\codeterm}[1]{\verb!#1!\xspace}
% \newcommand{\codeterm}[1]{\texttt{#1}\xspace}
% 
% \newcommand{\basesys}{BaseSys\xspace}
% 
% \newcommand{\smodetext}{$\star$\xspace}
% \newcommand{\xmodetext}{$\times$\xspace}
% \newcommand{\dmodetext}{$\diamond$\xspace}
% \newcommand{\cmodetext}{$\cdot$\xspace}
% \newcommand{\nullmodetext}{$\bowtie$\xspace}
% 
% \newcommand{\smode}{_\star}
% \newcommand{\xmode}{_\times}
% \newcommand{\dmode}{_\diamond}
% \newcommand{\cmode}{_\cdot}
% \newcommand{\nullmode}{_{\bowtie}}
% \newcommand{\nullhat}{\bowtie}
% 
% 
% 
% 
% \newcommand{\cB}{\textbf{\textsf{B}}}
% \newcommand{\cBx}{{\ensuremath{\textbf{\textsf{B}}_\times}}}
% \newcommand{\cBf}{>\!\cB\!\!}
% \newcommand{\cBb}{<\!\cB\!\!}
% 
% 
% \newcommand{\cT}{\textbf{\textsf{T}}}
% \newcommand{\cS}{\textbf{\textsf{S}}}
% \newcommand{\cH}{\textbf{\textsf{H}}}
% \newcommand{\bk}{\ensuremath{\backslash}}
% \newcommand{\ra}{\ensuremath{\rightarrow}}
% \newcommand{\Ra}{\ensuremath{\Rightarrow}}
% 
% \newcommand{\Sfapply}{\ensuremath{\!>\!}}
% \newcommand{\Sbapply}{\ensuremath{\!<\!}}
% \newcommand{\Sfcomp}{\ensuremath{\!>\!\!\cB\!}}
% \newcommand{\Sbcomp}{\ensuremath{\!<\!\cB\!}}
% \newcommand{\Sfxcomp}{\ensuremath{\!>\!\!\cB_{\!\times}\!}}
% \newcommand{\Sbxcomp}{\ensuremath{\!<\!\cB_{\!\times}\!}}
% \newcommand{\Sftype}{\ensuremath{\!>\!\!\cT\!}}
% \newcommand{\Sbtype}{\ensuremath{\!<\!\cT\!}}
% \newcommand{\Sfsubst}{\ensuremath{\!>\!\!\cS\!}}
% \newcommand{\Sbsubst}{\ensuremath{\!<\!\cS\!}}
% \newcommand{\Sfxsubst}{\ensuremath{\!>\!\!\cS_{\!\times\!}}}
% \newcommand{\Sbxsubst}{\ensuremath{\!<\!\cS_{\!\times\!}}}
% 
% \newcommand{\derivs}{\textsc{deriv}\xspace}
% \newcommand{\derivsbad}{\textsc{deriv}$_{0.1}$\xspace}
% \newcommand{\derivsrev}{\textsc{deriv}$_{0.0045}$\xspace}
% \newcommand{\derivsthree}{\textsc{deriv}$_{0.003}$\xspace}
% \newcommand{\derivsexp}{\textsc{deriv}$_{0.002}$\xspace}
% \newcommand{\hybrid}{\textsc{hybrid}\xspace}
% \newcommand{\hybridv}{\textsc{hybrid}$^v$\xspace}
% \newcommand{\optbeta}{\textsc{opt}\xspace}
% 
% \newcommand{\nltk}{\textsc{nltk}\xspace}
% \newcommand{\act}{\textsc{act}\xspace}
\renewcommand{\tabcolsep}{0.15cm}

%\setlength\titlebox{6.5cm}    % You can expand the title box if you
% really have to

\title{Rebanking: Grammar Engineering for Wide-Coverage Corpora}

\author{First Author\\
  Affiliation / Address line 1\\
  Affiliation / Address line 2\\
  {\tt email@domain}  \And
  Second Author\\
  Affiliation / Address line 1\\
  Affiliation / Address line 2\\
  {\tt  email@domain}}

% Various additions to existing treebanks have been created to
% partly solve this problem. 
% Treebanks tend to be used unchanged for many years despite any flaws, as
% they become community standards. Separate resources are then created
% to supply some of the information missing from the original corpus.
% This paper describes how these resources can be merged together. Our
% rebanking process involves crafting linguistic
% analyses that take advantage of all of the annotations available.

\date{}
\begin{document}
\maketitle
\begin{abstract}
Once released, treebanks tend to remain unchanged despite any shortcomings
in depth of linguistic analysis and coverage of specific
phenomena. Separate resources are then created to mitigate these problems.
In this paper we show how to improve the quality of a
treebank, by exploiting these existing resources, and
implementing improved analyses for specific constructions.

We demonstrate this rebanking process by creating an updated version of \ccgbank that
includes the  predicate-argument structure of both verbs and nouns,
base-NP brackets, verb-particle constructions,
and restrictive and non-restrictive nominal modifiers; and evaluate
the impact of each of these changes on a statistical parser.

\end{abstract}

\section{Introduction}

Progress in natural language processing relies on direct comparison
on shared data, discouraging improvements to the evaluation data. This means
that we often spend years competing to reproduce partially incorrect
annotations. It also encourages us to approach related problems as discrete
tasks, when a new data set that adds deeper information establishes a new
incompatible evaluation.

Direct comparison has been central to progress in statistical parsing, but it
has also caused problems. Treebanking is a difficult engineering task: coverage,
cost, consistency and granularity are all competing concerns that must be
balanced against each other when the annotation scheme is developed. The difficulty
of the task means that we ought to view treebanking as an ongoing process akin to grammar
development, such as the many years of work on the \erg \citep{erg}.

This paper demonstrates how a treebank can be \emph{rebanked} to incorporate 
novel analyses and information from existing resources. We chose to work
on \ccgbank \citep{hock:cl07}, a Combinatory Categorial Grammar \citep{steedman:00}
treebank acquired from the Penn Treebank \citep{marcus:93}. This work is
equally applicable to the corpora described by \citet{miyao:04}, \citet{shen:08}
or \citet{cahill:08}. 

Our first changes integrate four previously suggested improvements
to \ccgbank. We then describe a novel \ccg analysis of NP predicate-argument
structure, which we implement using \nombank \citep{nombank}.
Our analysis allows the distinction between core and peripheral
arguments to be represented for predicate nouns.

With this distinction, an entailment recognition system could recognise that
\emph{Google's acquisition of YouTube} entailed \emph{Google acquired YouTube},
because equivalent predicate-argument structures are built for both.
Our analysis also recovers non-local dependencies mediated by nominal predicates;
for instance, \emph{Google} is the agent of \emph{acquire}
in \emph{Google's decision to acquire YouTube}.

The rebanked corpus extends \ccgbank with:
\vspace{-0.03in}
\begin{enumerate}
\addtolength{\itemsep}{-.99\itemsep}
 \item NP brackets from \citet{vadas:08};
 \item Restored and normalised punctuation;
 \item Propbank-derived verb subcategorisation;
 \item Verb particle structure drawn from Propbank;
 \item Restrictive and non-restrictive adnominals;
 \item Reanalyses to promote better head-finding;
 \item Nombank-derived noun subcategorisation.
\end{enumerate}
\vspace{-0.03in}
Together, these changes modify 30\% of the labelled dependencies in
\ccgbank, demonstrating how multiple resources can be brought
together in a single, richly annotated corpus. We then train and evaluate
a parser for these changes, to investigate their impact on the accuracy
of a state-of-the-art statistical \ccg parser.

\section{Background and Motivation}
\label{background}

Formalisms like \hpsg \citep{pollard:94}, \lfg \citep{kaplan:82}, and \ccg
\citep{steedman:00} are \emph{linguistically motivated} in the sense that they
attempt to explain and predict the limited variation found in the
grammars of natural languages. They also attempt to specify how grammars
construct semantic representations from surface strings, which is why they are
sometimes referred to as \emph{deep} grammars. Analyses produced by these
formalisms are more detailed than those produced by skeletal phrase-structure
parsers, because they produce fully specified predicate-argument structures.

Unfortunately, statistical parsers do not take full advantage of this feature.
Statistical parsers induce their grammars from corpora, and the corpora being
used for this purpose currently do not contain high quality predicate-argument
annotation, because they were derived from the Penn Treebank
\citep[\penn][]{marcus:93}. Manually written grammars for these formalisms, such as the
\erg \hpsg grammar \citep{erg} and the \xle \lfg grammar \citep{xle} produce
far more detailed and linguistically correct analyses than any English
statistical parser, due to the comparatively coarse-grained annotation schemes of
the corpora statistical parsers are trained on. While rule-based parsers
use grammars that are carefully engineered, and can be updated to
reflect the best linguistic analyses, statistical parsers have so far had
to take what they are given.

What we suggest in this paper is that a treebank's grammar need not last its
lifetime. For a start, there have been many annotations of the \penn
that add much of the extra information needed to produce very high quality
analyses for a linguistically motivated grammar. There are also other
transformations which can be made with no additional information. That is,
sometimes the existing trees allow transformation rules to be written that
improve the quality of the grammar.

Linguistic theories are constantly changing, which means that there is a
substantial lag between what we (think we) understand of grammar and the
annotations in our corpora. The grammar engineering process we describe, which
we dub \emph{rebanking}, is intended to reduce this gap, tightening the
feedback loop between formal and computational linguistics.

\subsection{Combinatory Categorial Grammar}

Combinatory Categorial Grammar \citep[\ccg;][]{steedman:00}
is a lexicalised grammar, which means that all grammatical dependencies are
specified in the lexical entries and that the production of
derivations is governed by a small set of rules.

Lexical categories are either atomic (\cf{S}, \cf{NP}, \cf{PP}, \cf{N}),
or a functor consisting of a result, directional slash, and argument.
For instance, \emph{in} might head a 
\cf{PP}-typed constituent with one \cf{NP}-typed argument, written as
\cf{PP/NP}.

A category can have a functor as its result, so that a
word can have a complex valency structure. For instance, a verb phrase is
represented by the category \cf{S\bs NP}: it is a function from a leftward
\cf{NP} (a subject) to a sentence. A transitive verb requires an object to
become a verb phrase, producing the category \cf{(S\bs NP)/NP}.

A \ccg grammar consists of a small number of schematic rules, called combinators. 
\ccg extends the basic application rules of pure categorial
grammar with (generalised) composition rules and type raising.  The
most common rules are:
\vspace{0.1in}

\begin{tabular}{rrrrlc}
\cf{X/Y}   & \cf{Y}      &$\Rightarrow$& \cf{X}      & \hfill& (\Sfapply)\\
\cf{Y}     & \cf{X\bs Y} &$\Rightarrow$& \cf{X}      & \hfill&(\Sbapply)\\
\cf{X/Y}   & \cf{Y/Z}    &$\Rightarrow$& \cf{X/Z}    & \hfill&(\Sfcomp)\\
\cf{Y\bs Z}& \cf{X\bs Y} &$\Rightarrow$& \cf{X\bs Z} & \hfill&(\Sbcomp)\\
\cf{Y/Z}   & \cf{X\bs Y} &$\Rightarrow$& \cf{X/Z}    & \hfill&(\Sbxcomp)
\end{tabular}
\vspace{0.07in}
% \begin{eqnarray}
% %\addtolength{\itemsep}{-.8\itemsep}
% \cf{X/Y} & \cf{Y} & \Rightarrow_{\Sfapply}\;\;\;\;\;\;\; \cf{X}\nonumber\\
% \cf{Y} & \cf{X\bs Y} & \Rightarrow_{\Sbapply}\;\;\;\;\;\;\; \cf{X}\nonumber\\
% \cf{X/Y}    & \cf{Y/Z}    & \Rightarrow_{\Sfcomp}\;\;\;\;\; \cf{X/Z}\nonumber\\
% \cf{Y\bs Z} & \cf{X\bs Y} & \Rightarrow_{\Sbcomp}\;\;\;\; \cf{X\bs Z}\nonumber\\
% \cf{Y/Z} & \cf{X\bs Y} & \Rightarrow_{\Sbxcomp}\;\; \cf{X/Z}\nonumber
% \end{eqnarray}

\ccgbank \citep{hock:cl07} extends this compact set of combinatory rules
with a set of
type-changing rules, designed to strike a better balance between sparsity
in the category set and ambiguity in the grammar.
We mark type-changing rules \textbf{\textsc{TC}} in our derivations.

In wide-coverage descriptions, categories are generally modelled as
typed-feature structures \citep{shieber:86}, rather than atomic symbols.
This allows the grammar to include a notion of headedness, and to unify
under-specified features.

We occassionally must refer to these additional details, for which we employ
the following notation. Features are annotated in square-brackets, e.g.
\cf{S[dcl]}. Head-finding indices are annotated on categories in
 subscripts, e.g. \cf{(NP_y\bs NP_1)/NP_2}. The index of the word category
is assigned to is left implicit. We will sometimes
also annotate derivations with the heads of categories as they are being built, to help
the reader keep track of what lexemes have been bound to which categories.

\section{Combining CCGbank Corrections}
\label{sec:prev_corr}
There have been a few papers describing corrections to \ccgbank.
We bring these corrections
together for the first time, before building on them with our further changes.

\subsection{Compound Noun Brackets}

Compound noun phrases can nest inside each other, creating bracketing
ambiguities:

\begin{lexamples}
 \item (crude oil) prices\label{eg:bad_crude}
 \item crude (oil prices)
\end{lexamples}

The structure of such compound noun phrases is left underspecified in the Penn
Treebank (\penn), because the annotation procedure involved stitching together
partial parses produced by the Fidditch parser \citep{hindle:83}, which
produced flat brackets for these constructions. The bracketing
decision was also a source of annotator disagreement \citep{bies:95}.

When \citet{hock:lrec02} went to acquire a \ccg treebank from the \penn, this
posed a problem. There is no equivalent way to leave these structures
under-specified in \ccg, because derivations must be binary branching. They
therefore employed a simple heuristic: assume all such structures branch to
the right. Under this analysis, \emph{crude oil} is not a constituent,
producing an incorrect analysis as in (\ref{eg:bad_crude}).

\citet{vadas:07} addressed this by manually annotating all of the ambiguous
noun phrases in the \penn, and went on to use this information to correct 20,409
dependencies (1.95\%) in \ccgbank \citep{vadas:08}. Our changes build on this
corrected corpus.

\subsection{Punctuation Corrections}

The syntactic analysis of punctuation is notoriously difficult, and punctuation
is not
always treated consistently in the Penn Treebank \citep{bies:95}.
\citet{hock:thesis03} determined that quotation marks were particularly
problematic, and therefore removed them from \ccgbank altogether.
\citet{tse:08} produced a version of \ccgbank that restored the quotation marks,
and shifted commas so that they always attach to the constituent to their left.
This allowed a  grammar rule to be removed, preventing
a great deal of spurious ambiguity and improving the speed of the \candc parser
\citep{clark:cl07} by 37\%.

\subsection{Verb Predicate-Argument Corrections}

The adjuncts of verbal predicates are noisily distinguished from complements in
the \penn by function tags attached to some node labels. Complements are core
arguments of the predicate, from a role set specific to that word. Adjuncts are
peripheral arguments whose semantic role is drawn from a small set generic to
all predicates.

This distinction is represented in the surface syntax in \ccg, because the
category of a verb must specify its argument structure. In (\ref{verb_comp})
\emph{as a director} is annotated as a complement; in (\ref{verb_adj}) it
is an adjunct:

\begin{lexamples}
 \item \gll He joined as~a~director
       \cf{NP} \cf{(S\bs NP)/PP} \cf{PP}
       \gln 
       \glend \label{verb_comp}
\item \gll He joined as~a~director
       \cf{NP} \cf{S\bs NP} \cf{(S\bs NP)\bs (S\bs NP)}
       \gln 
       \glend \label{verb_adj}
\end{lexamples}

The noisy annotation in the \penn has been superceded by the predicate-argument
annotation layer provided by Propbank \citep{propbank}. These annotations have
since been used to correct the complement/adjunct distinctions in \ccgbank
\citep[Honnibal and Curran (2007);][]{boxwell:08}. As we are making similar
corrections using \nombank, we reimplemented the process, converting 1,543
complements to adjuncts and 13,256 adjuncts to complements.
These figures are quite close to those reported in the literature.

\subsection{Verb Particle Constructions}

Propbank also offers reliable annotation of verb-particle constructions. This
was not available in the \penn, so \citet{hock:cl07} annotated all intransitive
prepositions as adjuncts:

\begin{lexamples}
 \item \gll He woke up
       \cf{NP} \cf{S\bs NP} \cf{(S\bs NP)\bs (S\bs NP)}
       \gln 
       \glend \label{ccgbank_vpc}
\end{lexamples}

We follow \citet{constable:09} in exploiting the
\propbank annotations to add verb-particle distinctions to \ccgbank, by
introducing a new atomic category \cf{PT} for particles, and changing their status
from adjuncts to complements:

\begin{lexamples}
 \item \gll He woke up
       \cf{NP} \cf{(S\bs NP)/PT} \cf{PT}
       \gln 
       \glend \label{ccgbank_vpc}
\end{lexamples}

This analysis could be improved by adding extra head-finding logic to the
verbal category, to recognise the multi-word expression as the head.

\section{Noun Predicate-Argument Structure}
\label{sec:nom_parg}

\begin{figure*}
\centering
\scalebox{0.8}{
\deriv{7}{
\rm Rome & \rm 's & \rm gift & \rm of & \rm peace & \rm to & \rm Europe \\
\uline{1}&\uline{1}&\uline{1}&\uline{1}&\uline{1}&\uline{1}&\uline{1} \\
\cf{NP} &
\cf{(NP/(N/PP))\bs NP} &
\cf{(N/PP)/PP)/PP} &
\cf{PP/NP} &
\cf{NP} &
\cf{PP/NP} &
\cf{NP} \\
\bapply{2} && \fapply{2} & \fapply{2} \\
\mc{2}{\cf{N/(N/PP)}} && \mc{2}{\cf{PP}} & \mc{2}{\cf{PP}} \\
&& \fapply{3} \\
&& \mc{3}{\cf{(N/PP)/PP}} \\
&& \fapply{5} \\
&& \mc{5}{\cf{N/PP}} \\
\fapply{7} \\
\mc{7}{\cf{NP}}
}
}
\caption{\small Deverbal noun predicate with agent, patient and
beneficiary arguments.\label{fig:gift}}
\end{figure*}

Many common nouns in English can receive optional complements and adjuncts,
realised by prepositional phrases, genitive determiners, compound nouns, relative clauses, and for
some nouns, complementised clauses. For example, deverbal nouns
generally have argument structures similar to the verbs they are derived from:
\vspace{-0.03in}
\begin{lexamples}
\addtolength{\itemsep}{-.99\itemsep}
 \item Rome's destruction of Carthage\label{n_agent_patient}
 \item Rome destroyed Carthage\label{v_agent_patient}
\end{lexamples}
\vspace{-0.03in}
The semantic roles of \emph{Rome} and \emph{Carthage} are the same in
(\ref{n_agent_patient}) and (\ref{v_agent_patient}), but the noun cannot
case-mark them directly, so the genitive clitic and \emph{of} are pressed into
service. The semantic role depends on both the predicate and subcategorisation
frame:
%\vspace{-0.03in}
\begin{lexamples}
\addtolength{\itemsep}{-.99\itemsep}
 \item Carthage's$_\mathrm{p}$ destruction$_\mathrm{Pred.}$\label{gen_patient}
 \item Rome's$_\mathrm{a}$ destruction$_\mathrm{Pred.}$ of~Carthage$_\mathrm{p}$\label{gen_agent_patient}
 \item Rome's$_\mathrm{a}$ gift$_\mathrm{Pred.}$\label{agent_patient}
 \item Rome's$_\mathrm{a}$ gift$_\mathrm{Pred.}$ of~peace$_\mathrm{p}$ to~Europe$_\mathrm{b}$\label{agent_patient_ben}
\end{lexamples}

In (\ref{gen_patient}), the genitive introduces the patient, but when the patient is
supplied by the PP, it instead introduces the agent. The mapping differs for
\emph{gift}, where the genitive introduces the agent.

Peripheral arguments, which supply generically available modifiers of time,
place, cause, quality etc, can be realised by pre- and post-modifiers:
\vspace{-0.03in}
\begin{lexamples}
\addtolength{\itemsep}{-.99\itemsep}
 \item The portrait \emph{in the Louvre}
 \item The \emph{fine} portrait
 \item \emph{The Louvre's} portraits
\end{lexamples}
\vspace{-0.03in}
These are distinct from core arguments because their interpretation does not
depend on the predicate. The ambiguity can be seen in an NP such as
\emph{The nobleman's portrait}, where the genitive could mark
possession (peripheral), or it could introduce the patient (core).
The distinction between core and peripheral arguments is particularly difficult
for compound nouns, as pre-modification is very productive in English.




%Nominal predicates can also receive adjuncts, which canonically occur after
%the arguments, but a heavy movement rule appears to allow them to occur
%closer to the predicate:

%\begin{lexamples}
% \item The destruction of Carthage in 146 B.C.
% \item The destruction in 146 B.C. of a very phonetically heavy city-state
%       that must be moved to the right.\label{n_heavy}
%\end{lexamples}

\subsection{CCG Analysis}


\begin{figure*}
\centering
\scalebox{0.8}{
\deriv{6}{
\rm Google & \rm 's & \rm decision & \rm to & \rm buy & \rm YouTube \\
\uline{1}&\uline{1}&\uline{1}&\uline{1}&\uline{1}&\uline{1} \\
\cf{NP} &
\cf{(NP_y/(N_y/PP_z)_y)\bs NP_z} &
\cf{(N/PP_y)/(S[to]_z\bs NP_y)_z} &
\cf{(S[to]_y\bs NP_z)_y/(S[b]_y\bs NP_z)_y} &
\cf{(S[b]\bs NP_y)/NP_z} &
\cf{NP} \\
\bapply{2} &&& \fapply{2} \\
\mc{2}{\cf{NP_y/(N_y/PP_\mathrm{Google})_y}} &&& \mc{2}{\cf{S[b]\bs NP_y}} \\
\fcomp{3} & \fapply{3} \\
\mc{3}{\cf{NP_\mathrm{decision}/(S[to]_y\bs NP_\mathrm{Google})_y}} & \mc{3}{\cf{S[to]_\mathrm{buy}\bs NP_y}} \\
\fapply{6} \\
\mc{6}{\cf{NP}}
}
}
\caption{\small The coindexing on \emph{decision}'s category allows the hard-to-reach agent of
\emph{buy} to be recovered. A non-normal form derivation is shown so that instantiated
variables can be seen. \label{fig:decision_to}}\end{figure*}

We designed our analysis for transparency between the syntax and the predicate-argument
structure, by stipulating that all and only the core arguments should
be syntactic arguments of the predicate's category. This is fairly straightforward
for arguments introduced by prepositions:

\begin{center}
\deriv{3}{
\rm destruction & \rm of & \rm Carthage \\
\uline{1}&\uline{1}&\uline{1} \\
\cf{N/PP_y} &
\cf{PP_y/NP_y} &
\cf{NP} \\
& \fapply{2} \\
& \mc{2}{\cf{PP_\mathrm{Carthage}}} \\
\fapply{3} \\
\mc{3}{\cf{N_\mathrm{destruction}}}
}
\end{center}

In our analysis, the head of \emph{of Carthage} is \emph{Carthage}, as \emph{of} is
assumed to be a semantically transparent case-marker. We apply this analysis
to prepositional phrases that provide arguments to verbs as well --- a
departure from \ccgbank.

Prepositional phrases that introduce peripheral arguments are analysed as
syntactic adjuncts:

\begin{center}
\deriv{4}{
\rm The & \rm war & \rm in & \rm 149~B.C. \\
\uline{1}&\uline{1}&\uline{1}&\uline{1} \\
\cf{NP_y/N_y} &
\cf{N} &
\cf{(N_y\bs N_y)/NP_z} &
\cf{NP} \\
&& \fapply{2} \\
&& \mc{2}{\cf{(N_y\bs N_y)_\mathrm{in}}} \\
& \bapply{3} \\
& \mc{3}{\cf{N_\mathrm{war}}} \\
\fapply{4} \\
\mc{4}{\cf{NP_\mathrm{war}}}
}
\end{center}


Adjunct prepositional phrases remain headed by the preposition, as it is the
preposition's semantics that determines whether they function as temporal,
causal, spatial etc. arguments. We follow \citet{hock:cl07} in our analysis
of genitives which  realise peripheral arguments, such as the literal possessive:

 \begin{center}
\deriv{3}{
\rm Rome & \rm 's & \rm aqueducts \\
\uline{1}&\uline{1}&\uline{1} \\
\cf{NP} &
\cf{(NP_y/N_y)\bs NP_z} &
\cf{N} \\
\bapply{2} \\
\mc{2}{\cf{(NP_y/N_y)_\mathrm{'s}}} \\
\fapply{3} \\
\mc{3}{\cf{NP_\mathrm{aqueducts}}}
}
\end{center}

Arguments introduced by possessives are a little trickier, because the genitive
also functions as a determiner. We achieve this by having the noun subcategorise
for the argument, which we type \cf{PP}, and having the possessive subcategorise
for the unsaturated noun to ultimately produce an \cf{NP}:

\begin{center}
\deriv{3}{
\rm Carthage & \rm 's & \rm destruction \\
\uline{1}&\uline{1}&\uline{1} \\
\cf{NP} &
\cf{(NP_y/(N_y/PP_z)_y)\bs NP_z} &
\cf{N/PP_y} \\
\bapply{2} \\
\mc{2}{\cf{(NP_y/(N_y/PP_\mathrm{Carthage})_y)_\mathrm{'s}}} \\
\fapply{3} \\
\mc{3}{\cf{NP_\mathrm{destruction}}}
}
\end{center}

In this analysis, we regard the genitive clitic as a case-marker that performs
a movement operation roughly analogous to WH-extraction. Its category is
therefore similar to the one used in object extraction,  \cf{(N\bs N)/(S/NP)}.
Figure \ref{fig:gift} shows an example with multiple core arguments.

This analysis allows recovery of verbal arguments when the nominal predicate
subcategorises for a VP, as in Figure \ref{fig:decision_to}.
In this example, the agent of \emph{buy} is \emph{Google}, a fact which our
analysis accomodates effortlessly. The category assigned to \emph{decision}
can coindex the missing \cf{NP} argument of \emph{buy} with its own
\cf{PP} argument. When that argument is supplied by the genitive, it is also
supplied to the verb. This argument would be quite difficult to recover using a
shallow syntactic analysis, as the path between \emph{buy} and \emph{Google}
is quite long on a standard parse tree.

These analyses allow us to draw complement/adjunct distinctions for nominal
predicates, so that the surface syntax takes us very close to a full
predicate-argument analysis. The only information we are not specifying in the
syntactic analysis are the role labels assigned to each of the syntactic
arguments. We could go further and express these labels in the syntax, producing
categories like \cf{(N/PP[0]_y)/PP[1]_z} and \cf{(N/PP[1]_y)/PP[0]_z}, but we
expect that this would cause sparse data problems given the limited size of
the corpus. This experiment would be an interesting subject of future work.

The only local core arguments that we do not annotate as syntactic complements
are compound nouns, such as \emph{decision makers}. We avoided these arguments
because of the productivity of noun-noun compounding in English, which makes
these argument structures very difficult to recover.

We currently do not have an analysis that allows support verbs to supply noun
arguments, so we do not recover any of the long-range dependency structures
described by \citet{nombank}.

% Another challenge for our analysis is the apparent heavy-movement of nominal
% adjuncts. \ccg has an efficient analysis of this for verbal predicates, using
% the \Sbxcomp\xspace rule. However, \citet{hock:cl07} follow \citet{steedman:00}
% in restricting this rule to categories rooted in \cf{S}, to prevent
% over-generation. We must therefore introduce category ambiguity to
% generate this construction:
% 
% \begin{center}
% \deriv{3}{
% \rm destruction & \rm in~149~B.C. & \rm of~heavy~NP \\
% \uline{1}&\uline{1}&\uline{1} \\
% \cf{N/PP_1} &
% \cf{(N_1/PP_2)_1\bs (N_1/PP_2)_1} &
% \cf{PP} \\
% \bapply{2} \\
% \mc{2}{\cf{N_{destruction}/PP_1}} \\
% \fapply{3} \\
% \mc{3}{\cf{N_{destruction}}}
% }
% \end{center}



\subsection{Implementation and Statistics}

Our analysis requires semantic role labels for each argument of the
nominal predicates in the Penn Treebank --- precisely what \nombank
\citep{nombank} provides. We can therefore draw our
distinctions using the process described by \citet{honnibal:pacling07prop}.

\nombank follows the same format as \propbank, so the procedure is exactly the
same. First, we align \ccgbank and the Penn Treebank, and produce a version of
\nombank that refers to \ccgbank nodes. We then assume that any prepositional
phrase or genitive determiner annotated as a core argument in \nombank should
be analysed as a complement, while peripheral arguments and adnominals that receive
no semantic role label at all are analysed as adjuncts.

We converted 34,345 adnominal prepositional phrases to complements, leaving 18,919
as adjuncts. The most common preposition converted was \emph{of}, which was
labelled as a core argument 99.1\% of the 19,283 times it occurred as an adnominal.
The most common adjunct preposition was \emph{in}, which realised a peripheral
argument in 59.1\% of its 7,725 occurrences.

The frequent prepositions were more skewed towards core arguments. 73\% of
the occurrences of the 5 most frequent prepositions (\emph{of}, \emph{in},
\emph{for}, \emph{on} and \emph{to}) realised peripheral arguments, compared
with 53\% for other prepositions.

Core arguments were also more common than peripheral arguments for possessives.
There are 20,250 possessives in the corpus, of which 75\%
were converted to complements. The percentage was similar for both personal
pronouns (such as \emph{his}) and genitive phrases (such as
\emph{the boy's}).

%The converted corpus has a very high recall over Nombank core arguments that
%do not rely on support constructions. N\% of non-support core arguments have
%a corresponding argument on the lexical category. Most of the missing arguments
%(N\%) are realised by compound nouns and apposited noun phrases (N\%). Only N\%
%of syntactic arguments did not have a corresponding semantic argument in Nombank.

%\pagebreak
\section{Adding Restrictivity Distinctions}
\label{sec:restrictivity}

% While applying the changes to nominal argument structures described in Section
% \ref{sec:nom_parg}, we encountered another problem with the structure of
% noun-phrases in \ccgbank. This section describes the problem and our solution.
Adnominals can have either a restrictive or a non-restrictive (appositional)
interpretation, determining the potential reference of the noun phrase it
modifies. This ambiguity manifests itself in whether prepositional phrases,
relative clauses and other adnominals are analysed as modifiers of either N or NP,
yielding a restrictive or non-restrictive interpretation respectively.

In \ccgbank, all adnominals  attach to \cf{NP}s, producing
non-restrictive interpretations. We therefore move restrictive adnominals
to \cf{N} nodes:

\begin{center}
\deriv{5}{
\rm All & \rm staff & \rm on & \rm casual & \rm contracts \\
\uline{1}&\uline{1}&\uline{1}&\uline{1}&\uline{1} \\
\cf{NP/N} &
\cf{N} &
\cf{(N\bs N)/NP} &
\cf{N/N} &
\cf{N} \\
&&& \fapply{2} \\
&&& \mc{2}{\cf{N}} \\
&&& \tcrule{2} \\
&&& \mc{2}{\cf{NP}} \\
&& \fapply{3} \\
&& \mc{3}{\cf{N\bs N}} \\
& \bapply{4} \\
& \mc{4}{\cf{N}} \\
\fapply{5} \\
\mc{5}{\cf{NP}}
}
\end{center}

This corrects the previous interpretation, which stated
that there were no permanent staff.

\subsection{Implementation and Statistics}
\label{sec:restrictivity_implementation}
The Wall Street Journal's style guide mandates that 
this attachment ambiguity be often managed by bracketing
non-restrictive relatives with commas \citep[p. 82]{martin:02},
as in \emph{casual staff, who have no health insurance, support it}.
We thus use punctuation to make the attachment decision.

All \cf{NP\bs NP} modifiers that are not preceded by punctuation were
moved to the lowest \cf{N} node possible and relabelled \cf{N\bs N}. We select
the lowest (i.e. closest to leaf) \cf{N} node because some adjectives, such as
\emph{present} or \emph{former}, require scope over the qualified noun, making
it safer to attach the adnominal first.

Some adnominals in \ccgbank are created by the \cf{S\bs NP} $\rightarrow$ 
\cf{NP\bs NP} unary type-changing rule, which transforms reduced relative
clauses. We introduce a \cf{S\bs NP} $\rightarrow$ \cf{N\bs N} in its place,
and add a binary rule cued by punctuation to handle the relatively
rare non-restrictive reduced relative clauses.

%There are 79,680 \cf{NP\bs NP} and XXX \cf{N\bs N} nominal modifiers in \ccgbank.
%We moved 83\% of the \cf{NP\bs NP} nodes to attach at
%\cf{N}, to receive a restrictive interpretation or be changed to complements.

The rebanked corpus contains 34,134 \cf{N\bs N} restrictive modifiers, and
9,784 non-restrictive modifiers. Most (61\%) of the non-restrictive modifiers
were relative clauses.

\section{Reanalysing Partitive Constructions}

True partitive constructions consist of a quantifier (\ref{eg:some}), a 
cardinal (\ref{eg:four}) or demonstrative (\ref{eg:deictic}) applied to an NP
via \emph{of}.
There are similar constructions headed by common nouns (Ex. \ref{eg:common}):
\begin{lexamples}
\addtolength{\itemsep}{-.99\itemsep}
 \item Some of us\label{eg:some}
 \item Four of our members\label{eg:four}
 \item Those of us who smoke\label{eg:deictic}
 \item A glass of wine\label{eg:common}
\end{lexamples}
We regard the common noun partitives as headed by the initial noun, such as \emph{glass},
because this noun usually controls the number agreement. We therefore analyse
these cases as nouns with prepositional arguments. Above, \emph{glass} would be
assigned the category \cf{N/PP}.

True partitive constructions are different, however: they are always headed by the
head of the NP supplied by \emph{of}. The construction is quite common, because
it provides a way to quantify or apply a demonstrative to an already determinate NP.

Partitive constructions are not given special treatment in the \penn, and were
analysed as noun phrases with a PP modifier in \ccgbank:

\begin{center}
\deriv{4}{
\rm Four & \rm of & \rm our & \rm members \\
\uline{1}&\uline{1}&\uline{1}&\uline{1} \\
\cf{NP} &
\cf{(NP_y\bs NP_y)/NP_z} &
\cf{NP_y/N_y} &
\cf{N} \\
&& \fapply{2} \\
&& \mc{2}{\cf{NP_\mathrm{members}}} \\
& \fapply{3} \\
& \mc{3}{\cf{(NP_y\bs NP_y)_\mathrm{of}}} \\
\bapply{4} \\
\mc{4}{\cf{NP_\mathrm{Four}}}
}
\end{center}

This analysis does not yield the correct semantics, and may even hurt parser
performance, because the head of the phrase is incorrectly assigned. We correct
this with the following analysis, which takes the head from the NP argument
of the PP:

\begin{center}
\deriv{4}{
\rm Four & \rm of & \rm our & \rm members \\
\uline{1}&\uline{1}&\uline{1}&\uline{1} \\
\cf{NP_y/PP_y} &
\cf{PP_y/NP_y} &
\cf{NP_y/N_y} &
\cf{N} \\
&& \fapply{2} \\
&& \mc{2}{\cf{NP_\mathrm{members}}} \\
& \fapply{3} \\
& \mc{3}{\cf{PP_\mathrm{members}}} \\
\fapply{4} \\
\mc{4}{\cf{NP_\mathrm{members}}}
}
\end{center}

The cardinal is given the category \cf{NP/PP},  in analogy
with the standard determiner category which is a function from a noun to a noun
phrase (\cf{NP/N}).

\subsection{Implementation and Statistics}

We detect this construction by identifying NPs post-modified by an \emph{of}
PP. The NP's head must either have the \pos tag \textsc{cd}, or be one of the
following words, determined through manual inspection of Sections 02-21:
\vspace{-0.03in}
\begin{quote}
\small
 \emph{all, another, average, both, each, another, any,
anything, both, certain, each, either, enough, few, little, most, much,
neither, nothing, other, part, plenty, several, some, something, that, those}.
\end{quote}
\vspace{-0.03in}
Having identified the construction, we simply relabel the \cf{NP} to \cf{NP/PP},
and the \cf{NP\bs NP} adnominal to \cf{PP}. We identified 3,010 partitive
genitives in \ccgbank, all of which we reanalysed.

% \section{Corpus Analysis}
% 
% This section provides some descriptive statistics of the rebanked corpus, to
% examine the effect of our changes.

\section{Similarity to \ccgbank}

%The first question we asked was how much we had actually changed \ccgbank. We
%quantified the changes in two ways. First, we looked at how many lexical
%categories, labelled dependencies and unlabelled dependencies the rebanked
%corpus and the original had in common.

Table \ref{tab:intersect} shows the percentage of labelled dependencies
(L. Deps), unlabelled dependencies (U. Deps) and lexical categories (Cats)
that remained the same after each set of changes.

A labelled dependency is a 4-tuple consisting of the head,
the argument, the lexical category of the head, and the argument slot that the
dependency fills (e.g. the subject fills slot 1 and the object fills slot 2 on
the transitive verb category \cf{(S\bs NP)/NP}. This is why there are
more changes to labelled dependencies than lexical categories: one lexical
category change alters all of the dependencies headed by a predicate, because
they all depend on its lexical category. Unlabelled dependencies consist
of only the head and argument.

The biggest changes were those described in Sections
\ref{sec:nom_parg} and \ref{sec:restrictivity}. After the addition of
nominal predicate-argument structure, over 50\% of the labelled dependencies
were changed. Many of these changes involved changing an adjunct to a
complement, which also affects the unlabelled dependencies, as the head and
argument are inverted.

\begin{table}
\centering
\small
 \begin{tabular}{l|rrr}
\hline
  Corpus        & \textbf{\textsc{L. Deps}} & \textbf{\textsc{U. Deps}} & \textbf{\textsc{Cats}} \\
\hline\hline
 +NP brackets   & 97.2 & 97.7 & 98.5 \\
 +Quotes        & 97.2 & 97.7 & 98.5 \\
 +Propbank      & 93.0 & 94.9 & 96.7 \\
 +Particles     & 92.5 & 94.8 & 96.2 \\
 \hline
 +Restrictivity & 79.5 & 94.4 & 90.6 \\
 +Part. Gen.    & 76.1 & 90.1 & 90.4 \\
 +NP Pred-Arg   & 70.6 & 83.3 & 84.8 \\
\hline
% np_bracketing & .97095310136157337367 & .96744831064044377206
% quotes & .97014624306606152294 & .96664145234493192133
% propbank & .92960161371659102370 & .94866364094805849722
% restrictivity & .79483106404437720625 & .94379727685325264750
% (standard_in) 1: parse error
% (standard_in) 1: parse error
% part_gen &  & 
% vpc & .92450832072617246596 & .94783156833081190115
% vpc & .92450832072617246596 & .94783156833081190115
 \end{tabular}
\caption{\small Effect of the changes on \ccgbank,
by percentage of dependencies and categories
left unchanged in Section 00.\label{tab:intersect}}
\vspace{-.6in}
\end{table}

\section{Lexicon Statistics}

\begin{table}
\centering
\small
 \begin{tabular}{l|rrr}
\hline
  Corpus        & \textbf{\textsc{Cats}} & \multicolumn{1}{c}{Cats $\ge10$} & \textbf{\textsc{Cats/Word}} \\
\hline\hline
 \ccgbank       & 1286 & 425        & 8.6 \\
 +NP brackets & 1298 & 429        & 8.9 \\
 +Quotes        & 1300 & 431        & 8.8 \\
 +Propbank      & 1342 & 433        & 8.9 \\
 +Particles     & 1405 & 458        & 9.1 \\
 \hline
 +Restrictivity & 1447 & 471        & 9.3 \\
 +Part. Gen.    & 1455 & 474        & 9.5 \\
 +NP Pred-Arg   & 1574 & 511        & 10.1 \\
\hline
 \end{tabular}
\caption{\small Effect of the changes on the size of the lexicon.
\label{tab:lexicon}}
\vspace{-.6in}
\end{table}

%Because \ccg is a lexicalised formalism, the only change we have made to the
%grammar to support our new analyses are the modified type-changing rules for
%reduced relative clauses described in Section \ref{sec:restrictivity_implementation}.
%The rest of our changes are reflected in the lexicon.

Our changes make the grammar sensitive to new distinctions, which increases the
number of lexical categories required. Table \ref{tab:lexicon} shows the
number of lexical categories (Cats), the number of lexical categories that occur at
least 10 times in Sections 02-21 (Cats $\ge10$), and the average number of
categories available for assignment to each token in Section 00 (Cats/Word). We
followed \citepos{clark:cl07} process to determine the set of categories a word could
receive, which includes a part-of-speech back-off for infrequent words.

The lexicon steadily grew with each set of changes, because each added
information to the corpus. The addition of quotes only added two
categories (\cf{LQU} and \cf{RQU}), and the addition of the quote
tokens slightly decreased the average categories per word. The \propbank and
verb-particle changes both introduced rare categories for infrequent,
complicated argument structures.

The \cf{NP} predicate-argument structure modifications added the most
information. Head nouns were previously guaranteed the category \cf{N} in
\ccgbank; possessive clitics always received the category \cf{(NP/N)\bs NP};
and possessive personal pronouns were always \cf{NP/N}. Our changes introduce
new categories for these frequent tokens, which meant a substantial
increase in the number of possible categories per word.


\section{Parsing Evaluation}





Some of the changes we have made correct problems that have caused
the performance of a statistical \ccg parser to be over-estimated. Other changes
introduce new distinctions, which a parser may or may not find difficult to reproduce.
To investigate these issues, we trained and evaluated the \candc \ccg parser on
our rebanked corpora.

The experiments were set up as follows.
We used the highest scoring configuration \citet{clark:cl07} describe for our
experiments, the hybrid dependency model, using gold-standard \textsc{pos} tags.
We followed \citeauthor{clark:cl07} in excluding sentences that could not be parsed
from the evaluation. All models obtained similar coverage, between 99.0 and
99.3\%. The parser was evaluated using dependencies generated from the
gold-standard derivations (Boxwell, p.c., 2010).
% We examined how easily a parser could reproduce the distinctions we
% introduced by training and evaluating the \candc parser \citep{clark:cl07} on
% the rebanked corpus. Table \ref{tab:parser} shows the effect of each of our changes
% on the parser's accuracy, as measured by F-score over labelled (\textsc{lf})
% and unlabelled (\textsc{uf}) dependencies and lexical categories (Cat).

Table \ref{tab:parser} shows the accuracy of the parser on Sections 00 and 23.
The parser scored slightly lower as the NP brackets, Quotes,
Propbank and Particles corrections were added. This apparent decline in performance
is at least
partially an artefact of the evaluation. \ccgbank contains some dependencies
that are trivial to recover, because \citet{hock:cl07} was forced
to adopt a strictly right-branching analysis for NP brackets. 

\begin{table}
\centering
\small
 \begin{tabular}{l|rrr|rrr}

\hline
&\multicolumn{3}{c|}{\textsc{wsj} 00}&\multicolumn{3}{c}{\textsc{wsj} 23}\\
  Corpus        & \textbf{\textsc{lf}}   & \textbf{\textsc{uf}}  & \textbf{\textsc{Cat}} & \textbf{\textsc{lf}}   & \textbf{\textsc{uf}}  & \textbf{\textsc{Cat}}\\
\hline\hline
 \ccgbank       & 87.2 & 92.9 & 94.1 & 87.7 & 93.0 & 94.4 \\
 +NP brackets & 86.9 & 92.8 & 93.8 & 87.3 & 92.8 & 93.9 \\
 +Quotes        & 86.8 & 92.7 & 93.9 & 87.1 & 92.6 & 94.0 \\
 +Propbank      & 86.7 & 92.6 & 94.0 & 87.0 & 92.6 & 94.0\\
 +Particles     & 86.4 & 92.5 & 93.8 & 86.8 & 92.6& 93.8\\
\hline
 %+Restrictivity & 85.4 & 91.8 &  & & &\\
 %+Part. Gen.    &  &  &  & & &\\
All Rebanking   & 84.2 & 91.2 & 91.9 & 84.7 & 91.3 & 92.2 \\
\hline
 \end{tabular}
\caption{\small Parser evaluation on the rebanked corpora.
\label{tab:parser}}
%\vspace{-.6in}
\end{table}

\begin{table}
\centering
\small
 \begin{tabular}{l|rr|rr}
\hline
  Corpus        & \multicolumn{2}{c|}{Rebanked} & \multicolumn{2}{c}{\ccgbank} \\
                & \textbf{\textsc{lf}} & \textbf{\textsc{uf}}                      & \textbf{\textsc{lf}} & \textbf{\textsc{uf}} \\
\hline\hline
+NP brackets & 86.45 & 92.36 & 86.52 & 92.35\\
+Quotes      & 86.57 & 92.40 & 86.52 & 92.35\\
+Propbank    & 87.76 & 92.96 & 87.74 & 92.99\\
+Particles   & 87.50 & 92.77 & 87.67 & 92.93\\
\hline
%+Restrictivity & 86.6 & 92.3 & 87.8  & 92.9  \\
%+Part. Gen.    &  & & &   \\
All Rebanking & 87.23 & 92.71 & 88.02 & 93.51\\
\hline
 \end{tabular}
\caption{\small Comparison of parsers trained on \ccgbank and the rebanked corpora,
using dependencies that occur in both.
\label{tab:intersect}}
\vspace{-.6in}
\end{table}


There was a larger drop in accuracy on the fully rebanked corpus,
which included our analyses of restrictivity, partitive constructions and noun
predicate-argument structure. This might also be explained by the evaluation,
as the rebanked corpus includes much more fine-grained distinctions. The labelled
dependencies evaluation is particularly sensitive to this, as a single category
change affects multiple dependencies. This can be seen in the smaller gap in
category accuracy.

We investigated whether the differences in performance were due to the different
evaluation data by comparing the parsers' performance against the original parser
on the dependencies they agreed upon, to allow direct comparison. To do this,
we extracted the \ccgbank intersection of each corpus's Section 00 dependencies.

Table \ref{tab:intersect} compares the labelled and unlabelled recall of the
rebanked parsers we trained against the \ccgbank parser on these intersections.
Note that each row refers to a different intersection, so results are not
comparable between rows. This comparison shows that the
declines in accuracy seen in Table \ref{tab:parser} were largely confined to
the corrected dependencies. The parser's performance remained fairly stable on the
dependencies left unchanged.

The rebanked parser performed 0.8\% worse than the \ccgbank parser on the
intersection dependencies, suggesting that the fine-grained distinctions
we introduced did cause some sparse data problems. However, we did not change any of
the parser's maximum entropy features or hyper-parameters, which are
tuned for \ccgbank.

% The parser's accuracy steadily dropped as more information was added to the
% corpus. We suggest two factors that could explain this. First,
% the changes we introduced have increased the number of lexical categories,
% as shown in Table \ref{tab:lexicon}. This introduces sparse data problems,
% so the parser's model might fundamentally be weaker. The second factor is that
% the evaluation has become more difficult, due to the fine-grained
% distinctions introduced. For instance, every base NP was formerly
% right-branching, so the parsing task for these dependencies was quite easy.
% The \citet{vadas:08} NP rebracketing introduces a new way for the parser to be
% wrong.
% 
% To investigate the impact of each of these factors, we needed to compare each
% rebanked parser against the  \ccgbank parser on a common set of dependencies
% they were both trained to produce. For each rebanked corpus, we extracted the
% intersection of its dependencies and \ccgbank's dependencies. The parsers were
% evaluated on their recall over these sets of shared dependencies.
% 
% Table \ref{tab:intersection} shows the results from these intersection
% experiments. The final rebanking parser only performs 0.8\% worse on its
% intersection dependencies, so most of the difference in accuracy is over the
% difference. This suggests that the second factor, the changed evaluation,
% contributed more to the 3\% drop in accuracy than the sparse data problems.
% 
% One concern we had with this evaluation was that the dependencies
% in the intersection would be dominated by the easiest dependencies, such as
% determiner-head and adjective-head relations. However, the \ccgbank parser's
% score only changed by 0.8\% on the smallest intersection, suggesting
% that the evaluation set remained reasonably challenging.

\section{Conclusion}

Research in natural language understanding is driven by the datasets that we
have available. The most cited computational linguistics work to date is
the Penn Treebank \citep{marcus:93}\footnote{http://clair.si.umich.edu/clair/anthology/rankings.cgi}.
Propbank \citep{propbank} has also been very influential since its release, and
\nombank has been used for semantic dependency parsing in the CoNLL 2008 and 2009 shared
tasks.

This paper has described how these resources can be exploited
using a linguistically motivated theory of syntax and semantics. The semantic
annotations provided by \propbank and \nombank allowed us to build a corpus
that takes much greater advantage of the semantic transparency of a deep grammar,
using careful analyses and phenomenon-specific conversion rules.
%However, the resources themselves do not solve the challenging problem of
%representing the distinctions correctly. This is why some of the changes we
%ave made did not required , but no external resources.

The major areas of \ccgbank's grammar left to be improved are the analysis
of comparatives, and the analysis of named entities. English comparatives are
diverse and difficult to analyse. Even the \xtag grammar \citep{xtag}, which
deals with the major constructions of English in enviable detail, does not
offer a full analysis of these phenomena. Named entities are also difficult to
analyse, as many entity types obey their own specific grammars. This is another
example of a phenomenon that could be analysed much better in \ccgbank using an
existing resource, the \bbn named entity corpus.

Our rebanking has substantially improved \ccgbank, by increasing the granularity
and linguistic fidelity of its analyses. We achieved this by exploiting existing
resources and crafting novel analyses. The process we have demonstrated can be
used to train a parser that returns dependencies that abstract
away as much surface syntactic variation as possible --- including, now, even
whether the predicate and arguments are expressed in a noun phrase or a full
clause.

\begin{small}
\bibliography{thesis}
\end{small}
\bibliographystyle{aclnat}

\end{document}

% 
% we argue that there is a significant gap in linguistic
% fidelity between the parsers trained on adapted corpora, and the rule-based parsers written with linguistically
% motivated formalisms. It is this gap that this paper seeks to address.
% 
% % se adaptations rely on annotations that were applied less consistently than the main brackets,
% % and in many cases the analyses the \ptb adopted diverge considerably from the analysis a formalism like
% % \hpsg or \ccg would prefer. In other cases, the \ptb lacks distinctions the linguistically
% % motivated formalisms require. For instance, base noun phrases receive flat brackets in the Penn Treebank,
% % which removes distinctions that a formalism like \ccg requires. The crux of the problem is that the Penn
% % Treebank's annotation scheme was influenced by strategic decisions to control the cost of the project.
% % This makes it difficult to extract linguistically optimal distinctions from its annotations.
% 
% 
% 
% % It remains unclear whether statistical parsers are superior to the rule-based linguistically
% % motivated parsers. Statistical parsers appear to achieve higher accuracies at the shallow
% % bracketing they perform, particularly on text similar to their training data. On the other
% % hand, linguistically motivated parsers produce much more detailed output, and seem less
% % domain dependent.
% 
% 
% 
% 
% % All of these theories are lexicalised: each word is associated with an object
% % that specifies the word's valency and syntactic type. The objects are
% % constructed such that the grammar is reduced to a handful of general rules that
% % manipulate them.
% % 
% % One advantage of this division of labour is that the lexical objects allow
% % subcategorisation frames to be represented directly. This promotes an
% % isomorphism between the syntactic derivation and a predicate-argument structure
% % or compositional semantic analysis. This is why parsing with these formalisms is
% % sometimes called deep parsing, in contrast with the shallow parsing task of
% % replicating the Penn Treebank's skeletal brackets.
% % 
% % Linguistically motivated formalisms also allow much more detailed and accurate
% % grammars to be written. Work on hand-written context-free phrase-structure
% % grammars (\cfpsg) largely ceased with the release of the Penn Treebank (\penn)
% % \citep{marcus:93}, when \citepos{magerman:94} statistical \cfpsg parser
% % substantially outperformed a leading hand-written grammar that had been under
% % development for 10 years. In contrast, the \lingo \hpsg parser \citep{erg} is
% % still under active development, and the \xle \lfg parser \citep{xle} has been a
% % commercial success at Powerset\footnote{http://www.powerset.com}.
% % 
% % The analyses favoured by linguistically motivated formalisms diverge
% % considerably from those implemented in \penn. This is partly due to differences
% % in descriptive power, but much of it is due to problems with the \penn analyses.
% % This divergence has caused problems for the alternate approach to linguistically
% % motivated parsing, which involves converting the Penn Treebank.
% % 
% % The example we are concerned with is \ccgbank, although the approach taken in
% % this paper could equally be applied to \citepos{miyao:04} \hpsg corpus,
% % \citepos{chen:00} \ltag corpus, or automatic \lfg F-structure annotations
% % \citep{frank:03}.

% \section{NP Predicate-Argument Structure}
% \label{sec:np_parg}
% 
% There are several problems with the analysis of noun phrases in the Penn
% Treebank. The most obvious of these is the fact that base NPs are given flat
% brackets, and this has since been corrected by \citet{vadas:07}, who went on to
% explore how these annotations could be incorporated into \ccgbank and reproduced
% by the \candc parser \citep{vadas:08}. We use their NP-corrected version of
% \ccgbank as the starting point for our further corrections.
% 
% However, there are several NP structures that still receive unsatisfactory
% analyses. The biggest problem is that all prepositional phrases are analysed as
% adjuncts. We correct this issue using NomBank \citep{nombank}, which provides
% noun-phrase predicate-argument annotations of the Wall Street Journal.
% 
% Our process is similar to the one used by \citet{honnibal:pacling07prop} to
% correct verb phrase predicate-argument structures with PropBank. We use the same
% alignment procedure, but implement our changes using the generalised relabelling
% algorithm described in Section \ref{sec:relabel}.
% 
% \ccgbank already allows nouns like \emph{fact} to specify clausal arguments. Our
% changes allow nouns to specify arguments realised by PPs and genitives. However,
% we have avoided drawing argument/adjunct distinctions for compound nouns.
% Noun-noun compounding is very productive, and the task of distinguishing an
% argument compound like \emph{economic sanctions} from an adjunct compound like
% \emph{restrictive sanctions} is very difficult.
% 
% \begin{figure}
% \centering
% \deriv{3}{
% \rm destruction & \rm of & \rm Rome \\
% \uline{1}&\uline{1}&\uline{1} \\
% \cf{NP} &
% \cf{(NP\bs NP)/NP} &
% \cf{NP} \\
% & \fapply{2} \\
% & \mc{2}{\cf{NP\bs NP}} \\
% \bapply{3} \\
% \mc{3}{\cf{NP}}
% }
% \caption{\small \ccgbank analysis of a deverbal noun
% argument.\label{fig:npnp_destruction}}
% \end{figure}
% 
% \begin{figure}
% \centering
% \deriv{3}{
% \rm destruction & \rm of & \rm Rome \\
% \uline{1}&\uline{1}&\uline{1} \\
% \cf{NP/PP} &
% \cf{PP/NP} &
% \cf{NP} \\
% & \fapply{2} \\
% & \mc{2}{\cf{PP}} \\
% \fapply{3} \\
% \mc{3}{\cf{NP}}
% }
% \caption{\small Our analysis of a deverbal noun argument
% \label{fig:pp_destruction}}
% \end{figure}
% 
% \begin{figure}
% \centering
% \deriv{3}{
% \rm Rome & \rm 's & \rm destruction \\
% \uline{1}&\uline{1}&\uline{1} \\
% \cf{NP} &
% \cf{(NP/(NP/PP))\bs NP} &
% \cf{NP/PP} \\
% \bapply{2} \\
% \mc{2}{\cf{NP/(NP/PP)}} \\
% \fapply{3} \\
% \mc{3}{\cf{NP}}
% }
% \caption{\small Deverbal noun argument realised by
% genitive.\label{fig:gen_destroy}}
% \end{figure}
% 
% \begin{figure*}
% \centering
% \deriv{7}{
% \rm Rome & \rm 's & \rm gift & \rm of & \rm peace & \rm to & \rm Europe \\
% \uline{1}&\uline{1}&\uline{1}&\uline{1}&\uline{1}&\uline{1}&\uline{1} \\
% \cf{NP} &
% \cf{(NP/(NP/PP))\bs NP} &
% \cf{(NP/PP)/PP)/PP} &
% \cf{PP/NP} &
% \cf{NP} &
% \cf{PP/NP} &
% \cf{NP} \\
% \bapply{2} && \fapply{2} & \fapply{2} \\
% \mc{2}{\cf{NP/(NP/PP)}} && \mc{2}{\cf{PP}} & \mc{2}{\cf{PP}} \\
% && \fapply{3} \\
% && \mc{3}{\cf{(NP/PP)/PP}} \\
% && \fapply{5} \\
% && \mc{5}{\cf{NP/PP}} \\
% \fapply{7} \\
% \mc{7}{\cf{NP}}
% }
% \caption{\small Deverbal noun with multiple arguments.\label{fig:gift}}
% \end{figure*}
% 
% 
% A deverbal noun is a noun morphologically derived from a verb. Usually, the
% derivation preserves the semantics. In Figure \ref{fig:npnp_destruction}, the PP
% \emph{of Rome} is analysed as a modifier. However, the function of the phrase is
% actually to supply an argument to a predicate. It is therefore better analysed
% as a kind of case-marker.
% 
% Our analysis is shown in Figure \ref{fig:pp_destruction}. The head of the PP is
% taken from its argument, so that a dependency is created between the predicate,
% \emph{destruction}, and its argument, \emph{Rome}. This allows deverbal NPs to
% receive the same semantic analysis as their corresponding verb. Figure
% \ref{fig:gen_destroy} shows how we can apply this analysis to an argument
% realised by a genitive phrase. The analysis preserves the lexical categories of
% the content words, and relies on the categories of the function words to account
% for the surface movement. This analysis is similar to the analysis for
% superlative genitives suggested by \citet{bos:09}. Different prepositions are
% used to supply different arguments. Figure \ref{fig:gift} shows three arguments
% being supplied to a ditransitive deverbal noun. 
% 
% NomBank does not restrict its analysis of PP complements to deverbal nouns. For
% instance, in the NP \emph{a group of workers}, the head noun \emph{group}
% specifies a quantity or container, and the complement of the preposition
% supplies the semantic type. Another example are nominal predicates that are not
% morphologically derived from verbs but display similar semantics, such as
% \emph{trip to the zoo}.

% \begin{table*}
% \centering
% \small
% \begin{tabular}{l|ccccc}
% \hline
% \textbf{Production Type}           & \textbf{Old left}    & \textbf{Old right}  
%  & $\to$ & \textbf{New left}    & \textbf{New right}\\
% \hline
% \hline
% Argument Application               & \cf{A/Y}      & \cf{Y}         &$\to$&
% \cf{B/Y}       & \cf{Y}    \\
% \cf{PP} $\to$ \cf{S/S}             & \cf{PP/NP}    & \cf{NP}        &$\to$&
% \cf{(S/S)/NP}  & \cf{NP}   \\
% \hline
% Argument Compostion                & \cf{A_r/Y}    & \cf{Y/A_a}     &$\to$&
% \cf{B_r/Y}     & \cf{Y/B_a}\\
% \cf{(S\bs S)/S[em]} $\to$ NP/PP    & \cf{(S/S)/NP} & \cf{NP/S[em]}  &$\to$&
% \cf{NP/NP}     & \cf{NP/PP}\\
% \hline
% Adjunct Application     & \cf{A/A}      & \cf{A}         &$\to$& \cf{B/B}      
% & \cf{B}    \\
% \cf{VP} $\to$ \cf{VP/PP} & \cf{VP/VP}    & \cf{VP}        &$\to$&
% \cf{VP/VP}     & \cf{VP/PP}\\
% \hline
% Adjunct Composition                & \cf{A_r/A_r}  & \cf{A_r/A_a}   &$\to$&
% \cf{B_r/B_r}   & \cf{B_r/B_a}\\
% \cf{(VP/VP)/NP} $\to$ \cf{PP/NP}   & \cf{VP/VP}    & \cf{(VP/VP)/NP}&$\to$&
% \cf{PP/PP}     & \cf{PP/NP}\\
% \hline
% Conjunction                        & \cf{A}        & \cf{conj}      &$\to$&
% \cf{B}         & \cf{conj}\\
% \cf{NP\bs NP[conj]} $\to$ \cf{N\bs N[conj]} & \cf{NP\bs NP}& \cf{conj}     
% &$\to$&
% \cf{N\bs N}& \cf{conj}\\
% \hline
% Type-raising   & \multicolumn{2}{c}{\cf{T/(T\bs A)}}                 &$\to$&
% \multicolumn{2}{c}{\cf{T/(T\bs B)}}\\
% \cf{S/(S\bs NP)} $\to$ \cf{S/(S\bs PP)}   & \multicolumn{2}{c}{\cf{S/(S\bs NP)}}
%      &$\to$&  \multicolumn{2}{c}{\cf{S/(S\bs PP)}}\\
% \hline
% \end{tabular}
% \caption{The relabelling rule and an example for each production
% type.\label{relabel_rules}}
% \end{table*}
% 
% \section{Relabelling Nodes in \ccgbank}
% \label{sec:relabel}
% 
% Node labels in a \ccg tree are interdependent, so when we change one node label
% we need to propagate the change to other labels. We adopt a strategy of always
% propagating changes downwards. For instance, if we change a node labelled
% \cf{NP} to \cf{NP\bs NP}, we will need to make corresponding changes to its
% children, to ensure that we do not create an invalid production. 
% 
% It is also not enough to simply ensure that the production can be validated by
% some \ccg rule. If we have a tree where the left child is a functor and the
% right child is an argument, we must not produce child labels that reverse this
% relationship, or that change it by making one child an adjunct. For example, the
% tree on the right below is a valid \ccg production --- but a malformed analysis:
% 
% \begin{eqnarray}
% \ptbegtree
% \ptbeg \ptnode{\cf{PP}}
%   \ptleaf{\cf{PP/NP}}
%   \ptleaf{\cf{NP}}
% \ptend
% \ptendtree
% \longrightarrow
% \ptbegtree
% \ptbeg \ptnode{\cf{N\bs N}}
%   \ptleaf{\cf{PP/NP}}
%   \ptleaf{\cf{(N\bs N)\bs (PP\bs NP)}}
% \ptend
% \ptendtree
% \nonumber
% \end{eqnarray}
% 
% The correct transformation would change the result of the left side, instead of
% reversing the functor/argument relationship of the two:
% 
% \begin{eqnarray}
% \ptbegtree
% \ptbeg \ptnode{\cf{PP}}
%   \ptleaf{\cf{PP/NP}}
%   \ptleaf{\cf{NP}}
% \ptend
% \ptendtree
% \longrightarrow
% \ptbegtree
% \ptbeg \ptnode{\cf{N\bs N}}
%   \ptleaf{\cf{(N\bs N)/NP}}
%   \ptleaf{\cf{NP}}
% \ptend
% \ptendtree
% \nonumber
% \end{eqnarray}
% 
% Changing the corpus is much simpler if we have a generic label changing
% algorithm that does not make any assumptions about the relationship between the
% old label and the new label. We therefore need a more complicated process than
% previous work which has made quite restricted changes.
% 
% The first step is to identify the  \emph{production type}, which does not
% correspond one-to-one to the combinator used. For
% instance, both an adjunct and a function might use forward application, but we
% will need to treat the two productions differently. We sort binary productions
% into the following categories:
% argument application, argument composition, adjunct application, adjunct
% composition, conjunction, and punctuation. The node change rules for forward
% combinators are shown in Table \ref{relabel_rules}, with an example given below
% each rule type. The translation rules for backward combinators are directly
% analogous.
% 
% In general, the rules find the result $A_r$ and argument
% $A_a$ of the original parent $A$ and replace them with the appropriate of the
% new parent $B$. Adjunct categories are given special treatment. For instance, in
% the Adjunct Application case shown in Table \ref{relabel_rules}, the verb
% acquires an extra \cf{PP} argument. It is much better to use composition and
% preserve the form of the original adjunct category, instead of assigning it the
% category \cf{(VP/PP)/(VP/PP)} to allow application.
% 
% The type-changing rules allow us to make whatever changes we
% require to the tree without worrying about how our changes affect the subtrees
% of the nodes we are changing. The only time we make simple label substitutions
% are when a non-combinatory phrase-structure rule is encountered. Since these
% rules pair arbitrary parent and child combinations, we simply replace the parent
% and do not propagate the label change any further.