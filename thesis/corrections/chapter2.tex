% \documentclass[a4paper,10pt]{article}
% \usepackage[utf8x]{inputenc}
% \usepackage{xspace}
% \usepackage{natbib}
% \newcommand{\ccgbank}{CCGbank\xspace}
% \newcommand{\ccg}{\textsc{ccg}\xspace}
% \newcommand{\cg}{\textsc{cg}\xspace}
% 
% 
% %opening
% \title{Chapter 2 Corrections}
% \author{Matthew Honnibal}
% 
% \begin{document}
% 
% \maketitle
% 
% \begin{abstract}
% 
% \end{abstract}

\chapter{Chapter 2: Background}

\section{Jason Baldridge}

\subsection{The first formulation of CCG was Ades and Steedman (1982) ``On the
order of words"}

\emph{Citation corrected.}

\subsection{Fix N in trees to NP}

\emph{Derivation corrected.}

\subsection{Lambek didn't note slashes associative, made them so.}

\emph{Reworded to reflect this.}

\subsection{Composition provides associativity, not type-raising.}

\emph{Reworded to reflect this.}

\subsection{Careful about generation in PSG/CCG. Can get this bracketing in PSG}

\emph{Reviewed Gazdar, Pullum and Shieber literature to understand this
point in detail. Reworded to say it's ``uneconomical'' to generate this
bracketing in PSG.}

\subsection{Reference Baldridge's upcoming book for CG as a logic.}

\emph{Added footnote:}
\begin{quote}
The interpretation of a categorial grammar as a logic
dates to \citet{lambek:58}. We present here only a small and informal
illustration.
A more detailed explication of \cg as a logic can be found in
\citet{steedman:pedia}. 
\end{quote}


\subsection{Steedman and Baldridge is ``to appear"}

\emph{Citation corrected.}

\subsection{Use Bx in the rules fig 2.4 where crossing composition is used.}

\emph{Fixed, and same error corrected in Figure 2.5.}

\subsection{Reference Nobo Komagata about restrictions on type-raising}

\emph{Citations added.}

\subsection{Change variable to type}

\emph{Corrected.}

\subsection{Clarification of whether Baldridge defined the modes, or CTL did}

\emph{Reworded as:}
\begin{quote}
\citet{baldridge:thesis02} defined a hierarchy of seven modes, and a simplified
hierarchy of four modes sufficient for most purposes. 
\end{quote}


\subsection{Steedman noted conjunction overgeneration, not Baldridge}

\emph{Citation corrected.}

\subsection{Error in rule in 2.27}

\emph{Corrected.}

\subsection{Mention more OpenCCG stuff, including CoSy stuff and OpenCCG not
first based no CCGbank}

\emph{Corrected, with considerable elaboration on OpenCCG work.}

\subsection{Cite Birch, Osborne and Koehn for MT}

\emph{Done:}

\begin{quote}
\citet{birch:07} observed a more modest improvement on translation between
English and Dutch, and concluded that supertags improved the system's
reordering. 
\end{quote}


\subsection{CCGbank was created semi-automatically, not automatically}

\emph{Corrected.}

\section{Mark Steedman}

\subsection{Check all derivations and categories.}

\emph{Done.}

\subsection{Parasitic gaps actually are attested in \ccgbank.}

\emph{Removed this claim.}

\subsection{CFG's can handle WH-movement, cf. Gazdar.}

\emph{Baldridge made a similar point; this comment was addressed in that
correction.}

\subsection{Meta-rules don't actually challenge universal grammar}

\emph{Removed the problematic claim and replaced it with the following:}

\begin{quote}
These meta-rules would presumably be conventions adopted and learnt by speakers
of a language,
rather than fixed constraints in a universal (possibly innate) grammar.
\citet{baldridge:03} provide a way for such generalisations to
be represented in the formalism, which we will now describe.
\end{quote}

\subsection{It's better to use N instead of NP}

\emph{Corrected this and similar derivation errors.}

\subsection{Missing -]}

\emph{Corrected.}

\subsection{``as much as desired''}

\emph{Reworded this:}

\begin{quote}
Supertagging allows category ambiguity to be reduced, by forwarding only the
categories judged most likely to the parser. 
\end{quote}


\section{Stephen Clark}

\subsection{Context-freeness not proven by Chomsky}

\emph{Corrected by removing the reference to Chomsky and stating that a context
free grammar was ``considered at the time'' to be insufficient, with an
explanatory footnote:}

\begin{quote}
\citet{pullum:82} showed that the existing arguments for the non-context
freeness of natural language were flawed. However, the arguments were
taken seriously at the time, and \citet{wood:93} ascribes
the partial loss of interest in \cg to this. Natural languages were finally
shown to be non-context free by \citet{shieber:85}.
\end{quote}

\subsection{First formulated by Steedman (1988)}

\emph{Corrected the citation to Ades and Steedman (1982)}

\subsection{log-linear model parser is clumsy}

\emph{Reworded.}

\subsection{Mistake in derivation}

\emph{Corrected.}

\subsection{and and argument}

\emph{Corrected.}

\subsection{fairly minimal feature structures in \ccgbank}

\emph{Reworded the context of this point, and it is no longer necessary to
refer to the minimal feature structures in \ccgbank.}

\subsection{Which derivation?}

\emph{Reworded to `the \emph{Casey likes Pat} example'.}

\subsection{unification of Y instead of X}

\emph{Corrected.}

\subsection{Pointer is a bit too implementation-oriented}

\emph{Changed to coindexed. Other references to `implementation' of the grammar
have also been reworked through the thesis, apart from Chapter 6 where we detail
our implementation on the \candc parser.}

\subsection{`may be strongly equivalent'}

\emph{The strong equivalence of Greibach Normal Form \cfg and \abcg was due to
\citet{joshi:00}. As part of my corrections I have removed most references to
Strong Generative Power, as Steedman's corrections have convinced me that
I cannot support my previous claim that hat categories extend the SGP of CCG.
This reference to the SGP of CFG and CG has been removed also.}

\subsection{Why CF-PSG?}

\emph{The chapter discusses context-freeness results for CG, so it's clearer
to name the formalism CF-PSG instead of CFG, to distinguish the formalism from
its generative power.}

\subsection{What is base generation?}

\emph{Added an explanatory footnote:}

\begin{quote}
In the context of a transformational grammar, an analysis is \emph{base
generated} if it is generated by the core phrase-structure rules, with no
movement operations. We describe a \cg analysis as \emph{base generated} if it
relies on category ambiguity, instead of assigning canonical categories and
using associative and/or permutative combinators. 
\end{quote}


\subsection{Canonical their positions}

\emph{Corrected}

\subsection{Error in Figure 2.3}

\emph{Corrected.}

\subsection{allows the responsibility}

\emph{Corrected.}

\subsection{Unbounded extraction phenomena}

\emph{Reworded as `unbounded extraction.'}

\subsection{Type-raising clearly a tautology?}

\emph{My statement was in error: type-raised categories are not in fact
tautologous, but
can be shown to be valid. I have added a proof using a truth
table.}

\subsection{``makes just as much sense'' is too colloquial}

\emph{Reworded.}

\subsection{`a lot of over-generation'}

\emph{Reworded.}

\subsection{Chart constraints}

\emph{Reworded as `constraints on derivations'.}

\subsection{Introduce the Phi notation of coordination}

\emph{Done:}

\begin{quote}
Multi-modal \ccg can thus analyse coordination phenomena without introducing a
special
conjunction rule. We will largely be dealing with the \citet{steedman:00}
grammar,however, so most of our derivations will use the
\ccgbank analysis of coordination, which approximates a ternary conjunction
rule: 
\end{quote}


\subsection{Define head-dependencies}

\emph{Added subsection Instantiating Dependencies During Unification to the
unification section to cover this.}

\subsection{Extra definition of Eisner normal form constraints}

\emph{The earlier mention of ENF was part of an introduction to the section.
I've trimmed this mention of the normal-form constraints down a little so
that it's less redundant.}

\subsection{Are ENF useless in practice?}

\emph{Clarified: the paragraph was intended to note that the proof doesn't
apply. I added a note that empirically, the constraints are still very useful
in parsing \ccgbank.}

\subsection{Funny reference formatting}

\emph{Fixed.}

\subsection{Extend description of CCGbank conversion process}

\emph{Extended, see pages 46 and 47.}

\subsection{1.1 words}

\emph{Corrected.}

\subsection{More delicate brackets}

\emph{Corrected.}

\subsection{Say what TMP and CLR refer to.}

\emph{Done.}

\subsection{How do head finding heuristics help mitigate the bracketing issue?}

\emph{Deleted this sentence.}

\subsection{first.parser}

\emph{Fixed.}

\subsection{Why 0-crossing brackets?}

\emph{Added footnote explaining that this was how \citet{magerman:94} compared against
\citet{black:93}.}

\subsection{Rewrite the statistical parsing literature review}

\emph{This section has been completely rewritten, after thorough review of the
literature. See pages 42 to 45.}

\subsection{Description of \ccgbank conversion process is too short}

\emph{Extended the description of this by approximately one page.}

\subsection{lack of constituent type a consistent representation}

\emph{Fixed.}

\subsection{why `early' sentence from Section 00?}

\emph{Removed `early'.}

\subsection{Clarify the non-type change version of the derivation.}

\emph{Done:}

\begin{quote}
 The example \citet{hock:thesis03} used to demonstrate the modifier category
proliferation is a sentence from Section 00. Without type-changing rules,
the \ccg derivation for this sentence might look like this:
\end{quote}

A footnote points the reader to further discussion of problems
analysing reduced relative clauses in \ccg later in the thesis.

\subsection{Missing period}

\emph{Fixed.}

\subsection{Note that type-changing rules are presented bottom-up}

\emph{Done.}

\subsection{Introduce PSG notation}

\emph{Done.}

\subsection{Exceptional in what sense?}

\emph{Removed this sentence.}

\subsection{S[adj category}

\emph{Fixed.}

\subsection{can cause arguments to add or delete arguments}

\emph{Fixed.}

\subsection{core argument}

\emph{Fixed.}

\subsection{Why angle bracket?}

\emph{Explained in a footnote on page 47
that it's to make it easier to refer to rules with punctuation in running text.}

\subsection{Which dependency cannot be represented?}

\emph{Reworded to refer to the specific dependency.}

\subsection{Explain CCGbank dependencies earlier.}

\emph{Done.}

\subsection{Parent/child confusing}

\emph{Labels were inverted. Corrected.}

\subsection{Type raising rule in Table 2.3}

\emph{Removed the type-raising rules from the table.}

\subsection{simple maximum entropy tagger?}

\emph{Removed `simple'.}

\subsection{supertagger cannot make the sentence shorter}

\emph{Reworded this.}

\subsection{errors are passed along monotonically}

\emph{Reworded to `propagate along the pipeline.}

\subsection{the arg max}

\emph{Reworded this.}

\subsection{Clark and Curran don't do n-best tagging}

\emph{Reworded this.}

\subsection{perform the bare minimum work to make parsing tractable?}

\emph{Reworded.}

\subsection{The global model does make any limited horizon assumptions}

\emph{Reworded.}

\subsection{``another factor might be...''}

\emph{Agree that this does not stand up to scrutiny, so removed this suggestion.}

\subsection{Cause the chart to explode in size}

\emph{Reworded to `forces the parser to consider an enormous number of potential
analyses.'}

\subsection{Well principled question answering}

\emph{Removed the claim that it was `well principled'.}

\subsection{White and Baldridge (2003)}

\emph{Corrected.}

\subsection{Corrections for some of these issues have been proposed.}

\emph{Added citations.}
