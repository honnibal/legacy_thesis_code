\documentclass[phd,sit,logo,twoside,11pt,rightchapter,notimes,a4paper]{
style/infthesis}
%\documentclass[11pt,twoside,final,english]{ahudson-harvard_dave}
%\documentclass[11pt,twoside,final]{book}

%General formatting
\usepackage{latexsym}
\usepackage[english]{babel}
\usepackage{graphics,color}
\usepackage{xspace}
\usepackage{amsmath, amsthm, amssymb}
\setcounter{secnumdepth}{4}
\usepackage{pifont}
\usepackage{bm,float}
\usepackage{times}
\usepackage{url}
\usepackage{graphicx}
\usepackage{verbatim}
\usepackage{pgf}
\usepackage{tikz}
\usepackage{mathptmx}
\usepackage{latexsym}
\usepackage{natbib}
\usepackage{subfig}
\usepackage{rotating}
\usepackage{latexsym}
\usepackage[ot2enc]{inputenc}
\usepackage[english]{babel}

% define more useful colour command (with correct spelling)
\def\colour#1#2{\color{#1}#2\color{black}}

% from /usr/X11R6/lib/X11/rgb.txt
\definecolor{gold}{rgb}{1,0.84314,0}
\definecolor{purple}{rgb}{0.62745,0.12549,0.94118}
\definecolor{darkgreen}{rgb}{0,0.39216,0}
\definecolor{brown}{rgb}{0.64706,0.16471,0.16471}

\usepackage[
  a4paper,
  pdftitle={Hat Categories: Representing Form and Function Simultaneously in Combinatory Categorial Grammar},
  pdfauthor={Matthew Honnibal},
  citebordercolor={1.0 1.0 1.0},
  linkbordercolor={1.0 1.0 1.0},
  urlbordercolor={1.0 1.0 1.0},
  citecolor=red,
  linkcolor=blue,
  urlcolor=blue,
  bookmarks=true]{hyperref}
% \usepackage[pdftex,
% citebordercolor={1.0 1.0 1.0},
% linkbordercolor={1.0 1.0 1.0},
% urlbordercolor={1.0 1.0 1.0},
% bookmarks=true,
% pagebackref]{hyperref}



% Linguistics stuffs
\usepackage{lcovington}
\usepackage{avm}
\usepackage{mattLI}
\usepackage{parsetree}
\avmfont{\sc}
\avmoptions{sorted,active}
\avmvalfont{\rm}
\avmsortfont{\scriptsize\it}
% 
% \newcommand{\cf}[1]{\mbox{$\it{#1}$}}   % category font
% \newcommand{\assign}{\vdash}
% 
% \newcommand{\gis}{\textsc{gis}\xspace}
% \newcommand{\iis}{\textsc{iis}\xspace}
% \newcommand{\pos}{\textsc{pos}\xspace}
% \newcommand{\wsj}{\textsc{wsj}\xspace}
% \newcommand{\ccg}{\textsc{ccg}\xspace}
% %\newcommand{\tag}{\textsc{tag}\xspace}
% \newcommand{\ltag}{\textsc{ltag}\xspace}
% \newcommand{\hpsg}{\textsc{hpsg}\xspace}
% \newcommand{\psg}{\textsc{psg}\xspace}
% \newcommand{\lbfgs}{\textsc{l-bfgs}\xspace}
% \newcommand{\bfgs}{\textsc{bfgs}\xspace}
% \newcommand{\cfg}{\textsc{cfg}\xspace}
% \newcommand{\pcfg}{\textsc{pcfg}\xspace}
% \newcommand{\nlp}{\textsc{nlp}\xspace}
% \newcommand{\dop}{\textsc{dop}\xspace}
% \newcommand{\lfg}{\textsc{lfg}\xspace}
% \newcommand{\pp}{\textsc{pp}\xspace}
% \newcommand{\cky}{\textsc{cky}\xspace}
% \newcommand{\gb}{\textsc{gb}\xspace}
% \newcommand{\mb}{\textsc{mb}\xspace}
% \newcommand{\ram}{\textsc{ram}\xspace}
% \newcommand{\mpi}{\textsc{mpi}\xspace}
% \newcommand{\cpu}{\textsc{cpu}\xspace}
% \newcommand{\cpp}{\textsc{c++}\xspace}
% \newcommand{\parseval}{\textsc{parseval}\xspace}
% \newcommand{\trec}{\textsc{trec}\xspace}
% \newcommand{\epsrc}{\textsc{epsrc}\xspace}
% \newcommand{\dt}{\textsc{dt}\xspace}
% \newcommand{\hmm}{\textsc{hmm}\xspace}
% \newcommand{\ghz}{\textsc{ghz}\xspace}
% \newcommand{\rasp}{\textsc{rasp}\xspace}
% \newcommand{\ldc}{\textsc{ldc}\xspace}
% \newcommand{\gr}{\textsc{gr}\xspace}
% \newcommand{\qa}{\textsc{qa}\xspace} 
% \newcommand{\jj}{\textsc{jj}\xspace}
% \newcommand{\vbn}{\textsc{vbn}\xspace}
% \newcommand{\cd}{\textsc{cd}\xspace}
% \newcommand{\rp}{\textsc{rp}\xspace}
% \newcommand{\cc}{\textsc{cc}\xspace}
% \newcommand{\susanne}{\textsc{susanne}\xspace}
% \newcommand{\bandc}{\textsc{b{\small \&}c}\xspace}
% \newcommand{\ctl}{\textsc{ctl}\xspace}
% \newcommand{\ccgbank}{CCGbank\xspace}
% \newcommand{\depbank}{DepBank\xspace}
% \newcommand{\markedup}{\emph{markedup}\xspace}
% \newcommand{\candc}{\textsc{c{\scriptsize\&}c}\xspace}
% \newcommand{\cg}{\textsc{cg}\xspace}
% \newcommand{\penn}{\textsc{ptb}\xspace}
% \newcommand{\develtwo}{\textsc{devel-2}}
% \newcommand{\deps}{\mbox{\em deps}}
% \newcommand{\cdeps}{\mbox{\em cdeps}}
% \newcommand{\dmax}{\mbox{\em dmax}}
% \newcommand{\unify}{\equiv}
% \newcommand{\nounify}{\neq}
% \newcommand{\dest}{\textsc{dest}\xspace}
% \newcommand{\nom}{\textsc{nom}\xspace}
% \newcommand{\tccg}{\textsc{tccg}\xspace}
% \newcommand{\abcg}{\textsc{ab} categorial grammar\xspace}
% \newcommand{\mmccg}{\textsc{mmccg}\xspace}
% \newcommand{\cfpsg}{\textsc{cf-psg}\xspace}
% \newcommand{\evalb}{evalb\xspace}
% \newcommand{\dol}{\textsc{dol}\xspace}
% \newcommand{\xtag}{\textsc{xtag}\xspace}
% \newcommand{\erg}{\textsc{erg}\xspace}
% \newcommand{\enju}{\textsc{enju}\xspace}
% \newcommand{\bbn}{\textsc{bbn}\xspace}
% \newcommand{\xle}{\textsc{xle}\xspace}
% \newcommand{\lingo}{\textsc{lingo}\xspace}
% 
% 
% \newcommand{\lexpunct}{LexPunct\xspace}
% \newcommand{\nattach}{NAttach\xspace}
% 
% %\newcommand{\unhat}{\textsc{unhat}\xspace}
% % Constants and things
% \newcommand{\calo}{\mathcal{O}}
% \newcommand{\caln}{\mathcal{N}}
% 
% % commands for xyling
% \newcommand{\unode}[2][]{\K{#1$_{#2}$}}
% \newcommand{\bnode}[2][]{\K{#1$_{#2}$}\V}
% \newcommand{\vpmod}{}
% \newcommand{\bks}{$\backslash$}
% 
% 
% \newcommand{\cn}{\emph{\[citation needed\]}\xspace}
% \newcommand{\eqnpsrule}[3]{#1\;\;#2 & \longrightarrow & #3}
% 
% \newcommand{\psbinary}[3]{\ensuremath{\langle #1\;\; #2\;\longrightarrow\;#3\rangle}}
% \newcommand{\psunary}[3]{\ensuremath{\langle #1\;\longrightarrow\;#2\rangle}}
% 
% \newcommand{\term}[1]{\emph{#1}}
% %\newcommand{\comment}[1]{\quote{#1}}
% 
% 
% 
% % Corpus names
% \newcommand{\nounary}{\textsc{no unary}\xspace}
% 
% % \newcommand{\noabsorb}{NoAbsorb\xspace}
% % \newcommand{\commaconj}{CommaConj\xspace}
% % \newcommand{\hattraise}{HatTRaise\xspace}
% % \newcommand{\rrcsubcat}{RRCSubcat\xspace}
% % \newcommand{\tcpunct}{TCPunct\xspace}
% 
% % Small-capped constants for AVM fields
% \newcommand{\ressc}{\textsc{res}\xspace}
% \newcommand{\argsc}{\textsc{arg}\xspace}
% \newcommand{\slashsc}{\textsc{dir}\xspace}
% \newcommand{\headsc}{\textsc{head}\xspace}
% \newcommand{\catsc}{\textsc{cat}\xspace}
% \newcommand{\featsc}{\textsc{feat}\xspace}
% 
% \newcommand{\citepos}[1]{\citeauthor{#1}'s~(\citeyear{#1})\xspace}
% \newcommand{\scare}[1]{#1}
% 
% \newcommand{\wikieval}{\textsc{wiki200}\xspace}
% \newcommand{\hatsys}{\textsc{hat}\xspace}
% \newcommand{\trsys}{\textsc{hat+tr}\xspace}
% \newcommand{\lexpunctsys}{\textsc{lex-punct}\xspace}
% \newcommand{\puresys}{\textsc{no~unary}\xspace}
% 
% \newcommand{\sew}{\textsc{sew}\xspace}
% \newcommand{\few}{\textsc{few}\xspace}
% 
% %\newcommand{\codeterm}[1]{\verb!#1!\xspace}
% \newcommand{\codeterm}[1]{\texttt{#1}\xspace}
% 
% \newcommand{\basesys}{BaseSys\xspace}
% 
% \newcommand{\smodetext}{$\star$\xspace}
% \newcommand{\xmodetext}{$\times$\xspace}
% \newcommand{\dmodetext}{$\diamond$\xspace}
% \newcommand{\cmodetext}{$\cdot$\xspace}
% \newcommand{\nullmodetext}{$\bowtie$\xspace}
% 
% \newcommand{\smode}{_\star}
% \newcommand{\xmode}{_\times}
% \newcommand{\dmode}{_\diamond}
% \newcommand{\cmode}{_\cdot}
% \newcommand{\nullmode}{_{\bowtie}}
% \newcommand{\nullhat}{\bowtie}
% 
% 
% 
% 
% \newcommand{\cB}{\textbf{\textsf{B}}}
% \newcommand{\cBx}{{\ensuremath{\textbf{\textsf{B}}_\times}}}
% \newcommand{\cBf}{>\!\cB\!\!}
% \newcommand{\cBb}{<\!\cB\!\!}
% 
% 
% \newcommand{\cT}{\textbf{\textsf{T}}}
% \newcommand{\cS}{\textbf{\textsf{S}}}
% \newcommand{\cH}{\textbf{\textsf{H}}}
% \newcommand{\bk}{\ensuremath{\backslash}}
% \newcommand{\ra}{\ensuremath{\rightarrow}}
% \newcommand{\Ra}{\ensuremath{\Rightarrow}}
% 
% \newcommand{\Sfapply}{\ensuremath{\!>\!}}
% \newcommand{\Sbapply}{\ensuremath{\!<\!}}
% \newcommand{\Sfcomp}{\ensuremath{\!>\!\!\cB\!}}
% \newcommand{\Sbcomp}{\ensuremath{\!<\!\cB\!}}
% \newcommand{\Sfxcomp}{\ensuremath{\!>\!\!\cB_{\!\times}\!}}
% \newcommand{\Sbxcomp}{\ensuremath{\!<\!\cB_{\!\times}\!}}
% \newcommand{\Sftype}{\ensuremath{\!>\!\!\cT\!}}
% \newcommand{\Sbtype}{\ensuremath{\!<\!\cT\!}}
% \newcommand{\Sfsubst}{\ensuremath{\!>\!\!\cS\!}}
% \newcommand{\Sbsubst}{\ensuremath{\!<\!\cS\!}}
% \newcommand{\Sfxsubst}{\ensuremath{\!>\!\!\cS_{\!\times\!}}}
% \newcommand{\Sbxsubst}{\ensuremath{\!<\!\cS_{\!\times\!}}}
% 
% \newcommand{\derivs}{\textsc{deriv}\xspace}
% \newcommand{\derivsbad}{\textsc{deriv}$_{0.1}$\xspace}
% \newcommand{\derivsrev}{\textsc{deriv}$_{0.0045}$\xspace}
% \newcommand{\derivsthree}{\textsc{deriv}$_{0.003}$\xspace}
% \newcommand{\derivsexp}{\textsc{deriv}$_{0.002}$\xspace}
% \newcommand{\hybrid}{\textsc{hybrid}\xspace}
% \newcommand{\hybridv}{\textsc{hybrid}$^v$\xspace}
% \newcommand{\optbeta}{\textsc{opt}\xspace}
% 
% \newcommand{\nltk}{\textsc{nltk}\xspace}
% \newcommand{\act}{\textsc{act}\xspace}

\usepackage{bm,float}
\usepackage{times}
\usepackage{url}
\usepackage{graphicx}
\usepackage{verbatim}
\usepackage{pgf}
\usepackage{tikz}
\usepackage{mathptmx}
\usepackage{latexsym}
\usepackage{mattLI}
\usepackage{natbib}
\usepackage{parsetree}
\usepackage{xspace}
\usepackage{lcovington}
\usepackage{avm}
\usepackage{subfig}
\usepackage{rotating}

\avmfont{\sc}
\avmoptions{sorted,active}
\avmvalfont{\rm}
\avmsortfont{\scriptsize\it}

\usepackage{xyling_dave}

\usepackage{latexsym}
\usepackage[ot2enc]{inputenc}
\usepackage[english]{babel}
\usepackage{graphics,color}
\usepackage{xspace}
%\usepackage{longtable}
\usepackage{amsmath, amsthm, amssymb}
%\usepackage{caption}

\setcounter{secnumdepth}{4}
%\setcounter{tocdepth}{3}

%\setcounter{tocdepth}{2}

\newcommand{\degreemonth}{August}
\newcommand{\degreeyear}{2009}
\newcommand{\degree}{Doctor of Philosophy}
\newcommand{\field}{Computer Science}
\newcommand{\department}{School of Information Technologies}
\newcommand{\advisor}{James R. Curran}



\abstract{
Combinatory Categorial Grammar \citep[\ccg, ][]{steedman:00} is a lexicalised
grammar formalism, which means that each word is associated with a category that
specifies its argument structure and grammatical function. This accounts for the fact that
different words have different argument structures, and allows the grammar to
generate semantic analyses, instead of skeletal brackets. \ccg's lexicalisation has
been found to improve parsing speed and accuracy \citep{clark:cl07}, and constitutes
 a strong theoretical claim about what aspects of language are universal.

However, wide-coverage \ccg parsing and generation systems currently use
a set of type-changing rules that make their grammars less lexicalised.
The rules were included to address sparse data problems that wide-coverage
categorial grammars otherwise encounter. Without type-changing rules, categorial
grammars require a proliferation of modifier categories, because the grammar
is not always able to abstract away irrelevant detail about derivational context
when constructing modifier categories.

We analyse this modifier category proliferation problem in detail, and conclude
that the \ccg formalism makes it difficult to write efficient grammars for these
constructs. The inefficiencies
are introduced because the formalism has difficulty exploiting generalisations about
syntactic form. This motivates some system of type-changing rules, in order to
have more words receive categories that reflect their syntactic form. However,
the existing system of type-changing rules reduces the level of lexicalisation
in \ccg, weakening some of the formalism's most important properties.

We propose a solution that allows type-changing rules to be lexicalised. Our proposal
involves a modification to the category objects. \ccg
categories are composed of a number of attributes, including a type, a
directional slash, and a feature structure. We propose an additional attribute,
that represents the grammatical function that the category will
later perform in the derivation. This allows the immediate type to represent the
constituent's syntactic form, enabling better analyses of form/function discrepancies.
To avoid overusing \emph{function}, we dub this new attribute a \emph{hat}.

To investigate the modifier category proliferation problem and evaluate our solution,
we create and compare two versions of \ccgbank. In addition to our \hatsys solution,
we create a \nounary version of the corpus, which uses no type-changing rules.
We evaluate these corpora by training and evaluating a
state-of-the-art \ccg parser on the different versions of the corpus. We find
that the \nounary corpus makes parsing more difficult: accuracy decreased by
2.2\%.
The \hatsys version, however, achieved accuracy within 0.3\% of the original parser,
and increased in speed by 48\%.

In language processing, efficiency has often been contrasted with linguistic
desirability or depth, but this distinction is artificial. Linguistic theories
are keenly interested in how the human parser works as quickly as it does, and
lexicalised grammars may be part of the answer. We have identified a key problem
in the \ccg theory, one aspect of which has caused \ccg parsers to adopt a lower
level of lexicalisation. By restoring that lost lexicalisation, we bring \ccg
parsing practice back in line with the core claims of the theory, to its
immediate practical advantage.
}

\begin{document}

\title{Hat Categories: Representing Form and Function Simultaneously in Combinatory Categorial Grammar}
\author{Matthew Honnibal}
%***Update June
\maketitle
%\copyrightpage

% \standarddeclaration


%\declarationpage

%\begin{citations}
%\cite{vadas:07}
%\end{citations}


%***acknowledgements

%\begin{acknowledgments}
%\end{acknowledgments}

\cleardoublepage
\tableofcontents
\listoffigures
%***blank page?
\listoftables

\cleardoublepage
%\startarabicpagination



\chapter{Introduction}

Humans do not generate or interpret sentences as unstructured strings of words.
Instead, words are combined together to compose meaning as some sort of tree or
graph, until they form a sentence licensed by the language. The exact mechanisms
that govern the composition and generation of sentences in a natural language
continue to elude us, despite the unconscious ease with which children acquire
them, and our interest in the problem since at least Aristotle.

Without a theory of grammar, we
are limited to modelling sentences as strings of words, which means our systems
must deal with surface variation. Programs which
attempt to solve this problem by assigning a grammatical structure to a sentence
are called \emph{parsers}. Parsing researchers quickly discovered the
massive ambiguity of natural language grammars, and found that deterministic
approaches to parsing were infeasible. Natural grammars must work stochastically.

Research on statistical parsing for English began in earnest with the release of
a large sample of annotated text, the Penn Treebank \citep{marcus:93}.
\citet{magerman:95}, \citet{collins:96} and \citet{charniak:97} were the first
in a long line of research into building statistical parsing models from this
resource, which used an annotation that was not designed for any specific linguistic
theory.
These statistical parsers were a major departure from
the older parsing tradition that relied on carefully constructed hand-written
grammars, perhaps with a minimal probabilistic component. Soon after
\citeauthor{collins:96}'s models, a hybrid approach emerged. The newer strategy
was to convert the Penn Treebank into a treebank for a specific formalism,
allowing statistical parsing experiments informed by theories about the human
language faculty.

The first example of this was the conversion of the Penn Treebank into a
Tree-Adjoining Grammar \citep[\textsc{tag},][]{joshi:85} by \citet{xia:99},
\citet{xia:00} and later \citet{chen:06}. It was quickly followed by work on
a Lexical Functional Grammar \citep[\lfg,][]{kaplan:82} conversion. The first
to attempt this was \citet{genabith:99}, whose work was followed by \citet{sadler:00}
and \citet{frank:00}, before \citet{cahill:08} performed further work on the
issue.
In the meantime, \citet{hock:thesis03} released a Combinatory Categorial Grammar
\citep[\ccg,][]{steedman:00} conversion, and
\citet{miyao:04} did the same for Head-driven Phrase Structure Grammar
\citep[\hpsg,][]{pollard:94}.

The hypothesis behind this strategy is that linguistics and statistics are both
critical for successful wide-coverage parsing. The strategy has so far been very
successful. It has allowed statistical parsers to offer far more detailed
output, and also led to enormous increases in parse speeds. The \hpsg and \ccg
treebanks have produced two particularly successful statistical parsers, the
\enju system \citep{miyao:08} and the \candc system \citep{clark:cl07}. These
parsers share the same fundamental architecture. They make use of the two
principal advantages of modern linguistic theories over the context-free phrase
structure grammar (\cfpsg) used in the original Penn Treebank. First, the
interface between syntactic and semantic structure is fully
specified in both \hpsg and \ccg, allowing a parser to return logical forms.
This property of these formalisms is closely related to their
second big advantage: they are naturally
\emph{lexicalised} --- that is, each word is assigned responsibility for
specifying the structure of the part of the sentence that it governs. Together,
these properties have been exploited to produce parsers that are faster and 
produce more detailed output, and independent evaluations have shown they also offer
state-of-the-art accuracy \citep{kakkonen:08, rimell:09}. The success of these
parsers supports the hypothesis that well grounded grammatical theory can make
a substantial impact on parsing technology, suggesting that we can produce better
parsers by refining our linguistic theories.

In this thesis, we focus on a problem in Combinatory Categorial Grammar, and
update \ccgbank and the \candc parser to
make use of the solution we propose. The problem we focus on was first
identified when \citet{hock:lrec02} began the conversion of the Penn Treebank
into \ccgbank. They found that \ccg's
lexicalisation caused a proliferation of modifier categories. The problem
arises when a modifier (such as \emph{walking} in \emph{I tripped walking home})
is itself modified (such as \emph{via Cleveland Street} in \emph{I tripped walking home
via Cleveland Street}). The category of \emph{via} becomes sensitive to the fact
that \emph{walking} is a modifier, producing a proliferation of
modifier-modifier, modifier-modifier-modifier, etc.\@ categories.
To prevent this, \citet{hock:lrec02} introduced an auxiliary set of category-specific
type-changing rules into the grammar. This solved the sparse data problems, but
introduced a divide between the core claims of the linguistic theory and the
grammar that wide-coverage \ccg parsers actually implemented.

We suggest that the observation of problems like modifier category
proliferation should be treated as important evidence of a
theory's limitations. We hypothesise that if we capitalise on these
opportunities to improve the theory, we will develop better parsing technologies
than if we instead treat the issue as an engineering problem. This suggests an
appealing feedback loop. The use of linguistic theories
for parsing provides a stress test for a grammatical theory that is difficult to
replicate in the traditional methodology, which emphasises native speaker
intuition. Creating a corpus and training a parser forces us to consider
analyses for the interaction of constructions which otherwise might have only
been considered separately. The parsing phase is particularly critical, because
it forces us to model the \emph{process} of language understanding, instead of
focusing primarily on modelling the linguistic objects.
\citet{baldwin:05} discuss a similar feedback loop between data and intuition for
grammar development.

The modifier category proliferation problem is a good example of a hidden interaction
between constructions we might otherwise assume are independent. In a categorial
grammar, modifiers are assigned categories which refer to the function of their
head, because there is no special grammatical rule for adjunction --- a modifier
is interpreted as a function from a constituent, to an unchanged copy of that
constituent. \citeauthor{hock:lrec02} noticed that this created a proliferation of
modifier categories when modification interacted with form/function discrepancies,
in constructions such as gerund nominals or
reduced relative clauses.

This introduced a substantial inefficiency in the grammar that
\citet{hock:lrec02} feared would have a negative impact on parsing results.
To prevent this, the grammar was extended with a small set of carefully selected
type-changing rules to handle form/function discrepancies. These rules allow
a modifier to refer to its head's form, preventing the proliferation of
function-referencing modifier categories. This adjustment has not been
integrated into the \ccg theory --- it is perhaps seen as an engineering
solution specific to the Penn Treebank. We argue that
if we accept such `engineering solutions' in our grammar, we short-circuit the virtuous
cycle mentioned above. If we encounter a problem building the corpus, we should
take it seriously in the theory --- and the solution we adopt in the corpus
should be theoretically satisfactory.

We claim that modifier category proliferation is a general problem in \ccg,
caused by the inability of the grammar to handle form/function discrepancies
adequately. We also claim that the type-changing rule solution is
incompatible with the most important aspects of the \ccg theory, so we should
seek a better approach. We suggest an extension to the formalism that
addresses the problem without these drawbacks, implement it in \ccgbank, and
find that the cleaner resource allocation between grammar and lexicon it
promotes makes the parser over 40\% more efficient.

\section{The Thesis Proposed}

In Combinatory Categorial Grammar, every word is assigned a category that encodes
the function of the constituent it heads. This direct representation of constituent
function is behind many of the desirable properties of the formalism: it allows lexical
categories to be paired with semantic analyses, enables attractive
analyses of coordination constructions, and allows language-specific analysis to
be shifted into the lexicon. However, it also makes it difficult for a \ccg
grammar to exploit generalisations about constituent type --- the syntactic
category of a phrase, as distinct from its grammatical function. Missing
constituent type generalisations can cause over-generation, undesirable analyses,
and prevent the grammar from fully generating certain recursive structures
with a finite lexicon. These issues largely arise because the grammaticality
of modifier-head relationships is controlled by constituent type, not
constituent function.

The most apparent consequence of these problems is a proliferation of modifier
categories in the lexicon. \citeauthor{hock:lrec02} addressed this by adding
type-changing rules to the grammar. The rules perform the form-function
transformations required, but they do so at a substantial cost: they make the
grammar less lexicalised. This means the grammar is no longer language
universal, and grammatical rules must perform non-trivial semantic operations. The
rules are therefore unacceptable on a theoretical level, because they compromise the
main claims of the formalism. What we need is a mechanism that can do the same sort
of work --- form to function transformation --- while still preserving full
lexicalisation.

Our solution is to modify the construction of \ccg category objects, adding a
new attribute. This attribute contains a category that the original category must at
some point be transformed into. We call this new attribute the \emph{hat}, and categories
with the hat attribute filled we call \emph{hat categories}. Hat categories allow
advantageous linguistic analyses, without changing the grammar's weak generative
capacity. Although we have
designed hat categories for use in \ccg, the extension might be applied to any
categorial grammar.

We investigate the modifier category proliferation problem and our proposed
solution by adapting \ccgbank, creating a number of versions that each
implements a different analysis. This allows us to compare the analyses
directly. In addition to a version of \ccgbank that implements our hat category
solution, we create a version of the corpus that simply removes the
type-changing rules, replacing them with the purely combinatory analyses
described in the \ccg literature. This version of the corpus allows us to
confirm that modifier category proliferation does indeed present a practical
challenge for \ccg parsing.

Our main parsing result demonstrates that the type-changing rules in \ccgbank
are also problematic for a \ccg parser. The rules greatly increase the ambiguity
in the grammar, leading to larger charts and slower parse times. Because there
is this cost to additional type-changing rules, the \candc parser does not
implement all of the rules, which decreases the parser's coverage and may make
the parser more domain dependent. To investigate this, and to evaluate our
hat category solution on a more practical parsing task, we performed the first
 evaluation of parser accuracy on text from Wikipedia.

\section{Contributions}

The main contribution of this thesis is an extension to \ccg, \emph{hat
categories}, that enables simultaneous representation of form and function in
lexical categories. This allows us to report the first successful wide-coverage,
fully-lexicalised \ccg parsing results. We find that restoring full
lexicalisation to \ccgbank increases the efficiency of \ccg parsing
substantially. Additionally, full lexicalisation has
important theoretical benefits. The last language specific constraints have
recently been removed from the \ccg formalism \citep{baldridge:03}, suggesting
that all language, domain and analysis specific variation could be confined to
the lexicon. However, \ccg parsers have until now used grammars that are not
fully lexicalised, causing a disconnect between \ccg theory and practice.

Another contribution of this thesis is to show that without some kind of
type-changing operation, statistical \ccg parsing is a much harder problem.
Although the experiment had not been performed until now, it was suspected that a
\ccg grammar without the \ccgbank type-changing rules would suffer from
a prohibitively sparse category set. We confirm that this is the case,
by compiling out the type-changing rules
in \ccgbank so that only the standard \ccg combinators are used. This causes a
substantial drop in parsing accuracy. This result shows that although
lexicalisation is highly desirable, it is difficult to achieve with the current
\ccg formalism without impacting parsing results, motivating the extension we present.

This negative parsing result is supported by a detailed theoretical investigation of
why fully lexicalised \ccg analyses are ineffective without our hat categories.
We show that the proliferation of modifier categories is due to the difficulty
of representing generalisations about constituent type in \ccg grammars.

The results of our experiments bring into focus an interesting open question:
what is the right level of lexicalisation in a corpus? The experiments we describe
are, to our
knowledge, the first that compare analyses that vary in this respect
on a corpus, and evaluate their consequences. Having successfully lexicalised
the \ccgbank unary rules, we searched for an upper limit: what else could we shift
out of the grammar and into the lexicon? To examine this, we experimented with
an analysis that used our hat categories to lexicalise type-raising. This
contributes to the ongoing debate about whether type-raising rules should be
considered part of the grammar or part of the lexicon, by providing the first
concrete proposal for how type-raising can be lexically specified without
increasing modifier category ambiguity.  We found that lexicalising
type-raising in this way increased parsing efficiency, but at the cost of
decreased accuracy.


In summary, our parsing results suggest the lexicalisation strategy already
being exploited to great effect by the \candc parser can be pushed even further.
However, there is a potential drawback to doing this. What if increasing the
level of lexicalisation also increases the parser's domain dependence? This is a
risk because the lexicon must be acquired specifically from the training
material, while grammar rules can be manually specified. To investigate this, we
report the first accuracy evaluation of a parser on text from Wikipedia, a free online
encyclopedia.
We evaluate the hat category model on this out-of-domain data, and find
that it achieves accuracy within 0.1\% of the standard \candc parser, with a
28\% improvement in efficiency. This was similar to our finding on the in-domain
evaluation, where we observed a 0.2\% decrease in accuracy and a 45\%
improvement in efficiency. These results support our claim that there is a
practical advantage in pursuing theoretically desirable formal properties. 

This thesis argues that there is a long-standing problem with \ccg's ability to
express generalisations about constituency types. We propose grammatical machinery that
addresses this issue and implement our solution on a large corpus of \ccg
derivations. We confirm that this does not adversely affect a state-of-the-art
\ccg parser, and find that it actually causes a substantial increase in its
efficiency. We show that this result is robust to data outside the training
domain, and provide the first investigation of parser performance on one of the
most interesting modern resources for natural language processing, Wikipedia.

\section{Publications}

Our proposal to add hat categories to \ccg, with the corresponding parsing
results, has been published in the 2009 \emph{Proceedings of the Conference on
Empirical Methods in Natural Language Processing}. Our evaluation of the \candc
parser on Wikipedia was published in the 2009 \emph{Proceedings of the ACL/IJCAI
Workshop on Collaborative Semantic Resources}.

The research described in this thesis was supported by the results of earlier
work not described here. The integration of \ccgbank and PropBank described in
\citet{honnibal:pacling07prop}, and the adaptation of the Penn Treebank to
produce a Systemic Functional Grammar corpus described in
\citep{honnibal:dlp07sfl} helped us to develop the methodology for adapting
\ccgbank described in Chapter \ref{chapter:hat_corpus}. 

\section{Outline}

The remainder of the thesis is organised as follows:

\begin{itemize}
 \item Chapter 2 provides background in the form of an overview of 
Combinatory Categorial Grammar, and a
review of the prominent computational linguistics literature that makes use of
the theory.
 \item In Chapter 3, we argue that it is difficult to represent generalisations
about constituent type in \ccg, and that this causes a variety of problems for
\ccg analyses.
We discuss previous proposals which allow the formalism to address this,
but find that none of them adequately addresses the core issue.
 \item In Chapter 4, we propose an extension to the formalism that allows \ccg to
represent constituent types consistently and efficiently. The extension modifies
the category objects to allow them to specify a \emph{hat}, enabling categories
to specify disparate form and function.
 \item In Chapter 5, we describe the creation of new versions of \ccgbank to
investigate the impact of removing the type-changing rules from the corpus.
 \item In Chapter 6, we describe our parsing experiments with the \candc parser.
The focus of the chapter is an investigation of the impact of hat categories on
the parser's performance. The chapter also includes a series of experiments with
the hyper-parameters that must be adapted to suit the new corpus, and an
evaluation of the parser on a new domain, Wikipedia.
\end{itemize}


\chapter{Combinatory Categorial Grammar}
\label{chapter:background}
Categorial grammars (\cg) have their oldest roots in philosophy, as a logical
notation. Philosophy and logic have always maintained an interest in natural
language, so the suitability of the formalism for syntactic description was
commonly noted, even though it was not a central concern. \citet{wood:93} finds
early kernels of \cg ideas in \citepos{frege:1879} analysis of a proposition
into function and argument, rather than subject and predicate. The first full
formulation of a recognisable categorial grammar was presented by
\citet{ajdukiewicz:35}, but it was not until \citet{bar-hillel:53} that a \cg
description of a variety of syntactic phenomena was presented, motivated by the
post-war interest in machine translation that followed the success of digital
cryptanalysis \citep{wood:93}.

Interest in \cg decreased for some time after \citet{bar-hillel:60} proved that
categorial grammars could only generate context-free languages, while natural
languages were considered at the time to be context sensitive.\footnote{\citet{pullum:82}
showed that the existing arguments for the non-context freeness of
natural language were flawed, but they were nonetheless taken
seriously at the time, and \citet{wood:93} ascribes the partial loss of interest in \cg
to this. Natural languages were finally shown to be non-context free by \citet{shieber:85}
and \citet{culy:85}.}
It was considerably later that \cg's advantage --- a simple isomorphism between syntax and
semantics --- motivated
proposals that aimed to increase its weak generative capacity. The proposals can be divided
into two groups. Type-logic grammars \citep{lambek:88,benthem:88,morrill:94,moortgat:88}
extend \citepos{lambek:58} algebraic categorial calculus, and are largely concerned with the
grammar's formal and logical properties. The other approach is to extend the
\citeauthor{ajdukiewicz:35} and \citeauthor{bar-hillel:53} formulation, referred to as either
\textsc{ab} and pure categorial grammar, using combinatory rules drawn from \citet{curry:58}.

Combinatory Categorial Grammar (\ccg), first formulated by \citet{ades:80},
and presented more fully in \citet{steedman:00}, can be distinguished from the
categorical type-logic tradition by its close attention to linguistic and
psychological plausibility. The upshot of these concerns is a keen interest in
restricting the generative power of the formalism to precisely match the
complexity observed in natural languages. The restrictions also improve the
grammar's computability, making the formalism attractive for use in language
technology and arguably increasing its psychological plausibility
\citep{steedman:00}.

This thesis is concerned with the theoretical and practical properties of
combinatory categorial grammar, but we will start by defining the simpler
\textsc{ab} categorial grammar it is based on. We will then describe the rules
\citeauthor{steedman:96} introduced. These rules allow the grammar to handle the
long-distance dependencies that arise in coordination and extraction
constructions. We
also describe an important addition to \ccg that draws on categorical type-logic,
\citepos{baldridge:03} multi-modal slashes. This extension is the
limit of our interest in the type-logical tradition of work on categorial
grammars.

Once we have described the grammar, we turn to its recent use in natural language
engineering. The work most relevant to this thesis begins with the creation of a corpus of
\ccg derivations created by semi-automatically adapting
the Penn Treebank, \ccgbank \citep{hock:acl02}. This initiated a fruitful line of research in statistical \ccg parsing,
culminating in the \candc parser \citep{clark:cl07}, which we will use for our experiments
in Chapter~\ref{chapter:results}. We briefly review other prominent applications of \ccg in
natural language processing; for semantic analysis, natural language generation, and machine
translation.


\section{Category Definition}
\label{sec:ab_cat}

Categorial grammar (\cg) is a lexicalised framework, which means that the
majority of the syntactic information required to build a derivation is
specified in a sequence of composite objects paired with the input. One object
is assigned to each phonologically realised, meaning-bearing unit in a sentence,
which in English usually means one object per orthographic token. In a
categorial grammar, these composite objects are called \emph{categories}.

\cg lexical categories are composed of a result category, and potentially an
argument. Categories which specify an argument are called \emph{functors}, or
\emph{complex categories}; categories without an argument are referred to as
\emph{atomic}. In most \cg formulations, functors can specify the direction of
their argument. In the notation we will use throughout this thesis, the result
category is written on the left, with the argument to its right after a forward
(\cf{/}) or backward (\cf{\bs}) slash, for rightward and leftward arguments
respectively. We also occasionally use an undirected slash, \cf{|}, to
under-specify a slash from \cf{\{/, \bs\}}.

The \citet{ajdukiewicz:35} and \citet{bar-hillel:53} formalism
(termed \abcg) used just two atomic categories in its description,
\cf{N} (for substantives) and \cf{S} (for clauses).
This set of categories is generally considered
insufficient for a detailed linguistic description, but it is a useful
simplification. The \ccgbank grammar \citep{hock:man05} introduces two extra
categories, \cf{PP} (for prepositional phrases) and \cf{NP} (for noun phrases).
Arguably, a fifth category, \cf{PT}, would also
be useful, to distinguish verb particles from prepositions
\citep{constable:09}. Some punctuation symbols are also assigned
unique atomic categories, as are conjunctions.

Categories with multiple arguments can be specified by constructing a functor
whose result is itself a functor. Thus the category \cf{(S\bs NP)/NP} specifies
that the word requires two arguments --- an \cf{NP} to the right, and then an
\cf{NP} to the left --- to build an \cf{S} constituent. The relationship with
context-free phrase-structure rules is easy to see if we depict the
category's assignment to the verb \emph{likes} as a tree, and add some lexical
assignments for nouns:
\vspace{0.1in}
\begin{center}
\begin{tabular}{ccc}
\ptbegtree
\ptbeg \ptnode{\cf{NP}} \ptleaf{Casey} \ptend
\ptendtree
&
\ptbegtree
\ptbeg \ptnode{\cf{S}}
  \ptleaf{\cf{NP}}
  \ptbeg \ptnode{\cf{S[dcl]\bs NP}}
    \ptbeg \ptnode{\cf{(S[dcl]\bs NP)/NP}} \ptleaf{likes} \ptend
    \ptleaf{NP}
  \ptend
\ptend
\ptendtree
&
\ptbegtree
\ptbeg \ptnode{\cf{NP}} \ptleaf{Pat} \ptend
\ptendtree
\end{tabular}
\end{center}
\vspace{0.1in}
In this thesis, we define a category as a 5-tuple
$\langle\ressc, \featsc, \slashsc, \argsc, \headsc\rangle$, where:
\begin{itemize}
\addtolength{\itemsep}{-1mm}
 \item \ressc is a result category;
 \item \featsc is a feature structure;
 \item \slashsc is a directional slash;
 \item \argsc is an argument category;
 \item \headsc is a lexical head.
\end{itemize}

If the \argsc attribute is not empty, the category is complex; if only a \ressc
is specified the category is atomic. We assume that \featsc is a list of attribute-value
pairs, where values are always atomic. Attribute values can be underspecified, by assigning
the attribute an index variable. This allows attributes to be coindexed to each other,
providing a way for information to be shared across a category.

Figure~\ref{avm:likes} shows an \textsc{avm} representation of the category
\cf{(S[dcl]\bs NP)/NP}. The category is first broken down into its result
\cf{S[dcl]\bs NP} and its argument \cf{NP}, along with a slash denoting
the argument directionality, \cf{/}.
In general we follow \citet{hock:cl07} in not assigning features to
complex categories.
We have omitted attributes with empty
values, for brevity.

The head of the category is \emph{likes}, and its result is a complex category,
\cf{S[dcl]\bs NP}, also headed by \emph{likes}. The result has an \cf{NP} argument,
with a backward slash indicating it must occur to the left. We have specified a feature,
\emph{case}, for the \cf{NP} arguments, to illustrate the kind of linguistic information
typically encoded in the feature structures. However, we generally omit such
features from our analyses, and only assign features to \cf{S} categories. The
feature \cf{dcl} indicates that its category is a finite declarative.


\begin{figure}[t!]
\centering
\begin{avm}
[{}  \ressc  & [{} \ressc  & [{} \ressc  & \cf{S}\\
	                         \featsc & [{} \textsc{tense} & \emph{dcl}]\\
			         \headsc & \emph{likes}\\
		             ]\\
                    \slashsc& \bks \\
                    \argsc & [{} \ressc  & \cf{NP}\\
	                         \featsc & [{} \textsc{case} & \emph{nom}]\\
			         \headsc & $y$\\
		             ]\\
		\headsc & \emph{likes}\\
	    ]\\
     \slashsc& \cf{/} \\
     \argsc  & [{} \ressc  & \cf{NP} \\
                   \featsc & [{} \textsc{case} & \emph{acc}]\\
		   \headsc & $z$\\
	       ]\\
     \headsc & \emph{likes}\\
]
\end{avm}
\caption{Attribute-value matrix for \emph{likes} $\assign$ \cf{(S[dcl]\bs NP)/NP}
\label{avm:likes}}
\end{figure}

The unbound $y$ and $z$ variables that occupy the values of \headsc denote that
the two arguments must be filled by different lexical items. If two \headsc
attributes share an index variable, and one attribute is bound to a lexical item,
the other attribute will be bound to that lexical item too. It is this mechanism
that allows long-range dependencies to be created. The details of this process are
explained in Section~\ref{sec:unification}.

\section{Function Application}

\textsc{ab} categorial grammars use only one type of grammatical rule, function
application:

\begin{eqnarray}
\cf{X/Y} & \cf{Y} & \Rightarrow\;\; \cf{X}\label{rule:f.app}\\
\cf{Y} & \cf{X\bs Y} & \Rightarrow\;\; \cf{X}\label{rule:b.app}
\end{eqnarray}

These rules allow us to complete the derivation of the \emph{Casey likes Pat} example:

\begin{center}
\deriv{3}{
\rm Casey & \rm likes & \rm Pat \\
\uline{1}&\uline{1}&\uline{1} \\
\cf{NP} &
\cf{(S[dcl]\bs NP)/NP} &
\cf{NP} \\
& \fapply{2} \\
& \mc{2}{\cf{S[dcl]\bs NP}} \\
\bapply{3} \\
\mc{3}{\cf{S[dcl]}}
}\end{center}

The horizontal lines indicate the span of the constituent created, and the
symbol at the right edge of the line indicates the grammatical rule that
licenses the step of the derivation. The resulting category is listed below each
bracket. This tabular presentation of a \cg derivation will be used
interchangeably with tree-based representations throughout the thesis. We will
tend to use derivations where we wish to emphasise the rules being used,
while parse trees often provide an easier way to view attachment structure.



\section{Unification}
\label{sec:unification}
Unification is a mechanism for finding the union of the information represented in
two feature structures \citep{shieber:86}. Each \cg rule involves unifying two
categories during the string concatenation that the rule describes. In forward
application (\ref{rule:f.app}) and backward application (\ref{rule:b.app}), the
argument of the functor is unified with the argument category, and the functor's
result is returned as the product of the rule.

The argument unification can be seen in more detail using the \textsc{avm}
representation in Figure~\ref{avm:fwd_app}. The left-most \textsc{avm}
represents the category \cf{(S[dcl]\bs NP)/NP}, and to its right is the category
\cf{NP}, which will unify with the outer-most argument of \cf{(S[dcl]\bs
NP)/NP}. The result of this unification is shown in the \textsc{avm} on the
right. The missing values have been filled in from the \cf{NP} argument.
Specifically, the category now has a lexical head \emph{Pat}, and it has
inherited the feature and value pair \emph{type: proper}.

\begin{sidewaysfigure}
\begin{center}
\begin{avm}
[{}  \ressc  & [{} \ressc  & [{} \ressc  & \cf{S}\\
	                         \featsc & [{} \textsc{tense} & \emph{dcl}]\\
			         \headsc & \emph{likes}\\
		             ]\\
                    \slashsc& \bks \\
                    \argsc & [{} \ressc  & \cf{NP}\\
	                         \featsc & [{} \textsc{case} & \emph{nom}]\\
			         \headsc & $y$\\
		             ]\\
		\headsc & \emph{likes}\\
	    ]\\
     \slashsc& \cf{/} \\
     \argsc  & [{} \ressc  & \cf{NP} \\
                   \featsc & [{} \textsc{case} & \emph{acc}]\\
		   \headsc & $z$\\
	       ]\\
     \headsc & \emph{likes}\\
]
\end{avm}\;\;
\begin{avm}
\raisebox{-60mm}{[{} \ressc  & \cf{NP} \\
    \featsc & [{} \emph{type} & proper]\\
    \headsc & \emph{Pat}\\
]}
\end{avm}
\;\;\raisebox{-55mm}{\Huge $\longrightarrow$}\;\;
\begin{avm}
[{}  \ressc  & [{} \ressc  & [{} \ressc  & \cf{S}\\
	                         \featsc & [{} \textsc{tense} & \emph{dcl}]\\
			         \headsc & \emph{likes}\\
		             ]\\
                    \slashsc& \bks \\
                    \argsc & [{} \ressc  & \cf{NP}\\
	                         \featsc & [{} \textsc{case} & \emph{nom}]\\
			         \headsc & $y$\\
		             ]\\
		\headsc & \emph{likes}\\
	    ]\\
     \slashsc& \cf{/} \\
     \argsc  & [{} \ressc  & \cf{NP} \\
                   \featsc & [{} \textsc{case} & \emph{acc}]\\
				 \textsc{type} & \emph{proper}\\
		   \headsc & $z$\\
	       ]\\
     \headsc & \emph{likes}\\
]
\end{avm}
\end{center}
\caption{Attribute-value matrices for \cf{(S[dcl]\bs NP)/NP \;\; NP \;\;
\longrightarrow \;\; S[dcl]\bs NP}\label{avm:fwd_app}}
\end{sidewaysfigure}

During unification, each field in the two categories is compared. Where one
field's value is underspecified --- as is the \scare{type} value of the
functor's argument --- the union inherits the more specific value. If the two
values conflict, unification fails, preventing the rule from being applied.
Unification proceeds recursively, so if \argsc or \ressc contain complex
categories, or \featsc has a complex value, the components are unified as well.

\section{Logical Forms with Hybrid Dependency Logic Semantics}
\label{sec:hlds_background}

One of the key properties of categorial grammars is the ease with which they can
be made semantically transparent. A grammar is said to be semantically transparent if
it fully specifies an interface between syntax and semantics, such that every
syntactic derivation corresponds to exactly one semantic analysis.

Categorial grammars are made semantically transparent by associating every lexical
category with a logical form representation.
We follow \citet{baldridge:02} in using Hybrid Dependency Logic Semantics
\citep[\hlds, ][]{kruijff:01} to represent logical forms for \cg categories.
\hlds is a hybrid logic, which means
that it extends modal logic with \emph{nominals} to allow states to be referenced
explicitly. Formulas can be formed using both nominals and propositions, 
standard boolean operators, and the satisfaction operator $@$. A formula
$@_i\semf{p}$ states that the formula $\semf{p}$ holds at the state named by $i$.

Meanings are expressed as a conjunction of modalised terms, anchored by the head that
identifies the head's proposition:

\begin{eqnarray}
 @_h(\text{proposition} \wedge \langle\delta_i\rangle (d_i \vee dep_i))
\end{eqnarray}

Dependency relations are modelled as modal relations $\langle\delta_i\rangle$, and the
discourse referents connected by the dependency relations are each assigned a nominal $d_i$.
Because \hlds is an indexed representation, propositions can be flattened
to a conjunction of fixed-size elementary predications. We follow \citet{white:03} in
adopting this representation. In their approach, syntactic categories are paired with
a flattened representation, with coindexation connecting the two elementary predications:

\begin{center}
\begin{tabular}{lcrcl}
\emph{saw} & $\assign$ & \cf{(S[dcl]_x\bs NP_y)/NP_z} & $:$ &
$@_x \text{\textbf{see}} \wedge @_x\langle \textsc{tense}\rangle \text{\textbf{past}} \wedge @_x\langle \textsc{act}\rangle y \wedge @_x\langle \textsc{pat}\rangle z$\\
\emph{Bob} & $\assign$ & \cf{NP_b}  & $:$ & $@_b\text{\textbf{Bob}}$ \\
\emph{Gil} & $\assign$ & \cf{NP_g}  & $:$ & $@_g\text{\textbf{Gil}}$ \\
\end{tabular}
\end{center}

This flattened representation is a graph of
variable bindings and 
$\langle \text{parent}, \text{relation}, \text{child}\rangle$ triples, where each triple
is of the form $@_\text{head}\sematt{relation}\semf{child}$.
The relations $\sematt{act}$ and $\sematt{pat}$ stand for the semantic roles 
\textsc{actor} and \textsc{patient} respectively.

During a derivation, the variables are coindexed during unification, and the logical
forms are conjoined. For example:

\begin{center}
\deriv{3}{
\rm Bob & \rm saw & \rm Gil \\
\uline{1}&\uline{1}&\uline{1} \\
\cf{NP_b\;:} &
\cf{(S[dcl]_x\bs NP_y)/NP_z\;:} &
\cf{NP_g\;:} \\
\cf{@_b\text{\textbf{Bob}}} &
\cf{
  @_x \text{\textbf{see}}
  \wedge @_x\langle \text{\textsc{tense}}\rangle \text{\textbf{dcl}}
  \wedge @_x\langle \textsc{act}\rangle y
  \wedge @_x\langle \textsc{pat}\rangle z
} & \cf{@_g\text{\textbf{Gil}}} \\
& \fapply{2} \\
& \mc{2}{\cf{S[dcl]_x\bs NP_y\;:}} \\
& \mc{2}{\cf{
  @_x \text{\textbf{see}}
  \wedge @_x\langle \textsc{tense}\rangle \text{\textbf{dcl}}
  \wedge @_x\langle \textsc{act}\rangle y
  \wedge @_x\langle \textsc{pat}\rangle g
  \wedge @_g \text{\textbf{Gil}}}}\\
\bapply{3} \\
\mc{3}{\cf{S[dcl]_x\;:}}\\
\mc{3}{\cf{
  @_x \text{\textbf{see}}
  \wedge @_x\langle \textsc{tense}\rangle \text{\textbf{dcl}}
  \wedge @_x\langle \textsc{act}\rangle b
  \wedge @_x\langle \textsc{pat}\rangle g
  \wedge @_g \text{\textbf{Gil}} \wedge @_b \text{\textbf{Bob}}}}\\
}\end{center}

The first application rule coindexes the verb's outer argument variable to
$g$, and a term is added to the semantic representation binding $g$ to Gil.
The same happens when \emph{Bob} is found by backward application.

\section{Instantiating Dependencies During Unification}
% New

Instead of a full logical form, \ccgbank \citep{hock:cl07} represents
predicate-argument structure using dependency
graphs. These dependency graphs are used to evaluate the \candc parser \citep{clark:cl07}. Dependency
graphs are often convenient for our purposes, and are important for this thesis
as they are used in our experiments.

% New
We follow \citet{hock:cl07} in assuming that the arguments of a complex category are
numbered 1 to $n$, starting from the innermost argument, where $n$ is the arity of the
functor, e.g. \cf{((S[dcl]\bs NP_1)/NP_2)/PP_3}.

% New
A dependency is filled when an argument's \headsc feature has been filled with a lexical
item during unification. For instance, if the category \cf{(S[dcl]\bs NP_1)/NP_2}, assigned
to the word \emph{likes}, is the functor in a successful forward application rule with an
\cf{NP} headed by \emph{Pat}, its outer argument will be bound, and the following dependency
created:

\begin{center}
\begin{tabular}{lccr}
  likes & \cf{(S[dcl]\bs NP_1)/NP_2} & 2 & Pat\\
\end{tabular} 
\end{center}

In this thesis, a \emph{labelled \ccg dependency} is thus a 4-tuple consisting of the
functor word, category, argument number and argument word.  A dependency is unbounded
if the argument's head was mediated by one or more other categories during unification,
but we do not include this distinction in our dependency tuples. We sometimes refer to
unlabelled dependencies, which are a 2-tuple consisting of the functor word and the
argument word. 

\section{Inadequacy of AB Categorial Grammar}
\label{sec:ab_sucks}

% Updated
The formalism we have defined so far, using only the function application rules, is
the pure applicative categorial grammar elaborated by \citet{bar-hillel:53}.
The problem with this formalism is that it cannot adequately explain 
the long-distance dependencies that arise
from unbounded extraction --- where `explanatory adequacy' is taken to mean an analysis that
predicts the observed extraction asymmetries efficiently, without \emph{ad hoc}
additions to the grammar, a sense we adapt from \citet{chomsky:aspects}.
It also cannot provide an adequate explanation
of coordination, since spans which would not ordinarily be considered constituents
can be coordinated:

\begin{lexample}
Casey likes but Erin hates Pat.
\end{lexample}

% Updated
This sentence requires the subject to be bracketed with the verb, so that the two
equivalent brackets can be coordinated. An applicative categorial grammar can base
generate such a derivation, which for a \cg means relying on category ambiguity to
provide the alternative bracketing\footnote{In the context of a transformational grammar,
an analysis is \emph{base generated} if it is generated by the core phrase-structure rules,
with no movement operations. We describe a \cg analysis as \emph{base generated} if it
relies on category ambiguity, instead of assigning canonical categories and using
associative and/or permutative combinators.}:

\begin{center}
\deriv{6}{
\rm Casey & \rm likes & \rm but & \rm Erin & \rm hates & \rm Pat \\
\uline{1}&\uline{1}&\uline{1}&\uline{1}&\uline{1}&\uline{1} \\
\cf{NP} &
\cf{(S[dcl]/NP)\bs NP} &
\cf{(X\bs X)/X} &
\cf{NP} &
\cf{(S[dcl]/NP)\bs NP} &
\cf{NP} \\
\bapply{2} && \bapply{2} \\
\mc{2}{\cf{S[dcl]/NP}} && \mc{2}{\cf{S[dcl]/NP}} \\
&& \fapply{3} \\
&& \mc{3}{\cf{(S[dcl]/NP)\bs (S[dcl]/NP)}} \\
\bapply{5} \\
\mc{5}{\cf{S[dcl]/NP}} \\
\fapply{6} \\
\mc{6}{\cf{S[dcl]}}
}
\end{center}

This derivation assigns \emph{hates} the category \cf{(S[dcl]/NP)\bs NP}.
This is an associative variant of the standard transitive verb category
\cf{(S[dcl]\bs NP)/NP}. The coordination we observe appears to
require a novel category that has a specific logical relationship
to the canonical one, function associativity. However, the base generation
strategy does not predict any such relationship, as it does not admit any
constraints on the structure of the novel category. This makes it a weak
explanation of the phenomenon. A similar situation arises in the analysis of
extraction phenomena:

\begin{center}
\deriv{4}{
\rm Pat, & \rm who & \rm Erin & \rm hates \\
\uline{1}&\uline{1}&\uline{1}&\uline{1} \\
\cf{NP} &
\cf{(NP\bs NP)/(S[dcl]/NP)} &
\cf{NP} &
\cf{(S[dcl]/NP)\bs NP} \\
&& \bapply{2} \\
&& \mc{2}{\cf{S[dcl]/NP}} \\
& \fapply{3} \\
& \mc{3}{\cf{NP\bs NP}} \\
\bapply{4} \\
\mc{4}{\cf{NP}}
}
\end{center}

Once again, a construction that only requires an associative version of the canonical
category is analysed as though it could have received any category at all.
The limitations of relying on category ambiguity
are even more apparent in the following construction, where an object is extracted
from within an object clause:

\begin{center}
\deriv{6}{
\rm Pat, & \rm who & \rm Casey & \rm knows & \rm Erin & \rm hates \\
\uline{1}&\uline{1}&\uline{1}&\uline{1}&\uline{1}&\uline{1} \\
\cf{NP} &
\cf{(NP\bs NP)/(S[dcl]/NP)} &
\cf{NP} &
\cf{((S[dcl]/NP)/(S[dcl]/NP))\bs NP} &
\cf{NP} &
\cf{(S/NP)\bs NP} \\
&& \bapply{2} & \bapply{2} \\
&& \mc{2}{\cf{(S[dcl]/NP)/(S[dcl]/NP)}} & \mc{2}{\cf{S[dcl]/NP}} \\
&& \fapply{4} \\
&& \mc{4}{\cf{S[dcl]/NP}} \\
& \fapply{5} \\
& \mc{5}{\cf{NP\bs NP}} \\
\bapply{6} \\
\mc{6}{\cf{NP}}
}
\end{center}

Here the mediating verb, \emph{knows}, must be assigned a different category
to transmit the dependency, even though its own arguments are all in their canonical
positions. This extraction can occur from unbounded depth, so there could be any
number of such mediating verbs, all requiring novel categories due to a movement
phenomenon occurring outside their local argument structures.

\section{The \ccg Combinators}

Because the base generation strategy breaks down in these situations,
\citet{steedman:00} introduces a principle to restrict how much work category
ambiguity should adopt in the grammar:

\begin{headcat}
A single non-disjunctive lexical category for the head of a given construction
specifies both the bounded dependencies that arise when its complements
are in canonical position and the unbounded dependencies that arise when those
complements are displaced under relativization, co-ordination, and the like.
\end{headcat}

Most lexemes can head a great variety of different constructions, and will
require a variety of distinct syntactic categories, so the principle does not
stipulate a one-to-one mapping between lexical entries and categories. Natural
language syntax is ambiguous, so a lexicalised grammar requires ambiguous
lexical entries. The principle merely asserts that lexical ambiguity should be
minimised, which constrains its use as a generative strategy.

Having established this, we now need some extra machinery to generate extraction
and coordination constructions, using only the canonical categories. In \ccg,
this machinery comes in the form of a lexical operation, \emph{type raising}, which
can be used with an extra combinatory rule, \emph{composition}, to allow enough
associativity to solve the vast majority of the problematic long-distance dependencies.
\citeauthor{steedman:00} also introduces another type of combinatory rule,
\emph{substitution}, to handle parasitic gaps. However, this construction is rare in
English \citep{hock:cl07}, so we refer the reader to \citet{steedman:00} for a
discussion of the construction and the substitution rule.

\subsection{The Need for Associativity}

Some \cg categories can be defined so that they are associative functions.
If a category has two arguments, one to the left, and the
other to the right, it does not matter what order the arguments are applied in.
This means that the two bracketings below are equivalent:

\begin{equation}
\cf{(S[dcl]\bs NP)/NP} \;\equiv\; \cf{(S[dcl]/NP)\bs NP}
\end{equation}

Categories with sequences of arguments in a single
direction, are not associative, however, because switching the bracketing
would permute the sequence of arguments in the string. \citet{lambek:58} introduces a unary
\emph{associativity} operator to perform the transformation, given that the two
bracketings are logically equivalent. This operator would allow us to analyse
the sentences in Section \ref{sec:ab_sucks} using the canonical categories,
as we can see in this example of object extraction:

\begin{center}
\deriv{6}{
\rm Ashley & \rm likes & \rm Pat & \rm who & \rm Casey & \rm likes \\
\uline{1}&\uline{1}&\uline{1}&\uline{1}&\uline{1}&\uline{1} \\
\cf{NP} &
\cf{(S[dcl]\bs NP)/NP} &
\cf{NP} &
\cf{(NP\bs NP)/(S[dcl]/NP)} &
\cf{NP} &
\cf{(S[dcl]\bs NP)/NP} \\
&&&&& \asterisk{1} \\
&&&&& \mc{1}{\cf{(S[dcl]/NP)\bs NP}} \\
&&&& \bapply{2} \\
&&&& \mc{2}{\cf{S[dcl]/NP}} \\
&&& \fapply{3} \\
&&& \mc{3}{\cf{NP\bs NP}} \\
&& \bapply{4} \\
&& \mc{4}{\cf{S[dcl]\bs NP}} \\
& \fapply{5} \\
& \mc{5}{\cf{NP}} \\
\bapply{6} \\
\mc{6}{\cf{S[dcl]}}
}
\end{center}

The problem is that full associativity for all functors results in substantial
over-generation, and there is no obvious way to restrict the scope of a single
unary associativity operator. Instead, \ccg achieves partial binary
associativity using function composition.

\subsection{Binary Associativity with Composition}
\label{sec:associativity}
Instead of a unary grammatical rule, \ccg achieves function associativity with
a binary grammatical operation, composition.
Binary composition allows two functors to be merged,
if the argument of one functor matches the result of another in a direction
consistent with their slashes:

\begin{eqnarray}
\cf{X/Y}    & \cf{Y/Z}    & \Rightarrow_\cB\;\; \cf{X/Z}\\
\cf{Y\bs Z} & \cf{X\bs Y} & \Rightarrow_\cB\;\; \cf{X\bs Z}
\end{eqnarray}

Composition allows a functor access to arguments \scare{inside} another functor.
What we now need is a way to transform an argument into a functor of the
appropriate form. This is done with the type-raising operation, which takes a
category \cf{X} and transforms it into a functor over a functor over the
original category \cf{X} with the appropriate directionality:

\begin{eqnarray}
\cf{X} & \Rightarrow_\cT & \cf{T/(T\bs X)}\\
\cf{X} & \Rightarrow_\cT & \cf{T\bs (T/X)}
\end{eqnarray}

The directionality constraint is that the two slashes must be in the opposite
direction, to avoid permuting the order of a functor's arguments.

Figure \ref{fig:wh_movement} shows how partial associativity is used in an
analysis of a common case of extraction, WH-movement. The object of
\emph{hates}, \emph{Pat}, is displaced by the relativiser, \emph{who}. Partial
associativity allows the responsibility for the movement to be encapsulated in the
category assigned to \emph{who}, so that the other words can receive the same
categories they would if there were no extraction phenomena.
\emph{Casey}'s category, \cf{NP}, is type-raised to \cf{S/(S\bs NP)}.
In conjunction with the
composition rule, this allows the arguments of \emph{hates} to be applied in the
opposite order from the one specified by its category's bracketing.
Specifically, \emph{Casey} fills the leftward \cf{NP} argument, for the verb's subject.
This results in a constituent spanning \emph{Casey hates} with the category
\cf{S[dcl]/NP}.

\begin{figure}
\centering
\deriv{6}{
\rm Ashley & \rm likes & \rm Pat & \rm who & \rm Casey & \rm hates \\
\uline{1}&\uline{1}&\uline{1}&\uline{1}&\uline{1}&\uline{1} \\
\cf{NP} &
\cf{(S[dcl]\bs NP)/NP} &
\cf{NP} &
\cf{(NP\bs NP)/(S[dcl]/NP)} &
\cf{NP} &
\cf{(S[dcl]\bs NP)/NP} \\
&&&& \ftype{1} \\
&&&& \mc{1}{\cf{S/(S\bs NP)}} \\
&&&& \fcomp{2} \\
&&&& \mc{2}{\cf{S[dcl]/NP}} \\
&&& \fapply{3} \\
&&& \mc{3}{\cf{NP\bs NP}} \\
&& \bapply{4} \\
&& \mc{4}{\cf{NP}} \\
& \fapply{5} \\
& \mc{5}{\cf{S[dcl]\bs NP}} \\
\bapply{6} \\
\mc{6}{\cf{S[dcl]}}
}
\caption[Partial associativity provided by type-raising and
composition.]{Interaction of type-raising and composition to produce partial
associativity. This allows the WH-movement to be analysed with the canonical
category assignments.\label{fig:wh_movement}}
\end{figure}

% Updated
The logical basis\footnote{The interpretation of a categorial grammar as a logic
dates to \citet{lambek:58}. We present here only a small and informal illustration.
A more detailed explication of \cg as a logic can be found in \citet{baldridge:11}.}
 for these rules is easy to understand when a functor is considered as a kind of
conditional, where the result is the consequent and the argument is the antecedent.
If we ignore directionality for a moment, the application rule resembles a simple
\emph{modus ponens} deduction:
\begin{eqnarray}
NP \rightarrow S\\
NP\\
\therefore S
\end{eqnarray}
%\clearpage
Composition can be seen as a hypothetical syllogism:

\begin{eqnarray}
N\rightarrow NP\\
NP\rightarrow S\\
\therefore N\rightarrow S
\end{eqnarray}

And finally, the type-raising rule:

\begin{eqnarray}
 NP\\
\therefore (NP\rightarrow S) \rightarrow S
\end{eqnarray}

% Updated
Is proved valid by the following truth table:

% Updated
\begin{center}
 \begin{tabular}{cccc}
\hline
$NP$ & $S$ & $NP\rightarrow S$ & $(NP\rightarrow S) \rightarrow S$\\
\hline
\hline
\textbf{T}    & T   & T                 & \textbf{T} \\
\textbf{T}    & F   & F                 & \textbf{T} \\
F    & T   & T                 & T \\
F    & F   & T                 & F \\
\hline
 \end{tabular}
\end{center}

A conditional is false if and only if its antecedent is true and its consequent is false.
If $S$ is false, then the antecedent $NP \rightarrow S$ is false, so the conditional
$(NP \rightarrow S) \rightarrow S$ is true (due to false antecedent).
If $S$ is true, then the conclusion $(NP\rightarrow S) \rightarrow S$ is also true
(due to true consequent). The truth of \cf{NP} therefore guarantees the truth of the
conclusion. The conclusion is contingent, however, as it is false 
if both \cf{NP} and \cf{S} are false.

% There is therefore a logical intuition behind a non-traditional constituent like
% \emph{Casey hates}. This bracketing can be linguistically motivated in its own right,
% and has only been considered non-traditional because phrase-structure grammars encourage
% a choice between analyses that bracket together the subject and verb, and analyses that
% bracket together the verb and complements. Phrase-structure grammar analyses typically
% only assign brackets to the latter. The formalism cannot assign brackets to both
% simultaneously in one analysis, so traditional analyses selected the verb-object
% pairing because it allows better analyses for more frequent constructions. The
% lack of a subject-verb bracket is why subject-verb coordinations, such as
% \emph{Ashley likes and Casey hates Pat}, have received special attention and specific
% nomenclature (right node raising), when they are directly analogous to the more common
% verb-object coordinations, such as \emph{Ashley likes Casey and hates Pat}. Partial
% associativity does create some problems, such as the so-called spurious ambiguity
% issue discussed in Section \ref{sec:normal_form}. There are also movement phenomena
% which it cannot handle. For some constructions, we require a limited degree of
% \emph{permutativity} --- but only a little, or the grammar loses the ability to
% specify directionality in lexical categories. \ccg achieves limited permutativity
% with crossing composition rules. The restrictions that prevent scrambling are
% described in Section \ref{sec:cross_restrict}.

\subsection{Crossing Composition}
\label{sec:permutativity}
The forward and backward composition rules we have seen so far are order
preserving, or \term{harmonic}. However, there are some syntactic constructions
that reliably permute the order of constituents. Once again, the formalism must
decide whether it is preferable to base-generate these analyses, using category
ambiguity to handle the variation, or whether it is preferable to add some
grammatical machinery to perform the permutation using the canonical categories.

The order-permuting construction that \citet{steedman:pedia} discuss is heavy
\cf{NP} shift. In this construction, the order of arguments in a ditransitive
can be permuted, in order to minimise dependency distances:

\begin{lexamples}
\item I gave [to him] [a book that was very heavy and difficult to read if it is
close to the verb].
\item ? I gave [a book that was very heavy and difficult to read if it is close
to the verb] [to him].
\end{lexamples}

Verb-particle constructions present another example of argument order
permutation:

\begin{lexamples}
\item They gunned down [the very high-tech and expensive fighter plane].
\item ? They gunned [the very high-tech and expensive fighter plane] down.
\item * They gunned down it.
\end{lexamples}

`Heavy' (long) \cf{NP}s are likely to be pushed after the particle, while a
pronominal is forced to occur before the particle. We will assume in our
analyses that the canonical order is verb, particle, object; and that the
particle final construction is the permutation.

To analyse heavy \cf{NP} shift and verb particle structure alternations without
category ambiguity, we need composition rules that do not preserve the order of
a category's arguments. This is referred to as \emph{crossing composition}:

\begin{eqnarray}
\cf{X/Y} & \cf{Y\bs Z} & \Rightarrow_\cBx\;\; \cf{X/Z}\\
\cf{Y/Z} & \cf{X\bs Y} & \Rightarrow_\cBx\;\; \cf{X\bs Z}
\end{eqnarray}

The backward crossing composition rule can be used with the backward type
raising rule to analyse movement phenomena using the canonical category for
each word in the sentence.

\begin{figure}
\centering
\deriv{4}{
\rm They & \rm gave & \rm to~him & \rm a~very~heavy~book \\
\uline{1}&\uline{1}&\uline{1}&\uline{1} \\
\cf{NP} &
\cf{((S[dcl]\bs NP)/PP)/NP} &
\cf{PP} &
\cf{NP} \\
&& \btype{1} \\
&& \mc{1}{\cf{(S\bs NP)\bs ((S\bs NP)/PP)}} \\
&\bxcomp{2} \\
&\mc{2}{\cf{(S[dcl]\bs NP)/NP}} \\
&\fapply{3} \\
&\mc{3}{\cf{S[dcl]\bs NP}}\\
\bapply{4}\\
\mc{4}{\cf{S[dcl]}}
}
\caption[Heavy \cf{NP} shift with crossed composition.]{\ccg analysis of heavy
\cf{NP} shift. The analysis uses crossed composition to achieve the required
permutation of the verb's arguments.\label{fig:heavy_np}}
\end{figure}

\begin{figure}
\centering
\deriv{4}{
\rm They & \rm gunned & \rm it & \rm down \\
\uline{1}&\uline{1}&\uline{1}&\uline{1} \\
\cf{NP} &
\cf{((S[dcl]\bs NP)/PT)/NP} &
\cf{PP} &
\cf{NP} \\
&& \btype{1} \\
&& \mc{1}{\cf{(S\bs NP)\bs ((S\bs NP)/PT)}} \\
&\bxcomp{2} \\
&\mc{2}{\cf{(S[dcl]\bs NP)/NP}} \\
&\fapply{3} \\
&\mc{3}{\cf{S[dcl]\bs NP}}\\
\bapply{4}\\
\mc{4}{\cf{S[dcl]}}
}
\caption[Verb-particle reordering with crossed composition.]{\ccg analysis of
verb-particle reordering. The analysis uses crossed composition to achieve the
required permutation of the verb's arguments.\label{fig:vpc}}
\end{figure}

Figure \ref{fig:heavy_np} shows a \ccg derivation of heavy \cf{NP} shift. The
long argument, whose canonical position is just right of the verb, is shifted to
the end to minimise the surface distance between the verb and its arguments. The
permutation is achieved by type-raising the short \cf{PP} argument so that it
composes with the inner \cf{PP} argument of the verb, so that it unifies with
the same argument slot it would occupy had it occurred in its canonical
position. This requires a crossed composition rule, because the verb's slash is
rightward. Figure \ref{fig:vpc} shows an analogous analysis for verb-particle
reordering.

Finally, the composition rules must also be generalised, so that associativity
is preserved for functions of higher arity. This is shown in Figure
\ref{fig:gen_comp}. In this derivation, the argument of the category assigned to
\emph{may}, \cf{(S[dcl]\bs NP)/(S[b]\bs NP)}\footnote{The \cf{b} feature marks
bare inflection. See \citet{hock:cl07} for other feature values used during the
thesis.},
is an extra level deep inside the
category of \emph{give}. It is the result of the result, instead of just the
result. The generalised composition rules allow composition over directionally
consistent argument sequences, to handle such constructions.

\begin{figure}
\centering
\scalebox{0.8}{
\deriv{8}{
\rm Pat & \rm bought & \rm and & \rm may & \rm give & \rm Casey & \rm a & \rm
flower \\
\uline{1}&\uline{1}&\uline{1}&\uline{1}&\uline{1}&\uline{1}&\uline{1}&\uline{1}
\\
\cf{NP} &
\cf{((S[dcl]\bs NP)/NP)/NP} &
\cf{(X\bs X)/X} &
\cf{(S[dcl]\bs NP)/(S[b]\bs NP)} &
\cf{((S[b]\bs NP)/NP)/NP} &
\cf{NP} &
\cf{NP/N} &
\cf{N} \\
&&& \fcomp{2} && \fapply{2} \\
&&& \mc{2}{\cf{((S[dcl]\bs NP)/NP)/NP}} && \mc{2}{\cf{NP}} \\
&& \fapply{3} \\
&& \mc{3}{\cf{(\cf{((S[dcl]\bs NP)/NP)/NP})\bs (((S[dcl]\bs NP)/NP)/NP)}} \\
& \bapply{4} \\
& \mc{4}{\cf{((S[dcl]\bs NP)/NP)/NP}} \\
& \fapply{5} \\
& \mc{5}{\cf{((S[dcl]\bs NP)/NP}} \\
& \fapply{7} \\
& \mc{7}{\cf{S[dcl]\bs NP}} \\
\bapply{8} \\
\mc{8}{\cf{S[dcl]}}
}}
\caption[Example of generalised composition.]{An example of generalised
composition, between the auxiliary \emph{may} and the ditransitive verb
\emph{give}.\label{fig:gen_comp}}
\end{figure}


\section{Restricting Rule Productivity}

% Updated
The unrestricted interaction of the application, composition and type-raising rules
discussed so far introduces a great deal of syntactic ambiguity and over-generation.
A number of measures have been proposed to mitigate these problems. First,
\citet{steedman:00} follows \citet{komagata:97,komagata:99} in
placing a general restriction on type-raising, in order to ensure that the unary rule
is not completely unbounded. The restriction prevents a forward type raising rule from
producing a category \cf{T/(T\bs X)} where \cf{T\bs X} is not a valid category from
the fixed set of types occurring in the lexicon. An equivalent restriction is placed on
backward type-raising.

There is also strong motivation for constraining composition. Initially,
\citet{steedman:00} used the language specific constraints described in Section
\ref{sec:cross_restrict} to introduce the required restrictions. However,
\citet{baldridge:03} have since described how these rules can be
replaced with a more principled lexically sensitive specification. The mechanism
for doing this, multi-modal slashes, is now widely accepted as an important part
of the \ccg theory \citep{steedman:pedia}. Multi-modal \ccg is described in
Section \ref{sec:mmccg_background}.

% Updated
Finally, the partial associativity added by composition allows multiple equivalent
bracketings for the same semantic analysis. The alternative brackets allow information
structure to be expressed in the surface syntax, but for practical parsing purposes,
it is useful to suppress the ambiguity. We describe a mechanism suggested by
\citet{eisner:96} to do this in Section \ref{sec:normal_form}.


\subsection{Restrictions on Crossing Composition}
\label{sec:cross_restrict}
The crossing composition rules allow a functor's arguments to occur in a
different order from the one specified by the category. This amounts to local
scrambling --- a desirable property for free word order languages. The problem
is that we do not want scrambling in a configurational language like English:

\begin{center}
\deriv{6}{
\rm *I & \rm Ed & \rm think & \rm that & \rm saw & \rm Ann \\
\uline{1}&\uline{1}&\uline{1}&\uline{1}&\uline{1}&\uline{1} \\
\cf{NP} &
\cf{NP} &
\cf{(S[dcl]\bs NP)/S[em]} &
\cf{S[em]/S[dcl]} &
\cf{(S[dcl]\bs NP)/NP} &
\cf{NP} \\
&&&& \fapply{2} \\
&&&& \mc{2}{\cf{S[dcl]\bs NP}} \\
&&& \fxcomp{3} \\
&&& \mc{3}{\cf{S[em]\bs NP}} \\
&& \fxcomp{4} \\
&& \mc{4}{\cf{(S[dcl]\bs NP)\bs NP}} \\
& \bapply{5} \\
& \mc{5}{\cf{S[dcl]\bs NP}} \\
\bapply{6} \\
\mc{6}{\cf{S[dcl]}}
}
\end{center}

The problem is the forward crossed composition rule, which is not necessary in
English. \citet{steedman:00} therefore proposed language specific subsets of
the combinatory rules, which are suggested as a theory of universal grammar.
Additionally, \citet{steedman:00} introduced a restriction that
backward crossed composition should only be allowed for categories rooted in
\cf{S}. This prevents further over-generation:

\begin{center}
 \deriv{4}{
\rm *powerful & \rm by & \rm Rivaldo & \rm shots \\
\uline{1}&\uline{1}&\uline{1}&\uline{1} \\
\cf{N/N} &
\cf{(N\bs N)/NP} &
\cf{NP} &
\cf{N} \\
& \fapply{2} \\
& \mc{2}{\cf{N\bs N}} \\
\bxcomp{3} \\
\mc{3}{\cf{N/N}} \\
\fapply{4} \\
\mc{4}{\cf{N}}
}
\end{center}

% Updated
These constraints would presumably be conventions adopted and learnt by speakers of
a language, rather than fixed constraints in a universal (possibly innate) grammar.
\citet{baldridge:03} provide a way for such constraints to be represented in the
lexicon instead, which we will now describe.


\subsection{Multi-Modal \ccg}
\label{sec:mmccg_background}
\citepos{baldridge:03} solution was to import the approach to resource
sensitivity taken in the categorical type logic (\ctl) tradition
\citep{morrill:94,moortgat:97}. \ccg uses a single pair of slashes, $\lbrace /, \bs\rbrace$,
which means that there there is no way a specific category can restrict which
rules it can be used with. \ctl instead decorates its slashes with \emph{modes}.
The mode variables distinguish slashes that can participate in all rules from
slashes which can only participate in a certain subset.

\citeauthor{baldridge:03} divide the binary rules according to whether they
allow \emph{associativity}, \emph{permutativity}, or \emph{neither}. Table
\ref{tab:rule_props} shows how the rules break down according to these
properties. The harmonic composition rules allow the associativity required to
analyse the WH-movement and right node raising constructions described in
Section \ref{sec:associativity}, while the crossing composition rules allow the
permutativity required to analyse the heavy \cf{NP} shift and verb-particle
reordering constructions described in Section \ref{sec:permutativity}.

\begin{table}
 \centering

\begin{tabular}{llllcc}
\hline
 Direction & Type     & Combinator  &        & Associative? & Permutative?\\
\hline
\hline
 Forward   &          & Application & $>$    &   & \\
 Backward  &          & Application & $<$    &   & \\
 Forward   & Harmonic & Composition & $>\cB$   & $\checkmark$ & \\
 Backward  & Harmonic & Composition & $<\cB$   & $\checkmark$ & \\
 Forward   & Crossing & Composition & $>\cBx$ &   & $\checkmark$ \\
 Backward  & Crossing & Composition & $<\cBx$ &   & $\checkmark$ \\
\hline
\end{tabular}
\caption{Associativity and permutativity properties of the \ccg rules.
\label{tab:rule_props}}
\end{table}

The slashes used in these rules can therefore be decorated with types, referred
to as \emph{modes}, where each mode permits associativity, permutativity, both,
or neither. Lexical categories can then be assigned modes that reflect which rules
they are allowed to be used with. A slash decorated with the \dmodetext will allow
associative rules, \xmodetext decorated slashes allow permutativity, and \smodetext
decorated slashes allow neither. A fourth mode, \cmodetext, allows both.

\citet{baldridge:thesis02} defined a hierarchy of seven modes, and a simplified
hierarchy of four modes sufficient for most purposes. The simplified hierarchy
is shown in Figure \ref{fig:modes}. The most restrictive \emph{application-only}
mode (\smodetext) occurs at the top, with the more permissive tier of modes
(\dmodetext, \xmodetext) inheriting from it, and the most permissive mode (\cmodetext)
inheriting from them. A mode can unify with itself, or with any mode it subsumes.
This means that the standard unification mechanism can be exploited to restrict
rule productivity, once the combinators are updated with the appropriate modes
as shown in Table \ref{tab:ccg_rules}.

\begin{table}
 \centering
\begin{tabular}{lcclc}
\hline
\Sfapply & \cf{X/\smode Y}   & \cf{Y}            & $\Rightarrow$ & \cf{X}\\
\Sbapply & \cf{Y}            & \cf{X\bs\smode Y} & $\Rightarrow$ & \cf{X}\\
\Sfcomp  & \cf{X/\dmode Y}   & \cf{Y/\dmode Z}   & $\Rightarrow_\cB$ &
\cf{X/\dmode Z}\\
\Sbcomp  & \cf{Y\bs\dmode Z} & \cf{X\bs\dmode Y} & $\Rightarrow_\cB$ &
\cf{X\bs\dmode Z}\\
\Sfxcomp & \cf{X/\xmode Y}   & \cf{Y\bs\xmode Z} & $\Rightarrow_\cBx$ &
\cf{X\bs\xmode Z}\\
\Sbxcomp & \cf{Y/\xmode Z}   & \cf{X\bs\xmode Y} & $\Rightarrow_\cBx$ &
\cf{X/\xmode Z}\\
\Sftype  &                   & \cf{X}            & $\Rightarrow_\cT$ & \cf{T/_i
(T\bs_i X)} \\
\Sbtype  &                   & \cf{X}            & $\Rightarrow_\cT$ &
\cf{T\bs_i (T/_i X)} \\
\hline
\end{tabular}
\caption{Multi-modal \ccg rules.}
\label{tab:ccg_rules}
\end{table}

Let's look at a simple example of how these mode sensitive combinators allow rule
sensitivity to be restricted in the lexicon. We would like to handle conjunction
using our current inventory of combinatory rules, rather than using something
like the ternary conjunction rule described in \citet{steedman:00}. We can do
this by assigning coordinators categories of the form \cf{(X\bs X)/X}, where
\cf{X} can be any category.\footnote{This could be expanded into a series of
non-schematic categories in the lexicon, but for now we will work with the
schematic version.} \citet{steedman:00} shows that with this formulation,
we cannot prevent over-generation:

\begin{figure}
\centering

\begin{tikzpicture}
\path
  node (star) at (1, 2) {$\star$}
  node (times) at (0, 1) {$\times$}
  node (diamond) at (2, 1) {$\diamond$}
  node (dot) at (1, 0) {$\cdot$};

\draw [->] (star) -- (times);
\draw [->] (star) -- (diamond);
\draw [->] (times) -- (dot);
\draw [->] (diamond) -- (dot);
\end{tikzpicture}

\caption{Subsumption hierarchy of the four modes.\label{fig:modes}}
\end{figure}

\begin{center}
\deriv{6}{
\rm *player & \rm that & \rm shoots & \rm and & \rm he & \rm misses \\
\uline{1}&\uline{1}&\uline{1}&\uline{1}&\uline{1}&\uline{1} \\
\cf{N} &
\cf{(N\bs N)/(S[dcl]\bs NP)} &
\cf{S[dcl]\bs NP} &
\cf{(X\bs X)/X} &
\cf{N} &
\cf{S[dcl]\bs NP} \\
&&&& \bapply{2} \\
&&&& \mc{2}{\cf{S[dcl]}} \\
&&& \fapply{3} \\
&&& \mc{3}{\cf{S[dcl]\bs S[dcl]}} \\
&& \bcomp{4} \\
&& \mc{4}{\cf{S[dcl]\bs NP}} \\
& \fapply{5} \\
& \mc{5}{\cf{N\bs N}} \\
\bapply{6} \\
\mc{6}{\cf{N}}
}
\end{center}

However, if we make use of the multi-modal combinators to assign a more
restrictive category to \emph{and}, the invalid composition will be blocked.
Blocked rules are marked with an asterisk:

\begin{center}
\deriv{6}{
\rm *player & \rm that & \rm shoots & \rm and & \rm he & \rm misses \\
\uline{1}&\uline{1}&\uline{1}&\uline{1}&\uline{1}&\uline{1} \\
\cf{N} &
\cf{(N\bs N)/(S[dcl]\bs NP)} &
\cf{S[dcl]\bs NP} &
\cf{(X\bs\smode X)/\smode X} &
\cf{N} &
\cf{S[dcl]\bs NP} \\
&&&& \bapply{2} \\
&&&& \mc{2}{\cf{S[dcl]}} \\
&&& \fapply{3} \\
&&& \mc{3}{\cf{S[dcl]\bs\smode S[dcl]}} \\
&& \asterisk{4} \\
&& \mc{4}{\cf{S[dcl]\bs NP}}
}
\end{center}
%\clearpage
% Updated
Multi-modal \ccg can thus analyse coordination phenomena without introducing
a special conjunction rule. We will largely be dealing with the \citet{steedman:00}
grammar, however, so most of our derivations will use the \ccgbank analysis of
coordination, which approximates a ternary conjunction rule:

\begin{center}
 \deriv{6}{
\rm Casey & \rm and & \rm Pat & \rm sang & \rm and & \rm danced \\
\uline{1}&\uline{1}&\uline{1}&\uline{1}&\uline{1}&\uline{1} \\
\cf{NP} &
\cf{conj} &
\cf{NP} &
\cf{S[dcl]\bs NP} &
\cf{conj} &
\cf{S[dcl]\bs NP} \\
& \conj{2} && \conj{2}\\
& \mc{2}{\cf{NP[conj]}} & & \mc{2}{\cf{S[dc]\bs NP[conj]}}\\
\conj{3} & \conj{3} \\
\mc{3}{\cf{NP}} & \mc{3}{\cf{S[dcl]\bs NP}}\\
\bapply{5}\\
\mc{5}{\cf{S[dcl]}}
}
\end{center}

In multi-modal \ccg, the value of the \slashsc attribute of complex categories
is a typed feature structure with multiple properties, rather than a single atomic
value representing directionality. \citet{baldridge:03} suggest that it may be
useful to extend the definition of \slashsc beyond directionality and mode. They
propose two additional attributes, to specify the head-directionality of the slash
and whether the slash can be used in the primary functor of combinatory rules. The
extended definition of \slashsc might be presented as follows:

\begin{center}
\begin{avm}
[{} \slashsc & [{} \textsc{dir} & \cf{\bs}\\
                   \textsc{mode} & \smodetext\\
                   \textsc{depdir} & $\leftarrow$\\
                   \textsc{activity}  & !\\
               ]
]
\end{avm}
\end{center}

\citet{clark:acl07parseval} implement an equivalent of the \textsc{depdir}
attribute in the \candc parser to have it generate grammatical relations
\citep{briscoe:poster06}.
However, we do not perform a grammatical relations evaluation in our parsing experiments,
so we do not require the \textsc{depdir} slash attribute in this thesis.

The concept of inert slashes is useful in our analyses, and is important for our definition of
hat categories, as described in Section \ref{sec:null_mode}.
The inert slash allows an argument to be structurally defined, but unavailable 
for application or composition. Because \textsc{activity} is a separate dimension from
\textsc{mode}, \citet{hoyt:08} denote inert slashes using a superscript, rather
than a subscript. For instance, \citet{hock:cl07} uses the category
\cf{S[adj]\bs NP} for predicative adjuncts. The argument of this category
is a good candidate for an inert slash:

\begin{center}
\small
\deriv{7}{
\rm Pat & \rm is & \rm well & \rm read & \rm and & \rm handy & \rm
around~the~house \\
\uline{1}&\uline{1}&\uline{1}&\uline{1}&\uline{1}&\uline{1}&\uline{1} \\
\cf{NP} &
\cf{(S[dcl]\bs NP)/(S[adj]\bs^! NP)} &
\cf{(S\bs^i NP)/(S\bs^i NP)} &
\cf{S[adj]\bs^! NP} &
\cf{(X\bs X)/X} &
\cf{S[adj]\bs^! NP} &
\cf{(S\bs^i NP)\bs (S\bs^i NP)} \\
&& \fapply{2} && \bapply{2} \\
&& \mc{2}{\cf{S[adj]\bs^! NP}} && \mc{2}{\cf{S[adj]\bs^! NP}} \\
&&&& \fapply{3} \\
&&&& \mc{3}{\cf{(S[adj]\bs^! NP)\bs (S[adj]\bs NP)}} \\
&& \bapply{5} \\
&& \mc{5}{\cf{S[adj]\bs^! NP}} \\
& \fapply{6} \\
& \mc{6}{\cf{S[dcl]\bs NP}} \\
\bapply{7} \\
\mc{7}{\cf{S[dcl]}}
}
\end{center}

The copula, \emph{is}, is assigned the category \cf{(S[dcl]_z\bs
NP_y)/(S[adj]_z\bs^! NP_y)}. The inert argument offers several advantages. First, it
enables a dependency to be created between the subject \emph{Casey} and the
object of the copula, in this case both \emph{read} and \emph{handy}. The
dependency is created by coindexing the argument of the copula with the argument
of the predicate. Such a dependency would be impossible if the complement did
not specify an argument. The \cf{S[adj]\bs^! NP} category also predicts that the
predicate can be modified by verb phrase modifiers, using their canonical
categories \cf{(S\bs^i NP)\bs (S\bs^i NP)} and \cf{(S\bs^i NP)/(S\bs^i NP)}.
\footnote{Note that with the extended slash definition, modifiers must coindex their slashes
to ensure that they behave as functions that return their arguments unchanged.}
The inert slash would also allow the \cf{S\bs S} category to modify
\cf{S[adj]\bs^! NP}, as inert slashes can be the arguments of combinators, but
not their functors.

What we do not want in this analysis is for the predicate to apply its argument
directly. Assigning an inert mode prevents this, allowing the argument to fulfill
desirable structural criteria without increasing ambiguity in the grammar.

One subtlety to note is that inert slashes cannot be circumvented via type-raising.
At first it might seem that this would succeed:

\begin{center}
\deriv{2}{
\rm *Pat & \rm handy \\
\uline{1}&\uline{1} \\
\cf{NP} &
\cf{S[adj]\bs ^!NP} \\
\ftype{1} \\
\mc{1}{\cf{S/^i(S\bs^i NP)}} \\
\asterisk{2} \\
\mc{2}{\cf{S[adj]}}
}\end{center}

The crucial detail is that the type-raising rules are defined such that the two slashes
are coindexed. This leaves two possible forward type-raise to \cf{S} categories for Pat:
either both slashes can be inert, \cf{S/^!(S\bs^! NP)}, or both slashes can be active,
\cf{S/^+(S\bs^+ NP)}. If the slashes are inert, then the type-raise category cannot be
the functor, so the application is blocked. If the slashes are active, then unification
between \cf{S\bs^+NP} and \cf{S[adj]\bs^!NP} will fail because their slashes are
incompatible, and the categories still cannot combine.

\subsection{Normal Form Constraints}
\label{sec:normal_form}
Multi-modal slashes allow \ccg to control the undesirable effects of adding
permutative rules to the grammar --- namely, over-generation. From a processing
standpoint, associativity also has drawbacks, because it introduces multiple
equivalent bracketings of the same semantic analysis. For instance, the
alternative bracketing required in analysis \ref{bracket_1} can also be used in
analysis \ref{bracket_2}, where it is not required:

\begin{lexamples}
 \item (Erin hates) but (Casey likes) Pat. \label{bracket_1}
 \item (Erin hates) Pat. \label{bracket_2}
\end{lexamples}


\citet{steedman:00} introduces an explanation for this apparently
uneconomical situation, pointing out that even though the sentence only has one
valid semantic analysis, it does have two valid information structure analyses
--- which correspond to the two possible bracketings. \citet{white:10} exploit
this ability to express information structures in the surface syntax to produce
more natural speech generation. The ambiguity also allows
extreme left-corner analyses, which are useful for incremental parsing.

Nevertheless, the ambiguity can still be inconvenient. \ccg grammars can be
parsed using the \cky algorithm \citep{cocke:70,kasami:65,younger:67}, and the
extra analyses complicate the task of selecting the best semantic analysis.
\citet{eisner:96} offers a simple solution. He shows that for every semantic
analysis, there is exactly one \emph{normal form} derivation, and that
two simple constraints will ensure that only normal form derivations
are added to the chart.

The Eisner normal form constraints stipulate that if a category was produced by a
forward composition rule, it cannot immediately be used as the functor of forward
application; and similarly, if a category was produced by backward composition, it
cannot immediately be used as the functor of backward application.
\citeauthor{eisner:96} shows that these two rules are sufficient to prevent all spurious
ambiguity in a \ccg grammar that consists of application and composition rules,
but does not contain type-raising rules. \ccgbank's grammar does contain
type-raising and type-changing rules, so the normal form constraints do not
guarantee that derivations uniquely correspond to semantic analyses.
However \citet{clark:cl07} have found that the
Eisner normal form constraints are useful for increasing the efficiency of \ccg
parsing using the \ccgbank grammar.

\section{\ccgbank}

\citet{hock:01,hock:acl03} developed the first wide coverage \ccg parsing system.
The work that made this possible was the adaptation of the Penn Treebank
\citep{marcus:93} to \ccg \citep{hock:lrec02,hock:cl07}. This section describes the
corpus, which plays a central role in the thesis. First, we provide a brief overview
of the Penn Treebank, with a very brief look at the long history of statistical parsing
research it initiated. We will then describe the \ccgbank conversion process. Finally, we
describe the use of type-changing rules in \ccgbank,
to carry out type-changing operations. These type-changing rules are particularly important
for this thesis.

\subsection{The Penn Treebank}


The Penn Treebank (\penn) \citep{marcus:93} is a syntactically annotated corpus of English.
Version 3 of the corpus consists of approximately 1.2 million words of newswire text
from the Wall Street Journal, 900,000 words of text from a variety of written genres
from the Brown corpus, 1.1 million words of casual conversation between strangers from the
Switchboard corpus, and 10,000 words of spoken dialogue system queries from the
\textsc{atis} corpus. Syntactic parsing research has focused almost entirely on the
Wall Street Journal portion. The initial release of the corpus consisted of the Wall
Street Journal and \textsc{atis} corpora.

The corpus was annotated according to a theory-neutral annotation scheme intended
to maximise the cost effectiveness of the project. The exact scheme was determined
largely by practical considerations, especially the output format of the Fidditch
parser \citep{hindle:83}, which was used to provide annotators with tree fragments
to `glue' together. For instance, null elements were included in the annotation
because Fidditch already produced them, and it was found that they were not expensive
to correct \citep{marcus:93}. Without these null elements, the conversion to
\ccg would have been far more difficult.

Two other noteworthy aspects of the annotation scheme are the treatment of
complement-adjunct distinctions, and the flat noun phrase bracketing. Consistent
complement-adjunct distinctions were found to slow down annotation by roughly
150 words per hour, and could not be made consistently. A flat bracketing of noun
phrases was adopted because inserting more fine-grained brackets than those provided
by the Fidditch parser would have been quite costly. Full noun phrase bracketing was
finally added by \citet{vadas:07}, and integrated into \ccgbank by \citet{vadas:08}.
These additions to the \penn released were late in this project, so were not used in the
experiments described in Chapter \ref{chapter:results}.

% Updated
Figure \ref{fig:pierre_ptb} shows the annotation of the first sentence in the Wall
Street Journal portion of the corpus. This sentence contains two function tags,
\verb1-TMP1 and \verb1-CLR1, which show some predicate-argument information:
\verb1-TMP1 notes that the constituent functions as a temporal adjunct, while 
\verb1-CLR1 records (somewhat vaguely) that the constituent is `closely related'
to the verb. However, the distinctions are not drawn very consistently.
Full predicate-argument labels were later annotated and released as Propbank
\citep{propbank}. The Propbank annotation labels \emph{as a nonexecutive director}
as an adjunct, rather than a complement --- which is how \citet{hock:cl07} interprets
the \verb1-CLR1 label. This shows how difficult complement-adjunct distinctions
can be to annotate. In contrast, noun phrase brackets often involve very easy,
even trivial, annotation decisions. For instance, the noun phrase \emph{a nonexecutive
director} should obviously be right branching, but this structure is left unspecified.

\begin{figure}
 \begin{verbatim}
  ( (S
    (NP-SBJ
      (NP (NNP Pierre) (NNP Vinken) )
      (, ,)
      (ADJP
        (NP (CD 61) (NNS years) )
        (JJ old) )
      (, ,) )
    (VP (MD will)
      (VP (VB join)
        (NP (DT the) (NN board) )
        (PP-CLR (IN as)
          (NP (DT a) (JJ nonexecutive) (NN director) ))
        (NP-TMP (NNP Nov.) (CD 29) )))
    (. .) ))
 \end{verbatim}
\caption[A Penn Treebank bracketed sentence.]{The Penn Treebank analysis of the
first sentence in the corpus. The PP-CLR tag assigned to \emph{as a nonexecutive
director} produces a complement/adjunct error in the \ccgbank derivation. The
apposition between \emph{Pierre Vinken} and \emph{61 years old} produces a
binary type-changing rule.\label{fig:pierre_ptb}}
\end{figure}

\citet{magerman:95} was the first to exploit the new resource, using \textsc{spatter},
a decision tree-based statistical parser. \citet{magerman:94} had shown that
\textsc{spatter} already outperformed the leading manually written system, the
\textsc{ibm}-Lancaster parser, which had been under development for 10 years. When
trained on the Lancaster Computer Manuals corpus \citep{black:96}, \textsc{spatter}
achieved a 0-crossing score of 76\%, substantially higher than the 69\% reported by
\citeauthor{black:93}\footnote{\citet{magerman:94} followed \citet{black:93} in reporting 0-crossing for
direct comparison, even though this measure was by then considered less informative
than $F$-measure as defined by \parseval \citep{black:91}.}
Once the Penn Treebank was released, development of wide-coverage manually written
phrase-structure grammars largely ceased, although interest in manually written
grammars for other formalisms has continued \citep[e.g. ][]{xtag, erg, xle}.

% Updated
\citet{magerman:95} achieved an $F$-score of 84.1\% on Section 23 of the Penn Treebank on
sentences of 100 or fewer tokens. All subsequent Penn Treebank parsers have reported
results that can be compared directly against this figure, allowing us to trace 15 years
of improvements in statistical parsing.

% Updated
The first to improve on \citeauthor{magerman:95}'s model was \citet{collins:96}, which
defined an efficient conditional model using bilexical dependencies. Each constituent
was assigned a head using head-finding heuristics similar to \citet{magerman:95},
and base noun phrases (detected by a chunker) were represented solely by their heads.
Each parent and child node pair in the tree was then considered a bilexical dependency
labelled by the node label of the argument, the node label of the parent, and the node
label of the sibling that heads the parent. Dependencies were assigned probabilities
using maximum likelihood estimation, with part-of-speech backoff as described in
\citet{collins:95}. The best performing configuration achieved 85.5\% $F$-score.

% Updated
The first accurate generative grammar parser was described by \citet{charniak:97}.
The parser used a probabilistic context-free grammar (\pcfg) with several innovations
to weaken the problematic \pcfg independence assumptions. The most important innovation
was to extend each node label with the head word and part-of-speech, so that bilexical
dependency statistics could be estimated. Deleted interpolation smoothing was used to
mitigate sparse data problems. \citet{charniak:97} achieved 86.6\% $F$-score. After a
thorough analysis, he concluded that the most important factor in this
improvement in accuracy between Collins and Magerman was due to Collins' conditioning
on individual words more often than Magerman, due to Magerman's use of decision trees
to estimate probabilities. Charniak attributes his improvement over Collins to his superior
back-off probabilities and smoothing.

% Updated
\citet{collins:97} then defined three generative parsing models, ultimately improving
accuracy to 87.8\%. Model 1 was effectively a generative version of \citet{collins:96}.
Model 2 extended the parser to make complement/adjunct distinctions, using probabilities
over subcategorisation frames for headwords. Nodes labelled NP, SBAR or S directly under S;
nodes labelled VP, NP, SBAR or S directly under VP; and nodes labelled SBAR directly under S
were considered complements if they were not labelled with an adverbial function tag. There
are two motivations for drawing complement/adjunct distinctions. First, they offer potentially
useful distinctions to downstream applications. Second, they offer potentially
useful information for the model. \citeauthor{collins:97} found that conditioning over whole
subcategorisation frames, rather than assuming that arguments were generated independently,
improved accuracy slightly. Model 3 pursued further linguistic correctness by modelling traces
and WH-movement. Once again, the motivation is twofold: the desire for a more detailed
output, and the suggestion that moved constituents hamper the estimation of subcategorisation
frame probabilities. Model 3 achieved exactly the same accuracy as Model 2, and traces
representing non-local dependencies were recovered with only 67.4\% accuracy.

% Updated
The next breakthrough was made by \citet{ratnaparkhi:97}, who used a conditional
log-linear model, rather than a generative one. Conditional models allow arbitrary
features to be encoded --- although, for structured prediction problems such as parsing,
the features do have to be locally decidable to make the dynamic programming tractable.
\citeauthor{ratnaparkhi:97}'s model achieved an accuracy of 86.9\% $F$-measure, but
the introduction of conditional parsing opened the door for substantial subsequent
improvement.
\citet{charniak:00} followed \citet{ratnaparkhi:97} in using a conditional model,
and achieved an accuracy of 89.5\% $F$-measure. Charniak experimented with
a variety of features, and even performed a self-training experiment, in order to obtain
more robust estimates for bilexical features. The self-training experiment only improved
accuracy by 0.4\%.

% Updated
The next major innovation in this line of research was re-ranking. \citet{collins:00}
used a conditional model to select the best parse from the $k$-best produced by the
\citet{collins:97} model, and achieved accuracy comparable to \citet{charniak:00}. The
advantage of re ranking is that there are no restrictions on the features that can be
extracted from the parse tree, as it is a conditional model that does not require any
dynamic programming. \citet{charniak:05} applied re ranking to their statistical parser,
and achieved an $F$-score of 91.4\%. \citet{mcclosky:06} improved this to 92.1\% using
self-training, which remains the state-of-the-art.

% Updated
This line of research is but one of the many that have been pursued in English
statistical parsing. In particular, there is an extensive literature on dependency
parsing, using grammars extracted from the Penn Treebank. \citet{kubler:09} describes
this body of work well. Parsers have also been developed for corpora acquired from the
Penn Treebank.

% Updated
In Section \ref{sec:background_candc}, we provide an overview of one such parser.
The \candc parser \citep{clark:cl07} takes advantage of the fact that \ccg, as a
lexicalised formalism, naturally presents a way for a parser to include lexical
information and predicate-argument structure in its probability models. The most
important development for this approach to parsing was the development of a separate
supertagging phase \citep{srinivas:99}, where lexical categories are assigned by a
sequence tagger. \citet{clark:coling04} showed that a multi-tagger could be tightly
integrated with a chart parser, producing substantial improvements in speed and accuracy.
Similar findings have since been reported for \hpsg, another lexicalised formalism that
offers many of the same advantages for parsing as \ccg. In particular, the \enju parser
\citep{miyao:08} uses many of the same design features as the \candc model, and has been
found to be the most accurate parser at recovering long range dependencies \citep{rimell:09}.

\subsection{Conversion Process}

\ccgbank was created semi-automatically using a conversion algorithm that proceeds top-down over each tree:

\begin{center}
\begin{verbatim}
  foreach tree T:
    determineConstituentType(T);
    makeBinary(T);
    assignCategories(T);
\end{verbatim}
\end{center}

% Updated
This algorithm assumes that the input Penn Treebank trees conform to the
analyses desired for the \ccg corpus, which is often not the case. Analyses
diverge because of outright errors and inconsistencies in the Penn Treebank,
and the different capabilities of the two formalisms.
\citet{hock:thesis03} provides the following extended description of the conversion
process:

% Updated
\begin{center}
\begin{verbatim}
  foreach tree T:
    preprocessTree(T);
    preprocessArgumentClusters(T);
    determineConstituentType(T);
    makeBinary(T);
    percolateTraces(T);
    assignCategories(T);
    treatArgumentClusters(T);
    cutTracesAndUnaryRules(T);
\end{verbatim}
\end{center}

% Updated
\textbf{preprocessTree}. This stage corrects \pos tag errors and corrects analyses
for noun phrases, coordinate constructions and small clauses that diverge from the
desired \ccg analysis. It also eliminates quotation marks.
Quotation marks are attached inconsistently
in the Penn Treebank and are difficult to analyse in \ccg, partly because of editorial
conventions that force quotes outside punctuation, even when they logically belong
inside it.

% Updated
\textbf{preprocessArgumentClusters}. \ccg analyses allow argument clusters, as in
\emph{Give a policeman a rose and a fireman a violet}, to be bracketed together.
The Penn Treebank analysis of this phenomenon is quite different. The tree is thus
pre-processed to make use of the \ccg analysis.

% Updated
\textbf{determineConstituentType}. \emph{Constituent type} here refers to three labels:
\emph{head}, \emph{argument}, and \emph{adjunct}. These labels were assigned using slight
variations on the head finding heuristics described by \citet{magerman:95} and
\citet{collins:96}.

% Updated
\textbf{makeBinary}. Once the nodes were labelled as head, argument or adjunct,
the tree was binarised. Dummy nodes were inserted on the tree to the left of the
head, such that the tree right-branched towards the head. The same was done for
nodes to the right of the head.

% Updated
\textbf{percolateTraces}. The conversion algorithm relies on the Penn Treebank
\verb1*T*1 and \verb1*RNR*1 trace nodes to assign categories correctly when
arguments have been moved. This stage determines the category of the traces by following
them to their reference node, and percolates them up to their appropriate level
in the tree.

% Updated
\textbf{assignCategories}. The nodes are then assigned categories. The root node
and complement nodes are determined by a mapping from \penn node labels to \ccg categories.
Head nodes and adjuncts were then assigned labels based on their siblings.

% Updated
\textbf{treatArgumentClusters}. This stage inserts type-raising nodes and uses
composition rules to coordinate nodes in argument clusters, following the standard
\ccg analysis.

\textbf{cutTracesAndUnaryRules}. This stage cleans up the resulting tree by removing
traces, and deleting unary \cf{X}$\rightarrow$\cf{X} productions.

\citet{hock:cl07} report that the algorithm does not deal with 306, or 0.76\%, of the 39,832
trees. The main classes of trees not dealt with are unlike coordinated phrases where
the types of the conjuncts could not be identified, verb phrase gapping, and trees
rooted in \verb1X1. Gapping is a very difficult construction to analyse in \ccg,
and \verb1X1-rooted sentences can largely be considered performance noise, so it is unclear
how such sentences would have been analysed even by manual annotation.


\subsection{Type-Changing Rules in CCGbank}

% Updated
After applying the \ccgbank conversion process, \citet{hock:acl02} found
that it caused a proliferation of modifier categories. We explore this problem
in detail in Chapter \ref{chapter:ling_mot}, arguing that its root cause is the
difficulty of exploiting generalisations about constituent type in \ccg. Section
\ref{sec:ling_psg_rules} discusses the theoretical implications of
\citeauthor{hock:acl02}'s solution, which is to add type-changing rules
to the grammar.

The example \citet{hock:thesis03} used to demonstrate the modifier category
proliferation is a sentence from Section 00. Without type-changing rules,
the \ccg derivation for this sentence might look like this:
\footnote{See Sections \ref{sec:ling_rrc} and \ref{sec:pure_rrc} for further
discussion on how reduced relative clauses can be analysed in \ccg without
type-changing rules or hat categories.}
 
\begin{center}
\deriv{4}{
\rm a~form~of~asbestos & \rm once & \rm used & \rm to~make~cigarette~filters \\
\uline{1}&\uline{1}&\uline{1}&\uline{1} \\
\cf{NP} &
\cf{(NP\bs NP)/(NP\bs NP)} &
\cf{(NP\bs NP)/(S[to]\bs NP)} &
\cf{S[to]\bs NP} \\
&& \fapply{2} \\
&& \mc{2}{\cf{NP\bs NP}} \\
& \fapply{3} \\
& \mc{3}{\cf{NP\bs NP}} \\
\bapply{4} \\
\mc{4}{\cf{NP}}
}
\end{center}

The past participle, \emph{used}, receives a different category than it would
if it occurred in a main verb phrase. Adjuncts modifying
\emph{used} will also require different categories --- as will any words modifying
them. As we describe in Section \ref{sec:ling_rrc}, there are more attractive
analyses of this construction, but this analysis does illustrate how modifier categories
can proliferate when there are form/function discrepancies.

% Updated
\citeauthor{hock:acl02}'s solution was to introduce context-free type-changing rules
into the grammar. The example above can be analysed with the canonical categories by
employing the type-changing rule \cf{S[pss]\bs NP} $\rightarrow$ \cf{NP\bs NP}
\footnote{This and following production rules are presented in bottom-up notation.
Production rules often contain punctuation, so for clarity we often wrap single rules
in angle brackets.}. We mark such rules \textsc{tc}:

\begin{center}
\deriv{4}{
\rm a~form~of~asbestos & \rm once & \rm used & \rm to~make~cigarette~filters \\
\uline{1}&\uline{1}&\uline{1}&\uline{1} \\
\cf{NP} &
\cf{(S\bs NP)/(S\bs NP)} &
\cf{(S[pss]\bs NP)/(S[to]\bs NP)} &
\cf{S[to]\bs NP} \\
&& \fapply{2} \\
&& \mc{2}{\cf{S[pss]\bs NP}} \\
& \fapply{3} \\
& \mc{3}{\cf{S[pss]\bs NP}} \\
& \psgrule{3} \\
& \mc{3}{\cf{NP\bs NP}} \\
\bapply{4} \\
\mc{4}{\cf{NP}}
}
\end{center}

\begin{figure}
 \centering
\begin{center}
\deriv{10}{
\rm These & \rm actions & \rm are & \rm risky & \rm and & \rm not & \rm in & \rm
our & \rm best & \rm interests \\
\uline{1}&\uline{1}&\uline{1}&\uline{1}&\uline{1}&\uline{1}&\uline{1}&\uline{1}
&\uline{1}&\uline{1} \\
\cf{NP/N} &
\cf{N} &
\cf{(S[dcl]\bs NP)/(S[adj]\bs NP)} &
\cf{S[adj]\bs NP} &
\cf{conj} &
\cf{PP/PP} &
\cf{PP/NP} &
\cf{NP/N} &
\cf{N/N} &
\cf{N} \\
\fapply{2} &&&&&&& \fapply{2} \\
\mc{2}{\cf{NP}} &&&&&&& \mc{2}{\cf{N}} \\
&&&&&&& \fapply{3} \\
&&&&&&& \mc{3}{\cf{NP}} \\
&&&&&& \fapply{4} \\
&&&&&& \mc{4}{\cf{PP}} \\
&&&&& \fapply{5} \\
&&&&& \mc{5}{\cf{PP}} \\
&&&& \psgrule{6} \\
&&&& \mc{6}{\cf{S[adj]\bs NP[conj]}} \\
&&& \conj{7} \\
&&& \mc{7}{\cf{S[adj]\bs NP}} \\
&& \fapply{8} \\
&& \mc{8}{\cf{S[dcl]\bs NP}} \\
\bapply{10} \\
\mc{10}{\cf{S[dcl]}}
}
\end{center}
\caption[Conjunction cued binary type-changing rule in \ccgbank]{Derivation
using a conjunction cued binary type-changing rule to handle an unlike
coordinated phrase.\label{fig:conj_raising}}
\end{figure}

\begin{table}[h!]
\centering
\begin{tabular}{r|c|c||r|c|c}
\hline
Freq & Child & Parent & Freq & Child & Parent\\
\hline
\hline
142,530 & \cf{N} & \cf{NP} & 1,464 & \cf{S[to]\bs NP} & \cf{NP\bs NP} \\
4,052 & \cf{S[pss]\bs NP} & \cf{NP\bs NP} & 1,070 & \cf{S[dcl]/NP} & \cf{NP\bs NP} \\
1,818 & \cf{S[ng]\bs NP} & \cf{NP\bs NP} & 370 & \cf{S[ng]\bs NP} & \cf{NP} \\
1,617 & \cf{S[adj]\bs NP} & \cf{NP\bs NP} & 254 & \cf{S[ng]\bs NP} & \cf{S/S} \\
1,606 & \cf{S[to]\bs NP} & \cf{(S\bs NP)\bs (S\bs NP)} & 209 & \cf{S[dcl]} & \cf{NP\bs NP} \\
1,522 & \cf{S[ng]\bs NP} & \cf{(S\bs NP)\bs (S\bs NP)} & 192 & \cf{S[pss]\bs NP} & \cf{S/S} \\
1,476 & \cf{S[to]\bs NP} & \cf{N\bs N} & 154 & \cf{S[to]\bs NP} & \cf{S/S} \\
\hline
\end{tabular}
\caption{The most frequent unary type-changing rules in CCGbank.}
\label{tab:background_psg_rules}
\end{table}


Table \ref{tab:background_psg_rules} shows the most frequent unary type-changing rules in
\ccgbank. Most of the type-changing rules in \ccgbank are unary, but some of the
rules are cued by punctuation or a conjunction, making them binary. The
advantage of making these rules binary is that they introduce less ambiguity
into the grammar. Figure \ref{fig:conj_raising} shows an example where
a conjunction is used to cue a type-change. This involves a rather \emph{ad hoc}
type-changing rule:

\begin{eqnarray}
 \eqnpsrule{\cf{conj}}{\cf{PP}}{\cf{S[adj]\bs NP[conj]}}
\end{eqnarray}

This rule is also an example of how type-changing rules can allow categories
to add or delete arguments arbitrarily, breaking the isomorphism between lexically
assigned syntactic types and the resulting logical forms. This makes the
type-changing rules unacceptable on a
theoretical level, because they are incompatible with \citepos{steedman:00} core
argument. The first sentence of \citet{steedman:00} states the \ccg thesis as:

\begin{quote}
 This book argues that the Surface Syntax of natural language acts as a
completely transparent interface between the spoken form of the language,
including prosodic structure and intonational phrasing, and a compositional
semantic interface.
\end{quote}

This type-transparency is impossible to achieve if an atomic category is allowed
to transform into a functor, as the \psbinary{,}{NP}{S/S} rule used to handle
extraposition allows. An example of this rule is shown in
Figure~\ref{fig:extraposition}. In this example, the \cf{NP} \emph{our first
major decline} is coerced into a sentential modifier. However, there is no room
for an argument on its original lexical category, so there is no way to
represent the dependency between \emph{decline} and \emph{fell}. This causes the
\ccgbank dependency graph of the
sentence to be unconnected, as \emph{decline} does not have a head. The binary
type-changing rules are discussed in more detail in Chapter
\ref{chapter:hat_corpus}. The motivation for the rule is the modifier category
proliferation problem. If \emph{decline} were assigned a category that reflected
its function, the categories of its modifiers \emph{first} and \emph{major}
would have to change too, as would any post-modifiers that might attach, such as
\emph{this quarter}. We explore this problem with \ccg in more detail in Chapter
\ref{chapter:hat_cats}.

\begin{figure}

\begin{center}
\deriv{10}{
\rm Factory & \rm inventories & \rm fell & \rm 5 & \rm \% & \rm , & \rm our &
\rm first & \rm major & \rm decline \\
\uline{1}&\uline{1}&\uline{1}&\uline{1}&\uline{1}&\uline{1}&\uline{1}&\uline{1}
&\uline{1}&\uline{1} \\
\cf{N/N} &
\cf{N} &
\cf{(S[dcl]\bs NP)/NP} &
\cf{NP} &
\cf{NP\bs NP} &
\cf{,} &
\cf{NP/N} &
\cf{N/N} &
\cf{N/N} &
\cf{N} \\
\fapply{2} && \bapply{2} &&&& \fapply{2} \\
\mc{2}{\cf{N}} && \mc{2}{\cf{NP}} &&&& \mc{2}{\cf{N}} \\
\psgrule{2} & \fapply{3} &&& \fapply{3} \\
\mc{2}{\cf{NP}} & \mc{3}{\cf{S[dcl]\bs NP}} &&& \mc{3}{\cf{N}} \\
\bapply{3} &&&& \fapply{4} \\
\mc{3}{\cf{S[dcl]}} &&&& \mc{4}{\cf{NP}} \\
&&&&& \psgrule{5} \\
&&&&& \mc{5}{\cf{S\bs S}} \\
\bapply{10} \\
\mc{10}{\cf{S[dcl]}}
}
\end{center}
\caption[Derivation with binary type-changing rule in \ccgbank.]{Derivation
showing a binary \textsc{psg} rule cued by punctuation. The comma is used to cue
a rule that changes the \cf{NP} into a sentential modifier. This prevents the
formation of a connected dependency graph for the sentence, and results in a
loss of semantic transparency.\label{fig:extraposition}}
\end{figure}

\section{The \candc Parser}
\label{sec:background_candc}

The best \ccg parsing results to date have been achieved by the \candc parser.
The parser uses a global discriminative model, trained on \ccgbank. The most
similar system is probably the \citet{miyao:08} \hpsg parser, which also uses a
global model over a packed chart, and a supertagger to assign lexical
categories.

In this section we will briefly describe the parser, focusing on the aspects of
its design that are most important for this thesis. We will start with its
tightly integrated supertagger, before providing an overview of the chart parser
and the discriminative maximum entropy model used to select the most probable
parse.

\subsection{Supertagging}

\citet{srinivas:94} showed that \ltag lexical categories can be assigned in a
sequence tagging stage, using much the same sort of technique as part-of-speech
and named entity tagging. \citet{clark:tag02} showed that supertagging could be
applied to \ccg, and that the accuracy of a maximum entropy tagger
\citep{ratnaparkhi:96} on \ccgbank was comparable to the performance
\citet{srinivas:99} achieved on a manually constructed \ltag grammar, and
substantially higher than the performance \citet{chen:00} achieved on an
automatically extracted \ltag grammar.

The primary appeal of supertagging is that it allows much of the parsing work to
be transformed into a linear process. The tags can be assigned using well
understood Markov-based techniques that run in linear time with respect to the length
of the input sentence. Once the tags are
assigned, there is relatively little ambiguity remaining for the parser, which
is why \citeauthor{srinivas:99} referred to supertagging as `almost parsing'.

The efficiency improvement is explained by \citepos{sarkar:00} finding that
parsing efficiency is not solely related to sentence length. It is also related
to the syntactic ambiguity of the sentence, which in a lexicalised grammar is
largely predictable from the lexical category ambiguity. Supertagging allows
category ambiguity to be reduced, by forwarding only the categories judged most
likely to the parser.

One obvious configuration is to have the supertagger assign exactly one category
per word, eliminating all lexical category ambiguity. In this configuration, the
supertagger and parser interact in a simple cascade model: the supertagger makes
its predictions, which are fed directly to the parser. This configuration
suffers from the usual problems of cascaded classification. Errors propagate
along the pipeline, and the early model cannot benefit from the decisions of
subsequent models \citep{hollingshead:07}. The later models are also unable to take
into account any
uncertainty in the initial model, because they are given only the most likely label,
instead of the whole probability distribution \citep{finkel:06}.

The clearest problem occurs when the parser fails to assign a parse from the
supertag sequence that has been assigned. \citet{srinivas:94} address this by
\emph{multi-tagging}, assigning the $n$-best categories per word.
\citet{clark:cl07} take a similar approach, assigning all categories whose
probability is within a factor, $\beta$, of the maximum probability the
supertagger assigned to a single category for that word.\footnote{That is, the supertagger assigns all categories $t$ whose
probabilities $p(t) \geq \beta \cdot p\ast$, where $p\ast$
is the highest ranked tag's probability.}
Multi-tagging allows the \candc supertagger to make very favourable trade-offs between accuracy and
lexical ambiguity. 

\subsection{Integrating a Supertagger and Parser}

\citet{clark:emnlp03} and \citet{clark:coling04} present two ways the
multi-tagger can be integrated into a \ccg parsing system. The initial strategy
was to initialise $\beta$ at 0.01, and attempt to find a parse. If the ambiguity
caused too many categories to be added to the chart, the $\beta$ value was increased, tightening
the beam of categories sent to the parser, and the parser tried again.

This approach assigns as much work to the parser's model as possible, because it
has the supertagger forward as many categories as the parser can handle.
\citeauthor{clark:cl07} assumed that this configuration would
trade speed for accuracy, because it seemed safe to assume that the parser's
global model would be more accurate than the supertagger's sequence model.

\citet{clark:coling04} describe the critical finding that this was not the case.
Instead, the system performed both faster \emph{and} more accurately if the
supertagger was allocated much of the work. In this configuration, the $\beta$
level is initialised to $0.075$, and subsequently relaxed if parsing fails. This
is essentially the opposite strategy to the one described above, where the
$\beta$ levels proceed from low to high.

What is the explanation for this result? For a start, it seems that the limited
horizon assumption performs quite well --- most of the tagging decisions
actually are locally decidable. This is partially explained by
\citepos{gildea:07} finding that most dependencies are quite short-range: the
average dependency distance in English is 2.3. In fact, \citet{hawkins:90}
argues that the pressure to minimise dependency distances is a considerable
influence on the syntax of all human languages.

% Another factor might be that the number of competing analyses the parser's model
% must consider,  the chart size, grows exponentially with increases in supertag
% ambiguity. Even if the parser's model were much better suited to making
% decisions than the supertagger, its advantage might be overwhelmed by the number
% of competing analyses it has to decide between.


\subsection{Discriminative Parsing with Log-Linear Models}

After the supertagger has assigned one or more lexical categories to each word
in the sentence, the categories are assembled into \ccg derivations and
corresponding dependency analyses using the modified \cky algorithm first
described by \citet{hock:01}. Despite the reduction in ambiguity provided by the
supertagger, the length of the sentences in \ccgbank and the ambiguity in the
grammar necessary to produce high coverage mean that many sentences produce an
astronomical number of candidate analyses. For one sentence in Section 00, the
default configuration of version 1.02 of the parser leads to over
$6.39\times10^{23}$ unique parses. The sentence is:

\begin{quote}
For a while in the 1970s it seemed Mr. Moon was on a spending
spree, with such purchases as the former New Yorker Hotel and its
adjacent Manhattan Center; a fishing/processing conglomerate with
branches in Alaska, Massachusetts, Virginia and Louisiana; a former
Christian Brothers monastery and the Seagram family mansion
(both picturesquely situated on the Hudson River); shares in banks
from Washington to Uruguay; a motion picture production
company, and newspapers, such as the Washington Times, the New
York City Tribune (originally the News World), and the successful
Spanish-language Noticias del Mundo.
\end{quote}

At 108 tokens, this sentence is far from the longest in the Penn Treebank, but the long
coordinated noun phrase forces the parser to consider an enormous number of
possible analyses.

The \candc parser implements a modified \ccg grammar optimised for \ccgbank. The
most important modifications are implementations of the \ccgbank type-changing
rules, which are hard-coded into the parser. Type-raising rules are handled in
an equivalent way to unary type-changing rules, except they are not hard-coded
(possible type-raising operations are listed in text files read by the parser).
The parser also implements some restrictions on the composition rules, and some
of its models make use of the \citet{eisner:96} normal-form constraints. Some
models also use a list of productions seen in the training corpus to further
restrict the productivity of its grammar. Appendix \ref{appendix:type-changing}
lists each of the type-changing rules found in \ccgbank and the frequency of its
occurrence in Section 02-21.


\section{Applications of Combinatory Categorial Grammar}

There are a few other strands of \ccg work within the \nlp community that we
have not yet mentioned. Although we do not want to go into great detail about
these bodies of work, we will briefly mention the prominent projects, for
completeness.

\subsection{\ccg for Semantic Analysis}

\ccg's isomorphism between syntax and semantics has been used
to perform semantic role labelling. The first example of this was \citet{gildea:03},
who built a semantic role labelling system using the output of the
\citet{hock:acl03} \ccg parser. However, this research used an early version of
Propbank \citep{propbank}, which introduced a variety of
difficulties. For instance, arguments realised by prepositional phrases in \ccg
were listed as noun phrase arguments in the initial PropBank --- the preposition
was excluded from the argument span. Despite these issues,
\citeauthor{gildea:03} found that the semantic role labeller performed well on
core arguments with a simple feature set, suggesting that \ccg's recovery of
long-range dependencies and direct representation of verbal subcategorisation
frames simplified the task. \citet{boxwell:09}, who used the \candc parser
in a semantic role labelling system, reached a similar conclusion.

\ccg has also been used to recover logical form-based semantic analyses, as well
as the shallow semantic analysis provided by PropBank-style predicate-argument
structures. The first example of this was \citet{bos:coling04}, who
post-processed the output of the \candc parser to provide logical forms. The
system to do this, dubbed Boxer, can be downloaded and used along with the
\candc tools. Boxer has been used as the basis for competitive
question answering systems \citep{leidner:04,bos:06,bos:07}. \citet{bos:05}
showed that it could also be used for textual entailment. \citet{zettlemoyer:07}
used \ccg-based logical form parsing in a system that achieved competitive
results on the GeoQuery task, which is essentially a sort of natural language
database querying challenge. \citet{zettlemoyer:09} showed that \ccg could be
used as the basis of a system that retrieved logical forms for sentences in the
\textsc{atis} corpus with over 83\% accuracy.

\subsection{\ccg for Chart Realisation}

% Updated
Chart, or surface, realisation is the task of going from a (possibly underspecified)
semantic representation to one or more surface strings that can express it. It is
the inverse of the parsing task. Linguistically motivated formalisms have been a
popular choice for this task since the Penman system \citep{mann:85}, which later
merged with the \textsc{komet} system to become \kpml \citep{bateman:99}.
Surface realisers have also been developed for \hpsg \citep{wilcock:98} and
\lfg \citep{cahill:06}.

% Updated
Research on the use of \ccg for chart realisation has to some extent paralleled
research on \ccg parsing. \citet{white:03} describe the architecture of a
\ccg-based surface realiser. Their system, which was implemented as an extension
to the OpenCCG parser, uses a bottom-up approach to realising logical forms specified
in Hybrid Logic Dependency Semantics \citep{kruijff:01}.

% Updated
This early work on \ccg surface realisation used precise, manually developed grammars
for dialogue systems. \citet{white:07} began work on a wide-coverage \ccg realiser,
using a grammar extracted from \ccgbank. Research along these lines has progressed
rapidly, in two dimensions: modifications to \ccgbank to support more accurate logical
form recovery, and improvements to their statistical modelling. Modifications to
\ccgbank have included lexicalised treatment of punctuation \citep{white:punct08},
alignment with Propbank \citep{boxwell:08}, and integrating BBN named entity analyses
\citep{rajkumar:09}. Improvements to their statistical models have included a
perceptron reranker \citep{white:09} and a \emph{hypertagger}, a model which predicts
which lexical categories to use in the surface realisation. The
categories then closely constrain the space of possible surface strings, in much the
same way that the categories output by a supertagger constrain the chart space required
to build a semantic analysis from an input string.

% Updated
The hypertagger is inspired by the \citet{clark:tag02} \ccg supertagger, and improves the
chart realiser's speed and accuracy, and a reranker would likely improve the accuracy of
the \candc parser. Research on \ccg parsing and generation, then, has encountered similar
issues, and solutions developed for one problem can be profitably adapted to the other.
This is especially apparent for refinements to \ccgbank, the key resource for both lines
of research.

% Updated
OpenCCG has been used in a number of natural language dialogue applications: the
COMIC, FLIGHTS, CrAg, Methodius and INDIGO projects in Edinburgh; the DIALOG, SAMMIE and
CoSy projects in Saarbrucken; the JAST project in Munich; the AdaRTE project in Pavia;
and the STaR-UI project in Sydney. Many of these projects involve spoken language dialogue
systems. \ccg has a particular advantage for speech generation, because its non-standard
constituents allow prosodically salient portions of a sentence to be grouped together.
This allows the chart realiser to map information structure specifications onto surface
forms as prosodic tunes, so that a speech synthesiser can produce more natural sounding
output. \citet{white:10} demonstrated this capability in the FLIGHTS dialogue system,
showing that the information structure-aware version of the system scored significantly
higher in user ratings of naturalness, and that the system accurately produced prosodic
tunes that an expert recognised as contextually appropriate.\footnote{
The stimuli for these experiments are available at
http://www.ling.ohio-state.edu/~mwhite/flights-stimuli/} \ccg's flexible notion of
constituency has also been used to allow a dialogue system to interpret user utterances
incrementally, in the CoSY project \citep{kruijff:07}.


\subsection{\ccg for Machine Translation} 

% Updated
Finally, there has also been a line of research using \ccg for machine translation.
\citet{hassan:07} showed that the \candc supertagger could be used to add some syntactic
awareness to a language model, leading to state-of-the-art performance on Arabic to English
machine translation. \citet{birch:07} observed a more modest improvement on translation
between English and Dutch, and concluded that supertags improved the system's reordering.
\citet{hassan:09} then showed that an incremental \ccg parser could be integrated into a
phrase-based statistical machine translation system, leading to further improvements in
Arabic to English translation.

\section{Summary}

Combinatory categorial grammar is a lexicalised grammar formalism that provides
a transparent interface between syntax and semantics. A \ccg grammar mainly
consists of a set of lexical categories, which can be interpreted as functions
from some set of arguments to an atomic result category. The formalism uses a
very small inventory of rules, some of which were added to allow categories to
become partially associative. This associativity allows sentences to be
bracketed in a variety of ways, many of them featuring non-traditional
constituents --- for instance, subjects can be bracketed with verbs, in addition
to the traditional bracketing of verb and object. This so-called spurious ambiguity
has proven to be an advantage of the formalism, and the
complications it induces for practical parsing are easily overcome.

\ccg has become highly influential in computational linguistics, proving useful
in a variety of applications. It has become particularly prominent in statistical
parsing and semantic analysis. The most accurate wide-coverage \ccg parser, \candc,
is currently a leading parsing system, based on a variety of evaluations. The corpus
used to train the parser, \ccgbank, was created by semi-automatically converting the
Penn Treebank, a corpus that used a less informative representation than the target
analyses desired for \ccgbank. \ccgbank therefore includes certain some suboptimal
annotations. Corrections for some of these issues have been proposed
\citep[e.g. ][]{honnibal:pacling07prop,vadas:08,white:punct08}, but updated
versions of the corpus were released too late for use in this thesis.

The other problem we note with \ccgbank, the presence of unary and binary
type-changing rules, has received little attention. We will now explore the
problem that motivated the addition of these rules, and its consequences for a
wide-coverage \ccg grammar.



 \chapter{Form, Function and Modification in Combinatory Categorial Grammar}
\label{chapter:ling_mot}
%\section{Introduction}

Having provided an overview of Combinatory Categorial Grammar \citep[\ccg, ][]{steedman:00}
in Chapter~\ref{chapter:background}, we can begin the core argument of this thesis. Our
argument begins with the claim that there is a problem with some of the analyses
that result from a \ccg grammar that is limited to application, composition
and type-raising rules. We claim that this is why \citet{hock:cl07} included
type-changing rules in \ccgbank, but we argue that this is not a good
solution to the problem, which we characterise as follows.

In \ccg, every word is assigned a category that encodes the syntactic function of the
constituent it heads. This direct representation of
syntactic function is behind many of the desirable properties of the formalism:
it allows lexical categories to be paired with semantic analyses, enables attractive
analyses of coordination, and allows language-specific variation
to be confined to the lexicon.

However, it also makes it difficult to include a consistent treatment of
the other widely recognised dimension of regularity in syntax: syntactic form,
or constituent type.
In this chapter, we argue that difficulty expressing generalisations about
constituent type produces over-generation, undesirable analyses, and
prevents the grammar from fully generating certain recursive structures with a
finitely sized lexicon. These issues largely arise because the grammaticality
of modifier-head relationships is controlled by constituent type, not constituent
function.

The chapter is structured as follows.
First, we informally define constituent type and constituent function.
Next, we identify three problems the lack of constituent type generalisations
cause. We then investigate whether the issues we have raised can be solved by previous
proposals.

\section{Constituent Type and Constituent Function}

Grammarians have typically recognised two dimensions along
which words and the constituents they head can be grouped. One of these dimensions
describes words and constituents structurally. Along this dimension lie
part-of-speech labels like preposition, noun and verb; and constituent type
labels like prepositional phrase, noun phrase and small clause. The second
dimension describes how constituents interact with the rest of the sentence:
their syntactic function. Along this dimension lie function labels like subject,
complement and adjunct; and descriptions of argument structure such as transitive,
intransitive and ditransitive.

There are linguistically relevant generalisations along both dimensions.
For instance, English verbs agree in number with their subject, not with
the first occurring noun phrase (e.g. \emph{This week we are}, not \emph{* This week
we is}). This is a generalisation about constituent function. On the other hand,
subject, complement and adjunct noun phrases can all be modified in the same way by
adjectives, preposition and relative clauses. This is a generalisation about constituent
type.

This chapter argues that generalisations about the dimension of variation we are here
terming \emph{constituent type} are difficult to represent in \ccg.
The term constituent type is potentially problematic. A \emph{type} is, broadly,
simply a label assigned to a constituent. Thus the \ccg `constituent type' of 
a constituent is simply its category. However, \ccg categories reflect
generalisations along the dimension we are terming \emph{constituent function}.
Consider the case of prepositional phrases:

\begin{lexamples}
 \item \gll \underline{In~case~of~fire}, use~the~stairs.
            \cf{S/S} \cf{~}
       \gln
       \glend
 \item \gll The~man \underline{in~the~red~coat} used~the~stairs.
       \cf{~} \cf{N\bs N} \cf{~}
       \gln
       \glend
 \item \gll The~sky~was~filled \underline{with~smoke}.
       \cf{~} \cf{(S\bs NP)\bs (S\bs NP)}
       \gln
       \glend
\end{lexamples}

Each of these prepositional phrases is assigned a different category, so the
syntactic structure they share in common is not directly represented. As
we describe in Section~\ref{sec:proposals}, one way to capture the generalisation
is to have every category bear a feature, \textsc{type}, that reflects its
syntactic form. We offer three arguments for the necessity of constituent type
representation in the sections that follow, and suggest that a \textsc{type} feature
cannot always solve the problem.

\section{Under- and Over-generation with \ccg Modifiers}
\label{sec:over-generation}
The grammaticality of many attachment decisions --- notably
modification --- is best explained with reference to constituent type, not function. 
This means that generating modification structures with reference to the head's
function can produce ungrammatical attachments, leading to over-generation. This
problem arises when multiple constituent types can perform the same function,
but only one can be modified by a particular constituent type. An equivalent
under-generation problem occurs because the modifiers of a single constituent type
need to take on multiple categories, depending on the constituent's function.

For instance, in English there are many constituent types that can function adverbially,
such as temporal nouns, prepositional phrases, adverbs and participial clauses.
These constituent types have very different internal structures. Unfortunately,
\ccg modifiers refer to categories that reflect the constituent's function, not its form.
All modifiers of adverbially functioning constituents thus require the same
category, \cf{(VP\bs VP)\bs (VP\bs VP)}. This potentially 
licenses ungrammatical attachments:

\begin{lexamples}
\item \gll Robin slept very well
\cf{NP} \cf{S[dcl]\bs NP} \cf{(VP\bs VP)/(VP\bs VP)} \cf{VP\bs VP}
\gln
\glend
\item \gll Robin slept all~Tuesday
\cf{NP} \cf{S[dcl]\bs NP} \cf{VP\bs VP}
\gln
\glend
\item \gll *~Robin slept very all~Tuesday
\cf{NP} \cf{S[dcl]\bs NP} \cf{(VP\bs VP)/(VP\bs VP)} \cf{VP\bs VP}
\gln
\glend
\end{lexamples}

This kind of over-generation can be controlled by assigning every category
a \textsc{type} feature, that records its constituent type. For instance,
\emph{well} might be assigned the category \cf{(VP\bs VP)[mnr]} to reflect
its status as a manner adverbial, and \emph{very} could subcategorise for
the feature, with the category \cf{(VP\bs VP)[mnr]/(VP\bs VP)[mnr]}.

It is difficult to use features to account for under-generation, however. This
has typically been addressed in the lexicon, using inheritance classes or lexical
rules. The problem is that a syntactic type like adjective can perform many functions,
forcing its modifiers to adopt different categories as well:

\begin{lexamples}
 \item \gll The bike   is very fast
      \cf{NP/N} \cf{N} \cf{(S[dcl]\bs NP)/PE} \cf{PE/PE} \cf{PE}
 \gln
 \glend
      \gll The very fast bike
     \cf{NP/N} \cf{(N/N)/(N/N)} \cf{N/N} \cf{N}
 \gln
\glend
\end{lexamples}

We have followed \citet{carpenter:92} in adopting \cf{PE} for predicative elements,
but the problem is no different with the \citet{hock:cl07} category of \cf{S[adj]\bs NP}.
In either case, some mechanism needs to account for the multiple functions of an adjective
like \emph{fast}, and then generate the appropriate adjectival modifiers. Several proposals
to do this in the lexicon exist. For \ccg, both \citet{beavers:04} and \citet{mcconville:06}
suggest how a \ccg lexicon could be structured to support this kind of inheritance.
The issue has been a well-studied problem in \hpsg since at least
\citet{flickinger:thesis87}.

We are not suggesting that it is impossible to write a \ccg grammar that does not suffer
from such over- and under-generation problems. Our argument is that the solution to
these problems necessarily involves generalisations about constituent type; and we further
argue that the formalism makes such generalisations difficult to encode in a wide-coverage
grammar. The need for an easier way to represent constituent type generalisations motivates
the inclusion of some sort of type-changing operation. In Section~\ref{sec:ling_psg_rules}, we
argue that there are problems with the existing type-changing proposal, as presented by
\citet{hock:cl07}, and instead present our own in Chapter~\ref{chapter:hat_cats}.
But first, we describe two further problems that motivate type-changing rules in
\ccg.



\section{Recursive Modification Requires Infinite Categories}
\label{sec:infinite_categories}
In English, some constituent types can function as modifiers of their own type.
The result is unbounded recursion depth. This can be problematic for \ccg,
because it means depth sensitive categories are required. The result of this is
an inability to generate the full set of grammatical constituents with a finite
set of categories.

Compound nouns are the clearest example of this in English. Adverbial clauses
are another example. We assume that a phrase like \emph{management system} would
be analysed as noun-noun modification, with \emph{system} as head:

\begin{center}
\begin{parsetree}
(.\cf{N}.
  (.\cf{N/N}. `management')
  (.\cf{N}.   `system')
)
\end{parsetree}
\end{center}

\noindent The opposite ordering is also grammatical:

\begin{center}
\begin{parsetree}
(.\cf{N}.
  (.\cf{N/N}. `system')
  (.\cf{N}.   `management')
)
\end{parsetree}
\end{center}

Both words head constituents of the same type, which we will call \nom, to
distinguish it from the category, \cf{N}. One of the possible functions of \nom
constituents is leftward modification of another \nom. In the example above, the
category of the modifier constituent changed to reflect its function as a
modifier. If the whole constituent functions as a modifier of another \nom, both
of their categories must change:

\begin{center}
\ptbegtree
\ptbeg \ptnode{\cf{N}}
  \ptbeg \ptnode{\cf{N/N}}
    \ptbeg \ptnode{\cf{(N/N)/(N/N)}} \ptleaf{water} \ptend
    \ptbeg \ptnode{\cf{N/N}}  \ptleaf{meter} \ptend
  \ptend
  \ptbeg \ptnode{\cf{N}} \ptleaf{cover}\ptend
\ptend
\ptendtree
\end{center}

At each depth of modification, a new category is required. A longer
left-branching example, like \emph{water meter cover adjustment screw} would
require the category:

\begin{equation}
\cf{(((N/N)/(N/N))/((N/N)/(N/N)))/(((N/N)/(N/N))/((N/N)/(N/N))))}
\end{equation}

With an even slightly longer phrase, like \emph{hot water meter cover adjustment
screw}, the categories required become unprintable.

%\subsubsection{The recursion is infinite, so we will need infinite categories}

If we call one constituent that modifies another a \emph{modifier}, a
constituent that modifies the first one will be a \emph{modifier modifier},
which might be modified in turn by a \emph{modifier modifier modifier} --- and
so on, unbounded. Such a phrase of length $n$ will require $n$ different
categories. Since the phrase is grammatical at any length, a finite category set
is inadequate.

The crux of the problem is that the grammaticality of a \emph{(modifier, head)}
attachment is determined by the structural types of the two constituents, but this is not
how categorial grammars model modification. With no theory of constituent type,
a modifier instead refers to its head's function, which might be forced to refer
to \emph{its} head's function --- and so on.

%\subsubsection{Composition doesn't help}

At first glance, it might seem that the long categories are unnecessary, because
we can bracket the modifiers together using the composition rule:

\begin{center}
\deriv{3}{
\rm water & \rm meter & \rm cover \\
\uline{1}&\uline{1}&\uline{1} \\
\cf{N/N} &
\cf{N/N} &
\cf{N} \\
\fcomp{2} \\
\mc{2}{\cf{N/N}} \\
& \fapply{2} \\
& \mc{2}{\cf{N}}
}
\end{center}

However, this derivation does not produce the analysis we want, because of the
semantic annotation of the \cf{N/N} category:

\begin{lexample}
 \cf{(N_y/N_y)_x}
\end{lexample}

The \cf{x} variable is filled by the word that heads the modifier. When
\emph{water} composes with \emph{cover}, its argument unifies with
\emph{cover}'s result, which is unified with \emph{cover}'s argument. When this
argument unifies with \emph{meter}, we get the following dependencies:

\begin{eqnarray}
(water, \;\; \cf{N/N},\;\; 1, \;\; cover)\nonumber \\
(meter, \;\; \cf{N/N},\;\; 1, \;\; cover)\nonumber
\end{eqnarray}

The left-branching derivation using composition is therefore equivalent to the
right-branching derivation using application:

\begin{center}
\deriv{3}{
\rm water & \rm meter & \rm cover \\
\uline{1}&\uline{1}&\uline{1} \\
\cf{N/N} &
\cf{N/N} &
\cf{N} \\
& \fapply{2} \\
& \mc{2}{\cf{N}} \\
\fapply{3} \\
\mc{3}{\cf{N}}
}
\end{center}

While this approach is not fruitful for two modifiers, at one level deeper it can
constrain the number of categories. Ordinarily, modifiers receive symmetrically
indexed categories. Thus, the category for \emph{water} in \emph{water meter cover}
would be \cf{((N_y/N_y)_z/(N_y/N_y)_z)_w}, where $w$ is bound to \emph{water} and
$y$ and $z$ will be bound during the derivation. If we instead coindexed the category
so that the $w$ variable were exposed to composition, using the category
\cf{(N_y/N_y)_w/(N_y/N_y)_z}, 
we could get the correct 
analysis for \emph{hot water} without a modifier-modifier-modifier category:

\begin{center}
\deriv{2}{
\rm hot & \rm water \\
\uline{1}&\uline{1} \\
\cf{(N_y/N_y)_h/(N_y/N_y)_z} &
\cf{(N_y/N_y)_w/(N_y/N_y)_z} \\
\fcomp{2} \\
\mc{2}{\cf{(N_y/N_y)_h/(N_y/N_y)_z}}
}
\end{center}

This strategy gets the correct dependency between \emph{hot} and \emph{water},
but records the head of the constituent as \emph{hot}. This is problematic
if there is a modifier of \emph{water} to the left of \emph{hot}. For instance,
if the constituent were \emph{treated hot water meter cover}, \emph{treated}
must modify \emph{water}, as must \emph{hot}. With the categories assigned above,
only one is able to. Nevertheless, the OpenCCG surface realisation system employs
a variant of this mechanism \citep[][\S 4.1]{white:08punct}.

Adverbial examples of the recursive modification problem are less concise, because
the construction encounters semantic and pragmatic constraints. Verb phrases can
function as direct modifiers of other verb phrases, in the same way as noun-noun
compounding. For instance, \emph{Feeling} here modifies \emph{caught} directly,
and would receive a category such as \cf{(S/S)/(S[adj]\bs NP)}:

\begin{lexamples}
\item Feeling chilly, Robin caught a Taxi.
\end{lexamples}

\emph{Feeling} is non-finite, but its constituent type is still a verb phrase,
so it is susceptible to modification by its own clausal adjuncts, which can
themselves be modified by other clausal adjuncts:

\begin{lexamples}
 \item ((Feeling chilly) ((wearing a t-shirt) ((walking home) (carrying shopping
(...))))).
\end{lexamples}

In the intended reading, where \emph{carrying} modifies \emph{walking}, which
modifies \emph{wearing}, which modifies \emph{Feeling}, we will need to assign
the following category to \emph{carrying}:
\begin{equation}
 \cf{((((S/S)\bs (S/S))\bs ((S/S)\bs (S/S)))\bs (((S/S)\bs (S/S))\bs ((S/S)\bs
(S/S))))/NP}
\end{equation}


The pragmatic problem with this example is that the attachment ambiguity makes
the sentence very difficult to process, and the dependency distances become very
long because the verb phrases all have argument structures. There is also no
obvious way to construct an unbounded example, as we did with \emph{modifier
modifier}, because verb phrase modification typically involves temporal or
logical relations, and time and causation do not readily form loops. The closest
we can construct involves a sort of feedback loop. If we believed that
depression might act to make someone more depressed, we might express their mood
as \emph{((Feeling blue) ((feeling blue) ((feeling blue) (...))))}.

At any rate, even if a construction is impossible for pragmatic or semantic
reasons, if it is \emph{syntactically} licensed, it should be within the
coverage of the grammar. There will always be pragmatic constraints on
modification depth, if nothing else because speakers do not perform unbounded
utterances. The question is whether the mechanisms in our grammar seem to
model the way language is structured. These examples make clear that recursive
modification constructions pose a problem for \ccg, because full coverage of the
phenomenon requires an infinite lexicon.

\section{The Lack of Constituent Type Makes Description Difficult}
\label{sec:descriptive_power}
In Section~\ref{sec:ab_sucks}, we considered an \abcg analysis of extraction
that relied on category ambiguity instead of grammatical machinery:

\begin{center}
\deriv{4}{
\rm Pat, & \rm who & \rm Erin & \rm hates \\
\uline{1}&\uline{1}&\uline{1}&\uline{1} \\
\cf{NP} &
\cf{(NP\bs NP)/(S[dcl]/NP)} &
\cf{NP} &
\cf{(S[dcl]/NP)\bs NP} \\
&& \bapply{2} \\
&& \mc{2}{\cf{S[dcl]/NP}} \\
& \fapply{3} \\
& \mc{3}{\cf{NP\bs NP}} \\
\bapply{4} \\
\mc{4}{\cf{NP}}
}
\end{center}

In general, this is not a strategy we wish to adopt. Instead, we wish to assign
categories that place the arguments in canonical positions, and use the grammar
to account for predictable transformations.

Form-function discrepancies are another class of predictable transformations.
Form-function discrepancies can be handled by introducing additional category
ambiguity, but this treatment is not particularly satisfactory. We will pursue
two examples, both involving verbs: nominalisation, and reduced relative
clauses. There are many other constructions which force \ccg into an undesirable
analysis, such as predicative complements, adverbial nouns, and topicalisation.
What these constructions share in common is a mismatch between the underlying
type of the constituent and the category it must receive to function in the
derivation.

In general, there are two strategies for handling this mismatch: we can change
the category of the head of constituent, which will also force the categories of
its modifiers to change. Alternatively, we can change the category of its head,
usually by altering its argument structure.

\subsection{Nominal Clauses}
\label{ling_mot:nominal}
Any English verb can head a nominal clause, in gerund and infinitive forms:

\begin{lexamples}
\item Seeing things is believing things.\\
\item To see things is to believe them.
\end{lexamples}

A detailed analysis such as that provided by the \xtag grammar \citep{xtag}
identifies several varieties of gerund in English, but the important property
for our purposes is that they have the internal structure of sentences, but a
distribution roughly equal to other noun phrases \citep{rosenbaum:67}. A nominal
clauses can fill any \cf{NP} typed argument slot:

\begin{lexamples}
\item I gave \emph{doing things his way} a chance.\\
\item I gave a chance to \emph{doing things his way}.\\
\item \emph{Doing things his way} gave me a chance.
\end{lexamples}

This rules out one possible analysis, which would involve changing the argument structure
of the clause's head verb so that it subcategorises for gerund nominals specifically,
using a type such as \cf{S[nom]}.
This would introduce an extra category for every combination of \cf{NP} typed
arguments in the category:

\begin{eqnarray}
to   &\assign& \cf{PP/S[nom]}\nonumber\\
gave &\assign& \cf{((S[dcl]\bs S[nom])/NP)/NP}\nonumber\\
gave &\assign& \cf{((S[dcl]\bs NP)/S[nom])/NP} \nonumber\\
gave &\assign& \cf{((S[dcl]\bs NP)/NP)/S[nom]} \nonumber\\
gave &\assign& \cf{((S[dcl]\bs S[nom])/S[nom])/NP}\nonumber\\
etc  &    & \nonumber
\end{eqnarray}

Unfortunately, the only alternative is to type the inner-most result of the
nominalised clause as an \cf{NP[ger]}. The feature, for \emph{gerund}, is necessary
to prevent over-generation as discussed in Section~\ref{sec:over-generation}.
This introduces extra verbal categories, which
in turn introduces extra categories for any constituent that can function
adverbially:

\begin{center}
\deriv{5}{
\rm Seeing & \rm things & \rm clearly & \rm is & \rm important \\
\uline{1}&\uline{1}&\uline{1}&\uline{1}&\uline{1} \\
\cf{NP[ger]/NP} &
\cf{NP[n]} &
\cf{NP[ger]\bs NP[ger]} &
\cf{(S[dcl]\bs NP)/(S[adj]\bs NP)} &
\cf{S[adj]\bs NP} \\
\fapply{2} && \fapply{2} \\
\mc{2}{\cf{NP}} && \mc{2}{\cf{S[dcl]\bs NP}} \\
\bapply{3} \\
\mc{3}{\cf{NP[ger]}} \\
\bapply{5} \\
\mc{5}{\cf{S[dcl]}}
}
\end{center}

The \emph{n} feature assigned to the \cf{NP} \emph{things} records that \emph{things}
is not a gerund nominal, to block \emph{clearly} from attaching to it. As we argue in
Section~\ref{sec:over-generation}, it is difficult to formulate a concise and consistent
representation of constituent types as features, although not impossible.

Neither of the two analyses we have considered restrict the effect of the construction to
the gerund verb. Either the verb remains \cf{S}-typed and the argument structures
of its head must be altered, or the verb receives an \cf{NP}-type, and its modifiers must
change.

\subsection{Reduced Relative Clauses}
\label{sec:ling_rrc}
\begin{figure}
\centering
\deriv{6}{
\rm Ashley & \rm likes & \rm Pat & \rm who & \rm Casey & \rm hates \\
\uline{1}&\uline{1}&\uline{1}&\uline{1}&\uline{1}&\uline{1} \\
\cf{NP} &
\cf{(S[dcl]\bs NP)/NP} &
\cf{NP} &
\cf{(NP\bs NP)/(S[dcl]/NP)} &
\cf{NP} &
\cf{(S[dcl]\bs NP)/NP} \\
&&&& \ftype{1} \\
&&&& \mc{1}{\cf{S/(S\bs NP)}} \\
&&&& \fcomp{2} \\
&&&& \mc{2}{\cf{S[dcl]/NP}} \\
&&& \fapply{3} \\
&&& \mc{3}{\cf{NP\bs NP}} \\
&& \bapply{4} \\
&& \mc{4}{\cf{NP}} \\
& \fapply{5} \\
& \mc{5}{\cf{S[dcl]\bs NP}} \\
\bapply{6} \\
\mc{6}{\cf{S[dcl]}}
}
\caption[Partial associativity provided by type-raising and
composition.]{Interaction of type-raising and composition to produce partial
associativity. This allows the WH-movement to be analysed with the canonical
category assignments.\label{fig:ling_wh_movement}}
\end{figure}

\ccg offers an excellent analysis of extraction mediated by relative pronouns,
as shown in Figure~\ref{fig:ling_wh_movement}. The type-raising and composition
rules allow the extraction phenomenon to be represented entirely in the category
assigned to \emph{who}. Unfortunately, reduced relative clauses pose more of a
problem. The two closely related constructions are:

\begin{lexamples}
\item \textbf{WH-mediated}: \emph{asbestos that was once used for cigarette
filters}\\
\item \textbf{Reduced}: \emph{asbestos once used for cigarette filters}
\end{lexamples}

Without the WH item to coerce the clause into an \cf{NP} modifier, we must
either change the category of the noun, or the category of the verb. An analysis
that relies on changing some other constituent can be dismissed out of hand, as
the modifiers and non-extracted arguments of the verb are irrelevant to the 
construction. We consider each of the viable alternatives in turn.

\subsubsection{Changing the verb category}

Perhaps the most obvious solution is to change the verb category, so that its
category becomes \cf{(NP\bs NP)/PP}:

\begin{center}
\deriv{6}{
\rm asbestos & \rm once & \rm used & \rm for & \rm cigarette & \rm filters \\
\uline{1}&\uline{1}&\uline{1}&\uline{1}&\uline{1}&\uline{1} \\
\cf{NP} &
\cf{(NP\bs NP)/(NP\bs NP)} &
\cf{(NP\bs NP)/PP} &
\cf{PP} &
\cf{NP/NP} &
\cf{NP} \\
&&&& \fapply{2} \\
&&&& \mc{2}{\cf{NP}} \\
&&& \fapply{3} \\
&&& \mc{3}{\cf{PP}} \\
&& \fapply{4} \\
&& \mc{4}{\cf{NP\bs NP}} \\
& \fapply{5} \\
& \mc{5}{\cf{NP\bs NP}} \\
\bapply{6} \\
\mc{6}{\cf{NP}}
}
\end{center}

This analysis is undesirable for the same reasons as the \cf{NP}-rooted
nominalisation analysis described above. It forces an additional, undesirable
category onto its modifiers, and it breaks the assumption that the inner-most
result of a category characterises it in any meaningful way.

\subsubsection{Changing the noun category}

The alternative analysis that involves changing the noun's category was pointed
out to us by Baldridge and Steedman (p.c. 2007):

\begin{center}
\deriv{6}{
\rm asbestos & \rm once & \rm used & \rm for & \rm cigarette & \rm filters \\
\uline{1}&\uline{1}&\uline{1}&\uline{1}&\uline{1}&\uline{1} \\
\cf{NP/(S[pss]\bs NP)} &
\cf{(S\bs NP)/(S\bs NP)} &
\cf{(S[pss]\bs NP)/PP} &
\cf{PP} &
\cf{NP/NP} &
\cf{NP} \\
&&&& \fapply{2} \\
&&&& \mc{2}{\cf{NP}} \\
&&& \fapply{3} \\
&&& \mc{3}{\cf{PP}} \\
&& \fapply{4} \\
&& \mc{4}{\cf{S[pss]\bs NP}} \\
& \fapply{5} \\
& \mc{5}{\cf{S[dcl]\bs NP}} \\
\fapply{6} \\
\mc{6}{\cf{NP}}
}
\end{center}

This analysis allows the verb to keep its canonical category, and the noun's
inner-most result is preserved. Unfortunately, it is a rather unnatural
analysis. The omission of the relativiser does not change the clause from an
adjunct into an argument. The clause still has all the hallmarks of a modifier.
A noun can be modified by multiple relative clauses:

\begin{lexamples}
\item The lawsuit was based on (asbestos (linked to cancer) (used in cigarette
filters))
\end{lexamples}

The relationship between the noun and the relative clause is identical whether
the relative is bare or WH-mediated, suggesting that they are either both
adjuncts, or both arguments. So while this analysis is convenient, and certainly
better than the alternative, it is also unsatisfying.

In summary, both gerund nominals and reduced relative clauses can be assigned \ccg
analyses that correctly represent their semantics. However, the analyses required
are exceptional cases that do not directly resemble related phenomena.
Wide-coverage grammars are sufficiently difficult to write that we do not only require
a formalism technically capable of representing the relevant phenomena. The formalism
ought to be capable of representing \emph{natural} analyses, that meet our intuitions
about how phenomena are related. The analyses \ccg requires for 
form/function discrepancies are not currently satisfying in this sense.


\section{Existing Proposals}
\label{sec:proposals}
A great range of grammatical machinery has been proposed to extend a pure \abcg.
We briefly review three proposals that might mitigate the problems we have discussed.
The first is \citepos{lambek:58} division combinator, also known as the Geach
rule. We explain that this rule cannot be used to address the problems we have
identified. The second are the zero-morphemes of \citet{aone:90}. The third proposal
is \citepos{hock:lrec02} addition of type-changing rules to the grammar.

\subsection{Lambek's Division Combinator}
\label{division}

In the Lambek calculus \citep{lambek:58}, \emph{category division} is a unary
rule equivalent of \ccg's binary function composition. Lambek's rule is different from the
\textbf{D} combinator \citep{curry:58} proposed for use in \ccg by
\citet{hoyt:08}, a proposal which does not affect the issues we are concerned
with. Lambek's rule is:

\begin{equation}
\cf{X/Y} \;\;\Rightarrow\;\; \cf{X\$/Y\$}
\end{equation}

Where \$ is a variable denoting an arbitrary mono-directional argument
structure. The Lambek calculus uses four core rules (application, associativity,
composition, and raising), and Lambek only noted division in an aside,
commenting that it was provable under the system \citep{wood:93}. At first
glance, the rule seems to provide an attractive solution to the need for infinite
categories described in Section~\ref{sec:infinite_categories}:

\begin{center}
\deriv{3}{
\rm water & \rm meter & \rm cover \\
\uline{1}&\uline{1}&\uline{1} \\
\cf{N/N} &
\cf{N/N} &
\cf{N} \\
\division{1} \\
\mc{1}{\cf{(N/N)/(N/N)}} \\
\fapply{2} \\
\mc{2}{\cf{N/N}} \\
\fapply{3} \\
\mc{3}{\cf{N}}
}
\end{center}

The problem, however, is that this does not produce the correct dependencies. This
can be seen by annotating the rule with head indices:

\begin{equation}
\cf{X/Y} \;\;\Rightarrow\;\; \cf{(X_x\$)_x/(Y_y\$)_y}
\end{equation}

To form the correct dependency between \emph{water} and \emph{meter}, the category
assigned to \emph{water} must be indexed \cf{(N_y/N_y)_z/(N_y/N_y)_z}. The division
rule produces a category indexed \cf{(N_y/N_y)_y/(N_y/N_y)_y}. This means that
\emph{water} will form a dependency with \emph{cover}, not \emph{meter} --- just
as if \ccg's standard binary composition were used.

\subsection{Morpheme Categories}

One way to summarise the problems we have identified is that a single \ccg
category has conflicting demands: internal constituents want one category (based
on constituent type), while external constituents want another. \citet{aone:90}
put forward a proposal that partially addresses this, by assigning categories to
morphemes, and even to empty strings which they refer to as \emph{zero}
morphemes. Their proposal reduces category ambiguity by breaking up the
information categories specify into several pieces. \citet{bozsahin:02} 
shows that morphemic categories are essential for analysing a morphologically rich
language such as Turkish with \ccg, and \citet{cha:02} illustrates the same
point for Korean.

Morpheme categories could also be used to perform type-to-function coercions. A
morpheme based analysis of nominal clauses would assign a category to the
morphological suffix, allowing the open class lexical items to receive their
canonical categories:

\begin{center}
\deriv{3}{
\rm See & \rm -ing & \rm things \\
\uline{1}&\uline{1}&\uline{1} \\
\cf{(S[b]\bs NP)/\xmode NP} &
\cf{NP\bs\xmode (S[b]\bs NP)} &
\cf{NP} \\
\bxcomp{2} \\
\mc{2}{\cf{NP/NP}} \\
\fcomp{3} \\
\mc{3}{\cf{NP}}
}
\end{center}

The analysis requires crossed composition from a category rooted in \cf{NP}, so
we describe an analysis using modalised slashes, as described in Section
\ref{sec:mmccg_background}. This analysis closely corresponds to the seemingly
attractive analysis for infinitive nominalisations, which hangs the type-change
onto \emph{to}:

\begin{center}
\deriv{3}{
\rm to & \rm see & \rm things \\
\uline{1}&\uline{1}&\uline{1} \\
\cf{NP/(S[b]\bs NP)} &
\cf{(S[b]\bs NP)/NP} &
\cf{NP} \\
& \fapply{2} \\
& \mc{2}{\cf{S\bs NP}} \\
\fapply{3} \\
\mc{3}{\cf{NP}}
}
\end{center}

Unfortunately, things become difficult when adverbs are introduced:

\begin{lexamples}
\item \gll See -ing things clearly
\cf{(S\bs NP)/NP} \cf{NP\bs (S\bs NP)} \cf{NP} \cf{(S\bs NP)\bs (S\bs NP)}
\gln
\glend
\end{lexamples}

If the adverb \emph{clearly} is assigned its canonical category, it can no
longer apply to \emph{see}, and cannot compose with the \cf{NP}-rooted
\emph{-ing} morpheme. Similar problems occur when an adverb must right-modify an
infinitive (which is of questionable grammaticality in our dialect):

\begin{lexamples}
\item \gll ?~boldly to go
\cf{(S\bs NP)/(S\bs NP)} \cf{NP/(S\bs NP)} \cf{S\bs NP}
\gln
\glend
\end{lexamples}

Similar difficulties apply to zero morphemes. The zero category interferes with
modification, unless it is moved arbitrary distances away from the head --- a
highly unattractive solution. This makes \citepos{aone:90} argument that zero
morphemes can be compiled into the grammar as equivalent unary type-changing rules
problematic. A unary rule is not equivalent to a zero morpheme
lexical category, because unary rules do not have these interactions with the
linear order of the string.

In general, morpheme based categories are a promising concept --- so long as the
morpheme is explicitly realised, and so long as it does not interfere with
modification. When compiled into the grammar as unary type-changing operations,
as \citeauthor{aone:90} suggest, zero morphemes become unary type-changing rules.
We investigate such rules in the following section.

\subsection{Type-Changing Rules}
\label{sec:ling_psg_rules}




\begin{figure}
\begin{center}
\hspace*{-30mm}\scalebox{0.9}{
\ptbegtree
\pthorgap{7pt}
\ptnodefont{\small\rm}{11pt}{2pt}
\ptleaffont{\small\it}{11pt}{2pt}
\ptbeg \ptnode{\cf{S}}
  \ptbeg \ptnode{\cf{S}}
    \ptbeg \ptnode{\cf{NP}} \ptleaf{It} \ptend
    \ptbeg \ptnode{\cf{S[dcl]\bs NP}}
      \ptbeg \ptnode{\cf{(S[dc]\bs NP)/NP}} \ptleaf{is}  \ptend
      \ptbeg \ptnode{\cf{NP}}
        \ptbeg \ptnode{\cf{NP}}
          \ptbeg \ptnode{\cf{NP}} \ptleaf{the fourth time} \ptend
          \ptbeg \ptnode{\cf{NP\bs NP}}
            \ptbeg \ptnode{\cf{(NP\bs NP)/N}} \ptleaf{this} \ptend
            \ptbeg \ptnode{\cf{N}} \ptleaf{week} \ptend
          \ptend
        \ptend
        \ptbeg \ptnode{\cf{NP\bs NP}}
          \ptbeg \ptnode{\cf{S[dcl]}} \ptleaf{it has happened} \ptend
        \ptend
      \ptend
    \ptend
  \ptend
  \ptbeg \ptnode{\cf{S\bs S}}
    \ptbeg \ptnode{\cf{,}} \ptleaf{,}\ptend
    \ptbeg \ptnode{\cf{NP}}
      \ptbeg \ptnode{\cf{NP/NP}} \ptleaf{almost}\ptend
      \ptbeg \ptnode{\cf{NP}}
        \ptbeg \ptnode{\cf{NP}}
          \ptbeg \ptnode{\cf{NP/N}} \ptleaf{a}\ptend
          \ptbeg \ptnode{\cf{N}}\ptleaf{way}\ptend
        \ptend
        \ptbeg \ptnode{\cf{NP\bs NP}}
          \ptbeg \ptnode{\cf{(NP\bs NP)/NP}} \ptleaf{of}\ptend
          \ptbeg \ptnode{\cf{NP}}
            \ptbeg \ptnode{\cf{N}} \ptleaf{life}\ptend
          \ptend
        \ptend
      \ptend
    \ptend
  \ptend
\ptend
\ptendtree
}
\end{center}
\caption{\ccgbank derivation showing type-changing rules.}\label{full_sentence_ccgbank}
\end{figure}

\begin{figure}
\centering
\deriv{4}{
\rm John & \rm Paul & \rm Mary & \rm loves \\
\uline{1}&\uline{1}&\uline{1}&\uline{1} \\
\cf{NP} &
\cf{NP} &
\cf{NP} &
\cf{(S[dcl]\bs NP)/NP} \\
& \psgrule{1} & \ftype{1} \\
& \mc{1}{\cf{S/(S/NP)}} & \mc{1}{\cf{S/(S\bs NP)}} \\
&& \fcomp{2} \\
&& \mc{2}{\cf{S[dcl]/NP}} \\
& \fapply{3} \\
& \mc{3}{\cf{S[dcl]}} \\
& \psgrule{3} \\
& \mc{3}{\cf{NP\bs NP}} \\
\bapply{4} \\
\mc{4}{\cf{NP}}
}
\caption{Over-generation by \ccgbank rules.\label{fig:twisted_love}}
\end{figure}

\citet{hock:lrec02} includes a brief discussion of the modifier category
proliferation problem, and context-free type-changing rules to
address the situation. Appendix~\ref{appendix:type-changing} provides a list of
all unary and binary type-changing rules that occur more than 10 times in Sections 02-21
of \ccgbank. There are 204 type-changing rules in the training partition of
\ccgbank. 53 of the frequent rules produce modifier categories, 48 of which
transform verbal categories. The rules also handle a variety of other
constructions, such as form/function discrepancies like gerund nominals. By far
the most frequent rule (115,333 occurrences) is \psunary{\cf{N}}{\cf{NP}}, which
transforms bare nominals into noun phrases.

Figure~\ref{full_sentence_ccgbank} shows two such rules. The
\psunary{\cf{S[dcl]}}{\cf{NP\bs NP}} rule allows the reduced relative clause,
\emph{it has happened}, to function as a modifier while all words are assigned
their canonical categories. The other type-changing rule in the derivation,
\psbinary{\cf{,}}{\cf{NP}}{\cf{S\bs S}}, enables the extraposition of
\emph{almost a way of life}. This rule illustrates how type-changing rules prevent the
modifier category proliferation problem. The modifier \emph{almost} receives the
form-based category \cf{NP/NP}. The function of its head is factored away from
the modifier's category.

The rule is binary so that the punctuation can make the rule more precise.
Precision is important for type-changing rules, because they come at the cost of
over-generation. The type-changing rules introduced by \citet{hock:cl07} are
context-free, which makes it difficult to prevent this problem.
In the context of a statistical system, some over-generation is permissible, as
the statistical model should be able to filter out analyses unrepresented in the training
data. Over-generation is still undesirable, however, as the model will never be perfectly
accurate. This means that the type-changing rules must be
chosen judiciously, lest the solution become more costly than the initial
problem. 

Figure~\ref{fig:twisted_love} shows an example of over-generation caused by the
unary rule \psunary{\cf{NP}}{\cf{S/(S/NP)}}
\footnote{Note that this \ccgbank version of the topicalisation rule differs slightly
from \citepos{steedman:00} rule, which is \cf{S[t]/(S/NP)}. That is, it includes
a feature to mark topicalisation and constrain over-generation. This seems problematic,
as it blocks coordination such as \emph{I love Robin, but Pat I hate}, which we
suggest is grammatical.}, which is used to handle object
extraposition, as in the following:

\begin{center}
\deriv{7}{
\rm Robin & \rm I & \rm love~, & \rm but & \rm Pat & \rm I & \rm hate \\
\uline{1}&\uline{1}&\uline{1}&\uline{1}&\uline{1}&\uline{1}&\uline{1} \\
\cf{NP} &
\cf{NP} &
\cf{(S[dcl]\bs NP)/NP} &
\cf{(X\bs X)/X} &
\cf{NP} &
\cf{NP} &
\cf{(S[dcl]\bs NP)/NP} \\
\psgrule{1} & \ftype{1} &&& \psg{1} & \ftype{1} \\
\mc{1}{\cf{S/(S/NP)}} & \mc{1}{\cf{S/(S\bs NP)}} &&& \mc{1}{\cf{S/(S/NP)}} & \mc{1}{\cf{S/(S\bs NP)}} \\
& \fcomp{2} &&& \fcomp{2} \\
& \mc{2}{\cf{S[dcl]/NP}} &&& \mc{2}{\cf{S[dcl]/NP}} \\
\fapply{3} && \fapply{3} \\
\mc{3}{\cf{S[dcl]}} && \mc{3}{\cf{S[dcl]}} \\
&&& \fapply{4} \\
&&& \mc{4}{\cf{S[dcl]\bs S[dcl]}} \\
\bapply{7} \\
\mc{7}{\cf{S[dcl]}}
}
\end{center}

Adding this rule to the grammar without any restrictions allows local
scrambling, undoing all of the careful work to restrict the combinatory rules in
\citet{steedman:00}, and making \citepos{baldridge:03} replacement of them a
moot point. Many of the type-changing rules that are rare in \ccgbank offer similarly
unattractive trade-offs between descriptive power and over-generation. The
\psunary{\cf{S[dcl]}}{\cf{NP\bs NP}} rule is another example of this.

The ambiguity problem prevents type-changing rules from being a practical way to
consistently represent constituent type in the corpus. For instance,
prepositional phrases receive modifier categories directly, because unary rules
transforming prepositional phrases into nominal or verbal modifiers would be
very problematic. One problem with the addition of a rule such as
\psunary{\cf{PP}}{\cf{NP\bs NP}} to the grammar is that prepositional phrase
attachment would no longer be lexically specified. A word would receive the
category \cf{PP/NP}, and after applying its argument, it could then function as
an argument, nominal modifier, or verbal modifier. This loss of lexicalisation
is undesirable.

Perhaps the biggest problem with a type-changing rule like \psunary{\cf{PP}}{\cf{NP\bs
NP}} is the burden it places on type-changing rules to carry out non-trivial
semantic operations. All of the combinatory rules
perform very simple semantic operations, according to the following principle
stated by \citet[][p. 37]{steedman:00}:

\begin{combinatorytransparency}
 All syntactic combinatory rules are type-transparent versions of one of a small
 number of simple semantic operations over functions.
\end{combinatorytransparency}

Type-changing rules perform arbitrary category transformations, so can introduce
arbitrary semantic operations. The only way to avoid this is to pair a \cf{PP} category
with a logical form that takes into account its transformation to \cf{NP\bs NP}.
But since the syntactic category does not guarantee the transformation, this
introduces semantic ambiguities that are difficult to resolve, and may lead to valid
syntactic derivations that yield unpredictable semantic analyses.

For practical systems based on \ccgbank, type-changing operations have not presented
a significant obstacle to semantic transparency. Both \citet{bos:coling04} and \citet{white:03}
have produced logical forms from \ccgbank derivations, by pairing each type-changing
rule with a semantic operation. It is unsurprising that there should exist some engineering
solution for a small set of exceptional cases. This does not, however, suggest that
exceptional cases are a desirable addition to a linguistic theory.

Finally, type-changing rules also destroy the explanatory power of \ccg as a
model of the human language processor. One of the appeals of a lexicalised
grammar in this respect is that it makes a strong claim about exactly which part
of the human language faculty is innate (the grammar), and which part is
acquired (the lexicon). If the grammar is innate, it must be language universal,
confining all language specific variation to the lexicon. The type-changing
rules contradict this hypothesis, just when it seemed that the last language
specific exceptions to the grammar had been removed by \citepos{baldridge:03}
addition of resource sensitivity to the lexicon. We argue that although the
problem that type-changing rules address is real, we should seek a solution
that does not involve abandoning the central hypotheses of the \ccg theory.

\section{Summary}

A good way to introduce someone familiar with grammar to \ccg in a few minutes
is to tell them that in \ccg, a transitive verb is just a sentence missing two
arguments. Similarly, there is no such thing as a preposition or an adverb ---
there are just functions from, say, a noun phrase argument to a noun phrase
modifier. In this chapter, we have argued that this interesting design actually
poses some problems for the theory, because there really \emph{is} such a thing
as a prepositional phrase or an adverb in the language, and a grammatical theory
that cannot account for them loses economy.

Constituent type is not a semantically relevant property, so it is not strictly
necessary for a formalism that seeks to map surface forms directly to semantic
structures. However, constituent type is crucially relevant
\emph{syntactically}. This makes it essential for parsimonious linguistic
description. It is possible for \ccg grammars to account for constituent type,
but we have argued that the formalism does little to encourage generalisations
along this dimension. This has motivated the inclusion of type-changing rules
to the formalism to simplify linguistic description. We argue that unlexicalised
type-changing operations run contrary to the design principles of the theory,
and introduce over-generation that discourages the grammar writer from including
a consistent representation of constituent type.
In the following chapter, we suggest a way to unambiguously lexicalise type-changing
operations, using hat categories.
% 
  
\chapter{Hat Categories}
\label{chapter:hat_cats} 

This chapter describes hat \ccg: a Combinatory
Categorial Grammar minimally extended to include \emph{hat categories}.
Hat categories are a mechanism we have designed to allow \ccg
categories to represent form and function simultaneously to
address the problems described in Chapter~\ref{chapter:ling_mot}.

A hat category is a category of the form \cf{X^Y\$} that lexicalises a
unary production \psunary{\cf{X}}{\cf{Y}}. Hat categories are used so that the
\cf{X} category, termed the \emph{base}, represents the constituent's form; and
the \cf{Y} category, termed the \emph{hat}, represents the constituent's
function.

For example, the hat category \cf{(S[ng]\bs NP)^{N\bs N}/PP} has a base,
\cf{S[ng]\bs NP}, and a hat, \cf{N\bs N}. Once the outer argument has been
supplied, the unhat rule, \cH, transforms the base into the hat:

\begin{center}
\deriv{3}{
\rm people & \rm giving & \rm to~charities \\
\uline{1}&\uline{1}&\uline{1} \\
\cf{N} &
\cf{(S[ng]\bs NP)^{N\bs N}/PP} &
\cf{PP} \\
& \fapply{2} \\
& \mc{2}{\cf{(S[ng]\bs NP)^{N\bs N}}} \\
& \unhat{2} \\
& \mc{2}{\cf{N\bs N}} \\
\bapply{3} \\
\mc{3}{\cf{N}}
}
\end{center}

The hat category assigned to \emph{giving} should be read as a version of
its canonical category, \cf{(S[ng]\bs NP)/PP}, that specifies its function
as a \cf{N\bs N}. The resemblance to its canonical category is critical in
ensuring that modifiers can receive their canonical category. In the derivation
below, the modifier \emph{generously} is assigned a category that is not
sensitive to the function of \emph{giving}:

\begin{center}
\deriv{4}{
\rm people & \rm giving & \rm generously & \rm to~charities \\
\uline{1}&\uline{1}&\uline{1}&\uline{1} \\
\cf{N} &
\cf{(S[ng]\bs NP)^{N\bs N}/PP} &
\cf{(S\bs NP)\bs (S\bs NP)} &
\cf{PP} \\
& \bxcomp{2} \\
& \mc{2}{\cf{(S[ng]\bs NP)^{N\bs N}/PP}} \\
& \fapply{3} \\
& \mc{3}{\cf{(S[ng]\bs NP)^{N\bs N}}} \\
& \unhat{3} \\
& \mc{3}{\cf{N\bs N}} \\
\bapply{4} \\
\mc{4}{\cf{N}}
}\end{center}

This is the key property of hat categories: they allow the description of 
form/function discrepancies to be isolated within a single category. The other
categories in the derivation are unchanged.

Hat categories necessitate two changes to the formalism: the addition
of a \hatsc attribute to the category objects, and the addition of an 
\textsc{unhat} rule, \cf{X^Y}$\Rightarrow{_\cH}$ \cf{Y},
to perform the unary production. A set of stipulations constraining the permissible
values of the \hatsc attribute and the structure of \emph{base} categories
together guarantee that all hat categories are unhatted during valid derivations.
The \cH~rule can only be used if the base has no outer arguments. In the
following, the bare nominals \emph{people} and \emph{charities}
can be unhatted, but \emph{giving} cannot:

\begin{center}
\deriv{5}{
\rm people & \rm giving & \rm generously & \rm to & \rm charities \\
\uline{1}&\uline{1}&\uline{1}&\uline{1}&\uline{1} \\
\cf{N^{NP}} &
\cf{(S[ng]\bs NP)^{N\bs N}/PP} &
\cf{(S\bs NP)\bs (S\bs NP)} &
\cf{PP/NP} &
\cf{N^{NP}} \\
\unhat{1} & \asterisk{1} &&& \unhat{1} \\
\mc{1}{\cf{NP}} & \mc{1}{\cf{(N\bs N)/PP}} &&& \mc{1}{\cf{NP}}
}\end{center}

In order to ensure that the hat represents the category's function and that it is not
optional, arguments within the base category cannot be used. This prevents the derivation
of \cf{S[ng]} in the following:

\begin{center}
\deriv{5}{
\rm people & \rm giving & \rm generously & \rm to & \rm charities \\
\uline{1}&\uline{1}&\uline{1}&\uline{1}&\uline{1} \\
\cf{N^{NP}} &
\cf{(S[ng]\bs NP)^{N\bs N}/PP} &
\cf{(S\bs NP)\bs (S\bs NP)} &
\cf{PP/NP} &
\cf{N^{NP}} \\
\unhat{1} & \bxcomp{2} && \unhat{1} \\
\mc{1}{\cf{NP}} & \mc{2}{\cf{(S[ng]\bs NP)^{N\bs N}/PP}} && \mc{1}{\cf{NP}} \\
&&& \fapply{2} \\
&&& \mc{2}{\cf{PP}} \\
& \fapply{4} \\
& \mc{4}{\cf{(S[ng]\bs NP)^{N\bs N}}} \\
\asterisk{5} \\
\mc{5}{\cf{S[ng]}}
}\end{center}

A further stipulation ensures that the base category cannot be used as the
argument of non-modifier categories. This prevents the derivation of
\cf{S[dcl]} in the following:

\begin{center}
\deriv{6}{
\rm people & \rm are & \rm giving & \rm generously & \rm to & \rm charities \\
\uline{1}&\uline{1}&\uline{1}&\uline{1}&\uline{1}&\uline{1} \\
\cf{N^{NP}} &
\cf{(S[dcl]\bs NP)/(S[ng]\bs NP)} &
\cf{(S[ng]\bs NP)^{N\bs N}/PP} &
\cf{(S\bs NP)\bs (S\bs NP)} &
\cf{PP/NP} &
\cf{N^{NP}} \\
\unhat{1} && \bxcomp{2} && \unhat{1} \\
\mc{1}{\cf{NP}} && \mc{2}{\cf{(S[ng]\bs NP)^{N\bs N}/PP}} && \mc{1}{\cf{NP}} \\
&&&& \fapply{2} \\
&&&& \mc{2}{\cf{PP}} \\
&& \fapply{4} \\
&& \mc{4}{\cf{(S[ng]\bs NP)^{N\bs N}}} \\
& \asterisk{5} \\
& \mc{5}{\cf{S[dcl]\bs NP}}
}
\end{center}

Hat categories are defined such that the unhat rule must be applied to every hat in
order to form a valid derivation.
A hat category
\cf{(S[ng]\bs NP)^{N\bs N}} \emph{must} be unhatted and function as \cf{N\bs N}.
It cannot function as \cf{S[ng]\bs NP}, either as a functor or an argument.
A valid derivation using the \cf{(S[ng]\bs
NP)^{N\bs N}/PP} category is shown in Figure~\ref{fig:giving_rrc}.

The chapter is structured as follows. We first describe in Section
\ref{sec:hat_def} the modifications to the formalism we require to support
hat categories. In Section
\ref{sec:stipulations} we provide a way to guarantee that every hat category is
unhatted during a derivation. This is achieved with a set of stipulations on the
use of the \hatsc attribute and the structure of base categories.
We then describe the logical forms of hat categories, in
Section~\ref{sec:hat_semantics}.
Section~\ref{sec:interaction} describes how hat categories interact with the
combinators. Finally, Section~\ref{sec:hat_gp} argues that hat categories do
not alter the weak generative power of \ccg, and sketches an informal proof to that
effect. A full discussion of the
analyses hat categories enable is deferred to
Chapter~\ref{chapter:hat_corpus}, where we compare hat category analyses with
\ccgbank analyses and the analyses of a \ccg grammar that does not contain any
type-changing operations.

\begin{figure}
\small
\centering
\deriv{7}{
\rm people & \rm giving & \rm generously & \rm to & \rm charities & \rm are &
\rm happier \\
\uline{1}&\uline{1}&\uline{1}&\uline{1}&\uline{1}&\uline{1}&\uline{1} \\
\cf{N^{NP}} &
\cf{(S[ng]\bs NP)^{N\bs N}/PP} &
\cf{(S\bs NP)\bs (S\bs NP)} &
\cf{PP/NP} &
\cf{N^{NP}} &
\cf{(S[dcl]\bs NP)/(S[adj]\bs NP)} &
\cf{S[adj]\bs NP} \\
& \bxcomp{2} && \unhat{1} & \fapply{2} \\
& \mc{2}{\cf{(S[ng]\bs NP)^{N\bs N}/PP}} && \mc{1}{\cf{NP}} &
\mc{2}{\cf{S[dcl]\bs NP}} \\
&&& \fapply{2} \\
&&& \mc{2}{\cf{PP}} \\
& \fapply{4} \\
& \mc{4}{\cf{(S[ng]\bs NP)^{N\bs N}}} \\
& \unhat{4} \\
& \mc{4}{\cf{N\bs N}} \\
\bapply{5} \\
\mc{5}{\cf{N^{NP}}} \\
\unhat{5} \\
\mc{5}{\cf{NP}} \\
\bapply{7} \\
\mc{7}{\cf{S[dcl]}}
}
\caption[Derivation with hat category]{A valid derivation using the hat category
\cf{(S[ng]\bs NP)^{N\bs N}/PP} to analyse a reduced relative clause.
\label{fig:giving_rrc}}
\end{figure}




\section{Definition of Hat Combinatory Categorial Grammar}
\label{sec:hat_def}

\begin{figure}
\centering
\begin{avm}
[{}  \ressc  & [{} \ressc  & [{} \ressc  & \cf{S}\\
                                 \featsc & [{} \textsc{tense} & \emph{ng}]\\
			         \headsc & giving\\
                                 \hatsc  & $\nullhat$\\
		             ]\\
                    \slashsc& \bks$^!$ \\
                    \argsc & [{} \ressc  & \cf{NP}\\
                                 \headsc & $y$\\
                                 \hatsc  & $\nullhat$\\
		             ]\\
                    \headsc & giving\\
                    \hatsc & [{} \ressc & [{} \ressc  & \cf{N}\\
                                          \headsc & $y$\\
                                          \hatsc  & $w$\\
                                      ]\\
                                 \slashsc & \cf{\bs}\\
                                 \argsc & [{} \ressc  & \cf{N}\\
                                          \headsc & $y$\\
                                          \hatsc  & $w$\\
                                      ]\\
                                 \headsc & giving \\
                                 \hatsc  & $\nullhat$\\
                         ]\\
	    ]\\
     \slashsc& \cf{/} \\
     \argsc  & [{} \ressc  & \cf{NP} \\
		   \headsc & $z$\\
                   \hatsc  & $\nullhat$\\
	       ]\\
     \headsc & giving\\
     \hatsc  & $\nullhat$\\
]
\end{avm}
\caption[Attribute value matrix representation of a hat category.]{Attribute
value matrix representation of \emph{giving} $\assign$
\cf{(S[ng]\bs^! NP_y^\nullhat)^{N^w_y\bs N^w_y}/PP^\nullhat_z}.\\
The category's attributes are partially determined by two stipulations that help ensure
the unhat rule is always used by hat categories.
The \hatsc attributes of non-modifier arguments take on the null hat value $\nullhat$,
 as specified by the Null Hat Stipulation described in Section~\ref{sec:null_hats}.
 The \slashsc of \cf{S[ng]\bs^! NP} , as specified by
 the Inert Slash Stipulation described in Section~\ref{sec:null_mode}.}
\label{fig:hat_avm}
\end{figure}

This section describes the formal mechanisms required to support hat categories
within a multi-modal Combinatory Categorial Grammar such as the one described by 
\citet{steedman:pedia}. Hat categories could be added to other categorial grammars,
such as an \abcg, or something similar to the hat categories we describe might be
implemented in a categorial type logic using unary modalities or conjunctive
categories, in the sense of \citet{carpenter:98}. However, we confine
our discussion to adding
hat categories to an application, type-raising and composition Combinatory Categorial
Grammar that makes use of multi-modal slash types.

\subsection{Hat Category Definition}

A multi-modal \ccg category is defined as a 
5-tuple $\langle\ressc, \featsc, \slashsc, \argsc, \headsc\rangle$, where:
\begin{itemize}
 \item \ressc is a result category;
 \item \featsc is a feature structure;
 \item \slashsc is a modalised slash;
 \item \argsc is an argument category;
 \item \headsc is a lexical head.
\end{itemize}

Attribute values can be \emph{coindexed} to each other across a category. When
an attribute takes on a value, all attributes coindexed with it share that value. 

We extend this definition to include an additional attribute, \hatsc. 
Figure~\ref{fig:hat_avm} shows an attribute value matrix representation of
\emph{giving} $\assign$ \cf{(S[ng]\bs NP)^{N\bs N}/PP}. Attributes with empty values
have been omitted for brevity.

The \hatsc attribute of the base, \cf{S[ng]\bs NP}, stores the hat, \cf{N\bs N}.
Attributes within the hat category can be coindexed with attributes outside it.
When an attribute value is bound during a derivation, all coindexed attributes
take on the new value. During the derivation in Figure
\ref{fig:giving_rrc}, $y$ would be bound to \emph{people}, $z$ would be bound to
\emph{to}, and $w$ would be bound to \cf{NP}, to produce \cf{(S[ng]_{giving}\bs
NP_{people})^{N^{NP}_{people}\bs N^{NP}_{people}}/PP_{to}}.

Categories might also be defined with a single identity index, controlling
the identity of heads, hats, slashes and possibly features. We have chosen
the definition that allows grammar writers greater flexibility, even though
it makes the well-formedness stipulations described in Section~\ref{sec:stipulations} more
complicated. The implementation described in Chapter~\ref{chapter:results} uses the
\headsc index to determine hat coindexation, with a separate feature index.

\subsection{Unhat rule definition}

We add a single unary rule to the grammar to support hat categories, unhat:

\begin{eqnarray}
\cf{X^Y} & \Rightarrow_\cH\;\; \cf{Y}
\end{eqnarray}

The rule is an entirely syntactic operation. It has no impact on the category's
logical form, as described in Section~\ref{sec:hat_semantics}


\section{Stipulations to make hat categories non-disjunctive}
\label{sec:stipulations}

Hat categories have been designed to represent form/function discrepancies in \ccg.
They do this by allowing a form category to be specified in the base, and a function
category to be specified in the hat. This section describes how we ensure that the
constituent cannot instead function according to its base category, rather than its
hat. We do this by ensuring that every hat category must be unhatted in every derivation.

The hat category \cf{(S[ng]\bs NP)^{N\bs N}/PP} describes a verb with a \cf{PP} complement
heading a reduced relative clause. This construction represents a form/function discrepancy:
a verb phrase that functions as a nominal modifier. Because we want the grammar to be
fully lexicalised, we want the lexical category to describe exactly this construction,
and no others. This means that a verb assigned this category must not function as an
ordinary verb phrase. It must, at some point, be transformed into a nominal modifier.

To ensure that hat categories preserve full lexicalisation, we must design the grammar
so that the unhat rule always applies to every hat category. We state this principle as
follows:

\begin{hatpreservation}
 Hats may only be eliminated using the unhat rule. All other combinators must
preserve them.
\end{hatpreservation}

One way to make the grammar obey this principle would be to formulate a constraint
over derivations. However, this would mean modifying the existing combinators to 
add a post-constraint, checking whether the combinator preserved the hat category.
Instead, we propose a solution that places the constraints in the lexicon, using
stipulations that constrain what can be considered a well-formed category. In an
inheritance based lexicon, such as proposed by \citet{mcconville:06}, this would mean
stating the stipulations once at the top level of the category hierarchy. The decision
to place the stipulations in the lexicon follows the general principles described by
\citet{baldridge:03}, who argue that it is preferable to follow the approach to
resource-sensitivity (control over when combinators can apply) taken in Categorial
Type Logics, by placing rules and restrictions in the lexicon.

\subsection{Null Hat Stipulation}
\label{sec:null_hats}

The Null Hat Stipulation prevents hats from being eliminated through
argument application by ensuring that the arguments of predicate categories fail
to unify with hatted categories. It is stated as follows:

\begin{nullhatstip}
\hatsc attributes may not be empty. Legal values are a category,
coindexation between an argument and its result, or the null value $\nullhat$.
\end{nullhatstip}

A predicate category, such as \cf{(S[dcl]\bs NP)/(S[ng]\bs NP)},
is prevented from applying the hatted category \cf{(S[ng]\bs NP)^{N\bs N}} because
its argument's \hatsc is null valued. Null hat values successfully unify against each other,
allowing application of unhatted arguments as usual.
We will leave null hats implicit in our
derivations in future, but the following derivation shows all occurring null hats,
along with an application rule blocked by the null hat on the argument of 
\cf{(S[dcl]\bs NP)/(S[ng]\bs NP)}:

\begin{center}
\deriv{4}{
\rm Pat & \rm is & \rm collecting & \rm stamps \\
\uline{1}&\uline{1}&\uline{1}&\uline{1} \\
\cf{NP^\nullhat} &
\cf{((S[dcl]^\nullhat\bs NP^\nullhat)/(S[ng]^\nullhat\bs NP^\nullhat)^\nullhat)^\nullhat} &
\cf{((S[ng]^\nullhat\bs NP^\nullhat)^{(N^y\bs N^y)^\nullhat}/NP^\nullhat)^\nullhat} &
\cf{NP^\nullhat} \\
&& \fapply{2} \\
&& \mc{2}{\cf{(S[ng]^\nullhat\bs NP^\nullhat)^{(N^y\bs N^y)^\nullhat}}} \\
& \asterisk{3} \\
& \mc{3}{\cf{(S[dcl]^\nullhat\bs NP^\nullhat)^\nullhat}} \\
}
\end{center}

The null value prevents the \cf{(S[ng]\bs NP)^\nullhat} and \cf{(S[ng]\bs NP)^{N\bs N}}
categories from being unified, as their \hatsc values are incompatible.

The stipulation
allows hatted categories to be applied as arguments only when the hat would be preserved
through coindexation. For instance, the modifier category \cf{(S^y\bs NP^z)^w\bs (S^y\bs NP^z)^w}
preserves hats, due to the coindexation of its argument and result \hatsc values. The
following derivation makes hat coindexation explicit, so that all unspecified hats
should be assumed to be null:

\begin{center}
\deriv{4}{
\rm Pat & \rm tripped & \rm walking & \rm quickly \\
\uline{1}&\uline{1}&\uline{1}&\uline{1} \\
\cf{NP} &
\cf{S[dcl]\bs NP} &
\cf{(S[ng]\bs NP)^{(S^y\bs NP^z)^w\bs (S^y\bs NP^z)^w}} &
\cf{(S^y\bs NP^z)^w\bs (S^y\bs NP^z)^w} \\
&& \bapply{2} \\
&& \mc{2}{\cf{(S[ng]\bs NP)^{(S^y\bs NP^z)^w\bs (S^y\bs NP^z)^w}}} \\
&& \unhat{2} \\
&& \mc{2}{\cf{(S^y\bs NP^z)^w\bs (S^y\bs NP^z)^w}} \\
& \bapply{3} \\
& \mc{3}{\cf{S[dcl]\bs NP}}\\
\bapply{4}\\
\mc{4}{\cf{S[dcl]}}
}
\end{center}

In future, we will not always make hat indices explicit for modifier categories.
The argument of the modifier category \cf{(S^y\bs NP^z)^w\bs (S^y\bs NP^z)^w} unifies
with the hat category \cf{(S[ng]\bs NP)^{(S^y\bs NP^z)^w\bs (S^y\bs NP^z)^w}} during backward
application. The result category's attributes take its values from the argument
during unification, due to their coindexation.

The desirable property of the hat category analysis is that the form/function
discrepancy is entirely confined to the category assigned to \emph{walking}. All
other words in the derivation receive their canonical categories, without reference
to the special function of \emph{walking}. At the same time, the function of
\emph{walking} is fully specified. The category it is assigned does not allow it
to function as an ordinary \emph{-ing} inflected verb. The Null Hat Stipulation
goes part of the way to guaranteeing that, by ensuring that its hat category
is preserved when functors apply it as an argument.

To avoid cluttering our derivations, we will assume from here that the arguments
of predicate categories will receive null hats, and modifier categories will
recursively coindex the \hatsc values of their results and arguments. Null hat values
and hat indices will only be shown in our derivations when there is a particular need
for the more explicit presentation.


\subsection{Inert Slash Stipulation}
\label{sec:null_mode}

The Inert Slash Stipulation prevents complex categories that specify a \hatsc
value from applying their arguments, as this would not preserve their hats.
It does this by using the \emph{inert slash} described in Section \ref{sec:mmccg_background},
first defined by \citet{baldridge:02} and used by \citet{hoyt:08}.
The stipulation is stated as follows:

\begin{nullmodestip}
 If a complex category specifies a hat category, its \slashsc must be inert.
Combinators cannot use inert slashes, marked \cf{/^!} and \cf{\bs^!},
as their functors.
\end{nullmodestip}

Complex categories are allowed to specify hats so that categories such as 
\cf{(S[ng]\bs NP)^{N\bs N}/PP} can be formed. The \cf{NP} argument within
the base is specified to relate the category to its canonical version,
\cf{(S[ng]\bs NP)/PP}. This allows its modifier categories to receive their canonical
categories. However, we must prevent the \cf{NP} argument from being used,
lest the \cf{(S[ng]\bs NP)^{N\bs N}/PP} category be used to derive \cf{S[ng]}:

\begin{center}
\deriv{5}{
\rm people & \rm giving & \rm generously & \rm to & \rm charities \\
\uline{1}&\uline{1}&\uline{1}&\uline{1}&\uline{1} \\
\cf{N^{NP}} &
\cf{(S[ng]\bs^! NP)^{N\bs N}/PP} &
\cf{(S\bs_i NP)\bs (S\bs_i NP)} &
\cf{PP/NP} &
\cf{N^{NP}} \\
\unhat{1} & \bxcomp{2} && \unhat{1} \\
\mc{1}{\cf{NP}} & \mc{2}{\cf{(S[ng]\bs^! NP)^{N\bs N}/PP}} &&
\mc{1}{\cf{NP}} \\
&&& \fapply{2} \\
&&& \mc{2}{\cf{PP}} \\
& \fapply{4} \\
& \mc{4}{\cf{(S[ng]\bs^! NP)^{N\bs N}}} \\
\asterisk{5} \\
\mc{5}{\cf{S[ng]}}
}\end{center}

Backward application is blocked here because the slash on its functor, \cf{S[ng]\bs^! NP},
is inert. Note that type-raising cannot be used to circumvent the inert slash, for the
reason described in Section \ref{sec:mmccg_background}. The two slashes in a type-raised
category \cf{T_i/(T\bs_i X)} or \cf{T_i\bs (T/_i X)} are coindexed, so they must either
both be inert, or both be active. If they are both inert, then the type-raised category
cannot serve as a functor; if they are both active, then the \cf{T\bs^+ X} category
will fail to unify with the \cf{(T\bs^! X)^Y} hatted category, where \cf{\bs^+} denotes
an active slash.

Inert slashes can still be used as the argument by composition combinators.
This is important for allowing modification:

\begin{center}
\deriv{5}{
\rm people & \rm giving & \rm generously & \rm to & \rm charities \\
\uline{1}&\uline{1}&\uline{1}&\uline{1}&\uline{1} \\
\cf{N^{NP}} &
\cf{(S[ng]\bs^! NP)^{N\bs N}/PP} &
\cf{(S\bs^i NP)\bs (S\bs^i NP)} &
\cf{PP/NP} &
\cf{N^{NP}} \\
& \bxcomp{2} && \unhat{1} \\
& \mc{2}{\cf{(S[ng]\bs^! NP)^{N\bs N}/PP}} && \mc{1}{\cf{NP}} \\
&&& \fapply{2} \\
&&& \mc{2}{\cf{PP}} \\
& \fapply{4} \\
& \mc{4}{\cf{(S[ng]\bs^! NP)^{N\bs N}}} \\
& \unhat{4} \\
& \mc{4}{\cf{{N\bs N}}} \\
\bapply{5}\\
\mc{5}{\cf{N^{NP}}}\\
\unhat{5}\\
\mc{5}{\cf{NP}}
}\end{center}

The inert slash is allowed to be the argument in the backwards crossed composition, which
prevents the modifier \emph{generously} from requiring an argument-structure specific
category such as \cf{((S\bs NP)/PP) \bs ((S\bs NP)/PP)}.

Modification raises another subtlety. Modifiers of complex categories could potentially
circumvent our stipulations, by coindexing their \textsc{hat} values but not coindexing
their \slashsc values. A poorly formulated modifier category such as
\cf{(S\bs^i NP)^y\bs (S\bs^j NP)^y} both of our current stipulations, but would allow
\emph{giving} to shed its inert slash, and allow the derivation of \cf{S[ng]} that
we are trying to prevent. We prevent such ill-formed modifier categories
with the following auxiliary stipulation:

\begin{slashindexstip}
 If two complex categories' \hatsc values are coindexed, their \slashsc values
must be coindexed also.
\end{slashindexstip}

This stipulation ensures that a modifier category that obeys the Null Hat
Stipulation by coindexing its \hatsc values must also coindex its \slashsc
values, ensuring that inert slashes cannot be lost during the derivation

\section{Logical form of hat categories}
\label{sec:hat_semantics}

The logical forms of hat categories can be represented with the
Hybrid Logic Dependency Semantics \citep[\hlds, ][]{kruijff:01} described in Section
\ref{sec:hlds_background}. In some cases, hat categories might be paired
with exactly the same logical form as the unhatted equivalent:

\begin{tabular}{lcccc}
 people & $\assign$ & \cf{N}      & $:$  & $@_p \semf{person} \wedge @_p\sematt{num}\semf{pl}$\\
 people & $\assign$ & \cf{N^{NP}}  & $:$ & $@_p \semf{person} \wedge @_p\sematt{num}\semf{pl}$\\
 giving & $\assign$ & \cf{S[ng]\bs NP_y}  & $:$ & $@_x \semf{give}
                                            \wedge @_x \sematt{tense}\semf{ng}
                                            \wedge @_x \sematt{act}y$ \\
 giving & $\assign$ & \cf{(S[ng]\bs NP_y)^{NP}}  & $:$ & $@_x \semf{give}
                                                   \wedge @_x \sematt{tense}\semf{ng}
                                                   \wedge @_x \sematt{act}y$ \\
\end{tabular}

A grammar writer might choose to represent the fact that \emph{people} is a bare nominal or
\emph{giving} is a gerund in their hat categories' semantics, but they need not do
so if the distinction is not germane to their application. The variable assigned to
$\sematt{act}$ will go unfilled during the derivation, leaving the relation underspecified
in the semantics.

In the case of reduced relatives, \hlds allows the dual relationship between the verb and the
noun it modifies to be represented. A reduced relative verb modifies a noun that is also
its subject. The logical form of \emph{giving} as a
reduced subject relative is given below, with the logical form of a subject relativiser
given for comparison:

\begin{tabular}{lcccl}
seeing & $\assign$ & \cf{(S[ng]\bs NP_y)^{N_y\bs N_y}/NP_z} & $:$ &  $@_x \semf{see}
                                                               \wedge @_x \sematt{tense}\semf{ng}
                                                               \wedge @_x \sematt{act}y$\\
       &           &                                        &      &
                                                               $\wedge @_y \sematt{genrel}x
                                                               \wedge @_x \sematt{pat}z$\\
who    & $\assign$ & \cf{(N_y\bs N_y)/(S[dcl]_x\bs NP_y)} & $:$ &    $@_y \sematt{genrel}x$\\
\end{tabular}

The relativiser \emph{who} coindexes its noun argument to the missing subject of its verb
phrase argument. When it picks up the noun argument, that variable will be bound, allowing
the relation the verb has bound to that argument to be recovered:

\begin{center}
\deriv{4}{
\rm man & \rm who & \rm saw & \rm Gil \\
\uline{1}&\uline{1}&\uline{1}&\uline{1} \\
\cf{N_m} &
\cf{(N_y\bs N_y)/(S[dcl]_x\bs NP_y)} &
\cf{(S[dcl]_x\bs NP_y)/NP_z} &
\cf{NP_g} \\
\cf{@_m \semf{man}} & 
\cf{@_y \sematt{genrel}x} & 
\cf{@_x \semf{see}
  \wedge @_x\sematt{tense}\semf{past}
  \wedge @_x\sematt{act}y
  \wedge @_x\sematt{pat}z} &
\cf{@_g \semf{Gil}}\\
&& \fapply{2} \\
&& \mc{2}{\cf{S[dcl]_x\bs NP_y}} \\
&& \mc{2}{
     \cf{@_x \semf{see}
        \wedge @_x\sematt{tense}\semf{past}
        \wedge @_x\sematt{act}y
        \wedge @_x\sematt{pat}g
        \wedge @_g\semf{Gil}
     }
   }\\
& \fapply{3} \\
& \mc{3}{\cf{N_y\bs N_y}} \\
& \mc{3}{
    \cf{@_x \semf{see}
      \wedge @_x\sematt{tense}\semf{past}
      \wedge @_x\sematt{act}y
      \wedge @_x\sematt{pat}g
      \wedge @_g\semf{Gil}
      \wedge @_y\sematt{genrel}x
    }
  }\\
\bapply{4} \\
\mc{4}{\cf{N_m}}\\
\mc{4}{
    \cf{@_x \semf{see}
      \wedge @_x\sematt{tense}\semf{past}
      \wedge @_x\sematt{act}m
      \wedge @_x\sematt{pat}g
      \wedge @_g\semf{Gil}
      \wedge @_m\sematt{genrel}x
      \wedge @_m\semf{man}
    }
  }
}
\end{center}

The relativiser does two things: it coindexes the subject of \emph{saw} to the modified noun,
and records the modification relation, which we have termed $\sematt{genrel}$. The hat
category does both of these things, in addition to the verbal semantics:

\begin{center}
\deriv{3}{
\rm man & \rm seeing & \rm Gil \\
\uline{1}&\uline{1}&\uline{1} \\
\cf{N_m} &
\cf{(S[ng]_x\bs NP_y)^{N_y\bs N_y}/NP_z} &
\cf{NP_g} \\
\cf{@_m \semf{man}} & 
\cf{@_x \semf{see}
    \wedge @_x \sematt{tense}\semf{ng}
    \wedge @_x \sematt{act}y
    \wedge @_y \sematt{genrel}x
    \wedge @_x \sematt{pat}z
   } &
\cf{@_g \semf{Gil}}\\
& \fapply{2} \\
& \mc{2}{\cf{(S[ng]_x\bs NP_y)^{N_y\bs N_y}}} \\
& \mc{2}{
    \cf{@_x \semf{see}
      \wedge @_x \sematt{tense}\semf{ng}
      \wedge @_x \sematt{act}y
      \wedge @_y \sematt{genrel}x
      \wedge @_x \sematt{pat}g
      \wedge @_g \semf{Gil}
    }
  }\\
& \unhat{2} \\
& \mc{2}{\cf{N_y\bs N_y}} \\
& \mc{2}{
    \cf{@_x \semf{see}
      \wedge @_x \sematt{tense}\semf{ng}
      \wedge @_x \sematt{act}y
      \wedge @_y \sematt{genrel}x
      \wedge @_x \sematt{pat}g
      \wedge @_g \semf{Gil}
    }
  }\\
\bapply{3} \\
\mc{3}{\cf{N_m}}\\
\mc{3}{
  \cf{@_x \semf{see}
      \wedge @_x \sematt{tense}\semf{ng}
      \wedge @_x \sematt{act}m
      \wedge @_m \sematt{genrel}x
      \wedge @_x \sematt{pat}g
      \wedge @_g \semf{Gil}
      \wedge @_m\semf{man}
    }
  }
}\end{center}

The unhat rule is semantically inert: it returns exactly the same logical form as
was passed to it, just as type-raising does in \hlds.

The logical forms of the hat-driven reduced relative and the WH-relative
analysis are except for the tense of the verb. This accurately reflects
the fact that it is entirely a surface syntactic distinction.                                           
                                                               
                                                               



\section{Interaction with the Grammar}
\label{sec:interaction}

In this section, we describe how the existing unification algorithm and combinatory rules
behave with respect to the hat attribute we have added. This section does not
describe any modifications to the existing definitions --- all changes required to support
hat categories were described in Section~\ref{sec:hat_def}. However, some of the consequences
of these modifications may not be immediately apparent, so are made explicit here.

\subsection{Unification and the \hatsc attribute}

The \hatsc attribute is handled the same as any other attribute during unification.
If the two attributes have conflicting values, unification fails; and if one \hatsc
is underspecified, it inherits the \hatsc of the other category if unification
succeeds. In Chapter~\ref{chapter:hat_corpus}, we describe how we must
make a variety of changes to the \candc parser's implementation of unification
to support hat categories, but this is due to the efficiency compromises made in
that specific implementation. In theory, we do not require anything different
from the standard unification algorithm described by \citet{shieber:86}.

\subsection{Application and Adjunction}
\label{sec:app_interaction}

This section describes how the application rules interact with hat categories.
It also describes the interaction with modifier categories, where the result and
argument are coindexed. This behaviour is all predicted by the unification
semantics, so the adjunction interactions will also reoccur with composition.
Since forward and backward application work analogously, but with the order of
the categories reversed, we will confine our discussion to forward application.
During forward application, the argument of the left-side category is unified
with the right-side category. The result of the left-side category is returned,
with any attributes whose values were coindexed to the argument filled in:

% X^Y/Z Z --> X^Y
\begin{eqnarray}
\cf{S[ng]^{NP}/PP} & \cf{PP} & \Rightarrow\;\; S[ng]^{NP}
\end{eqnarray}

If the category on the right specifies a \hatsc value, unification will fail if the
argument has a null \hatsc value, but will pass if its \hatsc is underspecified
or the two \hatsc values match.

As discussed in Section~\ref{sec:null_hats},
we stipulate that the arguments and results of modifier categories have
underspecified \hatsc values coindexed to each other,
while complement arguments have a null hat. We have not used arguments which
specify a category value for their hat attribute in our analyses, but such an
argument would unify if and only if its hat attribute unified with the applicand:

% X/Z Z^Y --> X
\begin{eqnarray}
\cf{NP_y/NP_y} &\cf{NP^{S\bs S}}  & \Rightarrow\;\;\cf{NP^{S\bs
S}}\label{hat_adjunct} \\
\cf{(S[dcl]\bs NP)/NP^{\nullhat}} &\cf{NP^{S\bs S}}  & \Rightarrow\;\;\emptyset
\\
\cf{(S[dcl]\bs NP)/NP^{S\bs S}} & \cf{NP^{S\bs S}} & \Rightarrow\;\;S[dcl]\bs NP
\\
\cf{(S[dcl]\bs NP)/NP^{S/S}} & \cf{NP^{S\bs S}} & \Rightarrow\;\;\emptyset
\end{eqnarray}

When the left-side category is a modifier, as in~\ref{hat_adjunct}, the hat
category from the right-side category will be transmitted across to the result,
since the argument and result in a modifier are coindexed, and adjunct
categories specify empty hat values. This is much like the familiar example of a
modifier applying to a category with a feature: 

\begin{eqnarray}
 \cf{(S\bs NP)_y/(S\bs NP)_y} & \cf{S[ng]\bs NP} & \Rightarrow\;\; \cf{S[ng]\bs
NP}
\end{eqnarray}

The coindexation mechanism operates in the standard way to ensure that
modifiers are functions that return their arguments unchanged. Because the
coindexing is recursive, hat fields specified within the category's result are
also transmitted during unification:

\begin{eqnarray}
 \cf{(S\bs NP)_y/(S\bs NP)_y} & \cf{S[dcl]^{NP}\bs NP} & \Rightarrow\;\;
\cf{S[dcl]^{NP}\bs NP}
\end{eqnarray}


\subsection{Composition}
\label{sec:hat_composition}
The non-generalised forward and backward, harmonic and crossing composition rules
are rules of the form:

\begin{eqnarray}
 \cf{X/Y}    & \cf{Y|_iZ} & \Rightarrow \cf{X|_iZ}\\
 \cf{Y|_iZ} & \cf{X\bs Y} & \Rightarrow \cf{X|_iZ}
\end{eqnarray}

Where \cf{|} is either \cf{/} or \cf{\bs} (but must be the same slash for both instances of
\cf{|_i} in a rule). For ease of discussion, we introduce the following terms for the
different parts of these rules:

\begin{itemize}
 \item \textbf{Functor}: The \cf{X/Y} or \cf{X\bs Y} category;
 \item \textbf{Argument}: The \cf{Y|_iZ} category;
 \item \textbf{Functor slash}: The slash of the \cf{X/Y} or \cf{X\bs Y} category;
 \item \textbf{Argument slash}: The slash of the \cf{Y|_iZ} category.
 \item \textbf{Overlap}: The \cf{Y} category;
 \item \textbf{Product}: The \cf{X|_iZ} category;
 \item \textbf{Product result}: The \cf{X} category;
 \item \textbf{Product argument}: The \cf{Z} category;
\end{itemize}

In forward composition, the functor is on the left and the functor slash is \cf{/};
in backward composition, the functor is on the right and the functor slash is \cf{\bs}.
Composition is harmonic if the functor slash shares the same directionality as the
argument slash, and crossed if they do not.

We assume that the product inherits its attribute values from the argument, not the
functor. \citet{steedman:00} makes this clear for \slashsc, in
the Principle of Directional Consistency, but we must define the product's
\headsc, \hatsc and \featsc values as well. These values must be inherited
from the argument to produce linguistically satisfactory analyses. For instance,
 the head of the \cf{S[dcl]/NP} constituent in
the incremental derivation of \emph{they gave gifts} would be \emph{they},
rather than \emph{gave}. This can be seen by marking \headsc values
in a derivation, as subscripts:

\begin{center}
\deriv{2}{
\rm they & \rm gave \\
\uline{1}&\uline{1} \\
\cf{NP_{they}} &
\cf{((S[dcl]_{gave}\bs NP_y)_{gave}/NP_z)_{gave}} \\
\ftype{1} \\
\mc{1}{\cf{(S_y/(S_y\bs NP_{they})_y)_{they}}} \\
\fcomp{2} \\
\mc{2}{\cf{(S_{gave}/NP_z)_{gave}}}
}
\end{center}

The fact that the product inherits its attribute values from the argument, not the functor,
is important for determining the result of composition when the argument specifies a hat:

\begin{center}
\deriv{3}{
\rm gifts & \rm they & \rm gave \\
\uline{1}&\uline{1}&\uline{1} \\
\cf{N} &
\cf{NP} &
\cf{((S[dcl]\bs NP)/NP)^{N\bs N}} \\
& \ftype{1} \\
& \mc{1}{\cf{S/(S\bs NP)}} \\
& \fcomp{2} \\
& \mc{2}{\cf{(S[dcl]/NP)^{N\bs N}}} \\
& \unhat{2} \\
& \mc{2}{\cf{N\bs N}} \\
\bapply{3} \\
\mc{3}{\cf{N}}
}
\end{center}

The hat is inherited from the argument, producing the \cf{(S[dcl]/NP)^{N\bs N}} category.
This behaviour is exploited to analyse reduced object relatives in Chapter
\ref{chapter:hat_corpus}.

Composition always fails when the functor specifies a hat, because of the
Inert Slash Stipulation. As stated in Section~\ref{sec:null_mode}, complex
categories that specify a hat must receive an inert slash. This blocks
composition if the functor specifies a hat, as below:

\begin{center}
\deriv{2}{
\rm giving & \rm to \\
\uline{1}&\uline{1} \\
\cf{((S[ng]\bs^! NP^{\nullhat})/^! PP^{\nullhat})^{N\bs N}} &
\cf{PP/NP^{\nullhat}} \\
\asterisk{2} \\
\mc{2}{\cf{(S[ng]\bs^! NP^{\nullhat})/NP^{\nullhat}}}
}\end{center}

The slash assigned to \cf{S[ng]\bs^! NP} is inert, which means it cannot act as
the functor in application or composition.

For composition to succeed, the functor's overlap and the argument's overlap
must unify successfully. This unification will fail if the argument's overlap
is a hat category and the functor is a predicate, as the Null Hat Stipulation
(Section~\ref{sec:null_hats}) asserts that all arguments of predicate categories
must be assigned null hats. For example, this forward harmonic composition is blocked:

\begin{center}
\deriv{2}{
\rm giving & \rm to \\
\uline{1}&\uline{1} \\
\cf{(S[ng]\bs^! NP)/PP^{\nullhat}} &
\cf{PP^{(S\bs NP)\bs (S\bs NP)}/NP}\\
\asterisk{2} \\
\mc{2}{\cf{(S[ng]\bs^! NP)/NP}}
}\end{center}

Composition is therefore blocked if the functor or the overlap specify a hat,
but can succeed when the argument specifies a hat. If it does,
the argument's hat is preserved in the product, ensuring that the unhat rule
will eventually apply.

\subsection{Coordination}

There are several proposals for coordination in categorial grammars.
The accepted current \ccg analysis, from \citet{baldridge:03}, uses  application and
multi-modal slashes. However, the \citet{steedman:00}
analysis using a ternary conjunction combinator is relevant to this thesis,
as it is approximated in \ccgbank. The
analysis implemented in \ccgbank uses two binary rules, because the addition
of a ternary rule complicates the \cky chart-parsing algorithm \citep{hock:cl07}.

The \citet{baldridge:03} \mmccg analysis allows coordination to be implemented
using the application rule, by assigning the coordinator the category
\cf{(X\bs\smode X)/\smode X}. The multi-modal slashes are required to avoid
type-raised categories from composing with the coordinator category,
causing over-generation as described in Section
\ref{sec:mmccg_background}. This treatment of coordination uses the application
rules, so the interaction with hat categories is as expected from Section
\ref{sec:app_interaction}.

The \citet{steedman:00} ternary analysis uses two categories and a coordinator:

\begin{eqnarray}
 \cf{X} & conj\;\;\cf{X} & \Rightarrow\;\;\cf{X}
\end{eqnarray}

The two \cf{X} categories are unified, allowing a hat category to be transferred
from one side of a coordinated phrase to the other, as shown in~\ref{hat_coord}.
Coordination will be blocked if the hat categories do not unify, as in
\ref{hat_coord_conflict} and~\ref{hat_coord_block}:

\begin{eqnarray}
 \cf{NP}          & conj~\cf{NP^{S/S}}      & \Rightarrow \cf{NP^{S/S}}
\label{hat_coord}\\ 
 \cf{NP^{S\bs S}} & conj~\cf{NP^{S/S}}      & \Rightarrow \emptyset
\label{hat_coord_conflict}\\     
 \cf{NP^{S\bs S}} & conj~\cf{NP^{\nullhat}} & \Rightarrow \emptyset
\label{hat_coord_block}
\end{eqnarray}

Because ternary rules are undesirable for parsing, and the \mmccg analysis had
not yet been developed, \citet{hock:thesis03} implemented coordination using two
binary rules:
\begin{center}
\deriv{3}{
\rm Apples & \rm and & \rm oranges \\
\uline{1}&\uline{1}&\uline{1} \\
\cf{NP} &
\cf{conj} &
\cf{NP} \\
&\conj{2} \\
&\mc{2}{\cf{NP[conj]}} \\
\conj{3} \\
\mc{3}{\cf{NP}}
}
\end{center}

During the second rule application, the \cf{NP} categories are unified, ensuring
that hat categories interact with this implementation as we would expect.




\section{Consequences for Weak Generative Power}
\label{sec:hat_gp}
\begin{figure}
 \centering
\begin{parsetree}
(.\cf{P}.
  (.\cf{S}.)
  (.\cf{H}.
    (.\cf{B}.
      (.\cf{L}.
        (.\cf{LL}.)
        (.\cf{LR}.)
      )
      (.\cf{R}.)
    )
  )
)

\end{parsetree}
\caption[Parse tree with unary production.]{A parse tree with a unary
production, used to illustrate our proof that unary rules do not increase the
weak generative power of \ccg.\label{fig:unhat}}
\end{figure}


In the standard definition \citep{chomsky:aspects}, a grammar \emph{weakly}
generates a set of sentences (also called a language), and \emph{strongly}
generates a set of structural descriptions. In general, greater strong
generative power (\textsc{sgp}) is desirable if it does not lead to an increase
in weak generative power (\textsc{wgp}) \citep{joshi:00}. \textsc{sgp} allows
for more nuanced linguistic descriptions, giving the opportunity for more
efficient and less ambiguous analyses. On the other hand, an increase in
\textsc{wgp} is problematic. If a linguistic theory claims --- as \ccg does ---
that the formalism is a model of the human language faculty, rather than simply
a notational device, then the \textsc{wgp} of the formalism constitutes a
prediction about the set of languages that will naturally occur. No
natural languages have been shown to be more than mildly context-sensitive
\citep{shieber:85}, a theory with higher \textsc{wgp} must suggest some additional
constraint that explains this apparent upper bound. There may be many such plausible
explanations, so a theory with higher \textsc{wgp} cannot be ruled out; but,
all else being equal, we ought to prefer a theory that explains the observed
\textsc{wgp} of the world's languages more simply, by Occam's razor.

Our hat \ccg is weakly equivalent to non-hat \ccg if and only if it can generate
all and only the strings that the non-hat \ccg can generate. Since the hat \ccg
is a superset, there is no question of it under-generating. We therefore consider
whether it can generate any strings that the non-hat \ccg cannot.

The only rules added by hat categories are unary type-changing productions.
Furthermore, from a finitely sized lexicon, we will only add a finite number
of type-changing rules, as every rule must be lexically specified.
We will argue that for any \ccg derivation using such a finite set of
type-changing rules, there is an equivalent \ccg derivation with the type-changing
rules removed. \citet{carpenter:92} showed that if lexical rules are allowed to
generate an infinite lexicon, the grammar will generate all and only the recursively
enumerable languages. Since we have a bounded number of unary operations, this will
not be a problem.

Consider the example of a unary rule, shown in Figure~\ref{fig:unhat}.
We can convert this derivation into a standard \ccg analysis by
replacing the child node \cf{B} with its unary parent \cf{H}. This does not
affect the production \psbinary{\cf{P}}{\cf{S}}{\cf{H}}, so the only potential
problem is in the new \psbinary{\cf{H}}{\cf{L}}{\cf{R}} production. This
production might not be licensed by a \ccg combinator.

However, for any licensed production \psbinary{\cf{A}}{\cf{L}}{\cf{R}}, it is
always possible to switch the parent node \cf{A} for a new parent \cf{B} by
replacing one or both children, producing a valid production
\psbinary{\cf{B}}{\cf{L'}}{\cf{R'}}. In Section~\ref{sec:relabel_rule}, we
describe a simple set of rules to do this in a desirable way for our conversion
process.

For any \ccg category \cf{B}, it is trivial to construct a valid combinatory
rule that produces \cf{B} as its result. We will use forward application:

\begin{eqnarray}
 \cf{B/R} & \cf{R} & \Rightarrow \;\; \cf{B}
\end{eqnarray}

We can therefore flatten any unary rule and produce a valid \ccg derivation as
follows. First, replace the child \cf{B} with the parent \cf{H}. Next, replace
the left child of \cf{B} with a functor category \cf{H/R}, where \cf{H} is the
original righthand child of \cf{B}. Now propagate the change down the tree, by
replacing the lefthand child of what used to be \cf{L} with a category
\cf{(H/R)/LR}, where \cf{LR} was the original category of the righthand child of
\cf{L}. Continue propagating the changes in this way until you arrive at lexical
categories. The only unary rule that can be encountered is a type-raising
production, which has the form \cf{T/(T\bs TL)}. If this occurs, simply replace
the \cf{TL} part of the category of the parent and the child with the new
category, and continue propagating the changes down the tree.

This proves that a \ccg grammar extended with a finite set of unary rules cannot
generate any strings the original grammar could not. Since any unhat production
\cf{X^Y} $\rightarrow$ \cf{Y} can be rewritten as a unary type-changing rule
\cf{X} $\rightarrow$ \cf{Y}, it follows that the weak generative power is unaltered.

\section{Summary}

In this chapter, we introduced \emph{hat categories}, which allow unary
type-changing rules to be lexically represented. We have also described a set of
stipulations on category well-formedness that ensure hat categories are
non-disjunctive, offering total lexical control over the unary rule. The appeal
of hat categories is the ability to isolate a form/function discrepancy in a
single category, ensuring that modifiers can receive their canonical categories.
We have sketched an informal proof that hat categories do not extend the weak
generative power of the formalism.
%, and show that they offer substantial
%improvements in descriptive power.

We achieve these desirable properties with only minimal changes to the \ccg
machinery.
Hat categories require no change to the combinatory rules or the unification
algorithm, the addition of one new attribute to the category object, and only
one general grammatical rule that is limited to interaction with the new hat
attribute. In Chapter~\ref{chapter:hat_corpus}, we will see how hat categories
allow favourable analyses of some constructions common in \ccgbank.

% 
 
\chapter{Changing the Level of Lexicalisation of \ccgbank}

\label{chapter:hat_corpus}

Statistical parsing often involves some level of linguistic compromise. The
\candc parser \citep{clark:cl07} implements a subset of the \ccgbank grammar.
This subset was selected to reduce ambiguity, and the \ccgbank grammar
itself was largely determined by what could be cheaply acquired from the Penn
Treebank \citep{marcus:93}. The Penn Treebank grammar itself involved linguistic
compromises, as its annotation scheme was heavily influenced by concerns about
annotation cost \citep{bies:95}.

This chapter and the next explore perhaps the most fundamental theoretical
compromise in \ccgbank, the decision to relax the level of lexicalisation in the
corpus. We explore this issue by developing two versions of the corpus that
replace the type-changing rules in \ccgbank with lexicalised analyses.

One strategy for replacing these rules relies on the extension to the category
objects we described in Chapter~\ref{chapter:hat_cats}, hat categories. The hat
categories offer considerable flexibility, which makes them well suited to the
development of a minimally different version of \ccgbank, where we change almost
nothing but the level of lexicalisation. Hat categories let us do this with only
one altered category for each instance of type-changing.

The second strategy we implement compiles out the type-changing rules without
using hat categories, by replacing the child of a unary production with its
parent and updating the categories of its children appropriately. This corpus
allows us to investigate how well the hat categories control the modifier
category proliferation problem described in Chapter~\ref{chapter:ling_mot}. This
corpus is also intended to be minimally different from \ccgbank, which can
result in incorrect analyses. For some constructions, we modify the analysis
slightly, to prevent what would otherwise be a frequent, problematic analysis.
We also conduct an experiment where type-raising is lexically specified, using
hat categories.

The chapter is structured as follows.
First, we describe our approach to adapting \ccgbank. This includes a general
algorithm to propagate label changes down a \ccg derivation.
We then describe the \hatsys corpus, where we use hat categories to lexicalise
the type-changing rules in \ccgbank. We then provide a description of the
\nounary corpus, which implements lexicalised analyses using the \ccg grammar
described by \citet{steedman:00} --- in other words, a \ccg grammar that uses
application, composition, coordination and type-raising rules, but no
type-changing rules and no hat categories. We then describe some additional
lexicalisation experiments, and some alternative analyses we explored. Finally,
we summarise the corpora that we will use for our experiments in Chapter
\ref{chapter:results}.

\section{Adapting \ccgbank}

This section describes how we created the various corpora we required for the
experiments described in Chapter~\ref{chapter:results}. The corpora were all
created by adapting \ccgbank using conversion scripts we implemented, since the
original \ccgbank generation code was unavailable to us.

We will first provide a brief overview of how we approached the adaptation task,
and what we believe are the most important aspects of our current approach. We
then present the node relabelling algorithm we developed. We then describe our
process for annotating novel lexical categories and discuss how we generated the
predicate-argument files for the corpora we created, before briefly noting a
pre-processing step that unarises binary type-changing rules. The evaluation and
debugging processes we employed to validate the changes are discussed in Section
\ref{sec:validation}.

\subsection{Overview and General Lessons}

Adapting annotated corpora is a well studied problem
\citep{wang:94,lin:98,meyers:01,frank:03,miyao:04,shen:06,hock:cl07}, but it is
one that is very difficult to generalise from. Despite having experience with
this kind of problem
\citep{honnibal:04,honnibal:dlp07sfl,honnibal:pacling07prop}, our process
underwent several revisions as the research evolved. 

Conceptually, there are two ways to alter a \ccg derivation: top down, or bottom
up. The bottom up method involves changing the lexical categories and then
rebuilding the normal form derivation. The top down method involves changing a
production, and propagating changes down the tree to the lexical categories. We
found the top down method far simpler.

The other important developments were methodological. The use of well
structured, unit-tested libraries was important, because bugs which produce
suboptimal analyses can be difficult to catch, as we cannot manually review the
whole corpus. We also developed a framework for validating new versions of a
corpus according to its grammatical and lexical properties, as described in
Section~\ref{sec:validation}, and a labour saving device for the
frequent category-index annotation that was required.

The most important innovation in our  approach was the generalised node
relabelling algorithm described in Section~\ref{sec:relabel_rule}. Once this
issue was solved, specifying changes was much simpler. This algorithm also led
us to the proof that the \textsc{unhat} rule could not change the weak
generative power of the grammar, because it is always possible to replace a node
label with another arbitrary label without producing an invalid derivation.

\subsection{Relabelling CCG Nodes}
\label{sec:node_relabelling}
Node labels in a \ccg tree are interdependent, so when we change one node label
we need to propagate the change to other labels. We adopt a strategy of always
propagating changes downwards. For instance, if we change a node labelled
\cf{NP} to \cf{NP\bs NP}, we will need to make corresponding changes to its
children, to ensure that we do not create an invalid production. 

It is also not enough to simply ensure that the production can be validated by
some \ccg rule. If we have a tree where the left child is a functor and the
right child is an argument, we must not produce child labels that reverse this
relationship, or that change it by making one child a modifier. For example, the
tree on the right below is a valid \ccg production --- but a malformed analysis:

\begin{equation}
\ptbegtree
\ptbeg \ptnode{\cf{PP}}
  \ptleaf{\cf{PP/NP}}
  \ptleaf{\cf{NP}}
\ptend
\ptendtree
\longrightarrow
\ptbegtree
\ptbeg \ptnode{\cf{VP\bs VP}}
  \ptleaf{\cf{PP/NP}}
  \ptleaf{\cf{(VP\bs VP)\bs (PP/NP)}}
\ptend
\ptendtree
\end{equation}

The correct transformation would change the result of the left side, instead of
reversing the functor/argument relationship of the two:

\begin{equation}
\ptbegtree
\ptbeg \ptnode{\cf{PP}}
  \ptleaf{\cf{PP/NP}}
  \ptleaf{\cf{NP}}
\ptend
\ptendtree
\longrightarrow
\ptbegtree
\ptbeg \ptnode{\cf{VP\bs VP}}
  \ptleaf{\cf{(VP\bs VP)/NP}}
  \ptleaf{\cf{NP}}
\ptend
\ptendtree
\end{equation}

Changing the corpus is much simpler if we have a generic label changing
algorithm that does not make any assumptions about the relationship between the
old label and the new label. We therefore need a more complicated process than
previous work which has made quite restricted changes, such as
\citet{honnibal:pacling07prop} and \citet{boxwell:08}, who updated \ccgbank's
complement\slash adjunct labels to conform to Propbank's \citep{propbank}, or
\citet{vadas:08}, who reformed the structure of left-branching \cf{NP}s.

\subsubsection{Relabelling Binary Productions}
\label{sec:relabel_rule}
The first step is to identify the  \emph{production type}, which does not
correspond one-to-one to the combinatory rule used. For instance, both a
modifier and a predicate might use forward application, but we will need to treat
the two productions differently. We sort binary productions into the following
categories:
\begin{itemize}
\item Argument application;
\item Argument composition;
\item Modifier application;
\item Modifier composition;
\item Conjunction;
\item Punctuation.
\end{itemize}

In general, the rules for binary productions find the result $A_r$ and argument
$A_a$ of the original parent $A$ and replace them with the appropriate of the
new parent $B$. If one of the children is a modifier category, a different rule
is used. The node change rules for forward combinators are given below, with an
example given below each rule type. The translation rules for backward
combinators are directly analogous.

\begin{table}
\begin{tabular}{l|ccccc}
\hline
\textbf{Production Type}           & \textbf{Old left}    & \textbf{Old right}  
 & $\to$ & \textbf{New left}    & \textbf{New right}\\
\hline
\hline
Argument Application               & \cf{A/Y}      & \cf{Y}         &$\to$&
\cf{B/Y}       & \cf{Y}    \\
\cf{PP} $\to$ \cf{S/S}             & \cf{PP/NP}    & \cf{NP}        &$\to$&
\cf{(S/S)/NP}  & \cf{NP}   \\
\hline
Argument Composition                & \cf{A_r/Y}    & \cf{Y/A_a}     &$\to$&
\cf{B_r/Y}     & \cf{Y/B_a}\\
\cf{(S\bs S)/S[em]} $\to$ NP/PP    & \cf{(S/S)/NP} & \cf{NP/S[em]}  &$\to$&
\cf{NP/NP}     & \cf{NP/PP}\\
\hline
Modifier Application                & \cf{A/A}      & \cf{A}         &$\to$&
\cf{B/B}       & \cf{B}    \\
\cf{VP/NP} $\to$ \cf{(VP/NP)^{NP}} & \cf{VP/VP}    & \cf{VP}        &$\to$&
\cf{VP/VP}     & \cf{(VP/NP)^{NP}}\\
\hline
Modifier Composition                & \cf{A_r/A_r}  & \cf{A_r/A_a}   &$\to$&
\cf{B_r/B_r}   & \cf{B_r/B_a}\\
\cf{(VP/VP)/NP} $\to$ \cf{PP/NP}   & \cf{VP/VP}    & \cf{(VP/VP)/NP}&$\to$&
\cf{PP/PP}     & \cf{PP/NP}\\
\hline
Conjunction                        & \cf{A}        & \cf{conj}      &$\to$&
\cf{B}         & \cf{conj}\\
\cf{NP[conj]} $\to$ \cf{NP^{(S/S)}[conj]} & \cf{NP}& \cf{conj}      &$\to$&
\cf{NP^{(S/S)}}& \cf{conj}\\
\hline
\end{tabular}
\caption{The relabelling rule and an example for each production type.}
\end{table}

Modifier application and composition present the most complicated cases. Modifiers
should not specify features or hat categories, as these should be inherited from
the head. Where possible, we also remove arguments from the modifier category, to
prevent forming categories like \cf{((S\bs NP)/NP)/((S\bs NP)/NP)}. Instead, we
form the category \cf{(S\bs NP)/(S\bs NP)}, and the new production will rely on
a composition rule, rather than application. There are four exceptions where we
form a complex modifier, instead of trimming its arguments:

\begin{enumerate}
 \item If the head is a complex category with an innermost result that is not
\cf{S}, and a crossing composition rule would be needed, we form a complex
modifier;
 \item If the head is \cf{S\bs NP}, we do not strip the \cf{NP} argument. This
allows us to match the analyses in \ccgbank, which draws a distinction between
\cf{S/S} and \cf{(S\bs NP)/(S\bs NP)} for sentential and verb phrase modifiers
respectively;
 \item If the head is a modifier, a composition rule would produce incorrect
dependencies, because the category the new modifier's argument will unify with is
not coindexed with the modifier's lexical head. Essentially, the composition rule
cannot be used in place of a modifier-of-modifier category, as discussed in
Section~\ref{sec:infinite_categories}.
\end{enumerate}

\subsubsection{Relabelling Unary Productions}

There are two types of unary productions in \ccgbank: type raising rules, and
type-changing rules. The unary rules will be replaced by the
rules described in Sections~\ref{sec:hat_corpus} and~\ref{sec:nounary_corpus},
so we only need to relabel type-raising instances. The rule to relabel type-raising
follows a simple schema. To transform a type-raised category \cf{A} such as \cf{S/(S\bs A)}
into a type-raised category \cf{B}, we simply replace the original category:

\begin{eqnarray}
 \mbox{Forward  type raise} & \cf{T/(T\bs A)} & \to \cf{T/(T\bs B)}\\
 \mbox{Backward type raise} & \cf{T\bs(T/A)}  & \to \cf{T\bs(T/B)}
\end{eqnarray}

The only complication is when the new category is a modifier. Instead of generating a
type-raised category such as \cf{T\bs T/(S\bs S)}, we truncate the type-raising production,
and relabel its child node.


\subsection{Head and Dependency Annotation}
\label{mu_annotation}

Changing \ccg analyses introduces new lexical categories. These categories must
be assigned head indices and dependency markers consistent with how other
categories are annotated in the \candc parser, which does not always correspond
to the lexical categories that are specified in \ccgbank. We therefore manually
specify the head and dependency annotation for categories we introduce that
occur more than 10 times in the training portion of a corpus. This is the same
policy used by \citet{clark:cl07} when annotating the original \ccgbank
categories.

Head and dependency annotation is specified in the \markedup file. The file also
contains mappings from \ccgbank dependencies to \depbank dependencies, for the
extrinsic \depbank-based annotation described in \citet{clark:acl07parseval}. We
do not implement \depbank mappings for our categories, although these could be
acquired by extension through our mapping to \ccgbank dependency labels.


It is not clear that manually specifying the head and dependency annotation is
the best policy. In hindsight, it may have been better to attempt to manipulate
annotated versions of the categories from the start, allowing the annotation to
be naturally propagated down the derivation. A properly structured,
inheritance-based lexicon would make this task much easier, but the \candc
parser uses a flat lexicon, and we have largely followed their process for
category annotation. However, we have implemented a few generalisation
mechanisms, where it is obvious that the new categories we have created have a
predictable structure based on existing category annotations.

Over the various adaptations we experimented with, we added 901 new categories
to the \markedup file, which contained the 580 annotated categories created by
\citeauthor{clark:cl07}. Our policy was to manually annotate all categories that
occurred in a corpus 10 or more times before we ran parsing experiments, to
match the methodology of \citet{clark:cl07}. After doing this, we also
experimented with generalisation mechanisms. Of the categories we added, 211
were annotated manually, 509 were hat categories generalised from existing
\markedup categories, and 181 were generalisations from \cf{N} to \cf{NP} or
vice versa, using the method described in Section~\ref{sec:generalisation}.

Table~\ref{markedup_stats} shows statistics for the annotated categories of
different versions of the corpus. The statistics are calculated on the training
partitions of each corpus, sections 02-21. For each version, the table shows the
coverage with the existing \candc markedup categories, the coverage with the
manually-added entries for categories occurring 10 or more times, the coverage
achieved once the automatically annotated categories were
added, and the total number of categories occurring in the corpus. Coverage was
defined as the percentage of tokens in the training portion whose category was
annotated in the \markedup file. We investigated training coverage because we
were interested in how many of the possible annotations were found in the
\markedup file, and categories occurring in the development and testing sections
were not candidates for annotation.

Our addition of rare categories did not increase training coverage
substantially, compared to the original \ccgbank. \ccgbank's coverage changed
slightly because some categories that were too rare to receive annotation in
\ccgbank were more frequent in one or more versions of the corpus we created.
The absence of frequent categories from the \markedup file can affect
development and test scores, because they are unavailable for selection by the
supertagger. Rare categories can only affect training coverage, because
categories less frequent than 10 are filtered out of the category set for
parsing.

\begin{table}
\centering
 \begin{tabular}{l|r|r|r|r}
\hline
  Corpus   & Orig. Cover. & + New cats $>= 10$ & + All new cats & \# Cats\\
\hline
\hline
  CCGbank  & 99.82 & 99.83 & 99.83 & 1285\\
  \hatsys  & 85.73 & 99.67 & 99.86 & 1847 \\
  \nounary & 98.59 & 99.67 & 99.75 & 1646\\
\hline
 \end{tabular}
\caption{\markedup statistics for the various corpora.\label{markedup_stats}}
\end{table}


\subsubsection{Interface for Markedup Annotation}

A \candc \markedup entry consists of the bare category, the annotated category,
the number of argument slots in the annotated category, and a \depbank mapping
for each argument slot. Some example \markedup entries, using the \candc syntax
(extended to include a notation for hat categories):

\begin{quote}
\begin{verbatim}
(S[pss]\NP)^(N\N)
  1 (S[pss]{_}\NP{Y}<1>){_}^(N{Y}\N{Y}){_}
  1 ncsubj %l %f

((S[b]\NP)/(S[to]\NP))/NP
  3 (((S[b]{_}\NP{Y}<1>){_}/(S[to]{Z}<2>\NP{W*}){Z}){_}/NP{W}<3>){_}
  1 ncsubj %l %f _
  2 xcomp %f %l %k =(S[to]\NP)/(S[b]\NP)
  3 dobj %l %f

(S[dcl]\S[dcl])\NP
  2 ((S[dcl]{_}\S[dcl]{Y}<1>){_}\NP{Z}<2>){_}
  1 cmod _ %l %f
  2 ncsubj %l %f
\end{verbatim}
\end{quote}

Heads are labelled in curly brackets, with the lexical head listed as
\verb1{_}1. Arguments that produce a dependency are labelled with an argument
index between angled brackets. After the annotated category is given, each
argument is mapped to a \depbank dependency. \verb1%l1 and \verb1%f1 specify
whether the functor corresponds to the head of the dependency, or whether the
argument does.

This format makes annotating the categories manually quite inefficient, because
a simple alteration requires multiple changes to the annotated category string.
For instance, the variable ordering is significant, so coindexing one category
may require updating all of the other head indices. The same is true for the
argument indices, which also must be kept synchronised with the dependency
mapping information.

We developed a simple text-based \textsc{gui} to speed up the annotation
process, that also suggested matches based on previous entries. The program
searches previous entries for the longest matching result, and asks whether the
annotation is desirable. If it is rejected, it finds the next longest matching
result, until annotation begins with a desirable match or the innermost result
atom.

Once a match has been selected, the interface iterates through each category
that must receive a head or argument index, and prompts the user to supply one.
The user can either select an existing variable to coindex the category against,
or press return to introduce a new variable. At each argument, the user is
prompted to decide whether that argument should receive an argument index. The
interface sped up our annotation considerably, and helped ensure that our
annotations were consistent with previously annotated categories.

\subsection{Generalisation Mechanisms}
\label{sec:generalisation}

We developed two generalisation mechanisms to help with annotation. One
mechanism guesses how to annotate a hat category where annotation is known for
both the hat and the base category. The second mechanism generalises from
categories which only differ based on the distinction between \cf{N} and
\cf{NP}.

Given a hat category \cf{((S[pss]\bs NP)^{NP\bs NP}/PP)/NP}, where we know the
annotation for its component pieces, \cf{(((S[pss]_x\bs NP_y)_x/PP_z)_x/NP_y)_x}
and \cf{(NP_y\bs NP_y)_x}, we should be able to guess the annotation of the new
category. The only trick lies in determining the coindexing between the hat and
the base categories. This is approximated with two heuristics:

\begin{enumerate}
 \item The hat will always be coindexed with the category that projects it, so
the lexical head of the hat must be mapped to the head variable of the base
category.
 \item If there is an argument inside the base that matches a variable in the
hat (where \cf{N} and \cf{NP} are declared a match, and the matching is not
sensitive to features), those categories should be coindexed.
\end{enumerate}

The two heuristics allow us to guess the correct annotation for this category:
 
\begin{eqnarray}
\cf{(((S[pss]_x\bs NP_y)_x^{(NP_y\bs NP_y)_x}/PP_w)/NP_v)_x} 
\end{eqnarray}

 The annotation is correct in this case, but the process is not perfect. Since
incorrect annotation will lead to either incorrect dependencies or parse
failures, we manually annotated the hat categories that occurred more than 10
times. A comparison of our heuristic-based guesses against the manually
annotated categories shows that the heuristics produce the correct annotation
88\% of the time.

The second generalisation mechanism was based on the assumption that if two
categories differed only in the distinction between \cf{N} and \cf{NP}, their
annotations probably matched. The mechanism was straightforward to implement: we
simply mapped all \cf{N}s and \cf{NP}s to \cf{NOM}, looked up the annotated
category that matched the generalised version of the novel category, and then
mapped the annotated category's \cf{NOM}s back to the original \cf{N} or \cf{NP}
value. 

\subsection{Unarising Binary Type-changing Rules}
\label{sec:unarising}
The lexicalisation strategies described in Section~\ref{sec:hat_corpus} and
Section~\ref{sec:nounary_corpus} assume that all type-changing rules are
unary. To achieve this, we pre-process \ccgbank, converting binary productions
into unary productions. There are three types of binary productions in \ccgbank:
punctuation cued type-changes, conjunction cued type-changes, and a long tail of
miscellaneous arbitrary productions. Punctuation queued type changes are used
for form-to-function coercions that only occur alongside punctuation, such as
the extraposed appositives discussed in Section~\ref{sec:extraposition}.
Conjunction based type-changes are used to handle coordination between
mismatched constituents.

For the conjunction and punctuation cued productions, we insert a unary node
above the sibling. The label of this new node matches the label of the parent,
producing a valid production:

\begin{eqnarray}
\ptbegtree
\ptbeg \ptnode{\cf{S[adj]\bs NP[conj]}}
  \ptleaf{\cf{conj}}
  \ptleaf{\cf{PP}}
\ptend
\ptendtree
&
\longrightarrow
&
\ptbegtree
\ptbeg \ptnode{\cf{S[adj]\bs NP[conj]}}
  \ptleaf{\cf{conj}}
  \ptbeg \ptnode{\cf{S[adj]\bs NP}}
    \ptleaf{\cf{PP}}
  \ptend
\ptend
\ptendtree
\end{eqnarray}

The third type of binary type-changing rule in \ccgbank functions as a fall-back
strategy for sentences where \citeauthor{hock:cl07}'s \penn-to-\ccgbank
heuristics could not otherwise produce a derivation. This might occur because of
analysis mistakes in the Penn Treebank, source sentences of dubious
grammaticality, or exceptional cases the conversion heuristics did not cover.

These fall-back productions marry two arbitrary children to produce a parent
category. The important property of these rules is that they isolate the
problematic production rule, leaving the rest of the derivation relatively
intact. Lexicalising these rules causes this property to be lost, because the
new category has to be propagated down into the children's subtrees. We
therefore avoid changing binary productions which are not licensed by any
combinatory rules, and do not involve punctuation or conjunction. This means
that all of the noisy constructions, where the binary type-changing rule is
simply introduced to coerce a derivation, are left unchanged.

\section{The \hatsys Corpus}
\label{sec:hat_corpus}
The \hatsys corpus was created by lexically specifying the unary type-changing
productions using hat categories. Binary type-changing rules were unarised using
the rules described in Section~\ref{sec:unarising}. Converting the unary
type-changes was straightforward. All we have to do is add the parent as a hat
category on its immediate child:

\begin{equation}
\ptbegtree
\ptbeg \ptnode{\cf{S[adj]\bs NP[conj]}}
  \ptleaf{\cf{conj}}
  \ptbeg \ptnode{\cf{S[adj]\bs NP}}
    \ptleaf{\cf{PP}}
  \ptend
\ptend
\ptendtree
\longrightarrow
\ptbegtree
\ptbeg \ptnode{\cf{S[adj]\bs NP[conj]}}
  \ptleaf{\cf{conj}}
  \ptbeg \ptnode{\cf{S[adj]\bs NP}}
    \ptleaf{\cf{PP^{S[adj]\bs NP}}}
  \ptend
\ptend
\ptendtree
\end{equation}

The relabelling rules described in Section~\ref{sec:node_relabelling} handle the
propagation of the hat category down the subtree. 

\section{\hatsys Corpus Analyses}

The lexicalised type-changing scheme we have proposed offers many opportunities
for favourable analyses, because it allows form and function to be represented
simultaneously. However, we have limited our changes to replacing the existing
\ccgbank non-combinatory rules. This allows us to compare the two strategies for
controlling modifier category proliferation more closely.

In this section, we briefly examine some of the analyses that this strategy
produces. Some of these issues are discussed in more detail from a linguistic
perspective in Chapters~\ref{chapter:ling_mot} and~\ref{chapter:hat_cats}; the
discussion here focuses on the consequences of these analyses for parsing.

\subsection{Bare Noun Phrases}

By far the most frequent unary production in \ccgbank is the
\psunary{\cf{N}}{\cf{NP}}. rule, which handles bare noun phrases. The rule
ensures that nominals can always take the \cf{N} category, so compound nouns and
adjectives do not  need to be assigned the category \cf{NP/NP}, which is only
used for pre-determiners. Because adjectives and nouns are open lexical classes,
and bare noun phrases are fairly common, this reduction in category sparsity is
quite important.

When bare noun phrases perform a function other than \cf{NP}, a
\emph{hat-in-hat} category is used, like \cf{N^{NP^{S/S}}}. The \cf{NP} stage is
necessary because the phrase could be modified at either level:

\begin{center}
\ptbegtree
\ptbeg \ptnode{\cf{S[dcl]}}
\ptbeg \ptnode{\cf{S/S}}
  \ptbeg \ptnode{\cf{NP^{S/S}}}
    \ptbeg \ptnode{\cf{NP^{S/S}}}
      \ptbeg \ptnode{\cf{N^{NP^{S/S}}}}
        \ptbeg \ptnode{\cf{N/N}} \ptleaf{No} \ptend
        \ptbeg \ptnode{\cf{N^{NP^{S/S}}}} \ptleaf{dummies} \ptend
      \ptend
    \ptend
    \ptbeg \ptnode{\cf{NP\bs NP}}
      \ptbeg \ptnode{\cf{(NP\bs NP)/NP}} \ptleaf{at} \ptend
      \ptbeg \ptnode{\cf{NP}} \ptbeg \ptnode{\cf{N^{NP}}} \ptleaf{finance,}
\ptend \ptend
    \ptend
  \ptend
\ptend
\ptbeg \ptnode{\cf{S[dcl]}} \ptleaf{senators typically outperform top analysts
in the market.}
\ptend
\ptend
\ptendtree
\end{center}

We are dissatisfied with the attachment of the modifier at \cf{NP} rather than
\cf{N}, but it was the best available to us without changing the \ccgbank
analysis of the construction. Ideally, restrictive modifiers should attach to
unspecified \cf{N} nodes, while restrictive modifiers attach to specified noun
phrases.

Another solution would be to allow the distinction between \cf{N} and
\cf{NP} to be underspecified, as suggested by \citet{baldridge:02}. Instead of a
category-level distinction, a feature such as \textsc{spec} could be used.
Restrictive and non-restrictive relatives would then both receive the \cf{N\bs
N} category, and their interpretation would depend on their attachment height.

This analysis would still require a hat category for bare noun phrases, however.
There needs to be some way to provide two attachment levels to distinguish
between the restrictive and non-restrictive readings of \emph{policemen who are
severely obese} and \emph{policemen, who are severely obese}. We
therefore have no functional proposal  that would eliminate nested hat
categories such as \cf{N^{NP^{S/S}}}, which we regard as unfortunate.

\subsection{Hats and Determiner Categories}
\label{sec:hat_determiners}


As we describe in Chapter~\ref{chapter:hat_cats}, hat categories are transferred
during unification when \hatsc attributes are coindexed.
In our implementation on the \candc parser, we equate the \hatsc index with the \headsc
index, and assume that categories must be identical for hats to be passed across.
This compromise was made to simplify the implementation, as we wished to avoid
complicating the parser's unification algorithm, lest it impact efficiency.

One consequence of this compromise is that hats can only be transmitted by true
modifier categories, of the form \cf{X/X}. Other categories that coindex their
argument and result cannot, as their arguments and results fail the identity check.
The only category in English that this effects is the determiner, \cf{NP_y/N_y}.
This produces the following analysis for nouns functioning as modifiers, which
may or may not be suboptimal:

\begin{center}
\deriv{2}{
\rm these & \rm days \\
\uline{1}&\uline{1} \\
\cf{NP^{(S\bs NP)\bs (S\bs NP)}/N} &
\cf{N} \\
\fapply{2} \\
\mc{2}{\cf{NP^{(S\bs NP)\bs (S\bs NP)}}} \\
\unhat{2} \\
\mc{2}{\cf{(S\bs NP)\bs (S\bs NP)}}
}\end{center}

From a theoretical perspective, the biggest problem with this analysis is that
it prevents the dependencies from being specified in the head of the lexical
category. This is not a problem for our parsing evaluation, because we evaluate
against the \ccgbank \textsc{parg} files, which cannot capture these
dependencies either.

It is possible that this analysis is actually favourable for
parsing, given the specific mechanisms of the \candc parser. By placing the hat
on the determiner, some relevant information may be outside of the tagger's
context window, but at least the category is guaranteed to be available in the
word's tag dictionary, and there will be plenty of training examples that pair
the determiner with the hat category. On the other hand, it may be better to
condition the dependency on the head word.

\subsection{Reduced Relative Clauses}

The second most common class of type-changing rules in \ccgbank produce reduced
relative clauses. The following rules take verb phrases and convert them into
nominal post-modifiers:

\begin{eqnarray}
 \eqnpsrule{\cf{S[*]\bs NP}}{}{\cf{NP\bs NP}}\\
 \eqnpsrule{\cf{S[*]/NP}}{}{\cf{NP\bs NP}}
\end{eqnarray}

When these rules are lexicalised, they produce categories of the forms:

\begin{eqnarray}
\cf{(S[*]\bs NP)^{NP\bs NP}/\$}\\
\cf{((S[*]/NP)/\$)^{NP\bs NP}}
\end{eqnarray}

For example, the analysis of a reduced subject relative looks like this:

\begin{center}
\deriv{6}{
\rm asbestos & \rm once & \rm used & \rm in & \rm cigarette & \rm filters \\
\uline{1}&\uline{1}&\uline{1}&\uline{1}&\uline{1}&\uline{1} \\
\cf{N^{NP}} &
\cf{(S\bs NP)/(S\bs NP)} &
\cf{(S[pss]\bs NP)^{NP\bs NP}} &
\cf{((S\bs NP)\bs (S\bs NP))/NP} &
\cf{N/N} &
\cf{N^{NP}} \\
\unhat{1} & \fapply{2} && \fapply{2} \\
\mc{1}{\cf{NP}} & \mc{2}{\cf{(S[pss]\bs NP)^{NP\bs NP}}} && \mc{2}{\cf{N^{NP}}}
\\
&&&& \unhat{2} \\
&&&& \mc{2}{\cf{NP}} \\
&&& \fapply{3} \\
&&& \mc{3}{\cf{(S\bs NP)\bs (S\bs NP)}} \\
& \bapply{5} \\
& \mc{5}{\cf{(S[pss]\bs NP)^{NP\bs NP}}} \\
& \unhat{5} \\
& \mc{5}{\cf{NP\bs NP}} \\
\bapply{6} \\
\mc{6}{\cf{NP}}
}
\end{center}

The derivation for reduced object relatives involves the hat passing via
composition described in Section~\ref{sec:hat_composition}:

\begin{center}
\deriv{4}{
\rm The~asbestos & \rm cigarette~filters & \rm once & \rm used \\
\uline{1}&\uline{1}&\uline{1}&\uline{1} \\
\cf{NP} &
\cf{NP} &
\cf{(S\bs NP)/(S\bs NP)} &
\cf{((S[pss]\bs NP)/NP)^{NP\bs NP}} \\
&  \ftype{1} & \fcomp{2} \\
&  \mc{1}{\cf{S/(S\bs NP)}} & \mc{2}{\cf{((S[pss]\bs NP)/NP)^{NP\bs NP}}} \\
& \fcomp{3} \\
& \mc{3}{\cf{(S[pss]/NP)^{NP\bs NP}}} \\
& \unhat{3} \\
& \mc{3}{\cf{NP\bs NP}} \\
\bapply{4} \\
\mc{4}{\cf{NP}}
}
\end{center}

The hat categories enable matching analyses of subject and object reduced relatives.
The hat category analysis allows all modifiers and arguments to receive their canonical
categories, with only the verb receiving a different category.

\subsection{Gerund Nominals}

Gerund nominals are a relatively infrequent construction in the Wall Street
Journal corpus, with approximately 300 occurrences. The construction involves a
non-finite verb phrase functioning as a noun phrase. As we describe in Section
\ref{ling_mot:nominal}, their distribution is the same as ordinary \cf{NP}s.
\ccgbank handles gerund nominals by using \psunary{\cf{VP}}{\cf{NP}}
type-changing rules. The \candc parser does not implement this rule, because
the increase in coverage is not worth the ambiguity the rule would introduce.

Hat categories offer a way to mimic the \ltag analysis, where the constituent is
ultimately \cf{NP}-typed but has a \cf{VP}-like structure (or, in the terms
introduced in Chapter~\ref{chapter:ling_mot}, functions as an \cf{NP} but has a
\cf{VP} constituent type):

\begin{center}
\deriv{8}{
\rm I & \rm gave & \rm doing & \rm things & \rm his & \rm way & \rm a & \rm
chance \\
\uline{1}&\uline{1}&\uline{1}&\uline{1}&\uline{1}&\uline{1}&\uline{1}&\uline{1}
\\
\cf{NP} &
\cf{((S[dcl]\bs NP)/NP)/NP} &
\cf{(S[ng]\bs NP)^{NP}/NP} &
\cf{N^{NP}} &
\cf{NP^{VP\bs VP}/N} &
\cf{N} &
\cf{NP/N} &
\cf{N} \\
&&& \unhat{1} & \fapply{2} & \fapply{2} \\
&&& \mc{1}{\cf{NP}} & \mc{2}{\cf{NP^{VP\bs VP}}} & \mc{2}{\cf{NP}} \\
&& \fapply{2} & \unhat{2} \\
&& \mc{2}{\cf{(S[ng]\bs NP)^{NP}}} & \mc{2}{\cf{(S\bs NP)\bs (S\bs NP)}} \\
&& \bapply{4} \\
&& \mc{4}{\cf{(S[ng]\bs NP)^{NP}}} \\
&& \unhat{4} \\
&& \mc{6}{\cf{NP}} \\
& \fapply{5} \\
& \mc{5}{\cf{(S[dcl]\bs NP)/NP}} \\
& \fapply{7} \\
& \mc{7}{\cf{S[dcl]\bs NP}} \\
\bapply{8} \\
\mc{8}{\cf{S[dcl]}}
}
\end{center}

This analysis correctly captures all of the relevant properties of the
construction:

\begin{itemize}
\item The distribution of the constituent is identical to any noun phrase;
\item The constituent is headed by the head of the verb phrase;
\item The verb phrase can be modified as normal.
\end{itemize}


\subsection{Reported Speech}
\label{sec:hat_speech}
Written genres that involve a lot of direct speech (such as narratives and news
text) arrange quotations in a variety of ways, and often place the projecting
verb inside the quotation. \ccgbank follows the Penn Treebank in analysing the
quotation as the matrix clause, and handling the projecting verb as a
parenthetical \citep{bies:95}:

\begin{center}
\deriv{5}{
\rm ``That~is~an~issue\mbox{''}, & \rm he & \rm said, & \rm ``but & \rm
we~can~handle~it\mbox{''} \\
\uline{1}&\uline{1}&\uline{1}&\uline{1}&\uline{1} \\
\cf{S[dcl]} &
\cf{NP} &
\cf{(S[dcl]\bs NP)/S[dcl]} &
\cf{conj} &
\cf{S[dcl]} \\
& \ftype{1} \\
& \mc{1}{\cf{S/(S\bs NP)}} \\
& \fcomp{2} \\
& \mc{2}{\cf{S[dcl]/S[dcl]}} \\
& \psgrule{2} \\
& \mc{2}{\cf{S\bs S}} \\
\bapply{3} & \conj{2} \\
\mc{3}{\cf{S[dcl]}} & \mc{2}{\cf{S[dcl][conj]}} \\
\conj{5} \\
\mc{5}{\cf{S[dcl]}}
}
\end{center}

\noindent This rule is also used to handle quotations where the projecting verb
follows the whole quotation:

\begin{center}
\deriv{5}{
\rm ``That~is~an~issue, & \rm but & \rm we~can~handle~it\mbox{''}, & \rm he &
\rm said \\
\uline{1}&\uline{1}&\uline{1}&\uline{1}&\uline{1} \\
\cf{S[dcl]} &
\cf{conj} &
\cf{S[dcl]} &
\cf{NP} &
\cf{(S[dcl]\bs NP)/S[dcl]} \\
&&& \ftype{1} \\
&&& \mc{1}{\cf{S/(S\bs NP)}} \\
& \conj{2} & \fcomp{2} \\
& \mc{2}{\cf{S[dcl][conj]}} & \mc{2}{\cf{S[dcl]/S[dcl]}} \\
\conj{3} & \psgrule{2} \\
\mc{3}{\cf{S[dcl]}} & \mc{2}{\cf{S\bs S}} \\
\bapply{5} \\
\mc{5}{\cf{S[dcl]}}
}
\end{center}

There are several benefits to lexicalising this construction. The construction
is specific to verbs of speech, so only a few, frequent verbs will require the
altered category. The construction is also locally decidable, so the
supertagger should have little trouble assigning the correct category. The
parsing model is capable of making the same distinction, of course --- but less
efficiently. The hat analysis looks like this:
 
\deriv{5}{
\rm ``That~is~an~issue\mbox{''}, & \rm he & \rm said, & \rm ``but & \rm
we~can~handle~it\mbox{''} \\
\uline{1}&\uline{1}&\uline{1}&\uline{1}&\uline{1} \\
\cf{S[dcl]} &
\cf{NP} &
\cf{((S[dcl]\bs NP)/S[dcl])^{(S\bs S)}} &
\cf{conj} &
\cf{S[dcl]} \\
& \ftype{1} \\
& \mc{1}{\cf{S/(S\bs NP)}} \\
& \fcomp{2} \\
& \mc{2}{\cf{(S[dcl]/S[dcl])^{(S\bs S)}}} \\
& \unhat{2} & \conj{2} \\
& \mc{2}{\cf{S\bs S}} & \mc{2}{\cf{S[dcl][conj]}} \\
\conj{5} \\
\mc{5}{\cf{S[dcl]}}
}

For both of these constructions, the subject can occur after the verb, although
usually not when the subject is a pronoun (presumably due to information
structure constraints, such as those that force pronoun objects to occur before
verb particles):

\begin{lexamples}
\item ``That is an issue'', said John, ``but we can handle it''.
\item ``That is an issue, but we can handle it'', said John.
\item * Said John, ``that is an issue, but we can handle it''.
\end{lexamples}

The subject-verb alternation is handled with category ambiguity in \ccgbank, an
analysis which we follow.


\subsection{Extraposed NPs and Other Punctuation Cued Type-Changing}
\label{sec:extraposition}

Appositive noun phrases can be extraposed out of a sentence or verb phrase.
However, in standard written English, a
comma is required before or after the appositive:
\begin{lexamples}
 \item \emph{No dummies}, the drivers pointed out they still had space.
 \item Factory inventories fell 0.1\% in September, \emph{the first decline
since February 1987}.
\end{lexamples}

\ccgbank uses the comma as a cue for binary type-changing rules:

\begin{eqnarray}
\eqnpsrule{\cf{,}}{\cf{NP}}{\cf{S\bs S}}\\
\eqnpsrule{,}{\cf{NP}}{\cf{(S\bs NP)\bs (S\bs NP)}}\\
\eqnpsrule{\cf{NP}}{\cf{,}}{\cf{S/S}}
\end{eqnarray}

The binary rules introduce much less ambiguity than the equivalent unary rules,
such as \psunary{\cf{NP}}{\cf{S\bs S}}. One way to lexicalise these rules ---
and other punctuation cued type-changing --- is to assign a category to the
punctuation mark that performs the transformation, analysing it as a kind of
function word:

\begin{center}
\deriv{3}{
\rm Factory~inventories~fell~0.1\% & \rm , & \rm
the~first~decline~since~February~1987 \\
\uline{1}&\uline{1}&\uline{1} \\
\cf{S[dcl]} &
\cf{(S\bs S)/NP} &
\cf{NP} \\
& \fapply{2} \\
& \mc{2}{\cf{S\bs S}} \\
\bapply{3} \\
\mc{3}{\cf{S[dcl]}}
}
\end{center}

Assigning a complex category to the punctuation is acceptable because it is
phonologically realised. It represents an obligatory distinction in the way the
sentences are spoken. However, this analysis raises semantic problems. The
\cf{(S\bs S)/NP} category assigned to the punctuation does not allow a
dependency to be created between \emph{decline} and \emph{fell}, so the
dependency graph will not be connected, with the extraposed element left without
a head. This is actually what happens in the current \ccgbank analysis, because
the type-changing rule encounters a similar problem.

The hat category analysis puts the \cf{S} argument into the lexical category of
the \cf{NP}, allowing the dependency to be created:

\begin{center}
\deriv{2}{
\rm Factory~inventories~fell~0.1\%, & \rm the~first~decline~since~February~1987
\\
\uline{1}&\uline{1} \\
\cf{S[dcl]} &
\cf{NP^{S\bs S}} \\
& \unhat{1} \\
& \mc{1}{\cf{S\bs S}} \\
\bapply{2} \\
\mc{2}{\cf{S[dcl]}}
}
\end{center}

The punctuation becomes insignificant in the hat analysis, after the
type-changing rule is unarised as described in Section~\ref{sec:unarising}. The
punctuation symbol is no longer a hard constraint on the rule. Instead, the
supertagger will be able to use it as a contextual cue to decide that the coming
category might be extraposed, in much the same way that a human might. Factoring
punctuation out of the grammar is also desirable, because it allows the analyses
to be ported to domains where punctuation is used less reliably.

\citet{white:punct08} implements an alternate treatment of punctuation cued
type-changing in \ccgbank, using the analyses suggested for \ltag by
\citet{doran:98}. They find that these corrections improve the accuracy
of a statistical \ccg surface realiser, making their analyses
an interesting target for future work.

\subsection{Unlike Coordinated Phrases}

One of the strengths of a partially associative categorial grammar like \ccg is
the attractive analysis of non-constituent coordination, as we describe in
Section~\ref{sec:associativity}. However, some unlike coordinated phrases still
pose a problem. For instance, some verbs have arguments which can be realised by
multiple constituent types, allowing two arguments with different syntactic
categories to be conjoined based on their shared function:

\begin{lexamples}
\item These actions are risky and not in the best interests of the public.
\item Compound yields assume reinvestment of dividends and that the current
yield continues.
\end{lexamples}

These constructions could be analysed better with a type hierarchy, following
the proposal of \citet{mcconville:06}. This would allow the verb to specify a
less general argument type, with inheritance mechanisms ensuring that this does
not result in redundancy. Something similar can be seen in the analysis of the
second example, where the verb requires the argument \cf{S[adj]\bs NP}, which
can be realised by a variety of different constituent types.

\ccgbank uses a binary rule cued by the coordinator. The coordinator makes the
type-change rule less ambiguous, in much the same way punctuation is used to in
binary rules, as we describe in Section~\ref{sec:extraposition}. After the
binary rule is unarised, we perform the standard flattening of the unary rule
into a hat category. Here is an example:

\begin{center}
\deriv{5}{
\rm These~actions & \rm are & \rm risky & \rm and & \rm
not~in~our~best~interests \\
\uline{1}&\uline{1}&\uline{1}&\uline{1}&\uline{1} \\
\cf{NP} &
\cf{(S[dcl]\bs NP)/(S[adj]\bs NP)} &
\cf{S[adj]\bs NP} &
\cf{conj} &
\cf{PP^{(S[adj]\bs NP)}} \\
&&&& \unhat{1} \\
&&&& \mc{1}{\cf{S[adj]\bs NP}} \\
&&& \conj{2} \\
&&& \mc{2}{\cf{S[adj]\bs NP[conj]}} \\
&& \conj{3} \\
&& \mc{3}{\cf{S[adj]\bs NP}} \\
& \fapply{4} \\
& \mc{4}{\cf{S[dcl]\bs NP}} \\
\bapply{5} \\
\mc{5}{\cf{S[dcl]}}
}
\end{center}

Sometimes, this strategy still produces suboptimal analyses, as in the following
example, where the constituent \emph{that the yield continues} cannot reasonably
be analysed as an \emph{NP}:

\begin{center}
\deriv{5}{
\rm Those~figures & \rm assume & \rm reinvestment & \rm and & \rm
that~the~yield~continues \\
\uline{1}&\uline{1}&\uline{1}&\uline{1}&\uline{1} \\
\cf{NP} &
\cf{(S[dcl]\bs NP)/NP} &
\cf{NP} &
\cf{conj} &
\cf{S[em]^{NP}} \\
&&&& \unhat{1} \\
&&&& \mc{1}{\cf{NP}} \\
&&& \conj{2} \\
&&& \mc{2}{\cf{NP[conj]}} \\
&& \conj{3} \\
&& \mc{3}{\cf{NP}} \\
& \fapply{4} \\
& \mc{4}{\cf{S[dcl]\bs NP}} \\
\bapply{5} \\
\mc{5}{\cf{S[dcl]}}
}
\end{center}


\section{The \nounary Corpus}
\label{sec:nounary_corpus}

The \nounary corpus is a fully lexicalised corpus that does not employ any
extensions to the grammar. It implements a purely application, composition and
type-raising (\act) grammar. The \candc parser is somewhat ill-suited to a
purely \act analyses, as it was designed for \ccgbank. A wide-coverage, purely
combinatory corpus would greatly benefit from a structured lexicon, which the
\candc parser architecture does not currently support. Purely combinatory
analyses are also more sensitive to analysis quality and noise than the \ccgbank
grammar, so the corpus described in this section is far from ideal.

For the \hatsys corpus, we were able to translate the \ccg analysis directly.
This is not an option for the \nounary corpus, because the only unary
productions that are allowed in its grammar are instances of type-raising. This
means that we first have to decide on a set of analyses for the phenomena
\ccgbank handles using type-changing rules. We settle on a strategy that
largely adopts a simple transformation, so that the corpus is more closely
comparable to the Hat corpus and \ccgbank. The general strategy for compressing
a unary rule is to replace the child category with the parent category, relying
on the rules described in Section~\ref{sec:node_relabelling} to propagate the
changes down the subtree. This amounts to always assigning function-based
categories whenever there is a form/function distinction.

This section describes the most common constructions that are relabelled in the
\nounary corpus, paying particular attention to constructions where we have
deviated from the general relabelling strategy. We start by describing the
analysis for bare noun phrases, before describing the analysis for bare
relatives and reported speech. Many of the issues that come up here were also
discussed in Chapter~\ref{chapter:ling_mot}.

\subsection{Bare Noun Phrases}

Bare noun phrases are analysed with the default strategy: the \cf{N} is replaced
by an \cf{NP}:

\begin{eqnarray}
 \ptbegtree
   \ptbeg \ptnode{\cf{NP}}
     \ptbeg \ptnode{\cf{N}}
       \ptbeg \ptnode{\cf{N/N}} \ptleaf{dangerous} \ptend
       \ptbeg \ptnode{\cf{N}} \ptleaf{asbestos} \ptend
     \ptend
   \ptend
  \ptendtree
&
\longrightarrow
&
 \ptbegtree
   \ptbeg \ptnode{\cf{NP}}
     \ptbeg \ptnode{\cf{NP/NP}} \ptleaf{dangerous} \ptend
     \ptbeg \ptnode{\cf{NP}} \ptleaf{asbestos} \ptend
   \ptend
  \ptendtree
\end{eqnarray}

The modifier category ambiguity that results from this is problematic,
particularly if the analysis is left-branching, e.g. \emph{very dangerous
asbestos}. We see no way to avoid this ambiguity with a pure \ccg analysis.

\subsection{Reduced Relative Clauses}
\label{sec:pure_rrc}

As we describe in Section~\ref{sec:ling_rrc}, \ccg offers two possible ways to
analyse reduced relative clauses without extending the grammar. The best
approach is to have the noun subcategorise for the relative clause, because
changing the verb category to have an \cf{NP}-typed inner most result breaks the
assumptions that \citet{steedman:00} builds into the grammar about different
behaviours of \cf{NP\$} and \cf{S\$} types. The analysis looks like this:

\begin{center}
\deriv{6}{
\rm asbestos & \rm once & \rm used & \rm for & \rm cigarette & \rm filters \\
\uline{1}&\uline{1}&\uline{1}&\uline{1}&\uline{1}&\uline{1} \\
\cf{NP/(S[pss]\bs NP)} &
\cf{(S\bs NP)/(S\bs NP)} &
\cf{(S[pss]\bs NP)/PP} &
\cf{PP} &
\cf{NP/NP} &
\cf{NP} \\
&&&& \fapply{2} \\
&&&& \mc{2}{\cf{NP}} \\
&&& \fapply{3} \\
&&& \mc{3}{\cf{PP}} \\
&& \fapply{4} \\
&& \mc{4}{\cf{S[pss]\bs NP}} \\
& \fapply{5} \\
& \mc{5}{\cf{S[pss]\bs NP}} \\
\fapply{6} \\
\mc{6}{\cf{NP}}
}
\end{center}

This transformation is implemented as a set of rewrite rules that are run as a
preprocess before the main node changing algorithm. The rules are:

\begin{eqnarray}
\ptbegtree
\ptbeg \ptnode{\cf{NP}}
  \ptleaf{\cf{NP\bs NP}}
  \ptleaf{\cf{S[*]\bs NP}}
\ptend
\ptendtree
\longrightarrow
\ptbegtree
\ptbeg \ptnode{\cf{NP}}
  \ptleaf{\cf{NP/(S\bs NP)}}
  \ptleaf{\cf{S[*]\bs NP}}
\ptend
\ptendtree
\\
\ptbegtree
\ptbeg \ptnode{\cf{NP}}
  \ptleaf{\cf{NP\bs NP}}
  \ptleaf{\cf{S[*]/NP}}
\ptend
\ptendtree
\longrightarrow
\ptbegtree
\ptbeg \ptnode{\cf{NP}}
  \ptleaf{\cf{NP/(S/NP)}}
  \ptleaf{\cf{S[*]/NP}}
\ptend
\ptendtree
\end{eqnarray}

Where $*$ represents a feature value that is transmitted across to the rewritten
tree. This subcategorisation analysis of reduced relative clauses caused some
problems for the corpus. \ccgbank inherits its complement\slash adjunct distinctions
from the Penn Treebank, which results in very few nominal predicates. This means
that prepositional phrases that we would analyse as complements often occur
between the head noun and the verb phrase. When this happens, the prepositional
phrases are forced to subcategorise for the \cf{NP}'s argument structure,
causing problematic analyses like the following:

\begin{center}
\small
\deriv{7}{
\rm The & \rm effects & \rm of & \rm this & \rm policy & \rm seen & \rm today \\
\uline{1}&\uline{1}&\uline{1}&\uline{1}&\uline{1}&\uline{1}&\uline{1} \\
\cf{NP/N} &
\cf{N/(S[pss]\bs NP)} &
\cf{((NP/(S\bs NP))\bs ((NP/(S\bs NP)))/NP} &
\cf{NP/N} &
\cf{N} &
\cf{S[pss]\bs NP} &
\cf{(S\bs NP)\bs (S\bs NP)} \\
\fcomp{2} && \fapply{2} & \bapply{2} \\
\mc{2}{\cf{NP/(S[pss]\bs NP)}} && \mc{2}{\cf{NP}} & \mc{2}{\cf{S[pss]\bs NP}} \\
&&\fapply{3}\\
&&\mc{3}{\cf{NP/(S[pss]\bs NP)}}\\
\bapply{5} \\
\mc{5}{\cf{NP/(S[pss]\bs NP)}} \\
\fapply{7} \\
\mc{7}{\cf{NP}}
}
\end{center}

Prepositional phrases attach at the \cf{NP} level in \ccgbank, so our node
relabelling rules correctly cause the determiner to compose with \emph{effects}
to produce the \cf{NP/(S\bs NP)} category required. However, the prepositional
phrase is prevented from cross-composing into the \cf{NP}-rooted category by the
\citet{steedman:00} crossing composition constraints used in our grammar. \mmccg
would not help here either, because \cf{NP} modifiers cannot be assigned
permutative modes, lest they over-generate analyses for sentences such as
\emph{the tasty with anchovies pizza} that have \emph{with anchovies} modifying
\emph{pizza} \citep{baldridge:03}.

The root cause of our problem here is not the grammar, it is that the Penn
Treebank (and therefore \ccgbank) has the wrong analysis. \emph{of this policy}
is not an adjunct, it is a complement --- which is why it is allowed to occur
before the reduced relative clause (which we do believe is an adjunct). We could
reanalyse the prepositional phrases as arguments when this specific conflict
occurs, but then the corpus would be inconsistent, because nouns would only
subcategorise for prepositional phrases in the presence of reduced relative
clauses.

What we need is a way of reliably determining noun phrase argument structure.
Probably the best solution would be to use Nombank \citep{nombank} for this
purpose, employing the strategy used with Propbank by
\citet{honnibal:pacling07prop} and \citet{boxwell:08}. We have not pursued this
for the \nounary corpus, following the principle that we should not embark on
changes to \ccgbank outside the scope of the unary rules we are removing, to
ensure that our corpora are as comparable as possible.

The problems with this construction suggest that the additional descriptive
power provided by the \ccgbank type-changing rules or the hat categories we
have introduced help mitigate the problems caused by suboptimal analyses. In the
\nounary corpus, analysis problems tend to propagate through the derivation,
because of the increased sensitivity of modifiers to their heads' categories.

\subsection{Reported Speech}

The analyses described in Section~\ref{sec:hat_speech} would require a number of
extra categories in the \nounary corpus, which does not seem worthwhile to us.
Instead, we analyse these sentences as though the projecting verb were the head,
as it is when the quotation follows the verb. This introduces three argument
structure alternations for verbs of speech:

\begin{enumerate}
\item \cf{(S[dcl\bs NP)/S[dcl]}: \emph{John said, ``...''}
\item \cf{(S[dcl\bs NP)\bs S[dcl]}: \emph{``...'', John said}
\item \cf{(S[dcl/NP)\bs S[dcl]}: \emph{``...'', said John}
\end{enumerate}

These modifications are also implemented as preprocessing rewrite rules. The
rules are:

\begin{eqnarray}
&
\ptbegtree
\ptbeg \ptnode{\cf{S\bs S}}
  \ptbeg \ptnode{\cf{S[dcl]/S[dcl]}}
    \ptleaf{\cf{S/(S\bs NP)}}
    \ptleaf{\cf{(S[dcl]\bs NP)/S[dcl]}}
  \ptend
\ptend
\ptendtree
&\longrightarrow
\ptbegtree
\ptbeg \ptnode{\cf{S[dcl]\bs S[dcl]}}
  \ptleaf{\cf{S/(S\bs NP)}}
  \ptleaf{\cf{(S[dcl]\bs NP)\bs S[dcl]}}
\ptend
\ptendtree
\\
&
\ptbegtree
\ptbeg \ptnode{\cf{S\bs S}}
  \ptbeg \ptnode{\cf{S[dcl]/S[dcl]}}
    \ptleaf{\cf{(S[dcl]/S[dcl])/NP}}
    \ptleaf{\cf{NP}}
  \ptend
\ptend
\ptendtree
&\longrightarrow
\ptbegtree
\ptbeg \ptnode{\cf{S[dcl]\bs S[dcl]}}
  \ptleaf{\cf{(S[dcl]\bs S[dcl])/NP}}
  \ptleaf{\cf{NP}}
\ptend
\ptendtree
\end{eqnarray}

\section{The Hat-TR Corpus}
\label{sec:hattr_corpus}
\begin{figure}
 \centering
\deriv{6}{
\rm Casey & \rm likes & \rm but & \rm Erin & \rm hates & \rm Pat \\
\uline{1}&\uline{1}&\uline{1}&\uline{1}&\uline{1}&\uline{1} \\
\cf{NP^{S/(S\bs NP)}} &
\cf{(S[dcl]\bs NP)/NP} &
\cf{conj} &
\cf{NP^{S/(S\bs NP)}} &
\cf{(S[dcl]\bs NP)/NP} &
\cf{NP} \\
\unhat{1} &&& \unhat{1} \\
\mc{1}{\cf{S/(S\bs NP)}} &&& \mc{1}{\cf{S/(S\bs NP)}} \\
\fcomp{2} && \fcomp{2} \\
\mc{2}{\cf{S[dcl]/NP}} && \mc{2}{\cf{S[dcl]/NP}} \\
&& \conj{3} \\
&& \mc{3}{\cf{S[dcl]/NP[conj]}} \\
\conj{5} \\
\mc{5}{\cf{S[dcl]/NP}} \\
\fapply{6} \\
\mc{6}{\cf{S[dcl]}}
}
\caption{Example of forward type-raising lexicalised with a hat
category.\label{hattr_ftype}}
\end{figure}

Type-raising is an open ended unary rule: it can transform any \cf{NP},
\cf{S[adj]\bs NP} or \cf{PP} category into an unbounded number of other
logically equivalent categories. This unbounded productivity is problematic for a
chart parser, because local ambiguities can have a substantial impact on efficiency
even if they cannot be used to generate alternate analyses that span the whole sentence.
The \candc parser therefore implements type-raising as a pre-specified
set of unary transformations that cover most of the type-raising that occurs in
\ccgbank. This still introduces a lot of ambiguity, so we experiment with
lexically specifying the type-raise rules using hat categories, instead.

Lexicalising type-raising rules may bring smaller chart sizes, which should
produce faster parse times, and possibly higher accuracy. It also prevents the
% developer of a practical parsing system from having to decide whether to include
a given type-raising rule is worth the extra ambiguity it introduces.

The main question about this analysis is how locally decidable type-raising
really is. If the evidence required is not present inside the supertagger's
context window, then the tagger has little opportunity to make the right decision. This
will cause a reduction in speed and accuracy, as the required type-raise
category will not be highly ranked in the tagger's analysis.

The implementation of the hat-base reanalysis of type-raising was
straightforward. We simply collapse type-raising rules using the unary
collapsing function described in Section~\ref{sec:hat_corpus}. This causes the
type-raising rules to be lexicalised in exactly the same way as the unary
type-changing rules. An example of lexicalised forward type-raising is shown
in Figure~\ref{hattr_ftype}.

Figure~\ref{hattr_btype} shows an example of lexicalised backward type-raising.
The gist of this analysis is that the two argument clusters must coordinate,
which means that each cluster must form a constituent. They do this by both
type-raising into functions over the verb phrase category that requires their
argument, and then composing. This is the standard \ccg analysis of argument
cluster coordination, which is a particularly good example of \ccg's flexible
treatment of coordination \citep{steedman:00}.

\begin{sidewaysfigure}
 \centering
\deriv{8}{
\rm They & \rm held & \rm a & \rm treaty & \rm in & \rm one & \rm hand & \rm and~a~pistol~in~the~other \\
\uline{1}&\uline{1}&\uline{1}&\uline{1}&\uline{1}&\uline{1}&\uline{1}&\uline{1} \\
\cf{NP} &
\cf{((S[dcl]\bs NP)/PP)/NP} &
\cf{NP^{((S\bs NP)/PP)\bs (((S\bs NP)/PP)/NP)}/N} &
\cf{N} &
\cf{PP^{(S\bs NP)\bs ((S\bs NP)/PP)}} &
\cf{NP/N} &
\cf{N} &
\cf{(S\bs NP)\bs (((S\bs NP)/PP)/NP)[conj]} \\
&& \fapply{2} && \fapply{2} \\
&& \mc{2}{\cf{NP^{((S\bs NP)/PP)\bs (((S\bs NP)/PP)/NP)}}} && \mc{2}{\cf{NP}} \\
&& \unhat{2} & \fapply{3} \\
&& \mc{2}{\cf{((S\bs NP)/PP)\bs (((S\bs NP)/PP)/NP)}} & \mc{3}{\cf{PP^{(S\bs NP)\bs ((S\bs NP)/PP)}}} \\
&&&& \unhat{3} \\
&&&& \mc{3}{\cf{(S\bs NP)\bs ((S\bs NP)/PP)}} \\
&& \bcomp{5} \\
&& \mc{5}{\cf{(S\bs NP)\bs (((S\bs NP)/PP)/NP)}} \\
&& \conj{6} \\
&& \mc{6}{\cf{(S\bs NP)\bs (((S\bs NP)/PP)/NP)}} \\
&\bapply{7} \\
&\mc{7}{\cf{S[dcl]\bs NP}} \\
\bapply{8} \\
\mc{8}{\cf{S[dcl]}}
}
\caption{Example of backward type-raising lexicalised with a hat
category.\label{hattr_btype}}
\end{sidewaysfigure}


\section{Validating Changes}

\label{sec:validation}

Since we experimented with a variety of different corpus configurations, it was
not practical to construct a manually converted evaluation set to compare our
adaptation processes against. However, the conversion process can introduce
subtle problems, and an unintended analysis applied to a relatively small
portion of sentences can completely obscure results. We used a variety of
diagnostic measures to try to validate our changes.

\subsection{Rule Validation}

Rule validation involved checking whether each production in the derivation
could be produced by a \ccg combinatory rule. While it is not sufficient for a
derivation to be the product of the \ccg combinatory rules, it is certainly
necessary. Rule validation was mostly used to catch bugs in the conversion code
and node relabelling algorithm. We also performed a second rule validation,
using approximations of the composition heuristics implemented in the \candc
parser. This helped us determine whether our analyses would be invalid with the
parser's restricted implementation.

A minority of sentences persistently fail validation in each of the corpora, due
to the long tail of noisy derivations in \ccgbank. The noise is due to
inaccuracies in the Penn Treebank and imperfections in \ccgbank's conversion
heuristics.

\subsection{Examining the Lexicon and Grammar}

Once the corpus has been converted with a minority of rule failures, we inspect
the lexicon and grammar to see how many categories have been added, and how the
distribution of combinatory rules required has changed. This helps us to
determine whether our analysis has had unintended consequences. The first thing
we do is look at the category coverage of the corpus if we annotate a given
number of categories. What we are looking at here is how much more sparse our
analyses have made the corpus. We then look at the individual categories to be
added, as well as a frequency-sorted list of novel productions that have been
introduced. This helps us to find interactions between our rules and \ccgbank's
analyses that we had not anticipated.

For instance, an analysis of the novel categories led us to the problems
discussed in Section~\ref{sec:pure_rrc}. We were initially surprised by the
frequency of the following strange looking categories, which were assigned to
prepositions:

\begin{eqnarray}
 \cf{(NP/(S\bs NP))/(NP/(S\bs NP))/NP}\\
 \cf{(NP/(S/NP))/(NP/(S/NP))/NP}
\end{eqnarray}

When we looked at where these categories occurred, we encountered the
interaction between the complement\slash adjunct distinction errors in \ccgbank and
the new subcategorisation frames.

\subsection{Examining Parser Training Failures}


Once we had examined the grammar and lexicon, and added annotation for frequent
categories to the \markedup file, we attempted to train the parser. During
training, the \candc parser attempts to reproduce the gold standard \ccgbank
derivation for each sentence. If it cannot, the sentence is discarded from the
training set. There are a few ways the parser might be unable to reproduce the
correct analysis:

\begin{enumerate}
 \item If it contains a category with no \markedup entry;
 \item If it contains a production that cannot be produced with any implemented
rules;
 \item If the annotation on the categories' \markedup entries causes a head
conflict that makes the parse fail.
\end{enumerate}

As we saw in Section~\ref{mu_annotation}, the category coverages of our corpora
are very similar to \ccgbank's once we have updated the \markedup file, so
binary rule errors are the focus of our attention when examining training
failures. We debug the training failures by sorted the categories by the
percentage of sentences they occur in that fail. If a category causes parse
failures almost every time it occurs, there is likely to be something wrong with
the analyses that produce it, or with its \markedup annotation.

\section{Summary}

We have described the creation of three lexicalised corpora. During this
process, we encountered practical examples of many of the problems discussed in
Chapter~\ref{chapter:ling_mot}, particularly when creating the \nounary corpus.
We found that problems with the \ccgbank analyses for many constructions made it
very difficult to compile out unary rules while maintaining consistent,
linguistically desirable and efficient analyses using only application,
composition and type-raising rules. While the \ccgbank analyses might be
considered the root cause of many of the issues we have discussed for the
\nounary corpus, it was also clear that there was another factor involved. The
lack of descriptive power provided by the \nounary corpus's grammar made it very
sensitive to any annotation noise, because of the undesirable dependence between
modifier categories and their heads. If there was even a small problem with one
part of the analysis, it tended to spread throughout the rest of the derivation,
causing sparse data problems.

In contrast, the \hatsys corpus's grammar had almost as much descriptive power
as the original \ccgbank, because the unary rules could be easily replicated. We
did encounter some problems caused by our implementation of hat categories,
which relied on category unification rather than full coindexing. This caused
some undesirable analyses, with hats assigned to determiners instead of head
nouns. We presented several constructions where the \hatsys corpus was able to
implement a favourable, lexicalised analysis. These examples, and our
observations during the conversion process, suggest to us that the \hatsys
corpus is cleaner and more consistent than the \nounary corpus. We investigate
this in the following chapter, by comparing the performance of \candc parsers
adapted for use with these corpora against a parser trained on \ccgbank.

% 
\chapter{Parsing Experiments}
\label{chapter:results}

This chapter explores the consequences of increasing the level of lexicalisation
in \ccgbank on the performance of a \ccg parser. Chapter
\ref{chapter:hat_corpus} described the creation of three corpora implementing
different lexicalisation strategies, created by adapting \ccgbank. The first
corpus we evaluate lexicalises the \ccgbank type-changing rules using the hat
categories introduced in Chapter \ref{chapter:hat_cats}. We label this corpus
and parsers trained on it \hatsys. The second corpus lexicalises the rules
without hat categories, by compiling out the unary rules. We label this corpus
and the parsers trained on it \nounary, although it does use unary type-raising
rules. The third corpus lexicalises type-raising rules as well as
type-changing rules using hat categories. We label this corpus \trsys.

As well as performing experiments on the standard development and test set, we
address a possible problem with making the grammar more lexicalised. Increasing
lexicalisation might increase the domain dependence of the parser, because the
supertagger's model assumes that all valid pairings of words and categories have
been seen in the training data for moderately frequent words. To investigate
this, we perform the first evaluation of the \candc parser on Wikipedia. This
also serves as a more practical assessment of our progress on \ccg parsing. It
allows us to evaluate the parser on data that has immediate practical value, and
check how well our previous conclusions generalise to new data.

Before we describe our experiments, we briefly discuss some similar studies,
which have also involved training the \candc parser on altered versions of
\ccgbank. We then briefly review the experiments discussed in
\citet{clark:cl07}, to establish the parameters that we may need to alter to
accommodate our new analyses. Next, we describe how we can slightly improve their
results by searching for good values of important run-time parameters. We also
describe how we can compare results from an altered version of \ccgbank against
the original, by mapping dependency labels so that the output of the new model
matches the \ccgbank analyses.

Having dealt with these methodological preliminaries, we turn our attention to
the lexicalised corpora. First, we perform a series of development experiments,
to select the best configuration for each corpus. We then evaluate the selected
models on the \wsj test set. Finally, we describe the Wikipedia test set, before
performing the out-of-domain evaluation.


\section{Previous Experiments with \ccgbank Adaptations}
\label{sec:previous_problems}

There have been a few previous experiments that have trained the \candc parser
on altered versions of \ccgbank. \citet{honnibal:pacling07prop},
\citet{vadas:08} and \citet{tse:08} all corrected various problems with the
\ccgbank (predicate argument structure, noun phrase structure and punctuation,
respectively), and retrained and evaluated the parser. However, all of these
studies have shared subtle flaws that have made their results more difficult to
interpret.

\citet{honnibal:pacling07prop}, \citet{vadas:08} and \citet{tse:08} all present
results using one of the default \candc parsing models.
The model used for all of these experiments, described as \derivsbad in
\ref{training_betas}, achieves a labelled dependency $F$-score of 85.12\%
using gold standard part of speech tags. This is 1.62\% less accurate than a
model which differs only on one training parameter, and 2.2\% than the best
result \citet{clark:cl07} report, which uses a different probability model.
The use of this less accurate
parsing model makes results difficult to interpret. Changes in the corpus change
the difficulty of the inference problem, and we cannot easily predict how that
difference in difficulty interacts with a larger difference in model quality.

For instance, \citet{honnibal:pacling07prop} found that the Propbank-compatible
predicate-argument structures they produced made the parsing task more
difficult, causing a drop in accuracy. \citet{vadas:08} reported a similar
problem: the parsing task became more difficult when non-trivial noun phrase
brackets were introduced, and parsing performance dropped using the weaker
model. However, it is possible that the more difficult corpus would not have
been a problem for the more accurate model. \citet{vadas:08} also introduced new
features to mitigate the drop in performance. These features may have had more
impact with the better model, or they may have represented information the
model's additional features already took into account.

Another problem with previous \ccgbank adaptations is that little attention was
paid to the system's configuration parameters. In general, only the default
values were used, even though these defaults may not be optimal for the new
corpora. Some of the parameters, notably the $\beta$ levels described in Section
\ref{beta_k}, can have a substantial impact on performance.
%This is less of a concern than the use of the wrong model, but may also have
affected the conclusion by artificially lowering the results on the altered
corpora.

Finally, \citet{honnibal:pacling07prop} and \citet{vadas:08} also encountered a
methodological problem that was difficult to avoid. By changing \ccgbank, they
also changed the test set, making their results difficult to compare against the
original dependencies. In both of these cases, it was fairly clear that the test
set had become harder, but it was unclear whether the corpus had somehow become
noisier or less consistent, or even less noisy and more consistent. Two
variables had changed. In Section \ref{sec:dependency_mapping}, we describe how
we avoid this problem by mapping the new corpus's dependencies into \ccgbank's,
so that we can evaluate against the original \ccgbank.
\citeauthor{honnibal:pacling07prop} could have done something similar, by
producing a pivot mapping where all \cf{PP} complement labels were mapped to
adjunct labels, allowing a common comparison point against the same
dependencies.

These previous results illustrate a number of issues. First, we have to be
conscious of the fact that the \candc parser, like most \nlp systems of similar
complexity, involves many configuration options and models, and we have to make
sure we are choosing just the right configuration to enable informative
comparison. Second, we should reoptimise the relevant parameters ourselves,
rather than relying on the default values that may not be optimal for our
corpus. Finally, we should do our best to factor out the alternative explanation
for our results, and try to evaluate on the original \ccgbank where possible.

\section{Evaluation Framework}
\label{sec:evaluation_framework}
\begin{table}
\centering
 \begin{tabular}{c|llcc|l}
\hline
 &  \multicolumn{4}{c|}{Labelled}                        & Unlabelled \\
1&  Head      & Arg   & Functor                     &Slot& (Head, Arg)\\
\hline
\hline
2&  gave      & Pat   & \cf{((S[dcl]\bs NP_1)/PP_2)/NP_3} & 1  & (give, Pat)\\
3&  gave      & stamp & \cf{((S[dcl]\bs NP_1)/PP_2)/NP_3} & 3  & (give, stamp)\\
4&  gave      & to    & \cf{((S[dcl]\bs NP_1)/PP_2)/NP_3} & 2  & (give, to)\\
5&  yesterday & give  & \cf{(S\bs NP)\bs (S\bs NP)_1} & 1  & (yesterday, give)\\
6&  a         & stamp & \cf{NP/N_1}                   & 1  & (a, stamp)\\
7&  rare      & stamp & \cf{N/N_1}                    & 1  & (rare, stamp)\\
\hline
 \end{tabular}
\caption[Sample \ccgbank dependencies.]{Sample \ccgbank dependencies for the
sentence in Figure \ref{fig:dependencies}\label{tab:dependencies}}
\end{table}

\begin{figure}
\centering
 \deriv{8}{
\rm Pat & \rm gave & \rm a & \rm rare & \rm stamp & \rm to & \rm Robin & \rm
yesterday \\
\uline{1}&\uline{1}&\uline{1}&\uline{1}&\uline{1}&\uline{1}&\uline{1}&\uline{1}
\\
\cf{NP} &
\cf{((S[dcl]\bs NP)/PP)/NP} &
\cf{NP/N} &
\cf{N/N} &
\cf{N} &
\cf{PP/NP} &
\cf{NP} &
\cf{(S\bs NP)\bs (S\bs NP)} \\
&&& \fapply{2} & \fapply{2} \\
&&& \mc{2}{\cf{N}} & \mc{2}{\cf{PP}} \\
&& \fapply{3} \\
&& \mc{3}{\cf{NP}} \\
& \fapply{4} \\
& \mc{4}{\cf{(S[dcl]\bs NP)/PP}} \\
& \fapply{6} \\
& \mc{6}{\cf{S[dcl]\bs NP}} \\
& \bapply{7} \\
& \mc{7}{\cf{S[dcl]\bs NP}} \\
\bapply{8} \\
\mc{8}{\cf{S[dcl]}}
}
\caption{Derivation producing the dependencies shown in Table
\ref{tab:dependencies}\label{fig:dependencies}}
\end{figure}


We follow the \ccgbank evaluation framework of \citet{clark:cl07} to assess the
speed and accuracy of our parsing models. We use their evaluation scripts and
architecture, so our baseline speed and accuracy figures exactly match the
figures reported in \citet{clark:cl07}.

The main accuracy evaluation is labelled dependency $F$-score. A \ccgbank
labelled dependency is a tuple consisting of the head word, the argument word,
the category assigned to the head word (which must, necessarily, be a functor
category specifying some number of arguments\footnote{Some of the type-changing rules in
\ccgbank add arguments to a category. When this occurs, the dependency is
unrepresented. This problem is discussed in Section \ref{sec:ling_psg_rules}.}),
and the index of the argument slot that the dependency fills. All elements of
the tuple must be correct for the dependency to be counted as correct.
Unlabelled dependency scores are also calculated, consisting of just the head
word and its argument. The labelled and unlabelled dependencies that arise from
the derivation in Figure \ref{fig:dependencies} are shown in Table
\ref{tab:dependencies}.

The labelled dependency $F$-score is a rather punitive measure, because they are
conditioned on lexical categories. For instance, consider the dependencies that
arise in the sentence \emph{Pat gave a rare stamp to Robin yesterday},
especially the dependencies between \emph{give} and its three arguments. The
subject dependency between \emph{give} and \emph{Pat} is conditioned over the
whole category, so if the dependency between \emph{give} and \emph{to} is
mislabelled as an adjunct dependency, this dependency will be judged incorrect
--- even though the parser has returned the correct relationship between the two
words. The same is true for the object dependency. Unlabelled dependencies can
also be difficult to predict, because they still represent complement/adjunct
distinctions. If \emph{yesterday} were incorrectly judged a complement, rather
than an adjunct, the unlabelled dependency would become \emph{give, yesterday}.
\ccg unlabelled dependencies therefore make distinctions not captured in \penn
skeletal brackets.

On the other hand, many dependencies are fairly trivial. Dependencies 6 and 7 in
Table \ref{tab:dependencies} would be accurately assigned by even a very simple
model. There are a great many dependencies that are this easy to predict in
\ccgbank. This combination of very hard and very easy decisions can make the
magnitude of dependency accuracies look misleading. Almost all experimental
systems will score over 75\% labelled $F$-score, but no system yet developed
will score 90\%. The evaluation therefore has quite a small dynamic range. A
system that is far superior might only score 2\% better on labelled dependency
$F$-score.

Despite this short-coming, dependency $F$-score is the most direct evaluation on
\ccgbank. Measures like the grammatical relations evaluation developed by
\citet{briscoe:poster06} are good for comparing the parser to other systems, and
abstract away the undesirable specificity of the \ccg dependencies. The problem
is that converting the output of the parser is very difficult.
\citet{clark:acl07parseval} report an upper bound of 84.8\%, calculated by
converting the gold standard \ccgbank dependencies --- although, interestingly,
the \candc parser still outperformed the \rasp parser on this evaluation. Our
concern is that such a low upper bound will further compress the range of
results achieved by the systems we are comparing, adding an additional
complication. We therefore use labelled dependency scores as our evaluation
measure.

Table \ref{tab:ccgbank_dev_results} shows the format we will report our results
in. We choose slightly different auxiliary metrics from \citeauthor{clark:cl07}.
First, we have omitted separate precision and recall for brevity. The separate
measures are generally uninformative, because the parser has little opportunity
to trade off between precision and recall, as the number of dependencies
produced by any wrong analysis will be roughly the same as the number of
dependencies in the correct analysis. We use the space freed up by omitting
precision and recall figures to report some additional detail about the parser's
performance on automatic \pos tags. Additionally, the $L$-sent column shows the
percentage of sentences that were parsed with perfect accuracy, achieving
labelled $F$-score of 100\%. The cat column shows the percentage of words that
were assigned the correct category.

We generally report coverage in a separate table analysing the $\beta$ levels
that were used for a given corpus. Table \ref{tab:ccgbank_betas} is an example
of this. Coverage refers to the percentage of sentences for which the parser
returned an analysis. If no analysis is returned, the sentence is discarded, and
does not contribute to the accuracy scores. This means that low coverage can
make changes in accuracy more difficult to interpret, because the parser might
have risen in accuracy by simply discarding the hardest sentences. We omit
coverage when comparing models that all decide between the same set of spanning
analyses. When parsers only differ in the statistical model that selects the
most likely analysis, they all achieve the same coverage, so the figure is
uninformative.

Finally, we report parse speed in words per second. This measure does include
attempts to parse sentences that could not be analysed. Low coverage therefore
tends to be associated with lower speed. Generally, high speed and high accuracy
are also correlated, because the parser is faster and more accurate when it can
construct a parse on the first $\beta$ level. Failure at a $\beta$ level
generally occurs when the parser's chart has been seeded with an inaccurate set
of supertags, preventing the parser from constructing a parse. The parser must
then rebuild the chart with a larger set. The new chart will necessarily be
larger than the original chart, making the parser even less efficient.

\section{Review of the \candc Parsing Models}

\citet{clark:cl07} experiment with four statistical parsing models. By
\emph{parsing model}, we refer to the component that selects an analysis from
the candidate analyses generated by the grammar. We contrast this with the
larger \emph{parsing system} (or simply \emph{parser}); which in the case of the
\candc parser, includes a supertagger.  The other notable \ccg statistical
parsing model in the literature is the generative parsing model described by
\citet{hock:acl03}. The \citeauthor{hock:acl03} model performs worse than the
\citeauthor{clark:cl07} models, but it is not immediately clear which of the
several differences between the systems the difference in performance should be
attributed to.

The four \citeauthor{clark:cl07} models differ along two dimensions. The first
is the number of negative examples introduced for training, as determined by the
$\beta$ parameters and the use of grammatical constraints.
Secondly, \citeauthor{clark:cl07} also experimented with two derivation-based
chart constraints during parsing: \citet{eisner:96} normal-form parsing, and a
dictionary of valid productions gathered from the training data.
The second dimension is whether the parser models derivations, or
dependencies. We will briefly describe these dimensions, before laying out three
of the four models described in \citet{clark:cl07}. The configuration we omit is
the dependencies model without normal form or seen-rules derivational
constraints, labelled \emph{Dependency} in \citet{clark:cl07} Table 6. We omit
this configuration because it performs worse than the other models on every
dimension. We follow \citepos{clark:cl07} nomenclature in referring to the
dependency model that makes use of derivation constraints as the \emph{Hybrid}
model.

\subsection{Training $\beta$ Parameter}
\label{sec:training_examples}
\begin{table}
\centering
\small
\setlength{\tabcolsep}{1.5mm}
\renewcommand{\arraystretch}{0.85}
 \begin{tabular}{l|ccccc|ccccc}
\hline
                  & \multicolumn{5}{c|}{Gold \textsc{pos}}               &
\multicolumn{5}{c}{Auto \textsc{pos}}\\
Model             & $LF$  & $UF$  & $L$-sent & cat   & w/s & $LF$  & $UF$  &
$L$-sent & cat   & w/s\\
\hline
\hline
\derivsbad & 85.12 & 91.92 & 32.14 & 93.05 & 389.788 & 83.38 & 90.75 & 29.76 &
91.95 & 384 \\
\derivsrev & 86.74 & 92.72 & 35.15 & 94.04 & 375.017 & 84.78 & 91.45 & 31.77 &
92.84 & 392 \\
\derivsthree & 86.83 & 92.76 & 35.57 & 94.09 & 384.932 & 84.89 & 91.51 & 32.19 &
92.89 & 396 \\
\derivsexp & 86.85 & 92.78 & 35.57 & 94.10 & 389.054 & 84.90 & 91.52 & 32.24 &
92.89 & 396 \\
\hline
 \end{tabular}
\caption[$\beta$ parameter during training.]{Effect of $\beta$ parameter to
control number of negative examples during training using the normal-form
derivations model.\label{training_betas}}
\end{table}


The \candc parser uses a discriminative log-linear model over whole analyses. In
order to produce negative examples for this model, the parser is run with a set
of categories to which the supertagger has assigned high confidence. The parser
builds a forest of analyses for each sentence from these categories, allowing
feature functions to be computed over positive and negative examples. The
correct category is always added to the set of supertags, so the parser can
usually construct an analysis that matches the gold standard. If it cannot,
because of a problem with its grammar's coverage, the sentence is discarded as a
training failure.

The number of negative examples produced for each sentence can be controlled by
giving the parser larger or smaller sets of categories for each word. The sets
are created by taking all categories whose probability is judged to be within
some factor, $\beta$, of the highest confidence category. The optimal setting of
this $\beta$ parameter is an empirical issue. \citet{clark:cl07} report the
results of two configurations: $\beta=0.1$ and $\beta=0.0045$. Presumably, other
configurations were attempted and discarded, after evaluating on Section 00. A
minority of long sentences will produce enormous charts at $\beta=0.0045$. If
the chart size exceeds a given number of categories, the parser tries again with
a higher $\beta$ setting. A succession of increasing $\beta$ levels (0.0045,
0.0055, 0.01, 0.05, 0.1) is used for the $\beta=0.0045$ model.

Table \ref{training_betas} shows the performance for the two configurations of
the normal-form derivations model (henceforth \textsc{derivs}) that
\citet{clark:cl07} report, $\beta=0.1$ and $\beta=0.0045$, as well as two
additional experimental settings, $\beta=0.003$ and $\beta=0.002$, on Section
00. The $\beta=0.1$ and $\beta=0.0045$ results exactly match those reported in
\citet{clark:cl07}. Version 1.02 of the \candc system contains scripts to
replicate these experiments out of the box. The results refer to
derivations-based models with Eisner normal-form and production dictionary
constraints on the grammar during both training and parsing.

We performed the additional $\beta<0.0045$ experiments to investigate whether
performance would degrade when too many negative examples were used. The
substantial difference in performance between the 0.1 and 0.0045 settings
suggests we ought to investigate whether the optimum value is even lower than
0.0045. 0.002 is the lowest value we have been able to train the parser with,
for reasons specific to the implementation and our available hardware.
We used the same training hardware as \citet{clark:cl07}, an 18 node Beowulf
cluster, with 25\textsc{gb} of \textsc{ram} available in total.

Interestingly, performance improves slightly as $\beta$ gets lower, although the
differences become very small, and probably insignificant. A $K$ parameter is
also used during training. As it does during parsing, this controls the token
frequency at which the word-specific tag dictionary is used. We experimented
with changing this parameter during training, but found it had no effect,
presumably because the parameter is most useful for unseen words --- making it
insignificant for training coverage.
% 
% \subsection{Normal-form and Production Dictionary Constraints}
% 
% In addition to manipulating the $\beta$ parameter, \citet{clark:cl07} also use
% two types of grammatical constraint to control the number of noisy derivations
% during training. The first of these are the \citet{eisner:96} normal-form
% constraints, which control spurious ambiguity by preventing categories that are
% the result of forward composition from being the leftward functor in the next
% production, and, analogously, categories that are the result of backward
% composition from being the rightward functor in the next production.
% \citeauthor{eisner:96} provides a proof that these constraints are sufficient to
% produce exactly one derivation per dependency analysis, but this proof does not
% apply for the \ccgbank grammar, because of its type-raising and type-changing
% rules. These two constraints still substantially reduce spurious ambiguity, but
% we cannot assume that there is only one derivation per dependency analysis ---
% there may be a great many more.
% 
% The second constraint is the use of a rule dictionary, containing the
% production rules seen during training. Rules that were not seen at least once
% during training are prohibited, preventing the parser from exploring unlikely
% uses of the combinatory rules.
% 
% These constraints are used during training for both, the training $\beta$
% values below $0.1$ could not be used, because the grammar is more productive for
% a given set of categories, causing us to hit our memory limit more quickly. The
% constraints therefore allow the use of more informative negative examples, in
% place of negative examples created by noisy grammar rules that will not be
% enabled during parsing. Disabling the constraints causes the performance of the
% \derivsbad model (the only one that can be trained this way) to get even worse.

\subsection{Derivation vs Dependency Models}
\label{sec:deriv_deps}
\citeauthor{clark:cl07} describe two probability models. One defines the probability
of a single Eisner normal-form \citep{eisner:96} derivation, while the other defines
the probability of a dependency analysis. The models are estimated differently, require
different decoders, and use different feature functions (although they do share some
features in common).

The dependency model is more accurate than the derivation model, so long as the
normal-form constraints are used during training and testing. \citet{clark:cl07}
refer to this normal-form/dependency model as the \hybrid model. One possible
explanation for the importance of the normal form constraints is that they allow the
model to use a lower $\beta$ level during training, introducing more negative examples
of incorrect dependency analyses.

The \hybrid model and the \derivs model differ along several dimensions, so it
is difficult to be certain why the hybrid model is more accurate. One likely explanation
is that the hybrid model's objective function resembles the evaluation more closely, as
the parser is evaluated over labelled dependencies.

The derivations model is slightly faster than the hybrid model, however. This is related
to the different decoders used by the two models to select the most likely parse. Both
models use the Viterbi algorithm to find the most likely nodes in the chart. The hybrid
model must first use the inside-outside algorithm to calculate the probability of
the  dependencies on each node, decreasing its efficiency.
%This seems
%to be related to the different decoders used by the two models to select the most
%likely parse. This may be an implementation issue, however, as there is no obvious
%algorithmic explanation for the difference in performance.

\subsection{Summary of Training Configurations}
\label{sec:system_summary}
Having reviewed the space of training parameters, we can see that a few
combinations are of particular interest when parameterising our systems. First,
we obviously want to try the configuration which performed the best among the
models trained by \citet{clark:cl07}. We label this model \hybrid.
% We have also
% shown that a simple tweak to this configuration --- the use of the Viterbi
% decoder instead of the \citet{clark:cl07} dependency decoder --- improves its
% speed, and slightly improves its accuracy. We label this configuration \hybridv.
We also want to experiment with the best performing normal-form model, which we
label \derivsrev, as this was the fastest model reported by \citet{clark:cl07},
and achieved accuracy within 1\% of the \hybrid model. It will also be worth
reporting results for the normal-form model that uses a higher training $\beta$,
the \derivsbad model, which is the only model used in previous \ccgbank
adaptation experiments.
% Finally, we
% will also experiment with the other configuration we introduce, \derivsexp,
% which uses a slightly lower training $\beta$ than the \derivsrev model.

The results for these configurations are shown in Table
\ref{tab:ccgbank_dev_results}. The most accurate model is the \hybrid dependencies
model. However, the \derivs model is slightly faster.
% although the
% difference between it and \hybrid is small. \hybridv is faster, however, since
% it uses the Viterbi decoder, instead of the \citet{clark:cl07} dependencies
% decoder. The other \derivs models also use the Viterbi decoder, which is why
% they are also faster than the \hybrid model too. However, these models do not
% use the dependency-based features of the \hybrid model, which is why they are
% less accurate. The \derivs models differ with respect to the number of negative
% examples introduced during training, as described in Section
% \ref{sec:training_examples}.
 
\begin{table}
\centering
\small
\setlength{\tabcolsep}{1.5mm}
\renewcommand{\arraystretch}{0.85}
 \begin{tabular}{l|cccccc|cccccc}
\hline
                  & \multicolumn{6}{c|}{Gold \textsc{pos}}               &
\multicolumn{6}{c}{Auto \textsc{pos}}\\
Model             & $LF$  & $UF$  & $L$-sent & cat   & cover & w/s & $LF$  &
$UF$  & $L$-sent & cat   & cover & w/s\\
\hline
\hline
\hybrid    & 87.27 & 93.01 & 35.94 & 94.16 & 99.06 & 365 & 85.30 & 91.77 & 32.93
 & 93.00 & 99.06 & 359 \\
%\hybridv   & 87.35 & 93.07 & 36.78 & 94.17 & 99.06 & 391 & 85.31 & 91.77 & 33.30
% & 92.97 & 99.06 & 395 \\
\hline
\derivsbad & 85.12 & 91.92 & 32.14 & 93.05 & 99.06 & 386 & 83.38 & 90.75 & 29.76
 & 91.95 & 99.06 & 398 \\
\derivsrev & 86.74 & 92.72 & 35.15 & 94.04 & 99.06 & 383 & 84.78 & 91.45 & 31.77
 & 92.84 & 99.06 & 401 \\
%\derivsexp & 86.85 & 92.78 & 35.57 & 94.10 & 99.06 & 387 & 84.90 & 91.52 & 32.24
% & 92.89 & 99.06 & 398 \\
\hline
\end{tabular}
\caption[\ccgbank development results and table key.]{\small Development results
on \ccgbank for the systems summarised in Section \ref{sec:system_summary} using
gold and automatic part-of-speech tagging. $UF$=Unlabelled $F$-score,
$LF$=Labelled $F$-score, $L$-sent=Sentences with 100\% $LF$, cat=Category
accuracy, cover=Percentage of sentences parsed (unparsed sentences are excluded
from evaluation), w/s=Speed in words per second.\label{tab:ccgbank_dev_results}}
\end{table}

\section{$\beta$ and $K$ Parameters During Parsing}
\label{beta_k}

Having established a set of training configurations, we now consider run-time
parameters that affect the parser's speed, accuracy and coverage. The $\beta$
and $K$ settings control the supertagger-parser integration. During parsing,
several $\beta$ levels are used, with successive levels invoked if parsing
fails. As we described in Section \ref{sec:background_candc}, \citet{clark:cl07}
found that it was best to begin with a narrow beam of supertags, and then widen
it if a spanning analysis could not be constructed.

The corpora we have created have more lexical categories and fewer grammatical
rules, suggesting that the $\beta$ and $K$ parameters may need to be adjusted.
Information that was previously specified in the grammar using type-changing rules is now
specified in the lexicon, reallocating some of the work from the parsing model
to the supertagger. This change in the relative workloads of the models may mean
that a different configuration of the $\beta$ and $K$ parameters is optimal.
\citet{clark:cl07} only provide results for two configurations of these
parameters, noting that the final values for the successful `big to small'
$\beta$ levels were optimised on Section 00. We will briefly explore how to
solve this parameterisation problem in a principled way, so that we can select
good values for our lexicalised corpora.

\subsection{Corrected Greedy Search: A Search Strategy for $\beta$ Levels}

The $\beta$ and $K$ parameters are tricky to optimise exhaustively, because the
parser can use an arbitrary number of levels. One obvious solution to try is a
greedy approach. Each $\beta$ and $K$ combination would be used to parse the
original set of sentences. The accuracy of a $\beta$ and $K$ combination would be
measured according to the labelled dependency $F$-score accuracy of the
spanning analyses that it produced. We could then simply select the $\beta$ and $K$
combination that had the highest initial accuracy, remove the sentences that it
covered from consideration, and repeat the process to find the next level.

Unfortunately, the optimal sequence of $\beta$ levels does not necessarily begin
with the $\beta$ and $K$ combination that achieves the highest accuracy, because
it might do so by only succeeding on a particularly easy subset of sentences.
For instance, we might find that a configuration of $\beta=1.0$ and $K=1$
allows, say, 100 sentences to be parsed at 89\% accuracy, while a $\beta=0.075$
$K=20$ configuration finds a spanning analysis for 1500 parses, at an average
accuracy of 88\%. The problem is that the 100 sentences parsed at 89\% by the
$\beta=1.0$ level might have been parsed at an average accuracy of 92\% at the
$\beta=0.075$ level, as they were simply short, easy to parse sentences. In this
situation, there would be no advantage to selecting the $\beta=1.0$ level. We
would simply have been misled by its apparently favourable initial accuracy.

There is a simple way to control for this. Once we have selected an optimisation
level, we will have a set of sentences that it will cover. We must then check
whether that set of sentences can be parsed more accurately at any other
configuration. We term this the Corrected Greedy Search strategy.

For instance, instead of committing to the sub-optimal $\beta=1.0$, $K=1$
configuration as our first level, we would evaluate the accuracy of the other
configurations over that subset of 100 sentences. When we find that the
$\beta=0.075$ and $K=20$ configuration parses the subset more accurately, we
discard the $\beta=1.0$ $K=1$ configuration, and consider the second most
accurate configuration --- say, $\beta=0.8$ $K=5$ --- and perform the same check
over the set of sentences that were parsed at that configuration. Eventually, we
are guaranteed to settle on the configuration that parses a set of sentences
more accurately than any other configuration. We can then adopt that setting as
our first level, remove the sentences it covers from consideration, and repeat
the process over the reduced set of sentences to find the next level.

This search strategy provides a simple and comprehensive solution to optimise
these parameters. This is important, because any change to the parser or
supertagger model potentially requires different $\beta$ and $K$ values. For
instance, if new features were discovered that caused the supertagger to be much
more accurate, we might want to select much tighter $\beta$ levels, to reflect
our altered balance of trust between the two models --- and likewise for
improvements in the parsing model. Changes to the corpus will also require these
parameters to be re-optimised. The \candc strategy is currently the only known
way to integrate a supertagger and parser effectively, and it can be adopted by
any lexicalised formalism, making this an issue of general interest, rather than
one specific to \ccg or the particular \candc implementation.


\subsection{\ccgbank Development Results with Optimised Betas}
\label{sec:ccgbank_beta_opt}
\begin{table}
\centering
\small
\renewcommand{\arraystretch}{0.85}
\setlength{\tabcolsep}{1.5mm}
 \begin{tabular}{l|ccccc|ccccc}
\hline
                  & \multicolumn{5}{c|}{Gold \textsc{pos}}               &
\multicolumn{5}{c}{Auto \textsc{pos}}\\
Model             & $LF$  & $UF$  & $L$-sent & cat   & w/s & $LF$  & $UF$  &
$L$-sent & cat   & w/s\\
\hline
\hline
\hybrid            & 87.27 & 93.01 & 35.94 & 94.16 & 370 & 85.30 & 91.77 & 32.93
& 93.00  & 379 \\
\hybrid\optbeta    & 87.46 & 93.14 & 36.02 & 94.13 & 266  & 85.65 & 92.04 &
33.05 & 93.02 & 261 \\
\hline
%\hybridv           & 87.35 & 93.07 & 36.78 & 94.17 & 391 & 85.31 & 91.77 & 33.30
%& 92.97 & 395 \\
%\hybridv\optbeta   & 87.44 & 93.17 & 36.65 & 94.09 & 286 & 85.49 & 91.94 & 33.32
%& 92.92  & 280 \\
%\hline
%\derivsexp         & 86.85 & 92.78 & 35.57 & 94.10 & 387 & 84.90 & 91.52 & 32.24
%& 92.89 & 398 \\
%\derivsexp\optbeta & 86.87 & 92.81 & 35.33 & 93.98 & 287 & 85.12 & 91.69 & 32.05
%& 92.90 & 282 \\
%\hline
\derivsrev         & 86.74 & 92.72 & 35.15 & 94.04 & 383 & 84.78 & 91.45 & 31.77
& 92.84 & 401 \\
\derivsrev\optbeta & 86.75 & 92.72 & 34.81 & 93.92 & 287 & 84.89 & 91.57 & 31.42
& 92.80 & 282 \\
\hline
\derivsbad         & 85.12 & 91.92 & 32.14 & 93.05 & 396 & 83.38 & 90.75 & 29.76
& 91.95 & 398 \\
\derivsbad\optbeta & 86.75 & 92.72 & 34.81 & 93.92 & 287 & 84.89 & 91.57 & 31.42
& 92.80 & 281 \\
\hline

\end{tabular}
\caption[Comparison of optimised $\beta$s with default $\beta$s.]{Comparison of
optimised $\beta$s against default $\beta$s. Rows with Opt use the optimised
$\beta$ levels. $\beta$s were optimised using the \hybrid model.}
\end{table}

\begin{table}
\centering
\begin{tabular}{l|rrrr|rrrr}
\hline
      & \multicolumn{4}{c|}{Optimised}& \multicolumn{4}{c}{Default} \\
Level & $\beta$ & $K$ & Gold & Auto   & $\beta$ & $K$  & Gold & Auto \\
\hline
\hline
1     & 0.02    & 22      & 1857 & 1846   & 0.075    & 20  & 1812 & 1786 \\
2     & 0.05    & 50      & 13   & 10     & 0.03     & 20  & 35   & 45   \\
3     & 0.001   & 5       & 22   & 30     & 0.01     & 20  & 17    & 24   \\
4     & 0.001   & 10,000  & 7    & 11     & 0.005    & 20  & 8    & 18    \\
5     &         &         &      &        & 0.001    & 150 & 23   & 22   \\
\hline
Cover &         &         &99.27 & 99.17  &          &     &99.06 & 99.06\\
\hline
\end{tabular}
 \caption[Optimised $\beta$ levels for \ccgbank parser.]{Optimised $\beta$
levels for the \hybrid model of the \ccgbank parser. The gold and auto columns
show the number of sentences successfully parsed at each
level.\label{tab:ccgbank_betas}}
\end{table}

\begin{table}
\centering
\begin{tabular}{l|rrrrr}
\hline\hline
Parameter & \multicolumn{5}{c}{Levels}\\
\hline
$\beta$ & 1.0 & 0.8 & 0.6 & 0.4 &0.3 \\
        &0.2  & 0.1 &0.09 &0.08 &0.075 \\
        &0.07 &0.065&0.06 &0.05 &0.04 \\
        &0.03 &0.025&0.02 &0.015&0.01\\
        &0.007&0.005&0.001&0.0005&\\
        &0.0001     &     &      \\
\hline
$K$     &1    &5    &10   &15   &18\\
        &20   &22   &25   &30   &40\\
        &50   &80   &100  &150  &  \\
        &300  &1000 &10000&     &  \\
\hline
 \end{tabular}
\caption[$\beta$ and $K$ value search space.]{Search space of possible $\beta$
and $K$ values. The values were hand-selected to provide a wide range of
possible values, with fine granularity around the values \citet{clark:cl07}
selected.\label{tab:beta_k_search}}
\end{table}


We used the Corrected Greedy Search strategy on Section 00, using automatic \pos
tags.
We used automatic \pos tags because we found that when gold standard \pos tags
were used, the system tended to select $\beta$ levels that optimised the gold
\pos tag accuracy at the expense of the automatic \pos tag accuracy. It did this
by selecting very high $K$ parameter values, which meant that the supertagger
used the tag dictionary less often, relying on the \pos dictionary. The tag
dictionary is not keyed by the word's part of speech, so in the case of a
part-of-speech tagging error, the correct tag may be in the tag dictionary, but
not in the \pos dictionary. When we searched over automatic \pos tags, we found
that the parser's accuracy on automatic \pos tag was much better, with little
impact on the gold \pos evaluation.

We ran the search over 25 possible $\beta$ values and 17 possible $K$ values,
shown in Table \ref{tab:beta_k_search}. This meant that the 1915 sentences in
the development set were parsed 400 times each, for a total of 767,200. The
parsing took roughly 20 hours of \cpu time, with the subsequent search running
in negligible time. The parsing took much longer than it would take to parse
that many sentences using the standard strategy, because parsing succeeds at the
first $\beta$ level (which by default is $\beta=0.075$, $K=20$) for most
sentences. At this $\beta$ level on \ccgbank, the supertagger supplies an
average of only 1.27 categories per word to the parser, leading to low ambiguity
and high efficiency.

The $\beta$ levels selected, along with the number of sentences parsed at each
level using gold and automatic tags, are shown in Table \ref{tab:ccgbank_betas}.
The search strategy selected a lower initial $\beta$ level, allocating more work
to the parsing model than the default level of 0.075. More sentences are parsed
at this level, because the parser's chart is seeded with more categories per
word. However, this leads to much larger chart sizes, so the parser is slower
overall.

Every model improved in accuracy with the new $\beta$ values, with both
automatic and gold \pos tags. The improvement was larger on automatic \pos tags,
presumably because automatic \pos tags were used for the objective function. The
optimised $\beta$ levels were lower on average, preserving more ambiguity in the
pipeline. It is surprising that this improved the accuracy of all of the models
--- even the \derivsbad model, which we would have expected to perform better
when the parsing model had to do less work.

The accuracy improvements came at the expense of efficiency. The \hybrid model
was 39\% slower with the new $\beta$ values. Whether this trade-off is
worthwhile will depend on the use case, but for most applications the more
efficient configuration will probably be desirable. Parsing is likely to be the
speed bottleneck in an \nlp pipeline, and many \nlp pipelines have limited
processing resources, but almost unlimited text. In this situation, a 40\% drop
in parsing efficiency will translate to a 40\% drop in the text that the
application the application can make use of. For many applications, this extra
data will improve results more than the extra 0.35\% in accuracy on automatic
\pos tags. This consideration may be why \citet{clark:cl07} selected a higher
initial $\beta$ level. For future work, it would be interesting to incorporate
speed considerations into the objective function of our search strategy, so that
we could arrive at a good compromise between speed and accuracy.


\section{Adapting the \candc Parser for Hat Categories}
\label{sec:implementation}

So far, we have discussed the relevant parameters that we might need to change
to use the \candc parser with our new corpora. All of these parameters are
exposed as configuration options for the parser, either during training or
run-time. However, we also had to make several changes to the implementation.
The most significant change was to the unification engine, to pass the hat
categories around and deal with the addition of extra head variables. We also
had to modify the category objects, add the unhat rule, and disable the
type-changing rules. These changes were all applied to the standard 1.02
release of the parser.

\subsection{Changes to Category Objects}



There are two category classes in the \candc parser. The \codeterm{Cat} class
decouples the \ccg category from its context in the sentence, so that two
instances of the same category in the sequence of supertags will always be the
same. The \codeterm{Cat} class is used for both atomic and complex categories.
The \codeterm{Supercat} class is used for nodes in the chart. It stores lexical
heads, the last rule used to produce the category, and inside and outside
probabilities. \codeterm{Supercat} objects are passed a reference to a
\codeterm{Cat} object, so they do not need to know anything about hat categories
beyond the existence of the unhat rule.

A \texttt{hat} attribute was added to the \codeterm{Cat} class, as would be
expected from our outline in Chapter \ref{chapter:hat_cats}. As far as a complex
\codeterm{Cat} itself is concerned, the hats on its inner \codeterm{Category}
nodes are independent of one another. This means that the category-wide
restrictions we stipulate in Chapter \ref{chapter:hat_cats} are not enforced
directly. For instance, a \codeterm{Cat} object does not immediately know
whether it is an argument or an inner result of a category that already has a
hat specified, leading to unlicensed categories like \cf{PP/S[ng]^{NP}} or
\cf{(S[ng]^{NP\bs NP}\bs NP)^{S\bs S}}.

We regard these categories as syntactically well-formed but semantically
undesirable generalisations of the hat attribute. We have ensured that they do
not occur in the analyses in our corpora, and in our implementation we have made
some effort to prevent them from occurring as intermediate categories in the
chart. In general, however, we have prioritised performance rather than symbolic
correctness in our implementation, so we have steered clear of costly checks to
guarantee that poorly formed categories never occur. The parser can only return
lexical categories that were seen in the training data, so we can guarantee the
output is well-formed by ensuring that our analyses conform to our stipulations.

\subsection{Changes to Unification}

The \candc parser does not support a full unification algorithm, such as that
described by \citet{shieber:86} and assumed in our discussion in Chapter
\ref{chapter:hat_cats}. Unification can be computationally expensive, and
\ccgbank uses only a very limited number of features.

The unification algorithm in version 1.02 of the parser performs two tasks.
First, it recursively checks whether two categories can unify, by comparing
atomic categories, and the results, arguments and slashes of complex categories.
Two atomic categories match if they are of the same type (\cf{S}, \cf{N},
\cf{NP} etc) and if their features can be unified. Only features on \cf{S} typed
atomic features are considered. \cf{S} categories have exactly one feature,
which can be the variable \cf{X}, which inherits from any other feature upon
unification. Other conflicting feature pairs cause unification to fail.

We add a small extension to this phase of the unification algorithm to also
compare hat attributes, with unification failing if both categories have a hat
and the two cannot be unified. If unification succeeds, the hat must be shared
between the two categories, and any other categories unified with them. This is
the mechanism by which hats are preserved during adjunction:

\begin{eqnarray}
\eqnpsrule{\cf{S/S}}{\cf{S[ng]^{NP}}}{\cf{S[ng]^{NP}}}
\end{eqnarray}

The other aspect of the unification algorithm deals with head-passing. When a
combinatory rule is used to combine two categories, parts of the two categories
are unified. For instance, forward application involves unifying the result of
the functor with the argument category. During successful unification, the head
variables of the two categories are mapped to the corresponding variable on the
other category. For instance, consider this instance of forward
application:

\begin{eqnarray}
 \eqnpsrule{\cf{((S[X]_y\bs NP_z)_y/(S[X]_y\bs NP_z)_y))_x}}{\cf{(S[dcl]_x\bs
NP_y)_x}}{S[dcl]\bs NP}
\end{eqnarray}


When the \cf{S[X]\bs NP} and \cf{S[dcl]\bs NP} categories are unified, the head
variables (shown subscripted) are mapped onto each other as follows, where the
variables of the left side are shown in the rows, and the variables of the right
side are shown in the columns:

\begin{center}
\large
%\setlength{\tabcolsep}{3.5mm}
\begin{tabular}{l|ll}
  & $X$ & $Y$ \\
\hline
$X$ &   &   \\
$Y$ & $\checkmark$ &   \\
$Z$ &   & $\checkmark$ \\ 
\end{tabular}
\end{center}

The table tells us that the $Y$ variable in the left category and the $X$
variable on the right will represent the same lexical item. This mapping process
was updated to include variables from the \codeterm{hat}, so that dependencies
could be mediated through hat categories.


\subsection{Changes to Combinatory Rules}
\label{sec:combinator_implementation}
As we described in Chapter \ref{chapter:hat_cats}, hat categories do not require
any redefinition of the combinatory rules in the theory. However, the \candc
parser's implementation performs less of the work during unification than we
would expect in the theory. Instead, the new categories are constructed by the
combinatory rules, and each combinator implements the construction process
itself. We therefore had to change the forward and backward application rules,
and the forward, backward, and backward crossed combinators separately.
Generalised composition also required specific changes to ensure that
the hat attribute was passed along by the combinatory
rule where necessary.

We simulated the stipulations described in Chapter \ref{chapter:hat_cats} by
adding constraints to the combinatory rules. The Inert Slash Stipulation was
simulated by preventing hatted categories from being the functor of application
or composition rules. We implemented the stipulation this way because the \candc
parser does not currently support multi-modal \ccg. This implementation matched the
behaviour of the stipulation exactly, however.

The Null Hat Constraint was simulated by blocking non-modifier categories from applying
hatted arguments. A category was considered a modifier if its result exactly matched its
argument. This implementation assumes that all and only modifier categories had coindexed
hat values.

The final change to support hat categories was the addition of the
\codeterm{unhat} rule. This rule was straightforward to implement, as it is
conceptually so simple:

\begin{eqnarray}
\cf{\cf{X^Y}} \;\;\Rightarrow_\cH\;\; \cf{Y}
\end{eqnarray}


To implement the rule, we simply check whether a category has a hat, and if it
does, we return the hat as the result of the combinator, to be added to the same
cell of the chart. We also had to unpack recursive hat categories, so that for a
category like \cf{N^{NP^{S/S}}}, the two categories \cf{NP^{S/S}} and \cf{S/S}
are added to the same cell of the chart.



\subsection{Removal of Type-Changing Rules}

Once we had added support for hat categories, we disabled the non-combinatory
type-changing rules in the parser. The type-changing rules were hard
coded, but removing them did not cause any complications. The type-raising rules
are specified in text files, so these did not require code changes.

\subsection{Test Driven Development}

The \candc parser is a complex piece of software, consisting of approximately
40,000 lines of highly optimised \cpp code. Runtime always involves a great many
recursive calls. This makes debugging the system by inspecting run-time errors
fairly impractical. It is also difficult to navigate the chart structures, which
are populated by billions of derivations for corpus sentences. We therefore
adopted a test-driven development methodology. To do this, we used the system's
Python \textsc{api} to construct a series of unit tests combining categories
constructed in various ways. These tests also helped us to define the systems
behaviour for a variety of interesting edge cases, and serve as helpful
documentation for how hat categories are intended to behave.
% TODO? We have included a number of these test cases and some discussion of
%them in Appendix \ref{appendix:test_cases}.

\section{Dependency Label Mapping}
\label{sec:dependency_mapping}
As we describe in Section \ref{sec:previous_problems}, it is difficult to
compare a parser trained on an altered version of \ccgbank against the original.
When the corpus is changed for training, the testing portion is also altered. We
address this by creating mappings that translate the new corpus's labelled
dependencies into the original \ccgbank dependencies. This allows us to map the
parser's output to match the original \ccgbank, allowing us to evaluate against
the same set of dependencies.

A labelled dependency in \ccgbank is a 4-tuple containing the head token, the
head lexical category, the argument slot and the child lexical category. For
instance, the labelled dependency between \emph{Robin} and \emph{bought} in:

\begin{lexamples}
\item \gll The present Robin bought Kelly
\cf{NP/N} \cf{N} \cf{NP} \cf{(S[dcl]\bs NP)/NP)^{NP\bs NP}/NP} \cf{NP}
\gln
\glend
\end{lexamples}

would be:

\begin{equation}
(bought,\;\; \cf{(S[dcl]\bs NP_1)/NP_2)^{NP\bs NP_2}/NP_3},\;\; 1,\;\; Robin)
\end{equation}

The hat makes this dependency more specific than the equivalent in \ccgbank,
because it represents information that would not have been in the original
lexical category in the hat. This can make parsing results difficult to compare.
Additionally, one hat category can affect multiple dependencies, because a verb
like \emph{bought} governs multiple dependencies.

However, the fact that the hat version is \emph{more} specific means we can
accurately map down to the \ccgbank labels. All we need to do is align the
dependencies for the training partitions of the two corpora, and find the most
frequent \ccgbank label for each of the new labels. The mapping consists of the
new labels paired with their most frequent \ccgbank label. It is equivalent to
training a classifier that predicts the \ccgbank label using only one feature,
the new label.

This technique worked very well the \hatsys and \trsys corpora. We evaluated it
by counting how often the mapping assigned the correct label over the training
section. The \hatsys and \trsys mappings both scored over 99.5\% accuracy on
this measure.

The mapping was more problematic for the \nounary corpus, which scored 97.9\%.
While the categories in the \nounary corpus are, on average, more specific than
the \ccgbank categories, the re-analyses described in Section
\ref{sec:nounary_corpus} often reuses categories that occur elsewhere in the
corpus. By contrast, the changes in the \hatsys and \trsys always involved novel
categories.

The most common source of error occurs when bare nominals are relabelled. As we
described in Section \ref{sec:nounary_corpus}, the \psunary{\cf{N}}{\cf{NP}} is
compiled into the subtree, producing the following change:

\begin{eqnarray}
 \ptbegtree
   \ptbeg \ptnode{\cf{NP}}
     \ptbeg \ptnode{\cf{N}}
       \ptbeg \ptnode{\cf{N/N}} \ptleaf{dangerous} \ptend
       \ptbeg \ptnode{\cf{N}} \ptleaf{asbestos} \ptend
     \ptend
   \ptend
  \ptendtree
&
\longrightarrow
&
 \ptbegtree
   \ptbeg \ptnode{\cf{NP}}
     \ptbeg \ptnode{\cf{NP/NP}} \ptleaf{dangerous} \ptend
     \ptbeg \ptnode{\cf{NP}} \ptleaf{asbestos} \ptend
   \ptend
  \ptendtree
\end{eqnarray}

When this occurs, the \nounary version's dependency, \ref{nounary_dep} needs to
be mapped to the \ccgbank dependency \ref{ccgbank_dep}:

\begin{eqnarray}
(dangerous,\;\; \cf{NP/NP_1},\; 1,\;\; asbestos)\label{nounary_dep}\\
(dangerous,\;\; \cf{N/N_1},\;\; 1,\;\; asbestos)\label{ccgbank_dep}
\end{eqnarray}

The problem is that there is already a \cf{NP/NP} category in \ccgbank, which is
used for pre-determiners. Some examples of this category are:

\begin{lexamples}
 \item \gll Neither Lorillard   nor     the~researchers
         \cf{NP/NP} \cf{NP}   \cf{conj} \cf{NP}
\gln
\glend
 \item \gll Three~times the~expected~number
            \cf{NP/NP}  \cf{NP}
\gln
\glend
 \item \gll Formerly a~vice~chairman
         \cf{NP/NP}  \cf{NP}
\gln
\glend
\end{lexamples}

Mapping (\cf{NP/NP_1}, 1) dependencies to (\cf{N/N_1}, 1) produces errors for
all of these original uses of the category. There are other constructions which
display the same problem. This is why the mapping achieves only 97.9\% accuracy.
This problem means that only the unlabelled dependencies are properly comparable
for the \nounary corpus. The unlabelled mapping is 99.8\%.

\section{Lexicalised Parsing Models}

This section describes the parsing models we selected for the \hatsys, \nounary
and \trsys corpora. The models were selected by running a series of experiments
on Section 00 of the \wsj for each corpus. The purpose of these experiments is
to find a good configuration for evaluation on the test data. Evaluation will be
conducted on Section 23 of the \wsj and the \wikieval Wikipedia test set
described in Section \ref{sec:wikipedia_results}, using the configurations
decided on Section 00.

For each corpus, we first select an appropriate model from the five discussed in
Section \ref{sec:system_summary}, using the default $\beta$ levels. We then use
that model to search for better $\beta$ levels using the search strategy
described in Section \ref{beta_k} over Section 00. As we described in Section
\ref{sec:ccgbank_beta_opt}, we used accuracy over automatic \pos tags as the
objective criterion, because we found it produced better results than accuracy
over gold standard \pos tags.




\subsection{Selecting the \hatsys Model}
\label{sec:hat_dev}
\begin{table}
\centering
\small
\renewcommand{\arraystretch}{0.85}
\setlength{\tabcolsep}{1.5mm}

 \begin{tabular}{l|ccccc|ccccc}
\hline
            & \multicolumn{5}{c|}{Gold \textsc{pos}}               &
\multicolumn{5}{c}{Auto \textsc{pos}}\\
Model       & $LF$  & $UF$  & $L$-sent & cat   & w/s & $LF$  & $UF$  & $L$-sent
& cat   & w/s\\
\hline
\hline
\hybrid            & 86.66 & 92.55 & 34.86 & 92.92 & 403 & 84.81 & 91.39 & 32.52
& 91.74 & 413 \\
\hybrid\optbeta    & 87.09 & 92.79 & 35.46 & 93.20 & 430 & 85.04 & 91.58 & 32.68
& 91.93 & 435 \\
\hline
%\hybridv           & 86.60 & 92.56 & 35.97 & 92.98 & 443 & 84.70 & 91.39 & 32.95
%& 91.79 & 465 \\
%\hybridv\optbeta   & 87.00 & 92.80 & 36.41 & 93.21 & 472 & 84.96 & 91.60 & 33.16
% & 91.97& 498 \\
%& 87.00 & 92.80 & 36.41 & 93.21 & 458 & 84.96 & 91.60 & 33.16 & 91.97 & 482 \\
%\hline
%\derivsexp         & 86.23 & 92.37 & 34.33 & 92.94 & 455 & 84.31 & 91.14 & 31.36
%& 91.78 & 463 \\
%\derivsexp\optbeta & 86.60 & 92.61 & 34.88 & 93.16 & 449 & 84.58 & 91.37 & 31.84
%& 91.93 & 479 \\
%\hline
\derivsrev         & 86.28 & 92.40 & 34.76 & 92.99 & 447 & 84.41 & 91.17 & 31.52
& 91.85 & 448 \\
\derivsrev\optbeta & 86.67 & 92.63 & 35.25 & 93.22 & 461 & 84.69 & 91.42 & 31.94
& 92.03 & 491 \\
\hline
\derivsbad         & 83.68 & 90.91 & 32.07 & 91.23 & 435 & 81.92 & 89.81 & 29.62
& 90.11 & 467 \\
\derivsbad\optbeta & 83.79 & 91.17 & 31.14 & 91.39 & 465 & 81.84 & 89.98 & 28.35
& 90.11 & 504 \\
\hline
\end{tabular}
\caption[\hatsys development results.]{Development results for the \hatsys
parser with default and optimised $\beta$s. $\beta$s were optimised over
automatic \pos tags using the \hybrid model. \label{tab:hat_dev}}
\end{table}

\begin{table}
\centering
\begin{tabular}{l|rrrr|rrrr}
\hline
      & \multicolumn{4}{c|}{Optimised}& \multicolumn{4}{c}{Default} \\
Level & $\beta$ & $K$ & Gold & Auto   & $\beta$ & $K$  & Gold & Auto \\
\hline
\hline
1     & 0.05    & 22  & 1718 & 1677 & 0.075    & 20  & 1658   & 1621\\
2     & 0.05    & 40  & 32   &   29 & 0.03     & 20  & 89     & 99  \\
3     & 0.005   & 50  & 120  &  148 & 0.01     & 20  & 52     & 65  \\
4     & 0.001   & 10  & 14   &   21 & 0.005    & 20  & 26     & 26  \\
5     & 0.001   & 150 & 14   &   19 & 0.001    & 150 & 74     & 83  \\
\hline
Cover &         &     & 99.21& 98.9 &          &     & 99.27  & 99.01\\
\hline
\end{tabular}
 \caption[Optimised $\beta$ levels for \hatsys parser.]{Optimised $\beta$ levels
for \hatsys parser. The gold and auto columns show the number of sentences
successfully parsed at each level.\label{tab:hat_betas}}
\end{table}


The \hatsys corpus uses hat categories to lexicalise the \ccgbank
type-changing rules, as described in Section \ref{sec:hat_corpus}. Table
\ref{tab:hat_dev} shows the development result for the three models trained on
the \hatsys corpus, and evaluated on the \ccgbank dependencies of Section 00.
We selected the most accurate model, \hybrid, to optimise the $\beta$ levels.
%although the difference between its gold and
%automatic \pos tag $LF$ scores and \hybridv's might not be statistically
%significant. We selected the \hybrid model to optimise the $\beta$ levels.

The Corrected Greedy Search strategy selected the $\beta$ levels shown in Table
\ref{tab:hat_betas}. One disadvantage of the automated search is that it does
not optimise for speed at all. There are two $\beta$ levels which differ only in
their corresponding $K$ level. Most sentences which failed at the first level
will also fail at the second level, slowing down the parser. However, we decided
to accept the $\beta$ levels found by the search strategy as they were, to avoid
the error-prone manual tuning the search is designed to replace.
 
Table \ref{tab:hat_dev} shows the results using the new $\beta$ levels. The
labelled $F$-score results improve by similar amounts (roughly 0.5\%) for most
of the models, using both gold and automatic \pos tags. The exception is the
\derivsbad model, which only shows a slight improvement. This might be because
the parsing model is less accurate, while the supertagger accuracy is the same
for all models. The \derivsbad model would probably benefit from $\beta$ values
that trusted the supertagger more, by delivering a smaller set of categories to
the parser.

The new $\beta$ values produced roughly the same parsing speeds as the original
values. In \citet{honnibal:09}, we report results using hand-selected high $K$
values, producing parse speeds of 550 words per second using gold \pos tags,
with $LF$ accuracy of 87.12\%. However, the automatic \pos tag accuracy was only
84.85\%. This is an example of how $\beta$ levels can be selected to optimise
for gold standard \pos tag performance at the expense of the more likely use
case, automatically selected \pos tags.

We selected the \hybrid\optbeta model to represent the \hatsys corpus, as it
was the most accurate model on both gold and automatic \pos tags.
 %This model was only 0.08\% more accurate than the
%\hybridv\optbeta model, but was 15\% slower. We therefore selected the \hybridv
%model as the representative for the \hatsys corpus.

\subsection{Selecting the \nounary Model}
\label{sec:nounary_dev}

The \nounary corpus compiles out the type-changing rules from \ccgbank, using
no extra grammatical machinery beyond the application, composition and
type-raising rules, as described in Section \ref{sec:nounary_corpus}. Table
\ref{tab:nounary_dev} shows the development results for the parser trained on
the \nounary corpus and evaluated on the \ccgbank dependencies for Section 00.
As we describe in Section \ref{sec:dependency_mapping}, there is an upper bound
of 97.9\% on the labelled dependency accuracy for the \nounary corpus, because
of errors in the dependency mapping to \ccgbank. However, the labelled and
unlabelled accuracies in Table \ref{tab:nounary_dev} display similar trends.

Although we reserve detailed comparison for Section \ref{sec:wsj_evaluation}, it
is immediately noticeable that the performance on the \nounary corpus is
substantially lower than the development performance of the \ccgbank and \hatsys
parsers discussed above. One potential explanation is the lower label mapping
accuracy for the \nounary corpus, described in Section
\ref{sec:dependency_mapping}, but we see similar loss of performance in
unlabelled accuracy, suggesting that this is not the explanation.

We used the \hybrid model to search for the most accurate $\beta$ levels on
automatic \pos tags using the Corrected Greedy Search strategy described in
Section \ref{beta_k}. Table \ref{tab:nounary_betas} shows the $\beta$ levels
selected. There is no clear pattern in the levels selected, suggesting a degree
of over-fitting. The search strategy may be able to over-fit more often on the
\nounary corpus because the grammar is more restrictive and the lexical
categories in the corpus are more sparse, so the parser fails to find a spanning
analysis more often. This makes the set of sentences covered by each $\beta$ and $K$
combination smaller, giving the search strategy more freedom. 

For instance, the first level selected in Table \ref{tab:nounary_betas} is the
same as the first level selected for the original \ccgbank, in Table
\ref{tab:ccgbank_betas}. On \ccgbank, this level covers 1846 sentences with
automatic \pos tags. On the \nounary corpus it only covers 1680. The \nounary
corpus does not have the \ccgbank type-changing rules in its grammar, so the
parser has less freedom to construct a spanning analysis from a problematic set
of supertags. Sparse data problems in the corpus may also make the supertagger
less accurate, so the set of categories at a given $\beta$ and $K$ configuration may
also contain more errors.

\hybrid\optbeta is the most accurate of the \nounary models. The optimised
$\beta$ values bring a larger performance improvement on the \nounary models
than they did for the \hatsys and \ccgbank models, which could also be
interpretted as evidence of over-fitting. The \hybrid\optbeta model outperforms
the \hybrid default $\beta$s model by 0.8\% labelled $F$-score with gold \pos
tags, and 0.6\% with automatic \pos tags. The only model where the optimised
$\beta$ values perform more poorly is \derivsbad, where for the first time the
optimised $\beta$ levels perform worse than the defaults. This is probably
because of the 3.5\% gap in accuracy between the \derivsbad model, and the
\hybrid model used to select the $\beta$ values. The difference in accuracy of
these models mean they require different $\beta$ levels, because it is better to
rely on the supertagger when the parsing model is less accurate, by supplying
the parser fewer categories per word.


\begin{table}
\centering
\small
\renewcommand{\arraystretch}{0.85}
\setlength{\tabcolsep}{1.5mm}
 \begin{tabular}{l|ccccc|ccccc}
\hline
                  & \multicolumn{5}{c|}{Gold \textsc{pos}} &
\multicolumn{5}{c}{Auto \textsc{pos}}\\
Model             & $LF$  & $UF$   & $L$-sent & cat   & w/s & $LF$  & $UF$  &
$L$-sent & cat   & w/s\\
\hline
\hline
\hybrid            & 84.25 & 91.55 & 30.33 & 91.56 & 446 & 82.36 & 90.39 & 27.97
& 90.43 & 437 \\
\hybrid\optbeta    & 84.98 & 92.02 & 30.96 & 91.85 & 340 & 82.82 & 90.73 & 28.03
& 90.46 & 343 \\
\hline
%\hybridv           & 84.17 & 91.56 & 30.75 & 91.53 & 511 & 82.27 & 90.39 & 28.40
%& 90.39 & 512 \\
%\hybridv\optbeta   & 84.82 & 92.04 & 31.33 & 91.78 & 363 & 82.64 & 90.70 & 28.50
%& 90.46 & 376 \\
%\hline
%\derivsexp         & 84.08 & 91.33 & 30.33 & 91.60 & 514 & 82.22 & 90.15 & 27.81
%& 90.49 & 504 \\
%\derivsexp\optbeta & 84.59 & 91.73 & 30.90 & 91.82 & 357 & 82.58 & 90.47 & 27.97
%& 90.55 & 373 \\
%\hline
\derivsrev         & 83.95 & 91.27 & 30.17 & 91.52 & 509 & 82.07 & 90.06 & 27.60
& 90.36 & 502 \\
\derivsrev\optbeta & 84.41 & 91.67 & 30.43 & 91.69 & 364 & 82.36 & 90.39 & 27.50
& 90.39 & 376 \\
\hline
\derivsbad         & 80.64 & 89.26 & 27.78 & 89.26 & 512 & 78.95 & 88.17 & 25.63
& 88.24 & 493 \\
\derivsbad\optbeta & 78.62 & 88.65 & 22.03 & 87.78 & 346 & 76.57 & 87.35 & 20.52
& 86.44 & 374 \\
\hline
\end{tabular}
\caption[\nounary development results.]{Development results for the \nounary
parser with default and optimised $\beta$s. $\beta$s were optimised over
automatic \pos tags using the \hybrid model. \label{tab:nounary_dev}}
\end{table}

\begin{table}
\centering
 \begin{tabular}{l|rrrr|rrrr}
\hline
      & \multicolumn{4}{c|}{Optimised}& \multicolumn{4}{c}{Default} \\
Level & $\beta$ & $K$ & Gold & Auto  & $\beta$ & $K$  & Gold & Auto \\
\hline
\hline
1     & 0.02   & 22    & 1692 & 1680 & 0.075 & 20  & 1568 & 1522\\
2     & 0.1    & 40    & 27   & 23   & 0.03  & 20  & 89   & 114 \\
3     & 0.04   & 10000 & 61   & 53   & 0.01  & 20  & 71   & 83  \\
4     & 0.015  & 10    & 15   & 12   & 0.005 & 20  & 44   & 47  \\
5     & 0.001  & 80    & 82   & 101  & 0.001 & 150 & 114  & 111 \\
6     & 0.0005 & 1000  & 16   & 22   &       &     &      &     \\
\hline
Cover &        &       &98.95 & 98.85&       &     & 98.59& 98.12\\
\hline
\end{tabular}
\caption[Optimised $\beta$ levels for \nounary parser.]{Optimised $\beta$ levels
for \nounary parser. The gold and auto columns show the number of sentences
successfully parsed at each level.\label{tab:nounary_betas}}
\end{table}

\subsection{Selecting the \trsys Model}
\label{sec:hattr_dev}
Table \ref{tab:trsys_dev} shows the development results for the parser trained
on the \trsys corpus and evaluated on the \ccgbank dependencies for Section 00.
The corpus uses hat categories to lexicalise type-raising, in addition to the
type-changing rules. The corpus is described in Section
\ref{sec:hattr_corpus}. There were several strange development results for this
corpus. First, the \hybrid models performed substantially worse than the \derivs
models, which is the opposite of what we saw on the other corpora. We suspect
that the lexicalisation of the type-raising rules has interfered with some of
the \hybrid model's feature functions, but we have not been able to identify the
issue.

We used the \derivsrev model to search for $\beta$ levels that optimised
labelled $F$-score with automatic \pos tags using the Corrected Greedy Search
strategy described in Section \ref{beta_k}. The $\beta$ levels selected are
shown in Table \ref{tab:hattr_betas}. Much like the \nounary corpus, there is no
clear intuition behind the $\beta$ levels selected, which may indicate a degree
of over-fitting. The first level is particularly curious. A $K$ level of 1 means
that the supertagger will use its word-specific tag dictionary for every word in
the sentence, even if it has only occurred once in the training data. Arguably,
we should not have included $K$ levels below 10 in the search space, but we
wanted to make fewer assumptions about what would constitute good $\beta$ and
$K$ settings and see whether the search strategy was able to select good values.
As it turns out, the $\beta$ levels that the strategy selected generalised to
the Wikipedia data surprisingly well, as we discuss in Section
\ref{sec:wikipedia_results}.

The $\beta$ levels selected by the search strategy caused small decreases in
accuracy on gold \pos tags for the \derivsrev model,
although these differences may not be statistically significant. All of the
models improved in accuracy using automatic \pos tags, with the
\derivsrev\optbeta model performing the best. The improvement in accuracy was
similar to the improvement seen from $\beta$ level optimistion on the \hatsys
corpus, in Section \ref{sec:hat_dev}.

The first selected $\beta$ value, 0.2, is quite restrictive. This leads to
faster parsing for the \derivs models, because the parser is supplied a smaller
set of categories, leading to small charts and efficient parsing for the 60\% of
sentences for which the parser can construct a spanning analysis at this level.
The reduction in chart sizes is worth the cost of building a chart twice for the
other 40\% of sentences, because the chart sizes grow exponentially with respect
to the number of categories they are seeded with. This means that a sentence can
take many times longer to parse at a low $\beta$ level than it would to parse at
a high $\beta$ level.

%What is strange is that the \hybrid models get \emph{slower} with the new
% $\beta$s, despite the fact that the same chart parser is being seeded with the
% the same categories, and the only thing that is different is the model being
% used to select the most likely parse. This is another strong indication that
% there is an implementation issue with the \hybrid models for the \trsys corpus,
% affecting their results.

\begin{table}
\centering
\small
\renewcommand{\arraystretch}{0.85}
\setlength{\tabcolsep}{1.5mm}
 \begin{tabular}{l|ccccc|ccccc}
\hline
                  & \multicolumn{5}{c|}{Gold \textsc{pos}}               &
\multicolumn{5}{c}{Auto \textsc{pos}}\\
Model             & $LF$  & $UF$  & $L$-sent & cat    & w/s & $LF$  & $UF$  &
$L$-sent & cat   & w/s\\
\hline
\hline
\hybrid            & 85.15 & 91.36 & 31.63 & 92.18 & 678 & 83.09 & 90.02 & 28.80
& 90.87 & 537 \\
\hybrid\optbeta    & 85.34 & 91.52 & 32.21 & 92.26 & 581 & 83.48 & 90.33 & 29.73
& 91.12 & 502 \\
\hline
%\hybridv           & 84.76 & 91.03 & 30.53 & 92.14 & 662 & 82.73 & 89.69 & 27.64
%& 90.86 & 681 \\
%\hybridv\optbeta   & 84.93 & 91.15 & 30.58 & 92.19 & 636 & 83.08 & 89.94 & 28.30
%& 91.08 & 590 \\
%\hline
%\derivsexp         & 86.31 & 92.48 & 34.37 & 92.85 & 475 & 84.02 & 90.96 & 30.87
%& 91.44 & 510 \\
%\derivsexp\optbeta & 86.27 & 92.39 & 33.96 & 92.72 & 503 & 84.46 & 91.28 & 31.22
%& 91.59 & 552 \\
%\hline
\derivsrev         & 86.37 & 92.48 & 34.26 & 92.87 & 601 & 84.10 & 90.97 & 30.85
& 91.50 & 678 \\
\derivsrev\optbeta & 86.30 & 92.37 & 33.74 & 92.76 & 663 & 84.53 & 91.28 & 31.10
& 91.67 & 626 \\
\hline
\derivsbad         & 84.03 & 91.20 & 32.21 & 91.35 & 687 & 82.09 & 89.90 & 29.54
& 90.13 & 675 \\
\derivsbad\optbeta & 84.48 & 91.35 & 32.69 & 91.63 & 645 & 82.86 & 90.39 & 30.10
& 90.59 & 560 \\
\hline
\end{tabular}
\caption[\trsys development results.]{Development results for the \trsys parser
with default and optimised $\beta$s. $\beta$s were optimised over automatic \pos
tags using the \derivsrev model. \label{tab:trsys_dev}}
\end{table}

\begin{table}
\centering
 \begin{tabular}{l|rrrr|rrrr}
\hline
      & \multicolumn{4}{c|}{Optimised}& \multicolumn{4}{c}{Default} \\
Level & $\beta$ & $K$ & Gold & Auto  & $\beta$ & $K$  & Gold & Auto \\
\hline
\hline
1     & 0.2   & 1    & 1181 & 1161 & 0.075   & 20  & 1677 & 1632\\
2     & 0.04  & 30   & 591  &  572 & 0.03    & 20  & 89   & 110\\
3     & 0.06  & 40   &   8  &   10 & 0.01    & 20  & 70   & 73\\
4     & 0.1   & 150  &  20  &   20 & 0.005   & 20  & 19   & 29\\
5     & 0.007 & 15   &  74  &   93 & 0.001   & 150 & 48   & 52\\
6     & 0.001 & 50   &  29  &   38 &         &     &      &    \\
\hline
Cover &       &      & 99.47& 99.01&         &     & 99.48& 99.11\\
\hline
\end{tabular}
\caption[Optimised $\beta$ levels for \trsys parser.]{Optimised $\beta$ levels
for \trsys parser, selected using the \derivsrev model. The gold and auto
columns show the number of sentences successfully parsed at each
level.\label{tab:hattr_betas}}
\end{table}

\subsection{Summary of Performance on Section 00}
\label{sec:dev_summary}
\begin{table}
\centering
\small
\renewcommand{\arraystretch}{0.85}
\setlength{\tabcolsep}{1.5mm}
 \begin{tabular}{ll|cccccc|cccccc}
\hline
       &               & \multicolumn{6}{c|}{Gold \textsc{pos}}         &
\multicolumn{6}{c}{Auto \textsc{pos}}\\
Corpus & Model         & $LF$  & $UF$  & $L$-sent & cat& cover & w/s & $LF$  &
$UF$  & $L$-sent & cat& cover & w/s\\
\hline
\hline
\ccgbank & \hybrid         & 87.27 & 93.01 & 35.94 & 94.16 & 99.06 & 365 & 85.30
& 91.77 & 32.93  & 93.00 & 99.06 & 359 \\
%\ccgbank & \hybridv        & 87.35 & 93.07 & 36.78 & 94.17 & 99.06 & 391 & 85.31
%& 91.77 & 33.30  & 92.97 & 99.06 & 395 \\
\ccgbank & \hybrid\optbeta & 87.46 & 93.14 & 36.02 & 94.13 & 99.17 & 266 & 85.65
& 92.04 & 33.05 & 93.02 & 99.06 & 261 \\
\hline
\hatsys & \hybrid\optbeta  & 87.09 & 92.79 & 35.46 & 93.20 & 99.21 & 430 & 85.04 & 91.58 & 32.68
& 91.93 & 98.9 & 435 \\
\nounary & \hybrid\optbeta & 84.98 & 92.02 & 30.96 & 91.85 & 98.95 & 340 & 82.82
& 90.73 & 28.03 & 90.46 & 98.85 & 343 \\
\trsys   & \derivsrev\optbeta & 86.30 & 92.37 & 33.74 & 92.76 & 99.47 & 663 &
84.53 & 91.28 & 31.10 & 91.67 & 99.01 & 626 \\
\hline
\end{tabular}
\caption[Best \wsj development results.]{Development results for the best model
on each corpus on \wsj Section 00. The \ccgbank \hybrid model is the current
state-of-the-art \ccg parser.\label{tab:dev_summary}}
\end{table}

Table \ref{tab:dev_summary} summarises the results for our six models on Section
00. These are the models we have selected for evaluation on Section 23 and on
the \wikieval corpus. The systems were each trained on different corpora
described in Chapter \ref{chapter:hat_corpus}, but are compared on \ccgbank's
dependencies via the mapping discussed in Section \ref{sec:dependency_mapping}.
The current state-of-the-art is the \ccgbank \hybrid model, so we have also
included it in our comparison.
%We have also included the \ccgbank \hybridv
%model, which uses Viterbi decoding instead of the \citet{clark:cl07}
%dependencies decoder, as described in Section \ref{sec:deriv_deps}.

The three more lexicalised models showed substantial variation in performance. The
relatively low accuracy for the \trsys parser may be due to undetected
implementation problems with the \hybrid model, forcing us to use the \derivs
model which is generally less accurate, as discussed in Section
\ref{sec:hattr_dev}. The label mapping problems discussed in Section
\ref{sec:nounary_dev} may be a mitigating factor in the \nounary poor
performance of the \nounary parser. However, this issue would not affect the
unlabelled dependency scores, so it seems likely that the \nounary corpus simply
makes parsing more difficult. Section \ref{sec:nounary_corpus} discusses several
problematic analyses in the \nounary corpus, all the result of the lack of
descriptive power in the corpus's grammar once the \ccgbank type-changing
rules were compiled out.

The category accuracies for the more lexicalised corpora were considerably lower than
the category accuracy achieved by the \ccgbank parser. Category accuracies were
calculated with the native category sets for each corpus. They were not mapped
to \ccgbank categories, unlike the dependencies. The drop in category accuracy
is caused by the larger, more specific category set that is the result of full
lexicalisation. The fact that there was not a corresponding drop in dependency
accuracy indicates that there was a trade-off between the more difficult
supertagging phase, and a subsequently easier parsing phase, due to the
reduction in ambiguity from removing the type-changing rules from the grammar. The
difference between the category accuracy and dependency results thus supports
our hypothesis that the parsing phase is easier on the more lexicalised corpora.

The \hatsys corpus seems to be a superior strategy for lexicalising the
type-changing rules in \ccgbank. The \hatsys parser achieved accuracy scores
within 0.3\% of the current state-of-the-art model. This suggests that it is
possible to achieve the linguistically desirable properties of full
lexicalisation while maintaining parser accuracy. The \hatsys parser was
substantially faster than the \ccgbank \hybrid parser, however.
%This was partly
%the result of the Viterbi decoding, which is faster than the \citet{clark:cl07}
%dependencies decoder. The \hatsys parser was 26\% faster than the \ccgbank
%\hybrid model.

The most accurate parser was the \ccgbank\hybrid\optbeta model, which
outperformed the current state-of-the-art parser by 0.35\% with automatic \pos
tags. However, it achieved this improvement in accuracy at the expense of a 40\%
drop in efficiency. As we discuss in Section \ref{sec:ccgbank_beta_opt}, this
suggests that it would be desirable to have a way to balance speed and accuracy
in the objective function of the $\beta$ search strategy described in Section
\ref{beta_k}.
The \hatsys\hybrid\optbeta parser was 0.6\% less accurate than
the \ccgbank\hybrid\optbeta model when automatic \pos tags were used.
However, it was also 66\% faster.

\section{Evaluation on \wsj Section 23}
\label{sec:wsj_evaluation}
\begin{table}
\centering
\small
\renewcommand{\arraystretch}{0.85}
\setlength{\tabcolsep}{1.5mm}
 \begin{tabular}{ll|cccccc|cccccc}
\hline
       &                   & \multicolumn{6}{c|}{Gold \textsc{pos}}         &
\multicolumn{6}{c}{Auto \textsc{pos}}\\
Corpus & Model             & $LF$  & $UF$  & $L$-sent & cat   & cover & w/s &
$LF$  & $UF$  & $L$-sent & cat   & cover & w/s\\
\hline
\hline
\ccgbank & \hybrid           & 87.68 & 93.00 & 36.74 & 94.33 & 99.63 & 558 &
85.50 & 91.66 & 33.08  & 92.99 & 99.58 & 472 \\
\ccgbank & \hybrid\optbeta   & 87.73 & 92.99 & 36.84 & 94.18 & 99.92 & 314 &
85.55 & 91.73 & 33.35  & 92.88 & 99.79 & 318 \\
\hline
\hatsys  & \hybrid\optbeta  & 87.45 & 92.76 & 36.98 & 93.36 & 99.09 & 814 &
85.35 & 91.53 & 33.67  & 92.03 & 98.96 & 686 \\
\nounary & \hybrid\optbeta   & 85.24 & 91.78 & 31.59 & 91.90 & 99.42 & 501 &
83.22 & 90.57 & 28.61  & 90.55 & 99.17 & 450 \\
\trsys   & \derivsrev\optbeta& 86.76 & 92.26 & 35.58 & 92.95 & 99.38 & 789 &
84.58 & 90.96 & 31.92  & 91.56 & 99.17 & 673 \\
\hline
\end{tabular}
\caption[\wsj evaluation results.]{Evaluation results for the four models on
\wsj Section 23. The \ccgbank \hybrid model is the current state-of-the-art \ccg
parser.\label{tab:wsj23_results}}
\end{table}
%The standard test set for the \wsj corpus is Section 23. The test set is an
% attempt to model the system's behaviour on novel data, so it is desirable to
% perform as few experiments as possible, because we are interested in estimating
% the performance of a system we have configured on the development data on novel
% text.

Our in-domain evaluation experiments involved running the four models selected
in Section \ref{sec:dev_summary} on Section 23 of the Wall Street Journal. We
report the same figures we did for the development data: unlabelled dependency
$F$-score ($UF$), labelled dependency $F$-score ($LF$), percentage of
sentences with 100\% labelled dependency $F$-score ($L$-sent), category
accuracy (cat), coverage (cover), and speed in words per second (w/s). The
specifics of these measures are described in Section
\ref{sec:evaluation_framework}.

The results are shown in Table \ref{tab:wsj23_results}. The
\ccgbank\hybrid\optbeta system remains the most accurate, although the gap in
accuracy between it and the \ccgbank default $\beta$ configuration is smaller
than it was on the development data. The optimised $\beta$s bring an accuracy
improvement of only 0.05\% with automatic \pos tags, compared to the 0.34\%
improvement on the development data. This suggests that the $\beta$ search
strategy over-fitted to some extent.

One problem for accuracy comparison with these results is the difference in
coverage between the \ccgbank models and the \hatsys model. The
\ccgbank\hybrid\optbeta system achieves near perfect coverage on Section 23,
while the \hatsys system fails to find a spanning analysis for 1.04\% of
sentences using automatic \pos tags. The problem is that the sentences the
\hatsys parser rejects are not evaluated, following the evaluation methodology
of \citet{clark:cl07}. The 1.04\% of sentences rejected by
the \hatsys parser might be long and difficult, which would mean that the
\ccgbank parsers were penalised for returning an analysis for them. This might
be why the \hatsys parser is slightly closer in accuracy to the \ccgbank parsers
on Section 23 than it is on Section 00.

Having lower coverage is not an advantage for parse speed, however, because the
parser never rejects a sentence outright. For a sentence to fail, the parser
must try to find a spanning analysis at each $\beta$ level, so low coverage is
actually less efficient than high coverage. Despite this, the \hatsys parser
remains much more efficient than the \ccgbank parsers. It is 45\% faster than
the \ccgbank default $\beta$ parser, and 215\%
faster than the \ccgbank \optbeta parser, demonstrating the utility of full
lexicalisation for parsing.

The \hatsys parser is faster because it allows the supertagger to decide when to
activate the type-changing rules. In the \ccgbank grammar, a type-changing rule
such as \psunary{\cf{NP}}{\cf{S/(S/NP)}} is an option for every \cf{NP}, even though
many of these productions are unlikely given the context of the word. The \candc
parser creates a packed chart of all derivations generated by its grammar
using the lexical categories supplied to it by the supertagger. It does
not use a beam to search for only likely derivations.

Hat categories offer a way to prevent the parser from considering unlikely type-changing
rules, using the supertagger. If the supertagger cannot decide well between the \cf{NP}
and \cf{NP^{S/(S/NP)}} lexical categories, then both will be supplied to the parser.
The parser will be able to generate derivations using
either category, just as though the type-changing rule were in its grammar.
However, when the supertagger is confident, it might only supply \cf{NP}.
Without the type-changing rule in the grammar, the parser will then not generate the
unlikely analyses relying on \cf{NP^{S/(S/NP)}}.

This design, where the supertagger prunes the parser's search space ahead of time,
is why the \candc parser is far more efficient than comparable parsers. Hat categories
move the parser further in this direction, by making the set of grammatical rules that
are always available even smaller, and giving even more derivational control to the
supertagger.



\section{Cross-Domain Evaluation on Wikipedia}
\label{sec:wikipedia_results}

As we described in Chapter \ref{chapter:background}, statistical parsing results
have shown consistent improvement over the fifteen years or so since the Penn
Treebank was released. Having a large, high quality shared resource has been
hugely important for the development of the field, but it has also resulted in a
degree of domain dependence. Domain dependence raises two issues for our
evaluation, related to two ways evaluation can be used.

First, there is the basic question of how well our parser currently performs.
This kind of evaluation can address questions about how viable parsing is as an
approach to various tasks, although ultimately this can be tested empirically.
For this kind of evaluation, our main concern for domain dependence is
quantifying the loss of performance when the parser (or other \nlp tool) is
ported to a new domain.

The second question is whether conclusions we have previously reached in our
research on the training data hold true for other domains. In other words, has
our research agenda been overfit to the training domain, or could our methods be
applied equally well to a new domain given a sufficient quantity of labelled
data? This type of over-fitting is very important to uncover. Labelled data may
be costly, but research energy is far more expensive.

We prepared a cross-domain evaluation for the \candc parser to address the
second question --- the issue of whether our conclusions are valid in a
cross-domain context. We were concerned about this because lexicalisation
arguably increases the system's dependence on extracting a high coverage lexicon
from the training data, and that extraction process might be very domain
specific.

A cross-domain evaluation also lets us examine whether some of the previous
observations about the \candc parser hold true. For instance,
\citet{clark:acl04} came to the novel conclusion that a supertagging phase was
important for accuracy as well as speed, by showing that the parser was more
accurate when the supertagger supplied a smaller set of categories. We can also
see whether more subtle observations, such as that the hybrid model out-performs
the derivs model, hold true on the Wikipedia data.

We chose Wikipedia, a free online encyclopedia, as the target for our
cross-domain evaluation. We chose this domain because we wanted text which
conformed to standard English grammar, allowing us to factor out some of the
well observed problems that arise when newswire tools are applied to technical
or informal domains. We wanted a domain that was different from 1989 financial
news text, but one which was similar enough that we would hope our more general
research conclusions would remain valid.

Wikipedia is also an interesting domain in its own right. It has become a hugely
influential resource for \nlp, because it is large (the English Wikipedia dump
we used had approximately 1 billion words using our pre-processing method),
available in multiple languages, and contains a variety of interesting
meta-data. A review of the \nlp applications using Wikipedia is beyond the scope
of this thesis, but it has been particularly prominent for entity recognition
\citep{nothman:09} and disambiguation \citep{bunescu:06}, and lexical semantics
\citep{strube:06}. This means that we can also make a useful contribution to the
first kind of evaluation question, by investigating how well a state-of-the-art
parser performs on this domain. Our results are the first published accuracy
evaluation of a parser on Wikipedia. The only previous investigation of parsing
performance on Wikipedia was performed by \citep{ytrestol:09}, who ran the LinGo
\hpsg parser \citep{oepen:04} over Wikipedia, and found that the correct parse
was in the top 500 returned parses for 60\% of sentences.

Our cross-domain evaluation was conducted on 200 sentences, which were manually
annotated by a single annotator. This is far from ideal, but it is enough to
assess the issues we are most interested in. We first describe how we selected
and pre-processed the sentences for annotation. Next, we discuss the annotation
procedure, and how future \ccg manual annotation might be performed faster.
Finally, we perform the evaluation and discuss our findings.

\subsection{Preparing the Data}
\label{sec:wiki_preprocessing}

In general, our pre-processing strategy was to throw away any data that was
likely to be noisy or non-sentential. The evaluation set would by necessity be
small, and we wanted to avoid having any noisy sentences in the data. We
therefore chose not to expand occurrences of the Wikipedia templates, to avoid
including boilerplate sentences in our evaluation data. We also discarded any
non-paragraph text, such as lists, picture captions, etc.

The pre-processing was performed using the Wikipedia processing tools developed
by \citet{nothman:09}. These tools are largely an interface for database access
of articles, which are then parsed using \texttt{mwlib} \citep{mwlib}. The tools
also make the link graph available.

We used a simple relevance metric to avoid selecting sentences from the long
tail of less relevant articles in Wikipedia. We wanted to get sentences from the
types of articles people were likely to encounter when navigating Wikipedia. Our
metric is therefore based on the link-structure of the encyclopedia. We judged
an article's relevance according to how many other articles linked to it. This
was easier to compute than a more sophisticated measure like PageRank
\citep{pagerank}, and satisfied our requirements. We also imposed an additional
constraint on the articles we could draw sentences from. We were initially
interested in using the Simple English Wikipedia as self-training data,
hypothesising that the basic English used in that corpus would be easier to
parse. We therefore excluded all articles whose title matched a Simple English
Wikipedia article, to avoid evaluating on sentences whose subject matter was too
similar to our self-training data. Having applied this filter, we selected the
5,000 articles that were linked to the most from other Wikipedia pages.

Once we had a set of candidate articles, we set about extracting sentences from
them. We extracted paragraphs from the \texttt{mwlib} MediaWiki parse trees, and
split them into sentences using the Punkt sentence boundary detector
\citep{punkt}. Punkt uses an unsupervised algorithm, so it could be
parameterised on Wikipedia text. It was also a convenient option, because it is
freely distributed with \nltk \citep{nltkbook:09,nltkweb}. The sentences were
then tokenised according to the Penn Treebank standard, using custom regular
expressions.

\subsection{Annotating the Wikipedia Data}

The \wikieval annotation set consists of 200 Wikipedia sentences which we
manually annotated with \ccg derivations. Only one annotator was available.
Because the goal of the annotation was to evaluate the \candc parser, we tried
to stay as close to the \ccgbank annotation guidelines \citep{hock:thesis03} as
possible. The only exception to this policy was that we did not attempt to
recreate the known problems with \ccgbank annotation. Specifically, \ccgbank
contains incorrect noun phrase brackets, because there was no way to accurately
binarise the flat Penn Treebank noun phrase brackets automatically.

The most difficult annotation decisions involved complement/adjunct distinctions
for prepositional phrases that attached to verbs. \citet{propbank} showed that
the accuracy of these decisions can be substantially improved if they are made
with reference to a lexicon, rather than on a case-by-case basis. We therefore
referred to the \ccgbank and PropBank lexicons to guide our decisions. 

\citet{marcus:93} found that correcting parser output is both faster and less
error-prone than treebanking entirely by hand. We began by parsing the sentences
with the \candc parser. Four of the two hundred sentences could not be parsed.
For these sentences we just used the most likely categories returned by the
supertagger. Our strategy was to correct the supertags, which could then be
supplied back to the parser to produce very high quality parses. We did not
correct the part-of-speech tags. \citet{clark:cl07} report that with gold
standard supertags, the parser achieves 97\% $F$-score on labelled dependencies.
We hoped that this would mean most of the annotation could be done in a standard
text editor.

We corrected the supertags for the first 50 sentences of the annotation in 85
minutes. Annotation speeds remained fairly constant, and the supertag correction
phase was completed in just over five hours. In hindsight, it would have been
better to use the multi-tagging output of the supertagger, instead of the
categories selected by the parser. The multi-tagger returns the correct category
for 98\% of words, with an average of 3.4 categories returned per word. This
would have allowed the annotator to simply select the correct tag from three or
four alternatives, which should be possible fairly quickly. A comparison of
supertag annotation methods would be a useful contribution to the field, as
previous work \citep{clark:emnlp04,rimell:08} has shown that supertag annotation
is almost as useful as full \ccg treebanking.

Once the supertags were corrected, we supplied them to the parser in a
\textsc{gui} that displayed the derivation produced using the selected
supertags. If the analysis was incorrect, the annotator could change the
supertags, or add a constraint to the chart that prohibited the current
analysis. Three kinds of constraint were possible. With a \emph{require}
constraint, only derivations which bracketed together some sequence of words
could be constructed. A \emph{match} constraint was similar, but additionally
specified a label for the bracket. The third constraint, \emph{exclude},
prohibited a span of words from being bracketed together. Usually, the
constraints were used to resolve attachment ambiguities. The most common case
involved two adnominal prepositional phrases. For instance, consider the
sentence:

\begin{lexamples}
 \item We all know the man on the hill with the telescope.
\end{lexamples}

If we assign both \emph{on} and \emph{with} the category \cf{(NP\bs NP)/NP},
there is still an attachment ambiguity. \emph{with} could attach to \emph{hill}
or \emph{man}. A \emph{require} constraint will not help here, but if we
\emph{exclude} a bracket spanning \emph{the hill with the telescope}, the
attachment ambiguity is resolved, and the correct parse will be displayed. The
interactive interface was also useful for debugging remaining problems with the
supertags. Careless mistakes and syntax errors were immediately flagged, because
the supertags could not be used to construct a parse.

The annotation took approximately 18 hours in total. The two phase process
turned out to be less useful than we had hoped. Doing the annotation in two
passes meant that each sentence had to be mentally analysed twice, because the
initial annotation decisions were only dimly familiar by the time the sentence
came up again in the second pass. It would have been better to add a text box to
perform the supertagging to the chart constraint interface. This would have
allowed the annotator to perform the annotation in a single pass.


\subsection{Wikipedia Evaluation}

\begin{table}
\centering
\small
\renewcommand{\arraystretch}{0.85}
\setlength{\tabcolsep}{1.5mm}
 \begin{tabular}{ll|ccccc|ccccc}
\hline
         &                    & \multicolumn{5}{c|}{Wikipedia}       &
\multicolumn{5}{c}{\textsc{wsj} 23 Auto}\\
Corpus   & Model              & $LF$  & $UF$  &$L$-sent& cover & w/s & $LF$  &
$UF$  &$L$-sent& cover & w/s\\
\hline
\hline
\ccgbank & \derivsrev         & 80.86 & 88.73 & 27.78  & 98.71 & 231 & 85.06 &
91.41 & 33.08 & 99.58 & 545 \\
\ccgbank & \hybrid            & 81.22 & 88.75 & 26.77  & 98.71 & 225 & 85.50 &
91.66 & 33.08 & 99.58 & 472 \\
\ccgbank & \hybrid\optbeta    & 81.24 & 88.62 & 28.00  & 98.57 & 157 & 85.55 &
91.73 & 33.35 & 99.79 & 318 \\
\hline
\hatsys  & \hybrid\optbeta   & 81.12 & 88.41 & 26.53  & 98.47 & 288 & 85.35 &
91.53 & 33.67 & 98.96 & 686 \\
\nounary & \hybrid\optbeta    & 79.18 & 87.60 & 24.37  & 98.1  & 252 & 83.22 &
90.57 & 28.61 & 99.17 & 450 \\
\trsys   & \derivsrev\optbeta & 80.93 & 88.38 & 26.77  & 98.8  & 328 & 84.58 &
90.96 & 31.92 & 99.17 & 673 \\
\hline
\end{tabular}
\caption[Wikipedia evaluation results.]{Comparison of the more lexicalised \hatsys,
\nounary and \trsys models against \ccgbank trained parsers on
Wikipedia.\label{tab:wiki_results}}
\end{table}

We evaluated the four models that performed best on the development data, as
described in Section \ref{sec:dev_summary}, on the two hundred sentence
\wikieval evaluation set we manually annotated. We also evaluated two of
\citepos{clark:cl07} models, the \ccgbank \hybrid and \ccgbank \derivsrev
models. The results are shown in Table \ref{tab:wiki_results}. Because the test
set was only two hundred sentences, we performed separate coverage and speed
evaluation on a set of 10,000 unannotated sentences that were pre-processed and
selected in the same manner as the \wikieval sentences, described in Section
\ref{sec:wiki_preprocessing}.

The \ccgbank \hybrid and \ccgbank \hybrid \optbeta models achieved the highest
accuracy. However, the variation in performance was quite small, apart from the
\nounary corpus, which performed over 2\% worse than the \ccgbank models. The
\hatsys parser was only 0.1\% less accurate, and the \trsys was behind by only
0.3\%. The drop in performance between Wikipedia and the in-domain \wsj 23
evaluation was fairly constant, except for the \trsys result, which caught up to
the other models in accuracy.

% \begin{table}
% \centering
%  \begin{tabular}{lrr|cc|cc}
%      &       &     & \multicolumn{2}{c}{Wikipedia}      &
%\multicolumn{2}{c}{\wsj 00}\\
%      &$\beta$& $K$ & Cats per word & \% unparsed        & Cats per word & \%
%unparsed \\
% \hline
% \hline
% 1    & 0.075 & 20  & ??            & 100                &  ??             &
%100\\
% 2    & 0.03  & 20  & ??            & 7.7                &  ??             &
%6.7\\
% 3    & 0.01  & 20  & ??            & 5.4                &  ??             &
%4.2\\
% 4    & 0.005 & 20  & ??            & 3.5                &  ??             &
%3.0\\
% 5    & 0.001 & 150 & ??            & 2.9                &  ??             &
%2.1\\
% \hline
% %-    &       &     &              & 1.3                &               & 0.5
%  \end{tabular}
% \caption[Supertagger-parser integration in Wikipedia and \wsj]{Sentences to be
%parsed at each of the \candc-default $\beta$ levels in Wikipedia and \wsj
%Section 00.\label{tab:supertag_table}}
% \end{table}
%\subsubsection{Investigation of Loss of Speed}

We were surprised to find how much the parsers dropped in speed on the
out-of-domain data. All of the parsers were roughly twice as efficient on the
in-domain data. Wikipedia has fewer tokens per sentence than the \wsj using our
pre-processing (22.4 vs. 23.5), so we doubt that the text is simply harder to
parse. Instead, we hypothesised that the loss of speed was due to the way the
supertagger and parser are integrated in the \candc parser. As we described in
Section \ref{sec:background_candc} and Section \ref{beta_k}, the supertagger
supplies the parser with a set of categories per word, and the parser then
attempts to build a spanning analysis. If it cannot, the supertagger supplies
more categories, and the parser tries again.

There are two ways the \candc parser might become slower when the supertagger
encounters text outside of its training domain. First, out-of-domain data may
cause the supertagger to supply lower quality categories to the parser. There
will then be more cases where the parser will fail to build a spanning analysis,
causing it to drop down a $\beta$ level and try again. The second possibility is
that the supertagger will simply supply the parser with more categories per
word. The number of categories supplied for a given word is largely determined
by the way the supertagger distributes probability mass between its categories.
A uniform distribution will result in more categories being assigned to that
word. This means that if the supertagger assigns higher entropy probability
distributions to categories in the out-of-domain data, the parser will receive
more categories per word, on average.

%TODO: Fix this  To investigate how much each of these factors made an impact on
% the parse speeds on Wikipedia, we compared the average number of categories
% supplied per word and the percentage of sentences being parsed at each $\beta$
% level on Wikipedia against Section 00. We used the set of 10,000 unannotated
% Wikipedia sentences for this comparison, to make our estimates on Wikipedia more
% robust. We chose Section 00 to avoid over-analysing the parser's performance on
% Section 23. We used the \candc default $\beta$ levels and the default \ccgbank
% supertagger. Automatic \pos tags were used for both Wikipedia and the \wsj.
% 
% Table \ref{tab:supertag_table} presents the comparison. At each level, we are
% interested in the percentage of sentences that have not been parsed yet, and the
% average number of categories the supertagger supplies per word for that set of
% sentences. First, we see that the first $\beta$ level takes care of the majority
% of sentences for both Wikipedia and the \wsj, but the difference is only 1\%,
% and the gap in coverage does not grow as the sentences progress through the
% levels. The difference between the first level failures is thus almost entirely
% the result by the difference in coverage, which is 0.8\% lower on Wikipedia.
% TODO: discuss cats per word.



The most important result from the Wikipedia evaluation is that the more lexicalised
parsers are no more domain sensitive than those trained on \ccgbank. The
advantages and disadvantages of each more lexicalised model remained fairly constant,
except for fact that the \trsys parser is closer in accuracy than it is on the
\wsj. The \hatsys parser is 28\% faster than the \ccgbank \hybrid model,
maintaining its advantage of trading a small amount of accuracy for a large
increase in efficiency. It is also 83\% faster than the \ccgbank \optbeta
system, which was only slightly more accurate than the \ccgbank system with
default $\beta$s.

\section{Summary}

In this chapter, we described the evaluation of parsers trained on the \hatsys,
\nounary and \trsys corpora. The parsers were first parameterised on Section 00,
to select an appropriate model for each corpus, in order to avoid selecting
parameter values that were tuned for the original \ccgbank.

The main difference between the lexicalised corpora and \ccgbank was the level
of lexicalisation. The new corpora all used a much smaller grammar than
\ccgbank, because their grammars did not include the \ccgbank type-changing rules.
Instead the work performed by the unary rules was shifted into the lexicon. For
the \candc parser, this had the effect of making the supertagging phase more
difficult, but the chart parsing phase easier.

This new division of labour generally made the parser more efficient. The
exception was the \nounary corpus. We believe that the low supertagging accuracy
for this corpus explains this result. If lexicalisation is achieved at the
expense of too much category ambiguity, the supertagger supplies the parser with
larger, less accurate sets of categories. When this occurs, the parser fails to
find a spanning analysis more often, and the average chart sizes are larger.
This prevents the potential efficiency gains of a smaller grammar being
realised.

Performing more work in the supertagger can speed up parsing because parse
speeds are dominated by chart ambiguity. By making the grammar smaller, the
chart ambiguity can be reduced, so long as the supertagger is suitably accurate.
Another way to understand this interaction is to say that the work is shifted to
the supertagger, which runs in linear time, away from the parser, where the time
complexity is sensitive to the grammatical ambiguity.




% 
 \chapter{Conclusion}

This thesis has described a problem with Combinatory Categorial Grammar (\ccg),
and proposed a solution. We implemented our solution for use in a statistical
parser, alongside a control experiment. Our experiments investigated the effect
of removing a set of exceptional cases from \ccg, which had been added to the
grammar to address sparse data problems, but which contradict the central
hypothesis of the linguistic theory. We demonstrated via our control experiment
on the \nounary corpus
that without these exceptions the sparse data problems do become unmanageable.
This established that the existing choices for the \ccg theory were between
analyses that could not scale to a wide-coverage corpus, and an extension to the
grammar that contradicted its central hypothesis. Our solution provides an
analysis that scales just as well as the \citet{hock:cl07} type-changing rules
proposal, but is also compatible with the central claims of the theory.

The \ccg hypothesis is that surface syntax is a
completely transparent interface between the linguistic input signal and a
compositional interpretation of its meaning, where each syntactic operation
corresponds exactly to a simple semantic operation. \ccg also adopts the theory of
radical lexicalism: that the innate universal component of the human language
processor is exactly the grammar, and the acquired portion is exactly the
lexicon. These hypotheses are investigated by formulating proposals about the
objects that populate the lexicon and the innate grammar that manipulates them,
and testing whether they explain the data: observed variation in human
language. If we cannot describe the nature of a language faculty that conforms
to these hypotheses and explains our observations about language, we must
abandon them until we can.

\citet{baldridge:03} argued that a universal \ccg consisting of application,
composition and type-raising rules explained the bulk of our observations about
language very well. We have argued that this \ccg grammar actually encounters
problems that limit how efficiently it can be scaled beyond the limited set of
cases traditionally used to probe a grammar. The consequences of the problem we
identify had already been observed, when \citet{hock:lrec02} attempted to
implement a wide-coverage \ccg grammar on a large corpus of English. Since the
production of a practical corpus was the main aim of study,
\citet{hock:thesis03} designed a grammar that did scale well, and was well
suited to statistical parsing. However, the scalable grammar compromises some of
the key theoretical properties of the formalism. \citeauthor{hock:thesis03}'s grammar
is not language universal, and its syntactic rules are not necessarily paired with 
elementary semantic operations.

We argue that the adoption of this grammar for practical \ccg language
processing tasks has masked what should have been an interesting observation. This
issue, which we have explored in detail for the first time, is that the \ccg
formalism makes it difficult to encode some central, linguistically relevant
generalisations. This forces \ccg grammars to
adopt analyses that work well on the limited scale used in traditional methodology,
but are exposed as inadequate when they are implemented on a system capable of
handling the range of syntactic phenomena a human can process.

We address this by proposing an
updated model that is compatible with the \ccg hypotheses. The new model solves
the inefficiency and over-generation that caused the scalability problems the
previous application, composition and type-raising proposal encountered. The
main change in our proposal is to the structure of the linguistic objects the
\ccg rules manipulate, the \emph{categories}. We propose that \ccg categories
must contain an additional attribute, which we have dubbed \emph{hat}. The
addition of this attribute allows the grammar to account for a linguistically
relevant property it could not previously accommodate, constituent type.
Accounting for constituent type allows the grammar to make more efficient
generalisations, and ultimately provide more convincing explanations of
syntactic phenomena. We do add one rule to the grammar, which is used to
manipulate the new hat attribute. We show that this addition does not change the
weak generative power of the grammar, ensuring that our model still explains the
observed upper bound on the power required to generate natural languages.

We adopt a data-driven methodology to test our proposal. We do this because we
claim that the issue we have identified prevents the existing model from scaling
to a wide-coverage grammar. To test this, we adapted a large corpus of
\ccg-annotated sentences of English, \ccgbank \citep{hock:cl07}, to implement
analyses using the hat categories we proposed. We also implemented our
proposal on a state-of-the-art statistical parser \citep{clark:cl07}. Our corpus
and parsing experiments are an indirect way of testing our proposal, because
errors in the corpus and the parser's statistical model can act as
confounding variables. However, we compare our solution against an exacting
standard: the original \ccgbank grammar, which was designed to maximise the
efficiency of current statistical parsers on noisy analyses. The \ccgbank
grammar does not completely adopt two key \ccg hypotheses of combinatory
type-transparency and radical lexicalisation.

Our proposed, newly scalable \ccg model actually improved on the \ccgbank
grammar results, by achieving a substantial increase in efficiency at a very
small reduction in accuracy. That is, by adopting the constraints of the \ccg
hypothesis, a parser implementing our model achieves a more favourable trade-off
between speed and accuracy than a
parser implementing the grammar solely concerned with its practical properties.
In contrast, a parser trained using the prior \ccg model of application,
composition and type-raising rules performed substantially worse than the \ccgbank
grammar. The confounding variables of analysis noise and the specifics of the
\ccg parser we use mean that the negative result must be interpreted with
caution. It does not demonstrate that effective parsing with an application,
composition and type-raising model is impossible. But it does demonstrate that
\ccg theory-compliant parsing is difficult, and establishes that hat categories
are currently the best approach to the problem.

We summarise our key contributions as follows. First, we have proposed a grammar
conforming to the \ccg hypotheses that is more efficient than the current
proposal. Second, we have shown that conforming to these hypotheses makes a
statistical parser more efficient. Both of these contributions have interesting
implications.

% We achieve the increased efficiency by making it easier to express linguistically
% relevant generalisations in \ccg grammars.
% We noted in Chapter \ref{chapter:ling_mot} that this allowed \ccg to
% replicate the analysis of a related formalism, \ltag, for a construction which
% previously posed problems. One of the open questions for modern generative
% grammars is the relationship between the strong generative powers of the four
% most prominent syntactic theories: \ccg, \hpsg, \ltag and \lfg. Increasing the
% strong generative power of \ccg therefore raises interesting questions. Which
% desirable analyses from the other formalisms can we still not replicate in \ccg?
% Precisely which aspects of the \ccg machinery enable the \ccg analyses that
% other formalisms have trouble replicating? One way to pursue these questions is
% to investigate the extended domain of locality and factoring of recursion
% properties of \ccg categories more deeply. We touched on these issues in Chapter
% \ref{chapter:ling_mot}, but what we have not done is pursue our conjecture that
% hat categories offer a way to produce \ccg analyses with lexical categories that
% always exhibit the extended domain of locality property. This property has been
% identified as one of the most important aspects of the \ltag formalism, and the
% explanation for many of its advantages over context-free phrase structure
% grammars \citep{joshi:04}. Perhaps adopting this property for \ccg categories
% will allow \ccg to replicate all of the desirable \ltag analyses --- and if it
% does not, we will have obtained valuable clues about where the formalism might
% be lacking.

% Implementing the \ltag analyses described by the \xtag grammar \citep{xtag} will
% also allow us to reduce the confounding variables in our methodology.
% Specifically, it will help us to make further corrections to \ccgbank, reducing
% the noise in the corpus.
Our results suggest an appealing feedback loop
between the application of a linguistic theory and the proposals of the theory
itself. Paying close attention to the problems that arose when the formalism was
applied in a statistical parser led us to a weakness in the formalism.
Correcting this weakness improved the performance of the system. The implication
of this is that we should attempt to make the corpus and parser as faithful an
implementation of the theory as possible. Substantial progress has already been
made on this front. Some of the problematic analyses that \ccgbank inherited
from the Penn Treebank have been corrected
\citep{honnibal:pacling07prop,vadas:08,white:punct08}, and initial exploratory work on
adopting \mmccg in \ccgbank has been performed \citep{tse:honours}. These
updates are currently either incomplete, or currently awaiting release, so we
have not included them in our experiments. However, each of these resources
would reduce some of the problematic analysis noise we encountered. Further work
in this direction would include using Nombank \citep{nombank} to correct the
predicate-argument structure of nouns, and better analyses of multi-word
expressions in the grammar, particularly for named entities and verb-particle
constructions. It would also be interesting to explore the use of the \textbf{D}
combinator proposed by \citet{hoyt:08} in \ccgbank, to assess its interaction
with the other \ccg combinators in wide-coverage parsing.

These changes will make the corpus more theoretically sound, improving the
impact of the methodology we have demonstrated. The strength of this methodology
is that it provides a different type of evidence from the native speaker
intuition which has always been so important in syntactic inquiry. It provides a
way to build a precise model of the \emph{process} of the human language
faculty, where much of the effort in syntactic description has focused on the
objects being manipulated. It also has important technological implications. Our
processing power and the text we have available for processing continues to grow
exponentially, creating a pressing need for faster and more accurate ways of
extracting information from text, to capitalise on the opportunity modern
computing has made available. There is therefore a considerable \emph{practical}
benefit to developing better linguistic theories, as we have shown by improving
the efficiency of a state-of-the-art statistical parser. There is also a considerable
\emph{theoretical} benefit to crafting formal proposals precisely enough to be implemented,
so that they can be evaluated empirically.



% 
 \begin{appendix}
\chapter{Type-Changing Grammar of \ccgbank}

\label{appendix:type-changing}


 \begin{tabular}{ r | rcl }
\hline
Frequency & \multicolumn{3}{c}{Unary Type-Changing Rule} \\
\hline\hline
115533 & \cf{N}& $\longrightarrow$  &\cf{NP} \\
3323 & \cf{S[pss]\bs NP}& $\longrightarrow$  &\cf{NP\bs NP} \\
1506 & \cf{S[ng]\bs NP}& $\longrightarrow$  &\cf{NP\bs NP} \\
1333 & \cf{S[adj]\bs NP}& $\longrightarrow$  &\cf{NP\bs NP} \\
1323 & \cf{S[to]\bs NP}& $\longrightarrow$  &\cf{(S\bs NP)\bs (S\bs NP)} \\
1227 & \cf{S[ng]\bs NP}& $\longrightarrow$  &\cf{(S\bs NP)\bs (S\bs NP)} \\
1176 & \cf{S[to]\bs NP}& $\longrightarrow$  &\cf{NP\bs NP} \\
1174 & \cf{S[to]\bs NP}& $\longrightarrow$  &\cf{N\bs N} \\
873 & \cf{S[dcl]/NP}& $\longrightarrow$  &\cf{NP\bs NP} \\
309 & \cf{S[ng]\bs NP}& $\longrightarrow$  &\cf{NP} \\
206 & \cf{S[ng]\bs NP}& $\longrightarrow$  &\cf{S/S} \\
176 & \cf{S[dcl]}& $\longrightarrow$  &\cf{NP\bs NP} \\
141 & \cf{S[pss]\bs NP}& $\longrightarrow$  &\cf{S/S} \\
124 & \cf{S[to]\bs NP}& $\longrightarrow$  &\cf{S/S} \\
115 & \cf{S[adj]\bs NP}& $\longrightarrow$  &\cf{(S\bs NP)\bs (S\bs NP)} \\
103 & \cf{S[ng]\bs NP}& $\longrightarrow$  &\cf{(S\bs NP)/(S\bs NP)} \\
99 & \cf{(S[to]\bs NP)/NP}& $\longrightarrow$  &\cf{NP\bs NP} \\
87 & \cf{S[ng]\bs NP}& $\longrightarrow$  &\cf{S\bs S} \\
85 & \cf{S[pss]\bs NP}& $\longrightarrow$  &\cf{(S\bs NP)\bs (S\bs NP)} \\
62 & \cf{S[dcl]}& $\longrightarrow$  &\cf{S\bs S} \\
59 & \cf{S[adj]\bs NP}& $\longrightarrow$  &\cf{S/S} \\
43 & \cf{S[pss]\bs NP}& $\longrightarrow$  &\cf{(S\bs NP)/(S\bs NP)} \\
27 & \cf{NP}& $\longrightarrow$  &\cf{S/(S/NP)} \\
22 & \cf{S[dcl]\bs NP}& $\longrightarrow$  &\cf{NP\bs NP} \\
16 & \cf{S[dcl]}& $\longrightarrow$  &\cf{((S\bs NP)\bs (S\bs NP))\bs ((S\bs NP)\bs (S\bs NP))} \\
15 & \cf{S[dcl]}& $\longrightarrow$  &\cf{S/S} \\
15 & \cf{S[adj]\bs NP}& $\longrightarrow$  &\cf{(S\bs NP)/(S\bs NP)} \\
12 & \cf{S[dcl]/S[dcl]}& $\longrightarrow$  &\cf{S/S} \\
12 & \cf{S[adj]\bs NP}& $\longrightarrow$  &\cf{(S[adj]\bs NP)\bs (S[adj]\bs NP)} \\
12 & \cf{S[dcl]}& $\longrightarrow$  &\cf{(S/S)\bs (S/S)} \\
10 & \cf{S[ng]}& $\longrightarrow$  &\cf{(S\bs NP)\bs (S\bs NP)} \\
\hline
\end{tabular}

\begin{tabular}{ r | rcl }
\hline
Frequency & \multicolumn{3}{c}{Invalid Binary Rule} \\
\hline\hline
241 & \cf{S[dcl]/S[dcl]}& $\longrightarrow$  &\cf{S/S} \\
193 & \cf{, NP}& $\longrightarrow$  &\cf{(S\bs NP)\bs (S\bs NP)} \\
155 & \cf{S[dcl]/S[dcl]}& $\longrightarrow$  &\cf{(S\bs bs NP)/(S\bs bs NP)} \\
76 & \cf{S[dcl]/S[dcl]}& $\longrightarrow$  &\cf{(S\bs bs NP)\bs bs (S\bs bs NP)} \\
61 & \cf{conj S[em]}& $\longrightarrow$  &\cf{S[dcl][conj]} \\
50 & \cf{NP}& $\longrightarrow$  &\cf{S/S} \\
48 & \cf{S[dcl]/S[dcl]}& $\longrightarrow$  &\cf{S\bs bs S} \\
29 & \cf{S[dcl]\bs bs S[dcl]}& $\longrightarrow$  &\cf{S/S} \\
29 & \cf{NP[nb]/N NP\bs NP}& $\longrightarrow$  &\cf{NP[nb]/N} \\
20 & \cf{conj PP}& $\longrightarrow$  &\cf{S[adj]\bs NP[conj]} \\
18 & \cf{conj NP}& $\longrightarrow$  &\cf{S[adj]\bs NP[conj]} \\
18 & \cf{conj S[adj]\bs NP}& $\longrightarrow$  &\cf{NP[conj]} \\
17 & \cf{S[dcl]}& $\longrightarrow$  &\cf{S/S} \\
16 & \cf{S[dcl]\bs bs S[dcl]}& $\longrightarrow$  &\cf{(S\bs bs NP)/(S\bs bs NP)} \\
13 & \cf{(NP\bs NP)/(S[dcl]\bs NP) (NP\bs NP)\bs (NP\bs NP)}& $\longrightarrow$  &\cf{(NP\bs NP)/(S[dcl]\bs NP)} \\
12 & \cf{S[dcl]/S[dcl]}& $\longrightarrow$  &\cf{NP\bs bs NP} \\
11 & \cf{conj S[ng]\bs NP}& $\longrightarrow$  &\cf{S[adj]\bs NP[conj]} \\
11 & \cf{S[dcl]\bs bs S[dcl]}& $\longrightarrow$  &\cf{(S\bs bs NP)\bs bs (S\bs bs NP)} \\
10 & \cf{conj S[ng]\bs NP}& $\longrightarrow$  &\cf{S[pss]\bs NP[conj]} \\
10 & \cf{((S[dcl]\bs NP)/PP)/NP (S\bs NP)\bs (((S\bs NP)/PP)/NP)}& $\longrightarrow$  &\cf{S[dcl]\bs NP} \\
\hline
\end{tabular}

\end{appendix}



\bibliography{thesis}

\bibliographystyle{aclnat}

\end{document}
