\chapter{Combinatory Categorial Grammar}
\label{chapter:background}
Categorial grammars (\cg) have their oldest roots in philosophy, as a logical
notation. Philosophy and logic have always maintained an interest in natural
language, so the suitability of the formalism for syntactic description was
commonly noted, even though it was not a central concern. \citet{wood:93} finds
early kernels of \cg ideas in \citepos{frege:1879} analysis of a proposition
into function and argument, rather than subject and predicate. The first full
formulation of a recognisable categorial grammar was presented by
\citet{ajdukiewicz:35}, but it was not until \citet{bar-hillel:53} that a \cg
description of a variety of syntactic phenomena was presented, motivated by the
post-war interest in machine translation that followed the success of digital
cryptanalysis \citep{wood:93}.

Interest in \cg decreased for some time after \citet{bar-hillel:60} proved that
categorial grammars could only generate context-free languages, while natural
languages were considered at the time to be context sensitive.\footnote{\citet{pullum:82}
showed that the existing arguments for the non-context freeness of
natural language were flawed, but they were nonetheless taken
seriously at the time, and \citet{wood:93} ascribes the partial loss of interest in \cg
to this. Natural languages were finally shown to be non-context free by \citet{shieber:85}
and \citet{culy:85}.}
It was considerably later that \cg's advantage --- a simple isomorphism between syntax and
semantics --- motivated
proposals that aimed to increase its weak generative capacity. The proposals can be divided
into two groups. Type-logic grammars \citep{lambek:88,benthem:88,morrill:94,moortgat:88}
extend \citepos{lambek:58} algebraic categorial calculus, and are largely concerned with the
grammar's formal and logical properties. The other approach is to extend the
\citeauthor{ajdukiewicz:35} and \citeauthor{bar-hillel:53} formulation, referred to as either
\textsc{ab} and pure categorial grammar, using combinatory rules drawn from \citet{curry:58}.

Combinatory Categorial Grammar (\ccg), first formulated by \citet{ades:80},
and presented more fully in \citet{steedman:00}, can be distinguished from the
categorical type-logic tradition by its close attention to linguistic and
psychological plausibility. The upshot of these concerns is a keen interest in
restricting the generative power of the formalism to precisely match the
complexity observed in natural languages. The restrictions also improve the
grammar's computability, making the formalism attractive for use in language
technology and arguably increasing its psychological plausibility
\citep{steedman:00}.

This thesis is concerned with the theoretical and practical properties of
combinatory categorial grammar, but we will start by defining the simpler
\textsc{ab} categorial grammar it is based on. We will then describe the rules
\citeauthor{steedman:96} introduced. These rules allow the grammar to handle the
long-distance dependencies that arise in coordination and extraction
constructions. We
also describe an important addition to \ccg that draws on categorical type-logic,
\citepos{baldridge:03} multi-modal slashes. This extension is the
limit of our interest in the type-logical tradition of work on categorial
grammars.

Once we have described the grammar, we turn to its recent use in natural language
engineering. The work most relevant to this thesis begins with the creation of a corpus of
\ccg derivations created by semi-automatically adapting
the Penn Treebank, \ccgbank \citep{hock:acl02}. This initiated a fruitful line of research in statistical \ccg parsing,
culminating in the \candc parser \citep{clark:cl07}, which we will use for our experiments
in Chapter~\ref{chapter:results}. We briefly review other prominent applications of \ccg in
natural language processing; for semantic analysis, natural language generation, and machine
translation.


\section{Category Definition}
\label{sec:ab_cat}

Categorial grammar (\cg) is a lexicalised framework, which means that the
majority of the syntactic information required to build a derivation is
specified in a sequence of composite objects paired with the input. One object
is assigned to each phonologically realised, meaning-bearing unit in a sentence,
which in English usually means one object per orthographic token. In a
categorial grammar, these composite objects are called \emph{categories}.

\cg lexical categories are composed of a result category, and potentially an
argument. Categories which specify an argument are called \emph{functors}, or
\emph{complex categories}; categories without an argument are referred to as
\emph{atomic}. In most \cg formulations, functors can specify the direction of
their argument. In the notation we will use throughout this thesis, the result
category is written on the left, with the argument to its right after a forward
(\cf{/}) or backward (\cf{\bs}) slash, for rightward and leftward arguments
respectively. We also occasionally use an undirected slash, \cf{|}, to
under-specify a slash from \cf{\{/, \bs\}}.

The \citet{ajdukiewicz:35} and \citet{bar-hillel:53} formalism
(termed \abcg) used just two atomic categories in its description,
\cf{N} (for substantives) and \cf{S} (for clauses).
This set of categories is generally considered
insufficient for a detailed linguistic description, but it is a useful
simplification. The \ccgbank grammar \citep{hock:man05} introduces two extra
categories, \cf{PP} (for prepositional phrases) and \cf{NP} (for noun phrases).
Arguably, a fifth category, \cf{PT}, would also
be useful, to distinguish verb particles from prepositions
\citep{constable:09}. Some punctuation symbols are also assigned
unique atomic categories, as are conjunctions.

Categories with multiple arguments can be specified by constructing a functor
whose result is itself a functor. Thus the category \cf{(S\bs NP)/NP} specifies
that the word requires two arguments --- an \cf{NP} to the right, and then an
\cf{NP} to the left --- to build an \cf{S} constituent. The relationship with
context-free phrase-structure rules is easy to see if we depict the
category's assignment to the verb \emph{likes} as a tree, and add some lexical
assignments for nouns:
\vspace{0.1in}
\begin{center}
\begin{tabular}{ccc}
\ptbegtree
\ptbeg \ptnode{\cf{NP}} \ptleaf{Casey} \ptend
\ptendtree
&
\ptbegtree
\ptbeg \ptnode{\cf{S}}
  \ptleaf{\cf{NP}}
  \ptbeg \ptnode{\cf{S[dcl]\bs NP}}
    \ptbeg \ptnode{\cf{(S[dcl]\bs NP)/NP}} \ptleaf{likes} \ptend
    \ptleaf{NP}
  \ptend
\ptend
\ptendtree
&
\ptbegtree
\ptbeg \ptnode{\cf{NP}} \ptleaf{Pat} \ptend
\ptendtree
\end{tabular}
\end{center}
\vspace{0.1in}
In this thesis, we define a category as a 5-tuple
$\langle\ressc, \featsc, \slashsc, \argsc, \headsc\rangle$, where:
\begin{itemize}
\addtolength{\itemsep}{-1mm}
 \item \ressc is a result category;
 \item \featsc is a feature structure;
 \item \slashsc is a directional slash;
 \item \argsc is an argument category;
 \item \headsc is a lexical head.
\end{itemize}

If the \argsc attribute is not empty, the category is complex; if only a \ressc
is specified the category is atomic. We assume that \featsc is a list of attribute-value
pairs, where values are always atomic. Attribute values can be underspecified, by assigning
the attribute an index variable. This allows attributes to be coindexed to each other,
providing a way for information to be shared across a category.

Figure~\ref{avm:likes} shows an \textsc{avm} representation of the category
\cf{(S[dcl]\bs NP)/NP}. The category is first broken down into its result
\cf{S[dcl]\bs NP} and its argument \cf{NP}, along with a slash denoting
the argument directionality, \cf{/}.
In general we follow \citet{hock:cl07} in not assigning features to
complex categories.
We have omitted attributes with empty
values, for brevity.

The head of the category is \emph{likes}, and its result is a complex category,
\cf{S[dcl]\bs NP}, also headed by \emph{likes}. The result has an \cf{NP} argument,
with a backward slash indicating it must occur to the left. We have specified a feature,
\emph{case}, for the \cf{NP} arguments, to illustrate the kind of linguistic information
typically encoded in the feature structures. However, we generally omit such
features from our analyses, and only assign features to \cf{S} categories. The
feature \cf{dcl} indicates that its category is a finite declarative.


\begin{figure}[t!]
\centering
\begin{avm}
[{}  \ressc  & [{} \ressc  & [{} \ressc  & \cf{S}\\
	                         \featsc & [{} \textsc{tense} & \emph{dcl}]\\
			         \headsc & \emph{likes}\\
		             ]\\
                    \slashsc& \bks \\
                    \argsc & [{} \ressc  & \cf{NP}\\
	                         \featsc & [{} \textsc{case} & \emph{nom}]\\
			         \headsc & $y$\\
		             ]\\
		\headsc & \emph{likes}\\
	    ]\\
     \slashsc& \cf{/} \\
     \argsc  & [{} \ressc  & \cf{NP} \\
                   \featsc & [{} \textsc{case} & \emph{acc}]\\
		   \headsc & $z$\\
	       ]\\
     \headsc & \emph{likes}\\
]
\end{avm}
\caption{Attribute-value matrix for \emph{likes} $\assign$ \cf{(S[dcl]\bs NP)/NP}
\label{avm:likes}}
\end{figure}

The unbound $y$ and $z$ variables that occupy the values of \headsc denote that
the two arguments must be filled by different lexical items. If two \headsc
attributes share an index variable, and one attribute is bound to a lexical item,
the other attribute will be bound to that lexical item too. It is this mechanism
that allows long-range dependencies to be created. The details of this process are
explained in Section~\ref{sec:unification}.

\section{Function Application}

\textsc{ab} categorial grammars use only one type of grammatical rule, function
application:

\begin{eqnarray}
\cf{X/Y} & \cf{Y} & \Rightarrow\;\; \cf{X}\label{rule:f.app}\\
\cf{Y} & \cf{X\bs Y} & \Rightarrow\;\; \cf{X}\label{rule:b.app}
\end{eqnarray}

These rules allow us to complete the derivation of the \emph{Casey likes Pat} example:

\begin{center}
\deriv{3}{
\rm Casey & \rm likes & \rm Pat \\
\uline{1}&\uline{1}&\uline{1} \\
\cf{NP} &
\cf{(S[dcl]\bs NP)/NP} &
\cf{NP} \\
& \fapply{2} \\
& \mc{2}{\cf{S[dcl]\bs NP}} \\
\bapply{3} \\
\mc{3}{\cf{S[dcl]}}
}\end{center}

The horizontal lines indicate the span of the constituent created, and the
symbol at the right edge of the line indicates the grammatical rule that
licenses the step of the derivation. The resulting category is listed below each
bracket. This tabular presentation of a \cg derivation will be used
interchangeably with tree-based representations throughout the thesis. We will
tend to use derivations where we wish to emphasise the rules being used,
while parse trees often provide an easier way to view attachment structure.



\section{Unification}
\label{sec:unification}
Unification is a mechanism for finding the union of the information represented in
two feature structures \citep{shieber:86}. Each \cg rule involves unifying two
categories during the string concatenation that the rule describes. In forward
application (\ref{rule:f.app}) and backward application (\ref{rule:b.app}), the
argument of the functor is unified with the argument category, and the functor's
result is returned as the product of the rule.

The argument unification can be seen in more detail using the \textsc{avm}
representation in Figure~\ref{avm:fwd_app}. The left-most \textsc{avm}
represents the category \cf{(S[dcl]\bs NP)/NP}, and to its right is the category
\cf{NP}, which will unify with the outer-most argument of \cf{(S[dcl]\bs
NP)/NP}. The result of this unification is shown in the \textsc{avm} on the
right. The missing values have been filled in from the \cf{NP} argument.
Specifically, the category now has a lexical head \emph{Pat}, and it has
inherited the feature and value pair \emph{type: proper}.

\begin{sidewaysfigure}
\begin{center}
\begin{avm}
[{}  \ressc  & [{} \ressc  & [{} \ressc  & \cf{S}\\
	                         \featsc & [{} \textsc{tense} & \emph{dcl}]\\
			         \headsc & \emph{likes}\\
		             ]\\
                    \slashsc& \bks \\
                    \argsc & [{} \ressc  & \cf{NP}\\
	                         \featsc & [{} \textsc{case} & \emph{nom}]\\
			         \headsc & $y$\\
		             ]\\
		\headsc & \emph{likes}\\
	    ]\\
     \slashsc& \cf{/} \\
     \argsc  & [{} \ressc  & \cf{NP} \\
                   \featsc & [{} \textsc{case} & \emph{acc}]\\
		   \headsc & $z$\\
	       ]\\
     \headsc & \emph{likes}\\
]
\end{avm}\;\;
\begin{avm}
\raisebox{-60mm}{[{} \ressc  & \cf{NP} \\
    \featsc & [{} \emph{type} & proper]\\
    \headsc & \emph{Pat}\\
]}
\end{avm}
\;\;\raisebox{-55mm}{\Huge $\longrightarrow$}\;\;
\begin{avm}
[{}  \ressc  & [{} \ressc  & [{} \ressc  & \cf{S}\\
	                         \featsc & [{} \textsc{tense} & \emph{dcl}]\\
			         \headsc & \emph{likes}\\
		             ]\\
                    \slashsc& \bks \\
                    \argsc & [{} \ressc  & \cf{NP}\\
	                         \featsc & [{} \textsc{case} & \emph{nom}]\\
			         \headsc & $y$\\
		             ]\\
		\headsc & \emph{likes}\\
	    ]\\
     \slashsc& \cf{/} \\
     \argsc  & [{} \ressc  & \cf{NP} \\
                   \featsc & [{} \textsc{case} & \emph{acc}]\\
				 \textsc{type} & \emph{proper}\\
		   \headsc & $z$\\
	       ]\\
     \headsc & \emph{likes}\\
]
\end{avm}
\end{center}
\caption{Attribute-value matrices for \cf{(S[dcl]\bs NP)/NP \;\; NP \;\;
\longrightarrow \;\; S[dcl]\bs NP}\label{avm:fwd_app}}
\end{sidewaysfigure}

During unification, each field in the two categories is compared. Where one
field's value is underspecified --- as is the \scare{type} value of the
functor's argument --- the union inherits the more specific value. If the two
values conflict, unification fails, preventing the rule from being applied.
Unification proceeds recursively, so if \argsc or \ressc contain complex
categories, or \featsc has a complex value, the components are unified as well.

\section{Logical Forms with Hybrid Dependency Logic Semantics}
\label{sec:hlds_background}

One of the key properties of categorial grammars is the ease with which they can
be made semantically transparent. A grammar is said to be semantically transparent if
it fully specifies an interface between syntax and semantics, such that every
syntactic derivation corresponds to exactly one semantic analysis.

Categorial grammars are made semantically transparent by associating every lexical
category with a logical form representation.
We follow \citet{baldridge:02} in using Hybrid Dependency Logic Semantics
\citep[\hlds, ][]{kruijff:01} to represent logical forms for \cg categories.
\hlds is a hybrid logic, which means
that it extends modal logic with \emph{nominals} to allow states to be referenced
explicitly. Formulas can be formed using both nominals and propositions, 
standard boolean operators, and the satisfaction operator $@$. A formula
$@_i\semf{p}$ states that the formula $\semf{p}$ holds at the state named by $i$.

Meanings are expressed as a conjunction of modalised terms, anchored by the head that
identifies the head's proposition:

\begin{eqnarray}
 @_h(\text{proposition} \wedge \langle\delta_i\rangle (d_i \vee dep_i))
\end{eqnarray}

Dependency relations are modelled as modal relations $\langle\delta_i\rangle$, and the
discourse referents connected by the dependency relations are each assigned a nominal $d_i$.
Because \hlds is an indexed representation, propositions can be flattened
to a conjunction of fixed-size elementary predications. We follow \citet{white:03} in
adopting this representation. In their approach, syntactic categories are paired with
a flattened representation, with coindexation connecting the two elementary predications:

\begin{center}
\begin{tabular}{lcrcl}
\emph{saw} & $\assign$ & \cf{(S[dcl]_x\bs NP_y)/NP_z} & $:$ &
$@_x \text{\textbf{see}} \wedge @_x\langle \textsc{tense}\rangle \text{\textbf{past}} \wedge @_x\langle \textsc{act}\rangle y \wedge @_x\langle \textsc{pat}\rangle z$\\
\emph{Bob} & $\assign$ & \cf{NP_b}  & $:$ & $@_b\text{\textbf{Bob}}$ \\
\emph{Gil} & $\assign$ & \cf{NP_g}  & $:$ & $@_g\text{\textbf{Gil}}$ \\
\end{tabular}
\end{center}

This flattened representation is a graph of
variable bindings and 
$\langle \text{parent}, \text{relation}, \text{child}\rangle$ triples, where each triple
is of the form $@_\text{head}\sematt{relation}\semf{child}$.
The relations $\sematt{act}$ and $\sematt{pat}$ stand for the semantic roles 
\textsc{actor} and \textsc{patient} respectively.

During a derivation, the variables are coindexed during unification, and the logical
forms are conjoined. For example:

\begin{center}
\deriv{3}{
\rm Bob & \rm saw & \rm Gil \\
\uline{1}&\uline{1}&\uline{1} \\
\cf{NP_b\;:} &
\cf{(S[dcl]_x\bs NP_y)/NP_z\;:} &
\cf{NP_g\;:} \\
\cf{@_b\text{\textbf{Bob}}} &
\cf{
  @_x \text{\textbf{see}}
  \wedge @_x\langle \text{\textsc{tense}}\rangle \text{\textbf{dcl}}
  \wedge @_x\langle \textsc{act}\rangle y
  \wedge @_x\langle \textsc{pat}\rangle z
} & \cf{@_g\text{\textbf{Gil}}} \\
& \fapply{2} \\
& \mc{2}{\cf{S[dcl]_x\bs NP_y\;:}} \\
& \mc{2}{\cf{
  @_x \text{\textbf{see}}
  \wedge @_x\langle \textsc{tense}\rangle \text{\textbf{dcl}}
  \wedge @_x\langle \textsc{act}\rangle y
  \wedge @_x\langle \textsc{pat}\rangle g
  \wedge @_g \text{\textbf{Gil}}}}\\
\bapply{3} \\
\mc{3}{\cf{S[dcl]_x\;:}}\\
\mc{3}{\cf{
  @_x \text{\textbf{see}}
  \wedge @_x\langle \textsc{tense}\rangle \text{\textbf{dcl}}
  \wedge @_x\langle \textsc{act}\rangle b
  \wedge @_x\langle \textsc{pat}\rangle g
  \wedge @_g \text{\textbf{Gil}} \wedge @_b \text{\textbf{Bob}}}}\\
}\end{center}

The first application rule coindexes the verb's outer argument variable to
$g$, and a term is added to the semantic representation binding $g$ to Gil.
The same happens when \emph{Bob} is found by backward application.

\section{Instantiating Dependencies During Unification}
% New

Instead of a full logical form, \ccgbank \citep{hock:cl07} represents
predicate-argument structure using dependency
graphs. These dependency graphs are used to evaluate the \candc parser \citep{clark:cl07}. Dependency
graphs are often convenient for our purposes, and are important for this thesis
as they are used in our experiments.

% New
We follow \citet{hock:cl07} in assuming that the arguments of a complex category are
numbered 1 to $n$, starting from the innermost argument, where $n$ is the arity of the
functor, e.g. \cf{((S[dcl]\bs NP_1)/NP_2)/PP_3}.

% New
A dependency is filled when an argument's \headsc feature has been filled with a lexical
item during unification. For instance, if the category \cf{(S[dcl]\bs NP_1)/NP_2}, assigned
to the word \emph{likes}, is the functor in a successful forward application rule with an
\cf{NP} headed by \emph{Pat}, its outer argument will be bound, and the following dependency
created:

\begin{center}
\begin{tabular}{lccr}
  likes & \cf{(S[dcl]\bs NP_1)/NP_2} & 2 & Pat\\
\end{tabular} 
\end{center}

In this thesis, a \emph{labelled \ccg dependency} is thus a 4-tuple consisting of the
functor word, category, argument number and argument word.  A dependency is unbounded
if the argument's head was mediated by one or more other categories during unification,
but we do not include this distinction in our dependency tuples. We sometimes refer to
unlabelled dependencies, which are a 2-tuple consisting of the functor word and the
argument word. 

\section{Inadequacy of AB Categorial Grammar}
\label{sec:ab_sucks}

% Updated
The formalism we have defined so far, using only the function application rules, is
the pure applicative categorial grammar elaborated by \citet{bar-hillel:53}.
The problem with this formalism is that it cannot adequately explain 
the long-distance dependencies that arise
from unbounded extraction --- where `explanatory adequacy' is taken to mean an analysis that
predicts the observed extraction asymmetries efficiently, without \emph{ad hoc}
additions to the grammar, a sense we adapt from \citet{chomsky:aspects}.
It also cannot provide an adequate explanation
of coordination, since spans which would not ordinarily be considered constituents
can be coordinated:

\begin{lexample}
Casey likes but Erin hates Pat.
\end{lexample}

% Updated
This sentence requires the subject to be bracketed with the verb, so that the two
equivalent brackets can be coordinated. An applicative categorial grammar can base
generate such a derivation, which for a \cg means relying on category ambiguity to
provide the alternative bracketing\footnote{In the context of a transformational grammar,
an analysis is \emph{base generated} if it is generated by the core phrase-structure rules,
with no movement operations. We describe a \cg analysis as \emph{base generated} if it
relies on category ambiguity, instead of assigning canonical categories and using
associative and/or permutative combinators.}:

\begin{center}
\deriv{6}{
\rm Casey & \rm likes & \rm but & \rm Erin & \rm hates & \rm Pat \\
\uline{1}&\uline{1}&\uline{1}&\uline{1}&\uline{1}&\uline{1} \\
\cf{NP} &
\cf{(S[dcl]/NP)\bs NP} &
\cf{(X\bs X)/X} &
\cf{NP} &
\cf{(S[dcl]/NP)\bs NP} &
\cf{NP} \\
\bapply{2} && \bapply{2} \\
\mc{2}{\cf{S[dcl]/NP}} && \mc{2}{\cf{S[dcl]/NP}} \\
&& \fapply{3} \\
&& \mc{3}{\cf{(S[dcl]/NP)\bs (S[dcl]/NP)}} \\
\bapply{5} \\
\mc{5}{\cf{S[dcl]/NP}} \\
\fapply{6} \\
\mc{6}{\cf{S[dcl]}}
}
\end{center}

This derivation assigns \emph{hates} the category \cf{(S[dcl]/NP)\bs NP}.
This is an associative variant of the standard transitive verb category
\cf{(S[dcl]\bs NP)/NP}. The coordination we observe appears to
require a novel category that has a specific logical relationship
to the canonical one, function associativity. However, the base generation
strategy does not predict any such relationship, as it does not admit any
constraints on the structure of the novel category. This makes it a weak
explanation of the phenomenon. A similar situation arises in the analysis of
extraction phenomena:

\begin{center}
\deriv{4}{
\rm Pat, & \rm who & \rm Erin & \rm hates \\
\uline{1}&\uline{1}&\uline{1}&\uline{1} \\
\cf{NP} &
\cf{(NP\bs NP)/(S[dcl]/NP)} &
\cf{NP} &
\cf{(S[dcl]/NP)\bs NP} \\
&& \bapply{2} \\
&& \mc{2}{\cf{S[dcl]/NP}} \\
& \fapply{3} \\
& \mc{3}{\cf{NP\bs NP}} \\
\bapply{4} \\
\mc{4}{\cf{NP}}
}
\end{center}

Once again, a construction that only requires an associative version of the canonical
category is analysed as though it could have received any category at all.
The limitations of relying on category ambiguity
are even more apparent in the following construction, where an object is extracted
from within an object clause:

\begin{center}
\deriv{6}{
\rm Pat, & \rm who & \rm Casey & \rm knows & \rm Erin & \rm hates \\
\uline{1}&\uline{1}&\uline{1}&\uline{1}&\uline{1}&\uline{1} \\
\cf{NP} &
\cf{(NP\bs NP)/(S[dcl]/NP)} &
\cf{NP} &
\cf{((S[dcl]/NP)/(S[dcl]/NP))\bs NP} &
\cf{NP} &
\cf{(S/NP)\bs NP} \\
&& \bapply{2} & \bapply{2} \\
&& \mc{2}{\cf{(S[dcl]/NP)/(S[dcl]/NP)}} & \mc{2}{\cf{S[dcl]/NP}} \\
&& \fapply{4} \\
&& \mc{4}{\cf{S[dcl]/NP}} \\
& \fapply{5} \\
& \mc{5}{\cf{NP\bs NP}} \\
\bapply{6} \\
\mc{6}{\cf{NP}}
}
\end{center}

Here the mediating verb, \emph{knows}, must be assigned a different category
to transmit the dependency, even though its own arguments are all in their canonical
positions. This extraction can occur from unbounded depth, so there could be any
number of such mediating verbs, all requiring novel categories due to a movement
phenomenon occurring outside their local argument structures.

\section{The \ccg Combinators}

Because the base generation strategy breaks down in these situations,
\citet{steedman:00} introduces a principle to restrict how much work category
ambiguity should adopt in the grammar:

\begin{headcat}
A single non-disjunctive lexical category for the head of a given construction
specifies both the bounded dependencies that arise when its complements
are in canonical position and the unbounded dependencies that arise when those
complements are displaced under relativization, co-ordination, and the like.
\end{headcat}

Most lexemes can head a great variety of different constructions, and will
require a variety of distinct syntactic categories, so the principle does not
stipulate a one-to-one mapping between lexical entries and categories. Natural
language syntax is ambiguous, so a lexicalised grammar requires ambiguous
lexical entries. The principle merely asserts that lexical ambiguity should be
minimised, which constrains its use as a generative strategy.

Having established this, we now need some extra machinery to generate extraction
and coordination constructions, using only the canonical categories. In \ccg,
this machinery comes in the form of a lexical operation, \emph{type raising}, which
can be used with an extra combinatory rule, \emph{composition}, to allow enough
associativity to solve the vast majority of the problematic long-distance dependencies.
\citeauthor{steedman:00} also introduces another type of combinatory rule,
\emph{substitution}, to handle parasitic gaps. However, this construction is rare in
English \citep{hock:cl07}, so we refer the reader to \citet{steedman:00} for a
discussion of the construction and the substitution rule.

\subsection{The Need for Associativity}

Some \cg categories can be defined so that they are associative functions.
If a category has two arguments, one to the left, and the
other to the right, it does not matter what order the arguments are applied in.
This means that the two bracketings below are equivalent:

\begin{equation}
\cf{(S[dcl]\bs NP)/NP} \;\equiv\; \cf{(S[dcl]/NP)\bs NP}
\end{equation}

Categories with sequences of arguments in a single
direction, are not associative, however, because switching the bracketing
would permute the sequence of arguments in the string. \citet{lambek:58} introduces a unary
\emph{associativity} operator to perform the transformation, given that the two
bracketings are logically equivalent. This operator would allow us to analyse
the sentences in Section \ref{sec:ab_sucks} using the canonical categories,
as we can see in this example of object extraction:

\begin{center}
\deriv{6}{
\rm Ashley & \rm likes & \rm Pat & \rm who & \rm Casey & \rm likes \\
\uline{1}&\uline{1}&\uline{1}&\uline{1}&\uline{1}&\uline{1} \\
\cf{NP} &
\cf{(S[dcl]\bs NP)/NP} &
\cf{NP} &
\cf{(NP\bs NP)/(S[dcl]/NP)} &
\cf{NP} &
\cf{(S[dcl]\bs NP)/NP} \\
&&&&& \asterisk{1} \\
&&&&& \mc{1}{\cf{(S[dcl]/NP)\bs NP}} \\
&&&& \bapply{2} \\
&&&& \mc{2}{\cf{S[dcl]/NP}} \\
&&& \fapply{3} \\
&&& \mc{3}{\cf{NP\bs NP}} \\
&& \bapply{4} \\
&& \mc{4}{\cf{S[dcl]\bs NP}} \\
& \fapply{5} \\
& \mc{5}{\cf{NP}} \\
\bapply{6} \\
\mc{6}{\cf{S[dcl]}}
}
\end{center}

The problem is that full associativity for all functors results in substantial
over-generation, and there is no obvious way to restrict the scope of a single
unary associativity operator. Instead, \ccg achieves partial binary
associativity using function composition.

\subsection{Binary Associativity with Composition}
\label{sec:associativity}
Instead of a unary grammatical rule, \ccg achieves function associativity with
a binary grammatical operation, composition.
Binary composition allows two functors to be merged,
if the argument of one functor matches the result of another in a direction
consistent with their slashes:

\begin{eqnarray}
\cf{X/Y}    & \cf{Y/Z}    & \Rightarrow_\cB\;\; \cf{X/Z}\\
\cf{Y\bs Z} & \cf{X\bs Y} & \Rightarrow_\cB\;\; \cf{X\bs Z}
\end{eqnarray}

Composition allows a functor access to arguments \scare{inside} another functor.
What we now need is a way to transform an argument into a functor of the
appropriate form. This is done with the type-raising operation, which takes a
category \cf{X} and transforms it into a functor over a functor over the
original category \cf{X} with the appropriate directionality:

\begin{eqnarray}
\cf{X} & \Rightarrow_\cT & \cf{T/(T\bs X)}\\
\cf{X} & \Rightarrow_\cT & \cf{T\bs (T/X)}
\end{eqnarray}

The directionality constraint is that the two slashes must be in the opposite
direction, to avoid permuting the order of a functor's arguments.

Figure \ref{fig:wh_movement} shows how partial associativity is used in an
analysis of a common case of extraction, WH-movement. The object of
\emph{hates}, \emph{Pat}, is displaced by the relativiser, \emph{who}. Partial
associativity allows the responsibility for the movement to be encapsulated in the
category assigned to \emph{who}, so that the other words can receive the same
categories they would if there were no extraction phenomena.
\emph{Casey}'s category, \cf{NP}, is type-raised to \cf{S/(S\bs NP)}.
In conjunction with the
composition rule, this allows the arguments of \emph{hates} to be applied in the
opposite order from the one specified by its category's bracketing.
Specifically, \emph{Casey} fills the leftward \cf{NP} argument, for the verb's subject.
This results in a constituent spanning \emph{Casey hates} with the category
\cf{S[dcl]/NP}.

\begin{figure}
\centering
\deriv{6}{
\rm Ashley & \rm likes & \rm Pat & \rm who & \rm Casey & \rm hates \\
\uline{1}&\uline{1}&\uline{1}&\uline{1}&\uline{1}&\uline{1} \\
\cf{NP} &
\cf{(S[dcl]\bs NP)/NP} &
\cf{NP} &
\cf{(NP\bs NP)/(S[dcl]/NP)} &
\cf{NP} &
\cf{(S[dcl]\bs NP)/NP} \\
&&&& \ftype{1} \\
&&&& \mc{1}{\cf{S/(S\bs NP)}} \\
&&&& \fcomp{2} \\
&&&& \mc{2}{\cf{S[dcl]/NP}} \\
&&& \fapply{3} \\
&&& \mc{3}{\cf{NP\bs NP}} \\
&& \bapply{4} \\
&& \mc{4}{\cf{NP}} \\
& \fapply{5} \\
& \mc{5}{\cf{S[dcl]\bs NP}} \\
\bapply{6} \\
\mc{6}{\cf{S[dcl]}}
}
\caption[Partial associativity provided by type-raising and
composition.]{Interaction of type-raising and composition to produce partial
associativity. This allows the WH-movement to be analysed with the canonical
category assignments.\label{fig:wh_movement}}
\end{figure}

% Updated
The logical basis\footnote{The interpretation of a categorial grammar as a logic
dates to \citet{lambek:58}. We present here only a small and informal illustration.
A more detailed explication of \cg as a logic can be found in \citet{baldridge:11}.}
 for these rules is easy to understand when a functor is considered as a kind of
conditional, where the result is the consequent and the argument is the antecedent.
If we ignore directionality for a moment, the application rule resembles a simple
\emph{modus ponens} deduction:
\begin{eqnarray}
NP \rightarrow S\\
NP\\
\therefore S
\end{eqnarray}
%\clearpage
Composition can be seen as a hypothetical syllogism:

\begin{eqnarray}
N\rightarrow NP\\
NP\rightarrow S\\
\therefore N\rightarrow S
\end{eqnarray}

And finally, the type-raising rule:

\begin{eqnarray}
 NP\\
\therefore (NP\rightarrow S) \rightarrow S
\end{eqnarray}

% Updated
Is proved valid by the following truth table:

% Updated
\begin{center}
 \begin{tabular}{cccc}
\hline
$NP$ & $S$ & $NP\rightarrow S$ & $(NP\rightarrow S) \rightarrow S$\\
\hline
\hline
\textbf{T}    & T   & T                 & \textbf{T} \\
\textbf{T}    & F   & F                 & \textbf{T} \\
F    & T   & T                 & T \\
F    & F   & T                 & F \\
\hline
 \end{tabular}
\end{center}

A conditional is false if and only if its antecedent is true and its consequent is false.
If $S$ is false, then the antecedent $NP \rightarrow S$ is false, so the conditional
$(NP \rightarrow S) \rightarrow S$ is true (due to false antecedent).
If $S$ is true, then the conclusion $(NP\rightarrow S) \rightarrow S$ is also true
(due to true consequent). The truth of \cf{NP} therefore guarantees the truth of the
conclusion. The conclusion is contingent, however, as it is false 
if both \cf{NP} and \cf{S} are false.

% There is therefore a logical intuition behind a non-traditional constituent like
% \emph{Casey hates}. This bracketing can be linguistically motivated in its own right,
% and has only been considered non-traditional because phrase-structure grammars encourage
% a choice between analyses that bracket together the subject and verb, and analyses that
% bracket together the verb and complements. Phrase-structure grammar analyses typically
% only assign brackets to the latter. The formalism cannot assign brackets to both
% simultaneously in one analysis, so traditional analyses selected the verb-object
% pairing because it allows better analyses for more frequent constructions. The
% lack of a subject-verb bracket is why subject-verb coordinations, such as
% \emph{Ashley likes and Casey hates Pat}, have received special attention and specific
% nomenclature (right node raising), when they are directly analogous to the more common
% verb-object coordinations, such as \emph{Ashley likes Casey and hates Pat}. Partial
% associativity does create some problems, such as the so-called spurious ambiguity
% issue discussed in Section \ref{sec:normal_form}. There are also movement phenomena
% which it cannot handle. For some constructions, we require a limited degree of
% \emph{permutativity} --- but only a little, or the grammar loses the ability to
% specify directionality in lexical categories. \ccg achieves limited permutativity
% with crossing composition rules. The restrictions that prevent scrambling are
% described in Section \ref{sec:cross_restrict}.

\subsection{Crossing Composition}
\label{sec:permutativity}
The forward and backward composition rules we have seen so far are order
preserving, or \term{harmonic}. However, there are some syntactic constructions
that reliably permute the order of constituents. Once again, the formalism must
decide whether it is preferable to base-generate these analyses, using category
ambiguity to handle the variation, or whether it is preferable to add some
grammatical machinery to perform the permutation using the canonical categories.

The order-permuting construction that \citet{steedman:pedia} discuss is heavy
\cf{NP} shift. In this construction, the order of arguments in a ditransitive
can be permuted, in order to minimise dependency distances:

\begin{lexamples}
\item I gave [to him] [a book that was very heavy and difficult to read if it is
close to the verb].
\item ? I gave [a book that was very heavy and difficult to read if it is close
to the verb] [to him].
\end{lexamples}

Verb-particle constructions present another example of argument order
permutation:

\begin{lexamples}
\item They gunned down [the very high-tech and expensive fighter plane].
\item ? They gunned [the very high-tech and expensive fighter plane] down.
\item * They gunned down it.
\end{lexamples}

`Heavy' (long) \cf{NP}s are likely to be pushed after the particle, while a
pronominal is forced to occur before the particle. We will assume in our
analyses that the canonical order is verb, particle, object; and that the
particle final construction is the permutation.

To analyse heavy \cf{NP} shift and verb particle structure alternations without
category ambiguity, we need composition rules that do not preserve the order of
a category's arguments. This is referred to as \emph{crossing composition}:

\begin{eqnarray}
\cf{X/Y} & \cf{Y\bs Z} & \Rightarrow_\cBx\;\; \cf{X/Z}\\
\cf{Y/Z} & \cf{X\bs Y} & \Rightarrow_\cBx\;\; \cf{X\bs Z}
\end{eqnarray}

The backward crossing composition rule can be used with the backward type
raising rule to analyse movement phenomena using the canonical category for
each word in the sentence.

\begin{figure}
\centering
\deriv{4}{
\rm They & \rm gave & \rm to~him & \rm a~very~heavy~book \\
\uline{1}&\uline{1}&\uline{1}&\uline{1} \\
\cf{NP} &
\cf{((S[dcl]\bs NP)/PP)/NP} &
\cf{PP} &
\cf{NP} \\
&& \btype{1} \\
&& \mc{1}{\cf{(S\bs NP)\bs ((S\bs NP)/PP)}} \\
&\bxcomp{2} \\
&\mc{2}{\cf{(S[dcl]\bs NP)/NP}} \\
&\fapply{3} \\
&\mc{3}{\cf{S[dcl]\bs NP}}\\
\bapply{4}\\
\mc{4}{\cf{S[dcl]}}
}
\caption[Heavy \cf{NP} shift with crossed composition.]{\ccg analysis of heavy
\cf{NP} shift. The analysis uses crossed composition to achieve the required
permutation of the verb's arguments.\label{fig:heavy_np}}
\end{figure}

\begin{figure}
\centering
\deriv{4}{
\rm They & \rm gunned & \rm it & \rm down \\
\uline{1}&\uline{1}&\uline{1}&\uline{1} \\
\cf{NP} &
\cf{((S[dcl]\bs NP)/PT)/NP} &
\cf{PP} &
\cf{NP} \\
&& \btype{1} \\
&& \mc{1}{\cf{(S\bs NP)\bs ((S\bs NP)/PT)}} \\
&\bxcomp{2} \\
&\mc{2}{\cf{(S[dcl]\bs NP)/NP}} \\
&\fapply{3} \\
&\mc{3}{\cf{S[dcl]\bs NP}}\\
\bapply{4}\\
\mc{4}{\cf{S[dcl]}}
}
\caption[Verb-particle reordering with crossed composition.]{\ccg analysis of
verb-particle reordering. The analysis uses crossed composition to achieve the
required permutation of the verb's arguments.\label{fig:vpc}}
\end{figure}

Figure \ref{fig:heavy_np} shows a \ccg derivation of heavy \cf{NP} shift. The
long argument, whose canonical position is just right of the verb, is shifted to
the end to minimise the surface distance between the verb and its arguments. The
permutation is achieved by type-raising the short \cf{PP} argument so that it
composes with the inner \cf{PP} argument of the verb, so that it unifies with
the same argument slot it would occupy had it occurred in its canonical
position. This requires a crossed composition rule, because the verb's slash is
rightward. Figure \ref{fig:vpc} shows an analogous analysis for verb-particle
reordering.

Finally, the composition rules must also be generalised, so that associativity
is preserved for functions of higher arity. This is shown in Figure
\ref{fig:gen_comp}. In this derivation, the argument of the category assigned to
\emph{may}, \cf{(S[dcl]\bs NP)/(S[b]\bs NP)}\footnote{The \cf{b} feature marks
bare inflection. See \citet{hock:cl07} for other feature values used during the
thesis.},
is an extra level deep inside the
category of \emph{give}. It is the result of the result, instead of just the
result. The generalised composition rules allow composition over directionally
consistent argument sequences, to handle such constructions.

\begin{figure}
\centering
\scalebox{0.8}{
\deriv{8}{
\rm Pat & \rm bought & \rm and & \rm may & \rm give & \rm Casey & \rm a & \rm
flower \\
\uline{1}&\uline{1}&\uline{1}&\uline{1}&\uline{1}&\uline{1}&\uline{1}&\uline{1}
\\
\cf{NP} &
\cf{((S[dcl]\bs NP)/NP)/NP} &
\cf{(X\bs X)/X} &
\cf{(S[dcl]\bs NP)/(S[b]\bs NP)} &
\cf{((S[b]\bs NP)/NP)/NP} &
\cf{NP} &
\cf{NP/N} &
\cf{N} \\
&&& \fcomp{2} && \fapply{2} \\
&&& \mc{2}{\cf{((S[dcl]\bs NP)/NP)/NP}} && \mc{2}{\cf{NP}} \\
&& \fapply{3} \\
&& \mc{3}{\cf{(\cf{((S[dcl]\bs NP)/NP)/NP})\bs (((S[dcl]\bs NP)/NP)/NP)}} \\
& \bapply{4} \\
& \mc{4}{\cf{((S[dcl]\bs NP)/NP)/NP}} \\
& \fapply{5} \\
& \mc{5}{\cf{((S[dcl]\bs NP)/NP}} \\
& \fapply{7} \\
& \mc{7}{\cf{S[dcl]\bs NP}} \\
\bapply{8} \\
\mc{8}{\cf{S[dcl]}}
}}
\caption[Example of generalised composition.]{An example of generalised
composition, between the auxiliary \emph{may} and the ditransitive verb
\emph{give}.\label{fig:gen_comp}}
\end{figure}


\section{Restricting Rule Productivity}

% Updated
The unrestricted interaction of the application, composition and type-raising rules
discussed so far introduces a great deal of syntactic ambiguity and over-generation.
A number of measures have been proposed to mitigate these problems. First,
\citet{steedman:00} follows \citet{komagata:97,komagata:99} in
placing a general restriction on type-raising, in order to ensure that the unary rule
is not completely unbounded. The restriction prevents a forward type raising rule from
producing a category \cf{T/(T\bs X)} where \cf{T\bs X} is not a valid category from
the fixed set of types occurring in the lexicon. An equivalent restriction is placed on
backward type-raising.

There is also strong motivation for constraining composition. Initially,
\citet{steedman:00} used the language specific constraints described in Section
\ref{sec:cross_restrict} to introduce the required restrictions. However,
\citet{baldridge:03} have since described how these rules can be
replaced with a more principled lexically sensitive specification. The mechanism
for doing this, multi-modal slashes, is now widely accepted as an important part
of the \ccg theory \citep{steedman:pedia}. Multi-modal \ccg is described in
Section \ref{sec:mmccg_background}.

% Updated
Finally, the partial associativity added by composition allows multiple equivalent
bracketings for the same semantic analysis. The alternative brackets allow information
structure to be expressed in the surface syntax, but for practical parsing purposes,
it is useful to suppress the ambiguity. We describe a mechanism suggested by
\citet{eisner:96} to do this in Section \ref{sec:normal_form}.


\subsection{Restrictions on Crossing Composition}
\label{sec:cross_restrict}
The crossing composition rules allow a functor's arguments to occur in a
different order from the one specified by the category. This amounts to local
scrambling --- a desirable property for free word order languages. The problem
is that we do not want scrambling in a configurational language like English:

\begin{center}
\deriv{6}{
\rm *I & \rm Ed & \rm think & \rm that & \rm saw & \rm Ann \\
\uline{1}&\uline{1}&\uline{1}&\uline{1}&\uline{1}&\uline{1} \\
\cf{NP} &
\cf{NP} &
\cf{(S[dcl]\bs NP)/S[em]} &
\cf{S[em]/S[dcl]} &
\cf{(S[dcl]\bs NP)/NP} &
\cf{NP} \\
&&&& \fapply{2} \\
&&&& \mc{2}{\cf{S[dcl]\bs NP}} \\
&&& \fxcomp{3} \\
&&& \mc{3}{\cf{S[em]\bs NP}} \\
&& \fxcomp{4} \\
&& \mc{4}{\cf{(S[dcl]\bs NP)\bs NP}} \\
& \bapply{5} \\
& \mc{5}{\cf{S[dcl]\bs NP}} \\
\bapply{6} \\
\mc{6}{\cf{S[dcl]}}
}
\end{center}

The problem is the forward crossed composition rule, which is not necessary in
English. \citet{steedman:00} therefore proposed language specific subsets of
the combinatory rules, which are suggested as a theory of universal grammar.
Additionally, \citet{steedman:00} introduced a restriction that
backward crossed composition should only be allowed for categories rooted in
\cf{S}. This prevents further over-generation:

\begin{center}
 \deriv{4}{
\rm *powerful & \rm by & \rm Rivaldo & \rm shots \\
\uline{1}&\uline{1}&\uline{1}&\uline{1} \\
\cf{N/N} &
\cf{(N\bs N)/NP} &
\cf{NP} &
\cf{N} \\
& \fapply{2} \\
& \mc{2}{\cf{N\bs N}} \\
\bxcomp{3} \\
\mc{3}{\cf{N/N}} \\
\fapply{4} \\
\mc{4}{\cf{N}}
}
\end{center}

% Updated
These constraints would presumably be conventions adopted and learnt by speakers of
a language, rather than fixed constraints in a universal (possibly innate) grammar.
\citet{baldridge:03} provide a way for such constraints to be represented in the
lexicon instead, which we will now describe.


\subsection{Multi-Modal \ccg}
\label{sec:mmccg_background}
\citepos{baldridge:03} solution was to import the approach to resource
sensitivity taken in the categorical type logic (\ctl) tradition
\citep{morrill:94,moortgat:97}. \ccg uses a single pair of slashes, $\lbrace /, \bs\rbrace$,
which means that there there is no way a specific category can restrict which
rules it can be used with. \ctl instead decorates its slashes with \emph{modes}.
The mode variables distinguish slashes that can participate in all rules from
slashes which can only participate in a certain subset.

\citeauthor{baldridge:03} divide the binary rules according to whether they
allow \emph{associativity}, \emph{permutativity}, or \emph{neither}. Table
\ref{tab:rule_props} shows how the rules break down according to these
properties. The harmonic composition rules allow the associativity required to
analyse the WH-movement and right node raising constructions described in
Section \ref{sec:associativity}, while the crossing composition rules allow the
permutativity required to analyse the heavy \cf{NP} shift and verb-particle
reordering constructions described in Section \ref{sec:permutativity}.

\begin{table}
 \centering

\begin{tabular}{llllcc}
\hline
 Direction & Type     & Combinator  &        & Associative? & Permutative?\\
\hline
\hline
 Forward   &          & Application & $>$    &   & \\
 Backward  &          & Application & $<$    &   & \\
 Forward   & Harmonic & Composition & $>\cB$   & $\checkmark$ & \\
 Backward  & Harmonic & Composition & $<\cB$   & $\checkmark$ & \\
 Forward   & Crossing & Composition & $>\cBx$ &   & $\checkmark$ \\
 Backward  & Crossing & Composition & $<\cBx$ &   & $\checkmark$ \\
\hline
\end{tabular}
\caption{Associativity and permutativity properties of the \ccg rules.
\label{tab:rule_props}}
\end{table}

The slashes used in these rules can therefore be decorated with types, referred
to as \emph{modes}, where each mode permits associativity, permutativity, both,
or neither. Lexical categories can then be assigned modes that reflect which rules
they are allowed to be used with. A slash decorated with the \dmodetext will allow
associative rules, \xmodetext decorated slashes allow permutativity, and \smodetext
decorated slashes allow neither. A fourth mode, \cmodetext, allows both.

\citet{baldridge:thesis02} defined a hierarchy of seven modes, and a simplified
hierarchy of four modes sufficient for most purposes. The simplified hierarchy
is shown in Figure \ref{fig:modes}. The most restrictive \emph{application-only}
mode (\smodetext) occurs at the top, with the more permissive tier of modes
(\dmodetext, \xmodetext) inheriting from it, and the most permissive mode (\cmodetext)
inheriting from them. A mode can unify with itself, or with any mode it subsumes.
This means that the standard unification mechanism can be exploited to restrict
rule productivity, once the combinators are updated with the appropriate modes
as shown in Table \ref{tab:ccg_rules}.

\begin{table}
 \centering
\begin{tabular}{lcclc}
\hline
\Sfapply & \cf{X/\smode Y}   & \cf{Y}            & $\Rightarrow$ & \cf{X}\\
\Sbapply & \cf{Y}            & \cf{X\bs\smode Y} & $\Rightarrow$ & \cf{X}\\
\Sfcomp  & \cf{X/\dmode Y}   & \cf{Y/\dmode Z}   & $\Rightarrow_\cB$ &
\cf{X/\dmode Z}\\
\Sbcomp  & \cf{Y\bs\dmode Z} & \cf{X\bs\dmode Y} & $\Rightarrow_\cB$ &
\cf{X\bs\dmode Z}\\
\Sfxcomp & \cf{X/\xmode Y}   & \cf{Y\bs\xmode Z} & $\Rightarrow_\cBx$ &
\cf{X\bs\xmode Z}\\
\Sbxcomp & \cf{Y/\xmode Z}   & \cf{X\bs\xmode Y} & $\Rightarrow_\cBx$ &
\cf{X/\xmode Z}\\
\Sftype  &                   & \cf{X}            & $\Rightarrow_\cT$ & \cf{T/_i
(T\bs_i X)} \\
\Sbtype  &                   & \cf{X}            & $\Rightarrow_\cT$ &
\cf{T\bs_i (T/_i X)} \\
\hline
\end{tabular}
\caption{Multi-modal \ccg rules.}
\label{tab:ccg_rules}
\end{table}

Let's look at a simple example of how these mode sensitive combinators allow rule
sensitivity to be restricted in the lexicon. We would like to handle conjunction
using our current inventory of combinatory rules, rather than using something
like the ternary conjunction rule described in \citet{steedman:00}. We can do
this by assigning coordinators categories of the form \cf{(X\bs X)/X}, where
\cf{X} can be any category.\footnote{This could be expanded into a series of
non-schematic categories in the lexicon, but for now we will work with the
schematic version.} \citet{steedman:00} shows that with this formulation,
we cannot prevent over-generation:

\begin{figure}
\centering

\begin{tikzpicture}
\path
  node (star) at (1, 2) {$\star$}
  node (times) at (0, 1) {$\times$}
  node (diamond) at (2, 1) {$\diamond$}
  node (dot) at (1, 0) {$\cdot$};

\draw [->] (star) -- (times);
\draw [->] (star) -- (diamond);
\draw [->] (times) -- (dot);
\draw [->] (diamond) -- (dot);
\end{tikzpicture}

\caption{Subsumption hierarchy of the four modes.\label{fig:modes}}
\end{figure}

\begin{center}
\deriv{6}{
\rm *player & \rm that & \rm shoots & \rm and & \rm he & \rm misses \\
\uline{1}&\uline{1}&\uline{1}&\uline{1}&\uline{1}&\uline{1} \\
\cf{N} &
\cf{(N\bs N)/(S[dcl]\bs NP)} &
\cf{S[dcl]\bs NP} &
\cf{(X\bs X)/X} &
\cf{N} &
\cf{S[dcl]\bs NP} \\
&&&& \bapply{2} \\
&&&& \mc{2}{\cf{S[dcl]}} \\
&&& \fapply{3} \\
&&& \mc{3}{\cf{S[dcl]\bs S[dcl]}} \\
&& \bcomp{4} \\
&& \mc{4}{\cf{S[dcl]\bs NP}} \\
& \fapply{5} \\
& \mc{5}{\cf{N\bs N}} \\
\bapply{6} \\
\mc{6}{\cf{N}}
}
\end{center}

However, if we make use of the multi-modal combinators to assign a more
restrictive category to \emph{and}, the invalid composition will be blocked.
Blocked rules are marked with an asterisk:

\begin{center}
\deriv{6}{
\rm *player & \rm that & \rm shoots & \rm and & \rm he & \rm misses \\
\uline{1}&\uline{1}&\uline{1}&\uline{1}&\uline{1}&\uline{1} \\
\cf{N} &
\cf{(N\bs N)/(S[dcl]\bs NP)} &
\cf{S[dcl]\bs NP} &
\cf{(X\bs\smode X)/\smode X} &
\cf{N} &
\cf{S[dcl]\bs NP} \\
&&&& \bapply{2} \\
&&&& \mc{2}{\cf{S[dcl]}} \\
&&& \fapply{3} \\
&&& \mc{3}{\cf{S[dcl]\bs\smode S[dcl]}} \\
&& \asterisk{4} \\
&& \mc{4}{\cf{S[dcl]\bs NP}}
}
\end{center}
%\clearpage
% Updated
Multi-modal \ccg can thus analyse coordination phenomena without introducing
a special conjunction rule. We will largely be dealing with the \citet{steedman:00}
grammar, however, so most of our derivations will use the \ccgbank analysis of
coordination, which approximates a ternary conjunction rule:

\begin{center}
 \deriv{6}{
\rm Casey & \rm and & \rm Pat & \rm sang & \rm and & \rm danced \\
\uline{1}&\uline{1}&\uline{1}&\uline{1}&\uline{1}&\uline{1} \\
\cf{NP} &
\cf{conj} &
\cf{NP} &
\cf{S[dcl]\bs NP} &
\cf{conj} &
\cf{S[dcl]\bs NP} \\
& \conj{2} && \conj{2}\\
& \mc{2}{\cf{NP[conj]}} & & \mc{2}{\cf{S[dc]\bs NP[conj]}}\\
\conj{3} & \conj{3} \\
\mc{3}{\cf{NP}} & \mc{3}{\cf{S[dcl]\bs NP}}\\
\bapply{5}\\
\mc{5}{\cf{S[dcl]}}
}
\end{center}

In multi-modal \ccg, the value of the \slashsc attribute of complex categories
is a typed feature structure with multiple properties, rather than a single atomic
value representing directionality. \citet{baldridge:03} suggest that it may be
useful to extend the definition of \slashsc beyond directionality and mode. They
propose two additional attributes, to specify the head-directionality of the slash
and whether the slash can be used in the primary functor of combinatory rules. The
extended definition of \slashsc might be presented as follows:

\begin{center}
\begin{avm}
[{} \slashsc & [{} \textsc{dir} & \cf{\bs}\\
                   \textsc{mode} & \smodetext\\
                   \textsc{depdir} & $\leftarrow$\\
                   \textsc{activity}  & !\\
               ]
]
\end{avm}
\end{center}

\citet{clark:acl07parseval} implement an equivalent of the \textsc{depdir}
attribute in the \candc parser to have it generate grammatical relations
\citep{briscoe:poster06}.
However, we do not perform a grammatical relations evaluation in our parsing experiments,
so we do not require the \textsc{depdir} slash attribute in this thesis.

The concept of inert slashes is useful in our analyses, and is important for our definition of
hat categories, as described in Section \ref{sec:null_mode}.
The inert slash allows an argument to be structurally defined, but unavailable 
for application or composition. Because \textsc{activity} is a separate dimension from
\textsc{mode}, \citet{hoyt:08} denote inert slashes using a superscript, rather
than a subscript. For instance, \citet{hock:cl07} uses the category
\cf{S[adj]\bs NP} for predicative adjuncts. The argument of this category
is a good candidate for an inert slash:

\begin{center}
\small
\deriv{7}{
\rm Pat & \rm is & \rm well & \rm read & \rm and & \rm handy & \rm
around~the~house \\
\uline{1}&\uline{1}&\uline{1}&\uline{1}&\uline{1}&\uline{1}&\uline{1} \\
\cf{NP} &
\cf{(S[dcl]\bs NP)/(S[adj]\bs^! NP)} &
\cf{(S\bs^i NP)/(S\bs^i NP)} &
\cf{S[adj]\bs^! NP} &
\cf{(X\bs X)/X} &
\cf{S[adj]\bs^! NP} &
\cf{(S\bs^i NP)\bs (S\bs^i NP)} \\
&& \fapply{2} && \bapply{2} \\
&& \mc{2}{\cf{S[adj]\bs^! NP}} && \mc{2}{\cf{S[adj]\bs^! NP}} \\
&&&& \fapply{3} \\
&&&& \mc{3}{\cf{(S[adj]\bs^! NP)\bs (S[adj]\bs NP)}} \\
&& \bapply{5} \\
&& \mc{5}{\cf{S[adj]\bs^! NP}} \\
& \fapply{6} \\
& \mc{6}{\cf{S[dcl]\bs NP}} \\
\bapply{7} \\
\mc{7}{\cf{S[dcl]}}
}
\end{center}

The copula, \emph{is}, is assigned the category \cf{(S[dcl]_z\bs
NP_y)/(S[adj]_z\bs^! NP_y)}. The inert argument offers several advantages. First, it
enables a dependency to be created between the subject \emph{Casey} and the
object of the copula, in this case both \emph{read} and \emph{handy}. The
dependency is created by coindexing the argument of the copula with the argument
of the predicate. Such a dependency would be impossible if the complement did
not specify an argument. The \cf{S[adj]\bs^! NP} category also predicts that the
predicate can be modified by verb phrase modifiers, using their canonical
categories \cf{(S\bs^i NP)\bs (S\bs^i NP)} and \cf{(S\bs^i NP)/(S\bs^i NP)}.
\footnote{Note that with the extended slash definition, modifiers must coindex their slashes
to ensure that they behave as functions that return their arguments unchanged.}
The inert slash would also allow the \cf{S\bs S} category to modify
\cf{S[adj]\bs^! NP}, as inert slashes can be the arguments of combinators, but
not their functors.

What we do not want in this analysis is for the predicate to apply its argument
directly. Assigning an inert mode prevents this, allowing the argument to fulfill
desirable structural criteria without increasing ambiguity in the grammar.

One subtlety to note is that inert slashes cannot be circumvented via type-raising.
At first it might seem that this would succeed:

\begin{center}
\deriv{2}{
\rm *Pat & \rm handy \\
\uline{1}&\uline{1} \\
\cf{NP} &
\cf{S[adj]\bs ^!NP} \\
\ftype{1} \\
\mc{1}{\cf{S/^i(S\bs^i NP)}} \\
\asterisk{2} \\
\mc{2}{\cf{S[adj]}}
}\end{center}

The crucial detail is that the type-raising rules are defined such that the two slashes
are coindexed. This leaves two possible forward type-raise to \cf{S} categories for Pat:
either both slashes can be inert, \cf{S/^!(S\bs^! NP)}, or both slashes can be active,
\cf{S/^+(S\bs^+ NP)}. If the slashes are inert, then the type-raise category cannot be
the functor, so the application is blocked. If the slashes are active, then unification
between \cf{S\bs^+NP} and \cf{S[adj]\bs^!NP} will fail because their slashes are
incompatible, and the categories still cannot combine.

\subsection{Normal Form Constraints}
\label{sec:normal_form}
Multi-modal slashes allow \ccg to control the undesirable effects of adding
permutative rules to the grammar --- namely, over-generation. From a processing
standpoint, associativity also has drawbacks, because it introduces multiple
equivalent bracketings of the same semantic analysis. For instance, the
alternative bracketing required in analysis \ref{bracket_1} can also be used in
analysis \ref{bracket_2}, where it is not required:

\begin{lexamples}
 \item (Erin hates) but (Casey likes) Pat. \label{bracket_1}
 \item (Erin hates) Pat. \label{bracket_2}
\end{lexamples}


\citet{steedman:00} introduces an explanation for this apparently
uneconomical situation, pointing out that even though the sentence only has one
valid semantic analysis, it does have two valid information structure analyses
--- which correspond to the two possible bracketings. \citet{white:10} exploit
this ability to express information structures in the surface syntax to produce
more natural speech generation. The ambiguity also allows
extreme left-corner analyses, which are useful for incremental parsing.

Nevertheless, the ambiguity can still be inconvenient. \ccg grammars can be
parsed using the \cky algorithm \citep{cocke:70,kasami:65,younger:67}, and the
extra analyses complicate the task of selecting the best semantic analysis.
\citet{eisner:96} offers a simple solution. He shows that for every semantic
analysis, there is exactly one \emph{normal form} derivation, and that
two simple constraints will ensure that only normal form derivations
are added to the chart.

The Eisner normal form constraints stipulate that if a category was produced by a
forward composition rule, it cannot immediately be used as the functor of forward
application; and similarly, if a category was produced by backward composition, it
cannot immediately be used as the functor of backward application.
\citeauthor{eisner:96} shows that these two rules are sufficient to prevent all spurious
ambiguity in a \ccg grammar that consists of application and composition rules,
but does not contain type-raising rules. \ccgbank's grammar does contain
type-raising and type-changing rules, so the normal form constraints do not
guarantee that derivations uniquely correspond to semantic analyses.
However \citet{clark:cl07} have found that the
Eisner normal form constraints are useful for increasing the efficiency of \ccg
parsing using the \ccgbank grammar.

\section{\ccgbank}

\citet{hock:01,hock:acl03} developed the first wide coverage \ccg parsing system.
The work that made this possible was the adaptation of the Penn Treebank
\citep{marcus:93} to \ccg \citep{hock:lrec02,hock:cl07}. This section describes the
corpus, which plays a central role in the thesis. First, we provide a brief overview
of the Penn Treebank, with a very brief look at the long history of statistical parsing
research it initiated. We will then describe the \ccgbank conversion process. Finally, we
describe the use of type-changing rules in \ccgbank,
to carry out type-changing operations. These type-changing rules are particularly important
for this thesis.

\subsection{The Penn Treebank}


The Penn Treebank (\penn) \citep{marcus:93} is a syntactically annotated corpus of English.
Version 3 of the corpus consists of approximately 1.2 million words of newswire text
from the Wall Street Journal, 900,000 words of text from a variety of written genres
from the Brown corpus, 1.1 million words of casual conversation between strangers from the
Switchboard corpus, and 10,000 words of spoken dialogue system queries from the
\textsc{atis} corpus. Syntactic parsing research has focused almost entirely on the
Wall Street Journal portion. The initial release of the corpus consisted of the Wall
Street Journal and \textsc{atis} corpora.

The corpus was annotated according to a theory-neutral annotation scheme intended
to maximise the cost effectiveness of the project. The exact scheme was determined
largely by practical considerations, especially the output format of the Fidditch
parser \citep{hindle:83}, which was used to provide annotators with tree fragments
to `glue' together. For instance, null elements were included in the annotation
because Fidditch already produced them, and it was found that they were not expensive
to correct \citep{marcus:93}. Without these null elements, the conversion to
\ccg would have been far more difficult.

Two other noteworthy aspects of the annotation scheme are the treatment of
complement-adjunct distinctions, and the flat noun phrase bracketing. Consistent
complement-adjunct distinctions were found to slow down annotation by roughly
150 words per hour, and could not be made consistently. A flat bracketing of noun
phrases was adopted because inserting more fine-grained brackets than those provided
by the Fidditch parser would have been quite costly. Full noun phrase bracketing was
finally added by \citet{vadas:07}, and integrated into \ccgbank by \citet{vadas:08}.
These additions to the \penn released were late in this project, so were not used in the
experiments described in Chapter \ref{chapter:results}.

% Updated
Figure \ref{fig:pierre_ptb} shows the annotation of the first sentence in the Wall
Street Journal portion of the corpus. This sentence contains two function tags,
\verb1-TMP1 and \verb1-CLR1, which show some predicate-argument information:
\verb1-TMP1 notes that the constituent functions as a temporal adjunct, while 
\verb1-CLR1 records (somewhat vaguely) that the constituent is `closely related'
to the verb. However, the distinctions are not drawn very consistently.
Full predicate-argument labels were later annotated and released as Propbank
\citep{propbank}. The Propbank annotation labels \emph{as a nonexecutive director}
as an adjunct, rather than a complement --- which is how \citet{hock:cl07} interprets
the \verb1-CLR1 label. This shows how difficult complement-adjunct distinctions
can be to annotate. In contrast, noun phrase brackets often involve very easy,
even trivial, annotation decisions. For instance, the noun phrase \emph{a nonexecutive
director} should obviously be right branching, but this structure is left unspecified.

\begin{figure}
 \begin{verbatim}
  ( (S
    (NP-SBJ
      (NP (NNP Pierre) (NNP Vinken) )
      (, ,)
      (ADJP
        (NP (CD 61) (NNS years) )
        (JJ old) )
      (, ,) )
    (VP (MD will)
      (VP (VB join)
        (NP (DT the) (NN board) )
        (PP-CLR (IN as)
          (NP (DT a) (JJ nonexecutive) (NN director) ))
        (NP-TMP (NNP Nov.) (CD 29) )))
    (. .) ))
 \end{verbatim}
\caption[A Penn Treebank bracketed sentence.]{The Penn Treebank analysis of the
first sentence in the corpus. The PP-CLR tag assigned to \emph{as a nonexecutive
director} produces a complement/adjunct error in the \ccgbank derivation. The
apposition between \emph{Pierre Vinken} and \emph{61 years old} produces a
binary type-changing rule.\label{fig:pierre_ptb}}
\end{figure}

\citet{magerman:95} was the first to exploit the new resource, using \textsc{spatter},
a decision tree-based statistical parser. \citet{magerman:94} had shown that
\textsc{spatter} already outperformed the leading manually written system, the
\textsc{ibm}-Lancaster parser, which had been under development for 10 years. When
trained on the Lancaster Computer Manuals corpus \citep{black:96}, \textsc{spatter}
achieved a 0-crossing score of 76\%, substantially higher than the 69\% reported by
\citeauthor{black:93}\footnote{\citet{magerman:94} followed \citet{black:93} in reporting 0-crossing for
direct comparison, even though this measure was by then considered less informative
than $F$-measure as defined by \parseval \citep{black:91}.}
Once the Penn Treebank was released, development of wide-coverage manually written
phrase-structure grammars largely ceased, although interest in manually written
grammars for other formalisms has continued \citep[e.g. ][]{xtag, erg, xle}.

% Updated
\citet{magerman:95} achieved an $F$-score of 84.1\% on Section 23 of the Penn Treebank on
sentences of 100 or fewer tokens. All subsequent Penn Treebank parsers have reported
results that can be compared directly against this figure, allowing us to trace 15 years
of improvements in statistical parsing.

% Updated
The first to improve on \citeauthor{magerman:95}'s model was \citet{collins:96}, which
defined an efficient conditional model using bilexical dependencies. Each constituent
was assigned a head using head-finding heuristics similar to \citet{magerman:95},
and base noun phrases (detected by a chunker) were represented solely by their heads.
Each parent and child node pair in the tree was then considered a bilexical dependency
labelled by the node label of the argument, the node label of the parent, and the node
label of the sibling that heads the parent. Dependencies were assigned probabilities
using maximum likelihood estimation, with part-of-speech backoff as described in
\citet{collins:95}. The best performing configuration achieved 85.5\% $F$-score.

% Updated
The first accurate generative grammar parser was described by \citet{charniak:97}.
The parser used a probabilistic context-free grammar (\pcfg) with several innovations
to weaken the problematic \pcfg independence assumptions. The most important innovation
was to extend each node label with the head word and part-of-speech, so that bilexical
dependency statistics could be estimated. Deleted interpolation smoothing was used to
mitigate sparse data problems. \citet{charniak:97} achieved 86.6\% $F$-score. After a
thorough analysis, he concluded that the most important factor in this
improvement in accuracy between Collins and Magerman was due to Collins' conditioning
on individual words more often than Magerman, due to Magerman's use of decision trees
to estimate probabilities. Charniak attributes his improvement over Collins to his superior
back-off probabilities and smoothing.

% Updated
\citet{collins:97} then defined three generative parsing models, ultimately improving
accuracy to 87.8\%. Model 1 was effectively a generative version of \citet{collins:96}.
Model 2 extended the parser to make complement/adjunct distinctions, using probabilities
over subcategorisation frames for headwords. Nodes labelled NP, SBAR or S directly under S;
nodes labelled VP, NP, SBAR or S directly under VP; and nodes labelled SBAR directly under S
were considered complements if they were not labelled with an adverbial function tag. There
are two motivations for drawing complement/adjunct distinctions. First, they offer potentially
useful distinctions to downstream applications. Second, they offer potentially
useful information for the model. \citeauthor{collins:97} found that conditioning over whole
subcategorisation frames, rather than assuming that arguments were generated independently,
improved accuracy slightly. Model 3 pursued further linguistic correctness by modelling traces
and WH-movement. Once again, the motivation is twofold: the desire for a more detailed
output, and the suggestion that moved constituents hamper the estimation of subcategorisation
frame probabilities. Model 3 achieved exactly the same accuracy as Model 2, and traces
representing non-local dependencies were recovered with only 67.4\% accuracy.

% Updated
The next breakthrough was made by \citet{ratnaparkhi:97}, who used a conditional
log-linear model, rather than a generative one. Conditional models allow arbitrary
features to be encoded --- although, for structured prediction problems such as parsing,
the features do have to be locally decidable to make the dynamic programming tractable.
\citeauthor{ratnaparkhi:97}'s model achieved an accuracy of 86.9\% $F$-measure, but
the introduction of conditional parsing opened the door for substantial subsequent
improvement.
\citet{charniak:00} followed \citet{ratnaparkhi:97} in using a conditional model,
and achieved an accuracy of 89.5\% $F$-measure. Charniak experimented with
a variety of features, and even performed a self-training experiment, in order to obtain
more robust estimates for bilexical features. The self-training experiment only improved
accuracy by 0.4\%.

% Updated
The next major innovation in this line of research was re-ranking. \citet{collins:00}
used a conditional model to select the best parse from the $k$-best produced by the
\citet{collins:97} model, and achieved accuracy comparable to \citet{charniak:00}. The
advantage of re ranking is that there are no restrictions on the features that can be
extracted from the parse tree, as it is a conditional model that does not require any
dynamic programming. \citet{charniak:05} applied re ranking to their statistical parser,
and achieved an $F$-score of 91.4\%. \citet{mcclosky:06} improved this to 92.1\% using
self-training, which remains the state-of-the-art.

% Updated
This line of research is but one of the many that have been pursued in English
statistical parsing. In particular, there is an extensive literature on dependency
parsing, using grammars extracted from the Penn Treebank. \citet{kubler:09} describes
this body of work well. Parsers have also been developed for corpora acquired from the
Penn Treebank.

% Updated
In Section \ref{sec:background_candc}, we provide an overview of one such parser.
The \candc parser \citep{clark:cl07} takes advantage of the fact that \ccg, as a
lexicalised formalism, naturally presents a way for a parser to include lexical
information and predicate-argument structure in its probability models. The most
important development for this approach to parsing was the development of a separate
supertagging phase \citep{srinivas:99}, where lexical categories are assigned by a
sequence tagger. \citet{clark:coling04} showed that a multi-tagger could be tightly
integrated with a chart parser, producing substantial improvements in speed and accuracy.
Similar findings have since been reported for \hpsg, another lexicalised formalism that
offers many of the same advantages for parsing as \ccg. In particular, the \enju parser
\citep{miyao:08} uses many of the same design features as the \candc model, and has been
found to be the most accurate parser at recovering long range dependencies \citep{rimell:09}.

\subsection{Conversion Process}

\ccgbank was created semi-automatically using a conversion algorithm that proceeds top-down over each tree:

\begin{center}
\begin{verbatim}
  foreach tree T:
    determineConstituentType(T);
    makeBinary(T);
    assignCategories(T);
\end{verbatim}
\end{center}

% Updated
This algorithm assumes that the input Penn Treebank trees conform to the
analyses desired for the \ccg corpus, which is often not the case. Analyses
diverge because of outright errors and inconsistencies in the Penn Treebank,
and the different capabilities of the two formalisms.
\citet{hock:thesis03} provides the following extended description of the conversion
process:

% Updated
\begin{center}
\begin{verbatim}
  foreach tree T:
    preprocessTree(T);
    preprocessArgumentClusters(T);
    determineConstituentType(T);
    makeBinary(T);
    percolateTraces(T);
    assignCategories(T);
    treatArgumentClusters(T);
    cutTracesAndUnaryRules(T);
\end{verbatim}
\end{center}

% Updated
\textbf{preprocessTree}. This stage corrects \pos tag errors and corrects analyses
for noun phrases, coordinate constructions and small clauses that diverge from the
desired \ccg analysis. It also eliminates quotation marks.
Quotation marks are attached inconsistently
in the Penn Treebank and are difficult to analyse in \ccg, partly because of editorial
conventions that force quotes outside punctuation, even when they logically belong
inside it.

% Updated
\textbf{preprocessArgumentClusters}. \ccg analyses allow argument clusters, as in
\emph{Give a policeman a rose and a fireman a violet}, to be bracketed together.
The Penn Treebank analysis of this phenomenon is quite different. The tree is thus
pre-processed to make use of the \ccg analysis.

% Updated
\textbf{determineConstituentType}. \emph{Constituent type} here refers to three labels:
\emph{head}, \emph{argument}, and \emph{adjunct}. These labels were assigned using slight
variations on the head finding heuristics described by \citet{magerman:95} and
\citet{collins:96}.

% Updated
\textbf{makeBinary}. Once the nodes were labelled as head, argument or adjunct,
the tree was binarised. Dummy nodes were inserted on the tree to the left of the
head, such that the tree right-branched towards the head. The same was done for
nodes to the right of the head.

% Updated
\textbf{percolateTraces}. The conversion algorithm relies on the Penn Treebank
\verb1*T*1 and \verb1*RNR*1 trace nodes to assign categories correctly when
arguments have been moved. This stage determines the category of the traces by following
them to their reference node, and percolates them up to their appropriate level
in the tree.

% Updated
\textbf{assignCategories}. The nodes are then assigned categories. The root node
and complement nodes are determined by a mapping from \penn node labels to \ccg categories.
Head nodes and adjuncts were then assigned labels based on their siblings.

% Updated
\textbf{treatArgumentClusters}. This stage inserts type-raising nodes and uses
composition rules to coordinate nodes in argument clusters, following the standard
\ccg analysis.

\textbf{cutTracesAndUnaryRules}. This stage cleans up the resulting tree by removing
traces, and deleting unary \cf{X}$\rightarrow$\cf{X} productions.

\citet{hock:cl07} report that the algorithm does not deal with 306, or 0.76\%, of the 39,832
trees. The main classes of trees not dealt with are unlike coordinated phrases where
the types of the conjuncts could not be identified, verb phrase gapping, and trees
rooted in \verb1X1. Gapping is a very difficult construction to analyse in \ccg,
and \verb1X1-rooted sentences can largely be considered performance noise, so it is unclear
how such sentences would have been analysed even by manual annotation.


\subsection{Type-Changing Rules in CCGbank}

% Updated
After applying the \ccgbank conversion process, \citet{hock:acl02} found
that it caused a proliferation of modifier categories. We explore this problem
in detail in Chapter \ref{chapter:ling_mot}, arguing that its root cause is the
difficulty of exploiting generalisations about constituent type in \ccg. Section
\ref{sec:ling_psg_rules} discusses the theoretical implications of
\citeauthor{hock:acl02}'s solution, which is to add type-changing rules
to the grammar.

The example \citet{hock:thesis03} used to demonstrate the modifier category
proliferation is a sentence from Section 00. Without type-changing rules,
the \ccg derivation for this sentence might look like this:
\footnote{See Sections \ref{sec:ling_rrc} and \ref{sec:pure_rrc} for further
discussion on how reduced relative clauses can be analysed in \ccg without
type-changing rules or hat categories.}
 
\begin{center}
\deriv{4}{
\rm a~form~of~asbestos & \rm once & \rm used & \rm to~make~cigarette~filters \\
\uline{1}&\uline{1}&\uline{1}&\uline{1} \\
\cf{NP} &
\cf{(NP\bs NP)/(NP\bs NP)} &
\cf{(NP\bs NP)/(S[to]\bs NP)} &
\cf{S[to]\bs NP} \\
&& \fapply{2} \\
&& \mc{2}{\cf{NP\bs NP}} \\
& \fapply{3} \\
& \mc{3}{\cf{NP\bs NP}} \\
\bapply{4} \\
\mc{4}{\cf{NP}}
}
\end{center}

The past participle, \emph{used}, receives a different category than it would
if it occurred in a main verb phrase. Adjuncts modifying
\emph{used} will also require different categories --- as will any words modifying
them. As we describe in Section \ref{sec:ling_rrc}, there are more attractive
analyses of this construction, but this analysis does illustrate how modifier categories
can proliferate when there are form/function discrepancies.

% Updated
\citeauthor{hock:acl02}'s solution was to introduce context-free type-changing rules
into the grammar. The example above can be analysed with the canonical categories by
employing the type-changing rule \cf{S[pss]\bs NP} $\rightarrow$ \cf{NP\bs NP}
\footnote{This and following production rules are presented in bottom-up notation.
Production rules often contain punctuation, so for clarity we often wrap single rules
in angle brackets.}. We mark such rules \textsc{tc}:

\begin{center}
\deriv{4}{
\rm a~form~of~asbestos & \rm once & \rm used & \rm to~make~cigarette~filters \\
\uline{1}&\uline{1}&\uline{1}&\uline{1} \\
\cf{NP} &
\cf{(S\bs NP)/(S\bs NP)} &
\cf{(S[pss]\bs NP)/(S[to]\bs NP)} &
\cf{S[to]\bs NP} \\
&& \fapply{2} \\
&& \mc{2}{\cf{S[pss]\bs NP}} \\
& \fapply{3} \\
& \mc{3}{\cf{S[pss]\bs NP}} \\
& \psgrule{3} \\
& \mc{3}{\cf{NP\bs NP}} \\
\bapply{4} \\
\mc{4}{\cf{NP}}
}
\end{center}

\begin{figure}
 \centering
\begin{center}
\deriv{10}{
\rm These & \rm actions & \rm are & \rm risky & \rm and & \rm not & \rm in & \rm
our & \rm best & \rm interests \\
\uline{1}&\uline{1}&\uline{1}&\uline{1}&\uline{1}&\uline{1}&\uline{1}&\uline{1}
&\uline{1}&\uline{1} \\
\cf{NP/N} &
\cf{N} &
\cf{(S[dcl]\bs NP)/(S[adj]\bs NP)} &
\cf{S[adj]\bs NP} &
\cf{conj} &
\cf{PP/PP} &
\cf{PP/NP} &
\cf{NP/N} &
\cf{N/N} &
\cf{N} \\
\fapply{2} &&&&&&& \fapply{2} \\
\mc{2}{\cf{NP}} &&&&&&& \mc{2}{\cf{N}} \\
&&&&&&& \fapply{3} \\
&&&&&&& \mc{3}{\cf{NP}} \\
&&&&&& \fapply{4} \\
&&&&&& \mc{4}{\cf{PP}} \\
&&&&& \fapply{5} \\
&&&&& \mc{5}{\cf{PP}} \\
&&&& \psgrule{6} \\
&&&& \mc{6}{\cf{S[adj]\bs NP[conj]}} \\
&&& \conj{7} \\
&&& \mc{7}{\cf{S[adj]\bs NP}} \\
&& \fapply{8} \\
&& \mc{8}{\cf{S[dcl]\bs NP}} \\
\bapply{10} \\
\mc{10}{\cf{S[dcl]}}
}
\end{center}
\caption[Conjunction cued binary type-changing rule in \ccgbank]{Derivation
using a conjunction cued binary type-changing rule to handle an unlike
coordinated phrase.\label{fig:conj_raising}}
\end{figure}

\begin{table}[h!]
\centering
\begin{tabular}{r|c|c||r|c|c}
\hline
Freq & Child & Parent & Freq & Child & Parent\\
\hline
\hline
142,530 & \cf{N} & \cf{NP} & 1,464 & \cf{S[to]\bs NP} & \cf{NP\bs NP} \\
4,052 & \cf{S[pss]\bs NP} & \cf{NP\bs NP} & 1,070 & \cf{S[dcl]/NP} & \cf{NP\bs NP} \\
1,818 & \cf{S[ng]\bs NP} & \cf{NP\bs NP} & 370 & \cf{S[ng]\bs NP} & \cf{NP} \\
1,617 & \cf{S[adj]\bs NP} & \cf{NP\bs NP} & 254 & \cf{S[ng]\bs NP} & \cf{S/S} \\
1,606 & \cf{S[to]\bs NP} & \cf{(S\bs NP)\bs (S\bs NP)} & 209 & \cf{S[dcl]} & \cf{NP\bs NP} \\
1,522 & \cf{S[ng]\bs NP} & \cf{(S\bs NP)\bs (S\bs NP)} & 192 & \cf{S[pss]\bs NP} & \cf{S/S} \\
1,476 & \cf{S[to]\bs NP} & \cf{N\bs N} & 154 & \cf{S[to]\bs NP} & \cf{S/S} \\
\hline
\end{tabular}
\caption{The most frequent unary type-changing rules in CCGbank.}
\label{tab:background_psg_rules}
\end{table}


Table \ref{tab:background_psg_rules} shows the most frequent unary type-changing rules in
\ccgbank. Most of the type-changing rules in \ccgbank are unary, but some of the
rules are cued by punctuation or a conjunction, making them binary. The
advantage of making these rules binary is that they introduce less ambiguity
into the grammar. Figure \ref{fig:conj_raising} shows an example where
a conjunction is used to cue a type-change. This involves a rather \emph{ad hoc}
type-changing rule:

\begin{eqnarray}
 \eqnpsrule{\cf{conj}}{\cf{PP}}{\cf{S[adj]\bs NP[conj]}}
\end{eqnarray}

This rule is also an example of how type-changing rules can allow categories
to add or delete arguments arbitrarily, breaking the isomorphism between lexically
assigned syntactic types and the resulting logical forms. This makes the
type-changing rules unacceptable on a
theoretical level, because they are incompatible with \citepos{steedman:00} core
argument. The first sentence of \citet{steedman:00} states the \ccg thesis as:

\begin{quote}
 This book argues that the Surface Syntax of natural language acts as a
completely transparent interface between the spoken form of the language,
including prosodic structure and intonational phrasing, and a compositional
semantic interface.
\end{quote}

This type-transparency is impossible to achieve if an atomic category is allowed
to transform into a functor, as the \psbinary{,}{NP}{S/S} rule used to handle
extraposition allows. An example of this rule is shown in
Figure~\ref{fig:extraposition}. In this example, the \cf{NP} \emph{our first
major decline} is coerced into a sentential modifier. However, there is no room
for an argument on its original lexical category, so there is no way to
represent the dependency between \emph{decline} and \emph{fell}. This causes the
\ccgbank dependency graph of the
sentence to be unconnected, as \emph{decline} does not have a head. The binary
type-changing rules are discussed in more detail in Chapter
\ref{chapter:hat_corpus}. The motivation for the rule is the modifier category
proliferation problem. If \emph{decline} were assigned a category that reflected
its function, the categories of its modifiers \emph{first} and \emph{major}
would have to change too, as would any post-modifiers that might attach, such as
\emph{this quarter}. We explore this problem with \ccg in more detail in Chapter
\ref{chapter:hat_cats}.

\begin{figure}

\begin{center}
\deriv{10}{
\rm Factory & \rm inventories & \rm fell & \rm 5 & \rm \% & \rm , & \rm our &
\rm first & \rm major & \rm decline \\
\uline{1}&\uline{1}&\uline{1}&\uline{1}&\uline{1}&\uline{1}&\uline{1}&\uline{1}
&\uline{1}&\uline{1} \\
\cf{N/N} &
\cf{N} &
\cf{(S[dcl]\bs NP)/NP} &
\cf{NP} &
\cf{NP\bs NP} &
\cf{,} &
\cf{NP/N} &
\cf{N/N} &
\cf{N/N} &
\cf{N} \\
\fapply{2} && \bapply{2} &&&& \fapply{2} \\
\mc{2}{\cf{N}} && \mc{2}{\cf{NP}} &&&& \mc{2}{\cf{N}} \\
\psgrule{2} & \fapply{3} &&& \fapply{3} \\
\mc{2}{\cf{NP}} & \mc{3}{\cf{S[dcl]\bs NP}} &&& \mc{3}{\cf{N}} \\
\bapply{3} &&&& \fapply{4} \\
\mc{3}{\cf{S[dcl]}} &&&& \mc{4}{\cf{NP}} \\
&&&&& \psgrule{5} \\
&&&&& \mc{5}{\cf{S\bs S}} \\
\bapply{10} \\
\mc{10}{\cf{S[dcl]}}
}
\end{center}
\caption[Derivation with binary type-changing rule in \ccgbank.]{Derivation
showing a binary \textsc{psg} rule cued by punctuation. The comma is used to cue
a rule that changes the \cf{NP} into a sentential modifier. This prevents the
formation of a connected dependency graph for the sentence, and results in a
loss of semantic transparency.\label{fig:extraposition}}
\end{figure}

\section{The \candc Parser}
\label{sec:background_candc}

The best \ccg parsing results to date have been achieved by the \candc parser.
The parser uses a global discriminative model, trained on \ccgbank. The most
similar system is probably the \citet{miyao:08} \hpsg parser, which also uses a
global model over a packed chart, and a supertagger to assign lexical
categories.

In this section we will briefly describe the parser, focusing on the aspects of
its design that are most important for this thesis. We will start with its
tightly integrated supertagger, before providing an overview of the chart parser
and the discriminative maximum entropy model used to select the most probable
parse.

\subsection{Supertagging}

\citet{srinivas:94} showed that \ltag lexical categories can be assigned in a
sequence tagging stage, using much the same sort of technique as part-of-speech
and named entity tagging. \citet{clark:tag02} showed that supertagging could be
applied to \ccg, and that the accuracy of a maximum entropy tagger
\citep{ratnaparkhi:96} on \ccgbank was comparable to the performance
\citet{srinivas:99} achieved on a manually constructed \ltag grammar, and
substantially higher than the performance \citet{chen:00} achieved on an
automatically extracted \ltag grammar.

The primary appeal of supertagging is that it allows much of the parsing work to
be transformed into a linear process. The tags can be assigned using well
understood Markov-based techniques that run in linear time with respect to the length
of the input sentence. Once the tags are
assigned, there is relatively little ambiguity remaining for the parser, which
is why \citeauthor{srinivas:99} referred to supertagging as `almost parsing'.

The efficiency improvement is explained by \citepos{sarkar:00} finding that
parsing efficiency is not solely related to sentence length. It is also related
to the syntactic ambiguity of the sentence, which in a lexicalised grammar is
largely predictable from the lexical category ambiguity. Supertagging allows
category ambiguity to be reduced, by forwarding only the categories judged most
likely to the parser.

One obvious configuration is to have the supertagger assign exactly one category
per word, eliminating all lexical category ambiguity. In this configuration, the
supertagger and parser interact in a simple cascade model: the supertagger makes
its predictions, which are fed directly to the parser. This configuration
suffers from the usual problems of cascaded classification. Errors propagate
along the pipeline, and the early model cannot benefit from the decisions of
subsequent models \citep{hollingshead:07}. The later models are also unable to take
into account any
uncertainty in the initial model, because they are given only the most likely label,
instead of the whole probability distribution \citep{finkel:06}.

The clearest problem occurs when the parser fails to assign a parse from the
supertag sequence that has been assigned. \citet{srinivas:94} address this by
\emph{multi-tagging}, assigning the $n$-best categories per word.
\citet{clark:cl07} take a similar approach, assigning all categories whose
probability is within a factor, $\beta$, of the maximum probability the
supertagger assigned to a single category for that word.\footnote{That is, the supertagger assigns all categories $t$ whose
probabilities $p(t) \geq \beta \cdot p\ast$, where $p\ast$
is the highest ranked tag's probability.}
Multi-tagging allows the \candc supertagger to make very favourable trade-offs between accuracy and
lexical ambiguity. 

\subsection{Integrating a Supertagger and Parser}

\citet{clark:emnlp03} and \citet{clark:coling04} present two ways the
multi-tagger can be integrated into a \ccg parsing system. The initial strategy
was to initialise $\beta$ at 0.01, and attempt to find a parse. If the ambiguity
caused too many categories to be added to the chart, the $\beta$ value was increased, tightening
the beam of categories sent to the parser, and the parser tried again.

This approach assigns as much work to the parser's model as possible, because it
has the supertagger forward as many categories as the parser can handle.
\citeauthor{clark:cl07} assumed that this configuration would
trade speed for accuracy, because it seemed safe to assume that the parser's
global model would be more accurate than the supertagger's sequence model.

\citet{clark:coling04} describe the critical finding that this was not the case.
Instead, the system performed both faster \emph{and} more accurately if the
supertagger was allocated much of the work. In this configuration, the $\beta$
level is initialised to $0.075$, and subsequently relaxed if parsing fails. This
is essentially the opposite strategy to the one described above, where the
$\beta$ levels proceed from low to high.

What is the explanation for this result? For a start, it seems that the limited
horizon assumption performs quite well --- most of the tagging decisions
actually are locally decidable. This is partially explained by
\citepos{gildea:07} finding that most dependencies are quite short-range: the
average dependency distance in English is 2.3. In fact, \citet{hawkins:90}
argues that the pressure to minimise dependency distances is a considerable
influence on the syntax of all human languages.

% Another factor might be that the number of competing analyses the parser's model
% must consider,  the chart size, grows exponentially with increases in supertag
% ambiguity. Even if the parser's model were much better suited to making
% decisions than the supertagger, its advantage might be overwhelmed by the number
% of competing analyses it has to decide between.


\subsection{Discriminative Parsing with Log-Linear Models}

After the supertagger has assigned one or more lexical categories to each word
in the sentence, the categories are assembled into \ccg derivations and
corresponding dependency analyses using the modified \cky algorithm first
described by \citet{hock:01}. Despite the reduction in ambiguity provided by the
supertagger, the length of the sentences in \ccgbank and the ambiguity in the
grammar necessary to produce high coverage mean that many sentences produce an
astronomical number of candidate analyses. For one sentence in Section 00, the
default configuration of version 1.02 of the parser leads to over
$6.39\times10^{23}$ unique parses. The sentence is:

\begin{quote}
For a while in the 1970s it seemed Mr. Moon was on a spending
spree, with such purchases as the former New Yorker Hotel and its
adjacent Manhattan Center; a fishing/processing conglomerate with
branches in Alaska, Massachusetts, Virginia and Louisiana; a former
Christian Brothers monastery and the Seagram family mansion
(both picturesquely situated on the Hudson River); shares in banks
from Washington to Uruguay; a motion picture production
company, and newspapers, such as the Washington Times, the New
York City Tribune (originally the News World), and the successful
Spanish-language Noticias del Mundo.
\end{quote}

At 108 tokens, this sentence is far from the longest in the Penn Treebank, but the long
coordinated noun phrase forces the parser to consider an enormous number of
possible analyses.

The \candc parser implements a modified \ccg grammar optimised for \ccgbank. The
most important modifications are implementations of the \ccgbank type-changing
rules, which are hard-coded into the parser. Type-raising rules are handled in
an equivalent way to unary type-changing rules, except they are not hard-coded
(possible type-raising operations are listed in text files read by the parser).
The parser also implements some restrictions on the composition rules, and some
of its models make use of the \citet{eisner:96} normal-form constraints. Some
models also use a list of productions seen in the training corpus to further
restrict the productivity of its grammar. Appendix \ref{appendix:type-changing}
lists each of the type-changing rules found in \ccgbank and the frequency of its
occurrence in Section 02-21.


\section{Applications of Combinatory Categorial Grammar}

There are a few other strands of \ccg work within the \nlp community that we
have not yet mentioned. Although we do not want to go into great detail about
these bodies of work, we will briefly mention the prominent projects, for
completeness.

\subsection{\ccg for Semantic Analysis}

\ccg's isomorphism between syntax and semantics has been used
to perform semantic role labelling. The first example of this was \citet{gildea:03},
who built a semantic role labelling system using the output of the
\citet{hock:acl03} \ccg parser. However, this research used an early version of
Propbank \citep{propbank}, which introduced a variety of
difficulties. For instance, arguments realised by prepositional phrases in \ccg
were listed as noun phrase arguments in the initial PropBank --- the preposition
was excluded from the argument span. Despite these issues,
\citeauthor{gildea:03} found that the semantic role labeller performed well on
core arguments with a simple feature set, suggesting that \ccg's recovery of
long-range dependencies and direct representation of verbal subcategorisation
frames simplified the task. \citet{boxwell:09}, who used the \candc parser
in a semantic role labelling system, reached a similar conclusion.

\ccg has also been used to recover logical form-based semantic analyses, as well
as the shallow semantic analysis provided by PropBank-style predicate-argument
structures. The first example of this was \citet{bos:coling04}, who
post-processed the output of the \candc parser to provide logical forms. The
system to do this, dubbed Boxer, can be downloaded and used along with the
\candc tools. Boxer has been used as the basis for competitive
question answering systems \citep{leidner:04,bos:06,bos:07}. \citet{bos:05}
showed that it could also be used for textual entailment. \citet{zettlemoyer:07}
used \ccg-based logical form parsing in a system that achieved competitive
results on the GeoQuery task, which is essentially a sort of natural language
database querying challenge. \citet{zettlemoyer:09} showed that \ccg could be
used as the basis of a system that retrieved logical forms for sentences in the
\textsc{atis} corpus with over 83\% accuracy.

\subsection{\ccg for Chart Realisation}

% Updated
Chart, or surface, realisation is the task of going from a (possibly underspecified)
semantic representation to one or more surface strings that can express it. It is
the inverse of the parsing task. Linguistically motivated formalisms have been a
popular choice for this task since the Penman system \citep{mann:85}, which later
merged with the \textsc{komet} system to become \kpml \citep{bateman:99}.
Surface realisers have also been developed for \hpsg \citep{wilcock:98} and
\lfg \citep{cahill:06}.

% Updated
Research on the use of \ccg for chart realisation has to some extent paralleled
research on \ccg parsing. \citet{white:03} describe the architecture of a
\ccg-based surface realiser. Their system, which was implemented as an extension
to the OpenCCG parser, uses a bottom-up approach to realising logical forms specified
in Hybrid Logic Dependency Semantics \citep{kruijff:01}.

% Updated
This early work on \ccg surface realisation used precise, manually developed grammars
for dialogue systems. \citet{white:07} began work on a wide-coverage \ccg realiser,
using a grammar extracted from \ccgbank. Research along these lines has progressed
rapidly, in two dimensions: modifications to \ccgbank to support more accurate logical
form recovery, and improvements to their statistical modelling. Modifications to
\ccgbank have included lexicalised treatment of punctuation \citep{white:punct08},
alignment with Propbank \citep{boxwell:08}, and integrating BBN named entity analyses
\citep{rajkumar:09}. Improvements to their statistical models have included a
perceptron reranker \citep{white:09} and a \emph{hypertagger}, a model which predicts
which lexical categories to use in the surface realisation. The
categories then closely constrain the space of possible surface strings, in much the
same way that the categories output by a supertagger constrain the chart space required
to build a semantic analysis from an input string.

% Updated
The hypertagger is inspired by the \citet{clark:tag02} \ccg supertagger, and improves the
chart realiser's speed and accuracy, and a reranker would likely improve the accuracy of
the \candc parser. Research on \ccg parsing and generation, then, has encountered similar
issues, and solutions developed for one problem can be profitably adapted to the other.
This is especially apparent for refinements to \ccgbank, the key resource for both lines
of research.

% Updated
OpenCCG has been used in a number of natural language dialogue applications: the
COMIC, FLIGHTS, CrAg, Methodius and INDIGO projects in Edinburgh; the DIALOG, SAMMIE and
CoSy projects in Saarbrucken; the JAST project in Munich; the AdaRTE project in Pavia;
and the STaR-UI project in Sydney. Many of these projects involve spoken language dialogue
systems. \ccg has a particular advantage for speech generation, because its non-standard
constituents allow prosodically salient portions of a sentence to be grouped together.
This allows the chart realiser to map information structure specifications onto surface
forms as prosodic tunes, so that a speech synthesiser can produce more natural sounding
output. \citet{white:10} demonstrated this capability in the FLIGHTS dialogue system,
showing that the information structure-aware version of the system scored significantly
higher in user ratings of naturalness, and that the system accurately produced prosodic
tunes that an expert recognised as contextually appropriate.\footnote{
The stimuli for these experiments are available at
http://www.ling.ohio-state.edu/~mwhite/flights-stimuli/} \ccg's flexible notion of
constituency has also been used to allow a dialogue system to interpret user utterances
incrementally, in the CoSY project \citep{kruijff:07}.


\subsection{\ccg for Machine Translation} 

% Updated
Finally, there has also been a line of research using \ccg for machine translation.
\citet{hassan:07} showed that the \candc supertagger could be used to add some syntactic
awareness to a language model, leading to state-of-the-art performance on Arabic to English
machine translation. \citet{birch:07} observed a more modest improvement on translation
between English and Dutch, and concluded that supertags improved the system's reordering.
\citet{hassan:09} then showed that an incremental \ccg parser could be integrated into a
phrase-based statistical machine translation system, leading to further improvements in
Arabic to English translation.

\section{Summary}

Combinatory categorial grammar is a lexicalised grammar formalism that provides
a transparent interface between syntax and semantics. A \ccg grammar mainly
consists of a set of lexical categories, which can be interpreted as functions
from some set of arguments to an atomic result category. The formalism uses a
very small inventory of rules, some of which were added to allow categories to
become partially associative. This associativity allows sentences to be
bracketed in a variety of ways, many of them featuring non-traditional
constituents --- for instance, subjects can be bracketed with verbs, in addition
to the traditional bracketing of verb and object. This so-called spurious ambiguity
has proven to be an advantage of the formalism, and the
complications it induces for practical parsing are easily overcome.

\ccg has become highly influential in computational linguistics, proving useful
in a variety of applications. It has become particularly prominent in statistical
parsing and semantic analysis. The most accurate wide-coverage \ccg parser, \candc,
is currently a leading parsing system, based on a variety of evaluations. The corpus
used to train the parser, \ccgbank, was created by semi-automatically converting the
Penn Treebank, a corpus that used a less informative representation than the target
analyses desired for \ccgbank. \ccgbank therefore includes certain some suboptimal
annotations. Corrections for some of these issues have been proposed
\citep[e.g. ][]{honnibal:pacling07prop,vadas:08,white:punct08}, but updated
versions of the corpus were released too late for use in this thesis.

The other problem we note with \ccgbank, the presence of unary and binary
type-changing rules, has received little attention. We will now explore the
problem that motivated the addition of these rules, and its consequences for a
wide-coverage \ccg grammar.

