

\usepackage{LI_fix}
\usepackage{parsetree}
\usepackage{avm}
\usepackage{graphicx}
\usepackage{lcovington}
\usepackage{xcolor}
\usepackage{multirow}
%\usetikzlibrary{shapes,arrows}
% \newcommand{\gis}{\textsc{gis}\xspace}
% \newcommand{\iis}{\textsc{iis}\xspace}
% \newcommand{\wsj}{\textsc{wsj}\xspace}
 \newcommand{\ccg}{\textsc{ccg}\xspace}
% \newcommand{\ltag}{\textsc{ltag}\xspace}
% \newcommand{\hpsg}{\textsc{hpsg}\xspace}
% \newcommand{\lbfgs}{\textsc{l-bfgs}\xspace}
% \newcommand{\bfgs}{\textsc{bfgs}\xspace}
% \newcommand{\cfg}{\textsc{cfg}\xspace}
% \newcommand{\pcfg}{\textsc{pcfg}\xspace}
% \newcommand{\nlp}{\textsc{nlp}\xspace}
% \newcommand{\dop}{\textsc{dop}\xspace}
% \newcommand{\lfg}{\textsc{lfg}\xspace}
% \newcommand{\pp}{\textsc{pp}\xspace}
% \newcommand{\cky}{\textsc{cky}\xspace}
% \newcommand{\gb}{\textsc{gb}\xspace}
% \newcommand{\mb}{\textsc{mb}\xspace}
% %\newcommand{\ram}{\textsc{ram}\xspace}
% \newcommand{\mpi}{\textsc{mpi}\xspace}
% \newcommand{\cpu}{\textsc{cpu}\xspace}
% \newcommand{\parseval}{\textsc{parseval}}
% \newcommand{\trec}{\textsc{trec}\xspace}
% \newcommand{\epsrc}{\textsc{epsrc}\xspace}
% \newcommand{\dt}{\textsc{dt}\xspace}
% \newcommand{\hmm}{\textsc{hmm}\xspace}
% \newcommand{\ghz}{\textsc{ghz}\xspace}
% \newcommand{\rasp}{\textsc{rasp}\xspace}
% \newcommand{\ldc}{\textsc{ldc}\xspace}
% \newcommand{\gr}{\textsc{gr}\xspace}
% \newcommand{\qa}{\textsc{qa}\xspace}
% \newcommand{\jj}{\textsc{jj}\xspace}
% \newcommand{\vbn}{\textsc{vbn}\xspace}
% \newcommand{\cd}{\textsc{cd}\xspace}
% \newcommand{\rp}{\textsc{rp}\xspace}
% \newcommand{\cc}{\textsc{cc}\xspace}
% \newcommand{\susanne}{\textsc{susanne}\xspace}
% \newcommand{\bandc}{\textsc{b{\small \&}c}\xspace}
 \newcommand{\ccgbank}{CCGbank\xspace}
% \newcommand{\candc}{\textsc{C}\&\textsc{C}\xspace}
% \newcommand{\cg}{\textsc{cg}\xspace}
% \newcommand{\penn}{\textsc{ptb}\xspace}
 \newcommand{\semf}[1]{\ensuremath{\mathrm{#1}}}
 \newcommand{\sematt}[1]{\ensuremath{\langle \text{\textsc{#1}}\rangle}}
% %\newcommand{\bs}{\backslash}
 \newcommand{\cf}[1]{\mbox{$\it{#1}$}}   % category font
\newcommand{\wsd}{\textsc{wsd}\xspace}
% \newcommand{\lexent}[2]{\mbox{#1}~\mbox{:=}~\cf{#2}}   % lexical entry
% \newcommand{\dep}[1]{\langle\mbox{\small{#1}}\rangle}
% \newcommand{\head}[1]{_{\mbox{\tiny{\it #1}}}}
% \newcommand{\lhead}{\{\mbox{\small{\it l}\/}\}}
% \newcommand{\feat}[1]{[\mbox{\footnotesize{\it #1}\/}]}
% \newcommand{\tfeat}[1]{[\mbox{\tiny{\it #1}\/}]}
% \newcommand{\reln}[1]{\mbox{\small{\it #1}\/}}
% \newcommand{\hd}[1]{\{#1\}}
% \newcommand{\dy}[1]{\langle#1\rangle}
% \newcommand{\tab}{\hspace*{0.3cm}}

\newcommand{\agent}[1]{{\color{blue} #1}}
\newcommand{\patient}[1]{{\color{purple} #1}}
\newcommand{\price}[1]{{\color{green} #1}}
\newcommand{\predicate}[1]{\textbf{#1}}

\newcommand{\clrone}[1]{{\color{red} #1}}
\newcommand{\clrtwo}[1]{{\color{blue} #1}}
\newcommand{\clrthree}[1]{{\color{violet} #1}}
\newcommand{\clrfour}[1]{{\color{olive} #1}}
\newcommand{\glow}[1]{\textbf{#1}}

\title{Matt's grand plan for natural language understanding (with \ccg)}

\author{Matthew Honnibal}

\date{\today}

\begin{document}

\titleslide

\section{Introduction}

\begin{points}{NLU is currently divided into several tasks}
\begin{block}{Example}
 \emph{The company was persuaded to buy Power Set.}
\end{block}

\p \glow{Syntax:} {\small ((The company) (was (persuaded (to buy (Power Set)))))}
\p \glow{Semantics:} persuade(x, company, buy(company, Power Set))
\p \glow{Sense Disambiguation:} \emph{company} as in \emph{army unit}?
\p \glow{Named Entity Recognition and Disambiguation:}\\ Power Set $\rightarrow$ {\small \url{http://en.wikipedia.org/PowerSet}}
\p \glow{Coreference? Sentiment? Discourse? More?}
\end{points}

\begin{plain}{Pros and cons of dividing up the task}
\begin{block}{Pros}
 \begin{itemize}
  \item \textbf{Reductionism:} It may be easier to make progress on the tasks in isolation.
  \item \textbf{Modularity:} Don't like one parser? Just plug in another!
 \end{itemize}
\end{block}
\begin{block}{Cons}
\begin{itemize}
 \item \textbf{Accuracy:} Information can only flow in one direction.
 \item \textbf{Efficiency:} The same work is repeated many times.
 \item \textbf{Plausibility:} Is a pipeline a realistic model of natural language
                             understanding? Should we be trying to find one?
\end{itemize}

\end{block}

\end{plain}

\begin{points}{Intuition behind joint modelling}
 \p $H(W_s)$: information to disambiguate the words in $s$
 \p $H(R_s)$: information to assign semantic role labels to $s$
 \p If word senses are good features for SRL, then $H(R_s|W_s) < H(R_s)$
 \p But if $H(R_s|W_s) < H(R_s)$, then $H(W_s|R_s) < H(W_s)$
 \p \textbf{If WSD helps SRL, then SRL must be able to help
    WSD.}
 \p \textbf{So: model $P(R_s, W_s)$ instead of $P(W_s)$ and $P(R_s|W_s)$}
 \p The grand plan: jointly model all the sentence understanding tasks by bringing all
    the information into a \ccg parse.
\end{points}

\section{CCG}

\begin{plain}{Categorial Grammar: few rules, complex categories}
\begin{table}
\vspace*{-10mm}
\centering
\caption{Categorial Grammar has only 2 rule schemas, and 3 atomic types.}
\begin{tabular}{cccc|c}
 \multicolumn{4}{c|}{Rules} & Types \\
\hline
  \cf{X} & $\rightarrow$ & \cf{X/Y} & \cf{Y} & \cf{N}\\
 \cf{X} & $\rightarrow$ & \cf{Y} & \cf{Y\bs X} & \cf{PP} \\
&&&&\cf{S}
\end{tabular}
\vspace*{-10mm}
\end{table}
\begin{table}
 \centering
\caption{Production rules get `translated' into complex categories.}
 \begin{tabular}{cccc|c}
   \multicolumn{4}{c|}{\textsc{psg}} & \textsc{cg} \\
\hline
  \cf{NP} & $\rightarrow$ & \cf{DT} & \cf{N'} & \cf{NP/N}\\
  \cf{PP} & $\rightarrow$ & \cf{IN} & \cf{NP} & \cf{PP/NP}\\
  \cf{S} & $\rightarrow$ & \cf{NP} & \cf{VP} & \cf{S\bs NP}\\
  \cf{VP} & $\rightarrow$ & \cf{V} & \cf{NP} & \cf{(S\bs NP)/NP}\\
  \cf{VP} & $\rightarrow$ & \cf{VP} & \cf{ADVP} &\cf{(S\bs NP)\bs (S\bs NP)}\\
 \end{tabular}
\end{table}
\end{plain}

\begin{plain}{Example Categorial Grammar Derivation}
 \begin{center}
\deriv{5}{
\rm The & \rm company & \rm bought & \rm Power & \rm Set \\
\uline{1}&\uline{1}&\uline{1}&\uline{1}&\uline{1} \\
\cf{NP/N} &
\cf{N} &
\cf{(S\bs NP)/NP} &
\cf{NP/NP} &
\cf{NP} \\
\fapply{2} && \fapply{2} \\
\mc{2}{\cf{NP}} && \mc{2}{\cf{NP}} \\
&& \fapply{3} \\
&& \mc{3}{\cf{S\bs NP}} \\
\bapply{5} \\
\mc{5}{\cf{S}}
}\end{center}
\end{plain}

\begin{plain}{The pay-off: semantic transparency}
  \begin{center}
\deriv{5}{
\rm The & \rm company & \rm bought & \rm Power & \rm Set \\
\uline{1}&\uline{1}&\uline{1}&\uline{1}&\uline{1} \\
\cf{\clrone{NP_w}/\clrone{N_w}} &
\cf{\clrone{N}} &
\cf{(S\bs \clrone{NP_x})/\clrtwo{NP_y}} &
\cf{\clrtwo{NP_z}/\clrtwo{NP_z}} &
\cf{\clrtwo{NP}} \\
\textrm{spec(\clrone{w})}  &
\textrm{\clrone{company}} &
\textrm{buy(\clrone{x}, \clrtwo{y})}&
\textrm{\clrtwo{Power\_z}}&
\textrm{\clrtwo{Set}} \\
\fapply{2} && \fapply{2} \\
\mc{2}{\cf{\clrone{NP}}} && \mc{2}{\cf{\clrtwo{NP}}} \\
\mc{2}{\textrm{\clrone{company}}} && \mc{2}{\textrm{\clrtwo{Power\_Set}}} \\
&& \fapply{3} \\
&& \mc{3}{\cf{S\bs \clrone{NP_x}}} \\
&& \mc{3}{\textrm{buy(\clrone{x}, \clrtwo{Power\_Set})}} \\
\bapply{5} \\
\mc{5}{\cf{S}}\\
\mc{5}{\textrm{buy(\clrone{company}, \clrtwo{Power\_Set})}}
}\end{center}
\end{plain}

\begin{plain}{\ccg adds more rules to reduce category ambiguity}
 \begin{figure}
\small
\centering
\deriv{5}{
\rm The & \rm company & \rm which & \rm they & \rm bought \\
\uline{1}&\uline{1}&\uline{1}&\uline{1}&\uline{1} \\
\cf{NP/N} &
\cf{N} &
\cf{(NP\bs NP)/(S/NP)} &
\cf{NP} &
\cf{(S/NP)\bs NP} \\
\fapply{2} && \bapply{2} \\
\mc{2}{\cf{NP}} && \mc{2}{\cf{S/NP}} \\
&& \fapply{3} \\
&& \mc{3}{\cf{NP\bs NP}} \\
\bapply{5} \\
\mc{5}{\cf{NP}}
}\end{figure}

\begin{figure}
\small
\centering
\deriv{5}{
\rm The & \rm company & \rm which & \rm they & \rm bought \\
\uline{1}&\uline{1}&\uline{1}&\uline{1}&\uline{1} \\
\cf{NP/N} &
\cf{N} &
\cf{(NP\bs NP)/(S/NP)} &
\cf{NP} &
\cf{(S\bs NP)/NP} \\
\fapply{2} && \ftype{1} \\
\mc{2}{\cf{NP}} && \mc{1}{\cf{S/(S\bs NP)}} \\
&&& \fcomp{2} \\
&&& \mc{2}{\cf{S/NP}} \\
&& \fapply{3} \\
&& \mc{3}{\cf{NP\bs NP}} \\
\bapply{5} \\
\mc{5}{\cf{NP}}
}\end{figure}
\end{plain}

\section[CCG SRL]{Semantic Role Labelling with CCG}

\begin{points}{PropBank and NomBank: Penn Treebank SRL layers}
 \p A \textbf{predicate} heads a proposition (but might not assert it)
 \p \textbf{Arguments} can be \textbf{core} or \textbf{peripheral}.
\begin{block}{Example predicate-argument structures}
 \begin{lexamples}
  \item \gll Google bought YouTube October~2006 for~1.6bn
             Arg-0  Predicate Arg-1 Arg-TMP Arg-3
  \gln
  \glend
   \item \gll Google paid 1.6bn for~YouTube October~2006
             Arg-0  Predicate Arg-1 Arg-3 Arg-TMP
  \gln
  \glend
  \item \gll Google's 1.6bn acquisition of~YouTube October~2006
             Arg-0    Arg-1 Predicate      Arg-3   Arg-TMP
        \gln
        \glend
 \end{lexamples}
\end{block}

\p \textbf{PropBank:} Propositions headed by \textbf{verbs}  in the PTB.
\p \textbf{NomBank:} Propositions headed by \textbf{nouns} in the PTB.
\end{points}

\begin{points}{Integrating PropBank annotation into CCG}
 \p Target: CCG derivations that map unambiguously to PropBank analyses.
 \p \textbf{Predicates} will be identified by the \textbf{semantic
                                category} assigned to them.
 \p \textbf{Core arguments} will be \textbf{syntactic complements}.
    Argument labels will be assigned by the syntax-semantics mapping.
 \p \textbf{Peripheral arguments} will be \textbf{syntactic adjuncts}. Their type will
    be specified in their semantics.
\end{points}

\begin{plain}{Distinguishing core and peripheral arguments in CCG}
\centering
 \begin{center}
\small
\deriv{6}{
\rm Google & \rm paid & \rm 1.6b & \rm for & \rm YouTube & \rm Oct.~2006 \\
\uline{1}&\uline{1}&\uline{1}&\uline{1}&\uline{1}&\uline{1} \\
\cf{NP} &
\cf{((S\bs NP)/PP)/NP} &
\cf{NP} &
\cf{PP/NP} &
\cf{NP} &
\cf{(S\bs NP)\bs (S\bs NP)} \\
& \fapply{2} & \fapply{2} \\
& \mc{2}{\cf{(S\bs NP)/PP}} & \mc{2}{\cf{PP}} \\
& \fapply{4} \\
& \mc{4}{\cf{S\bs NP}} \\
& \bapply{5} \\
& \mc{5}{\cf{S\bs NP}} \\
\bapply{6} \\
\mc{6}{\cf{S}}
}\end{center}
\end{plain}

\begin{plain}{Compositional semantics for SRL with CCG}
\small
\begin{center}
\scalebox{0.90}{
\deriv{6}{
\rm YT, & \rm which & \rm Google & \rm paid & \rm 1.6b & \rm for \\
\uline{1}&\uline{1}&\uline{1}&\uline{1}&\uline{1}&\uline{1} \\
\cf{\clrone{NP}} &
\cf{(\clrone{NP_y}\bs \clrone{NP_y})/(S/\clrone{NP_y})} &
\cf{\clrtwo{NP}} &
\cf{((S\bs \clrtwo{NP_y})/\clrone{PP_z})/\clrthree{NP_w}} &
\cf{\clrthree{NP}} &
\cf{\clrone{PP_y}/\clrone{NP_y}} \\
&
 &
&
\textrm{pay(\clrtwo{y}, \clrthree{w}, u, \clrone{z})} &
&
 \\
&& \ftype{1} & \fapply{2} \\
&& \mc{1}{\cf{S_y/(S_y\bs \clrtwo{NP})}} & \mc{2}{\cf{(S\bs \clrtwo{NP_y})/\clrone{PP_z}}} \\
&&& \mc{2}{\textrm{pay(\clrtwo{y}, \clrone{z}, \clrthree{1.6b})}}\\
&&& \fcomp{3} \\
&&& \mc{3}{\cf{(S\bs \clrtwo{NP_y})/\clrone{NP_z}}} \\
&&& \mc{3}{\textrm{pay(\clrtwo{y}, \clrthree{1.6b}, u, \clrone{z})}}\\
&& \fcomp{4} \\
% && \mc{4}{\cf{S/\clrone{NP_z}}} \\
% && \mc{4}{\textrm{pay(\clrtwo{Google}, \clrthree{1.6b}, u, \clrone{z})}}\\
% & \fapply{5} \\
& \mc{5}{\cf{\clrone{NP_y}\bs \clrone{NP_y}}} \\
\bapply{6} \\
\mc{6}{\cf{\clrone{NP}}}\\
\mc{6}{\textrm{pay(\clrtwo{Google}, \clrthree{1.6b}, u, \clrtwo{YT})}}\\
}
}\end{center}
\end{plain}



\begin{plain}{Compositional semantics for nominal predicates in CCG}
\begin{center}
\deriv{3}{
\rm acquisition & \rm of & \rm YouTube \\
\uline{1}&\uline{1}&\uline{1} \\
\cf{N_{acquisition}/\clrone{PP_b}} &
\cf{\clrone{PP_c}/\clrone{NP_c}} &
\cf{\clrone{NP_\semf{YouTube}}} \\
\cf{\text{acquire(\emph{a, \clrone{b}})}} & & \\
& \fapply{2} \\
& \mc{2}{\cf{\clrone{PP_\semf{YouTube}}}} \\
\fapply{3}\\
\mc{3}{\cf{N_\semf{acquisition}}}\\
\mc{3}{\cf{\text{acquire(\emph{a}, \clrone{YouTube})}}}\\
}
\end{center}
\end{plain}

\begin{plain}{Nominal predicates require some creative analyses}
\begin{center}
\deriv{3}{
\rm YouTube & \rm 's & \rm acquisition \\
\uline{1}&\uline{1}&\uline{1} \\
\cf{\clrone{NP_\semf{YouTube}}} &
\cf{(\clrtwo{NP_a}/(\clrtwo{N_a}/\clrone{PP_b}))\bs \clrone{NP_b}} &
\cf{\clrtwo{N_\semf{acquisition}}/\clrone{PP_d}}\\
& & \cf{\text{acquire(\emph{c}, \emph{\clrone{d}})}}\\
\bapply{2} \\
\mc{2}{\cf{\clrtwo{NP_a}/(\clrtwo{N_a}/\clrone{PP_\semf{YouTube}})}} \\
\fapply{3} \\
\mc{3}{\cf{\clrtwo{NP_\semf{acquisition}}}}\\
\mc{3}{\cf{\text{acquire(\emph{c}, \clrone{YouTube})}}}\\
}
\end{center}
\end{plain}

\begin{plain}{Nominal predicates with support verbs}
\begin{center}
 \deriv{4}{
\rm Google & \rm made & \rm a & \rm decision \\
\uline{1}&\uline{1}&\uline{1}&\uline{1} \\
\cf{\clrone{NP_\semf{Google}}} &
\cf{(S\bs \clrone{NP_a})/(\clrtwo{NP_b}/\clrone{PP_a})} &
\cf{\clrtwo{NP_c}/\clrtwo{N_c}} &
\cf{\clrtwo{N_\semf{decision}}/\clrone{PP_d}} \\
& & & \cf{\text{decide(\emph{\clrtwo{d}}, \emph{e})}}\\
&& \fcomp{2} \\
&& \mc{2}{\cf{\clrtwo{NP_\semf{decision}}/\clrone{PP_d}}} \\
&& \mc{2}{\cf{\text{decide(\clrone{\emph{d}}, \emph{e})}}}\\
& \fapply{3} \\
& \mc{3}{\cf{S\bs \clrone{NP_d}}} \\
& \mc{3}{\cf{\text{decide(\clrone{\emph{d}}, \emph{e})}}}\\
\bapply{4} \\
\mc{4}{\cf{S}}\\
\mc{4}{\cf{\text{decide(\clrone{Google}, \emph{e})}}}\\
}
\end{center}
\end{plain}

\section[CCG NER]{Named Entity Recognition in CCG (aka Nicky's PhD project)}

\begin{points}{Joint Named Entity Recognition and PTB parsing}
 \p Named entities recognition is usually modelled as a \textbf{sequence tagging} task, e.g. 
   \begin{itemize} \item \textbf{Power$|$ORG Set$|$ORG}\end{itemize}
 \p This makes it difficult to account for \textbf{nested named entities}, e.g.
   \begin{itemize}
     \item \textbf{New York Stock Exchange}
     \item \textbf{Sydney, Australia}
     \item \textbf{David and Melissa Smith}
   \end{itemize}
 \p Finkel and Manning (2009) joint \textsc{ner} and parsing:
    \begin{itemize}
     \item Up to 1.36\% F-measure parsing improvement;
     \item Up to 9\% F-measure \textsc{ner} improvement.
    \end{itemize}
\end{points}

\begin{plain}{Named entities screw up CCG parses if handled naively}
\begin{center}
\scalebox{0.8}{
\deriv{3}{
\rm October & \rm 26 & \rm 2006 \\
\uline{1}&\uline{1}&\uline{1} \\
\cf{(((VP\bs VP)/(VP\bs VP)))/(((VP\bs VP)/(VP\bs VP)))} &
\cf{(VP\bs VP)/(VP\bs VP)} &
\cf{VP\bs VP} \\
\fapply{2} \\
\mc{2}{\cf{(VP\bs VP)/(VP\bs VP)}} \\
\fapply{3} \\
\mc{3}{\cf{VP\bs VP}}
}}\end{center}

\begin{center}
\scalebox{0.8}{
\deriv{5}{
\rm The & \rm Grand & \rm Rapids, & \rm MI & \rm man \\
\uline{1}&\uline{1}&\uline{1}&\uline{1}&\uline{1} \\
\cf{NP/N} &
\cf{((N/N)/(N/N))/((N/N)/(N/N))} &
\cf{(N/N)/(N/N)} &
\cf{N/N} &
\cf{N} \\
&& \fapply{2} \\
&& \mc{2}{\cf{(N/N)/(N/N)}} \\
& \fapply{3} \\
& \mc{3}{\cf{N/N}} \\
& \fapply{4} \\
& \mc{4}{\cf{N}} \\
\fapply{5} \\
\mc{5}{\cf{NP}}
}}\end{center}

\end{plain}

\begin{plain}{Integrating NER into CCG with Hat Categories}
\begin{center}
\scalebox{0.8}{
\deriv{3}{
\rm October & \rm 26 & \rm 2006 \\
\uline{1}&\uline{1}&\uline{1} \\
\cf{MON/DAY} &
\cf{DAY} &
\cf{DATE^{VP\bs VP}\bs MON} \\
\fapply{2} \\
\mc{2}{\cf{MON}} \\
\bapply{3} \\
\mc{3}{\cf{DATE^{VP\bs VP}}} \\
\unhat{3} \\
\mc{3}{\cf{VP\bs VP}}
}}\end{center}
 \begin{center}
\scalebox{0.8}{
\deriv{5}{
\rm The & \rm Grand & \rm Rapids & \rm MI & \rm man \\
\uline{1}&\uline{1}&\uline{1}&\uline{1}&\uline{1} \\
\cf{NP/N} &
\cf{CITY/CITY} &
\cf{CITY^{N/N}/STATE} &
\cf{STATE} &
\cf{N} \\
&& \fapply{2} \\
&& \mc{2}{\cf{CITY^{N/N}}} \\
& \fapply{3} \\
& \mc{3}{\cf{CITY^{N/N}}} \\
& \unhat{3} \\
& \mc{3}{\cf{N/N}} \\
& \fapply{4} \\
& \mc{4}{\cf{N}} \\
\fapply{5} \\
\mc{5}{\cf{NP}}
}}
\end{center}
\end{plain}

\section[WSD CCG]{Word Sense Disambiguation and CCG}

\begin{points}{Tentative thoughts on Word Sense Disambiguation}
\p Full word sense disambiguation involves many fine-grained labels
\p Integrating these labels into \ccg category sets may cause sparse data problems
\p What if I just use super senses and WordNet Domains?
\p 
\begin{itemize}
 \item 41 supersenses e.g. noun.food, noun.group, verb.cognition.
 \item 46 domains, e.g. economy, sport, fashion, sexuality
\end{itemize}

\end{points}

\begin{plain}{Adding SuperSenses and domains as category features}
 \begin{figure}
\deriv{5}{
\rm The & \rm board & \rm grasped & \rm the & \rm problem \\
\uline{1}&\uline{1}&\uline{1}&\uline{1}&\uline{1} \\
\cf{NP/N} &
\cf{N[group,econ]} &
\cf{(S[cog]\bs NP)/NP} &
\cf{NP/N} &
\cf{N[cog]} \\
\fapply{2} && \fapply{2} \\
\mc{2}{\cf{NP[group,econ]}} && \mc{2}{\cf{NP[cog]}} \\
&& \fapply{3} \\
&& \mc{3}{\cf{S[cog]\bs NP}} \\
\bapply{5} \\
\mc{5}{\cf{S[cog]}}
}\end{figure}
\end{plain}

\begin{plain}{WordNet senses for `board' and `problem'}
\scalebox{0.8}{
 \begin{tabular}{r|c|p{8.5cm}}
\hline
Sense & Super sense & Definition \\
\hline
\hline
1 &  noun.group & A committee having supervisory powers \emph{the board has seven members}\\
2 & noun.substance & A stout length of sawn timber; made in a wide variety of sizes and used for many purposes \\
4 & noun.food      & Food or meals in general \emph{room and board}\\
9 & noun.artifact & A flat portable surface (usually rectangular) designed for board games. \emph{he got out the board and set up the pieces}\\
\hline
1     & noun.state  & A state of difficulty that needs to be resolved: \emph{she and her husband are having problems}\\
2     &noun.communication & A question raised for consideration or solution: \emph{our homework consisted of ten problems to solve}\\
3     & noun.cognition    & A source of difficulty \emph{one trouble after another delayed the job}\\
\hline
 \end{tabular}
}
\end{plain}

\section{Conclusion}

\begin{points}{Progress so far}
 \p PropBank/CCGbank integration complete
 \p Most difficult NomBank/CCGbank integration complete
 \p Preliminary parsing experiments on modified corpora
 \p Nicky has BBN and CCG aligned and is working on the integration
 \p Mike White's group have done something with CCG and discourse parsing
\end{points}

\begin{points}{Current priorities}
 \p Get oracle figures for CCGbank-to-SRL
 \p Error analysis over oracle errors. Further improvements? Problems with CCG?
 \p Parse with SRL-CCGbank to get joint model performance.
 \p Tinker with WSD/CCG ideas at some point.
\end{points}

\begin{points}{Conclusion}
 \p I am focussing on a representation problem, rather than the learning problem. But
    do these tasks all fit in one hypothesis space? Will the task be tractable?
 \p It's currently very difficult to deploy a system that makes use of all the NLU modules.
 \p If my approach works, it will produce a very efficient all-singing-all-dancing NLU solution.
 \p The project also raises a lot of questions about our current theories of compositional semantics.
  
\end{points}


\end{document}

